% ==========================================
% BAB II STUDI LITERATUR
% ==========================================
\chapter{STUDI LITERATUR}

Bab ini membahas konsep dan penelitian terdahulu yang menjadi landasan bagi analisis pengaruh \textit{user persona} terhadap perilaku \textit{large language model}. Pembahasan disusun secara bertahap, dimulai dari uraian mengenai model bahasa modern, mekanisme pemrosesan instruksi, konsep dasar persona, serta temuan empiris mengenai sensitivitas model terhadap identitas pengguna. Selain itu, bab ini meninjau isu bias dan metode evaluasi penalaran yang relevan bagi perancangan penelitian ini.

% -------------------------------------------------------------
\section{Large Language Model}

%-- 2.1 Large Language Model --
\subsection{Konsep dan Karakteristik Dasar}

\textit{Large language model} (LLM) merupakan model generatif berbasis arsitektur transformator yang dilatih menggunakan data dalam skala sangat besar. Model ini mempelajari pola bahasa melalui hubungan antartoken, sehingga mampu membangun representasi yang mencakup makna, hubungan semantik, serta isyarat pragmatik yang muncul dalam teks. Dengan skala pelatihan yang luas, LLM dapat digunakan pada berbagai tugas tanpa memerlukan penyesuaian khusus untuk setiap tugas.

Secara konseptual, LLM bekerja dengan memprediksi token berikutnya berdasarkan konteks sebelumnya. Namun, proses prediksi ini tidak sekadar berbasis frekuensi kata, melainkan menggunakan representasi kontekstual yang memungkinkan model memahami instruksi, gaya penulisan, maupun kecenderungan komunikasi. Model seperti GPT, LLaMA, Mistral, dan Gemini mengadopsi pendekatan ini dan menunjukkan kemampuan generalisasi yang kuat terhadap tugas bahasa yang kompleks.

Karakteristik utama LLM antara lain fleksibilitas dalam mengikuti instruksi, kemampuan menyusun penalaran, serta penyesuaian terhadap pola komunikasi pengguna. Kemampuan ini muncul dari kombinasi arsitektur dasar transformator, skala parameter yang besar, dan keragaman data pelatihan. Karena model tidak dibuat untuk satu domain tertentu, tetapi dilatih pada data lintas konteks, gaya, dan situasi, LLM dapat mengadaptasi perilaku komunikasinya berdasarkan variasi kecil dalam instruksi.

\subsection{Representasi Bahasa dan Pemahaman Instruksi}

LLM memproses teks melalui beberapa tahapan representasi internal. Teks diuraikan menjadi token, kemudian dipetakan ke dalam ruang representasi berdimensi tinggi melalui \textit{embedding}. Representasi awal ini kemudian diperkaya melalui lapisan-lapisan transformator yang memanfaatkan mekanisme perhatian untuk menentukan hubungan antar token dalam konteks yang lebih luas. Hasilnya adalah representasi kontekstual yang mencerminkan interpretasi model terhadap instruksi atau percakapan.

Representasi ini tidak bersifat statis. Makna sebuah token dapat berubah bergantung pada cara pengguna menyampaikan instruksi. Perbedaan gaya penulisan, urutan informasi, atau tingkat formalitas dapat menghasilkan representasi internal yang berbeda, sehingga memunculkan respons yang berbeda pula. Penelitian Zhou et al. \parencite{zhou2023largemodelsensitive} menunjukkan bahwa perubahan kecil dalam framing, seperti perbedaan nada atau cara bertanya, dapat menggeser perhatian model dan mengubah struktur jawaban yang dihasilkan.

Sebagai ilustrasi, perbedaan instruksi berikut sering kali menghasilkan respons yang berbeda meskipun inti pertanyaannya sama:
\begin{itemize}
    \item “Jelaskan secara singkat apa itu regularisasi.”
    \item “Saya sedang menulis laporan akademik. Bisakah Anda menjelaskan secara formal apa yang dimaksud dengan regularisasi?”
\end{itemize}
Instruksi kedua biasanya memicu model untuk memberikan penjelasan yang lebih panjang, lebih berhati-hati, dan lebih formal. Perbedaan ini mencerminkan bagaimana representasi instruksi terbentuk berdasarkan konteks linguistik dan pragmatik.

\subsection{Penalaran dan Dinamika Perilaku Model}

Selain pemahaman instruksi, LLM juga menunjukkan kemampuan melakukan penalaran. Model dapat menyelesaikan soal penalaran numerik sederhana, menjawab pertanyaan berbasis pengetahuan umum, hingga memberikan penilaian terhadap skenario sosial atau moral. Namun, kemampuan ini tidak sepenuhnya stabil. Turpin et al. \parencite{turpin2023language} menemukan bahwa penalaran yang dihasilkan model dapat berubah hanya karena variasi kecil pada bentuk instruksi, walaupun substansi tugas tetap sama.

Hal ini terjadi karena model tidak melakukan penalaran melalui prosedur logis eksplisit, tetapi melalui dinamika representasi internal yang sensitif terhadap konteks. Sebuah instruksi yang lebih panjang atau lebih formal dapat memicu struktur penalaran yang lebih sistematis, sementara instruksi yang lebih langsung dapat menghasilkan jawaban tanpa uraian langkah-langkah penalaran yang jelas. Perubahan ini memperlihatkan bahwa struktur penalaran yang muncul merupakan fungsi dari konteks interaksi, bukan semata-mata fungsi dari logika masalah yang diberikan.

Ketidakstabilan ini penting untuk dipahami karena berhubungan langsung dengan penelitian mengenai \textit{user persona}. Jika perubahan kecil pada instruksi dapat mengubah penalaran, maka variasi identitas pengguna yang tersirat dalam tulisan juga berpotensi memicu perubahan serupa.

\subsection{Dimensi Sosial dalam Pemrosesan Bahasa}

Model bahasa modern tidak hanya mempelajari struktur dan makna bahasa, tetapi juga pola interaksi sosial yang tercermin dalam data pelatihan. Weidinger et al. \parencite{weidinger2021ethical} menunjukkan bahwa LLM dapat menginternalisasi norma sosial, stereotip, serta pola komunikasi yang umum digunakan manusia. Dalam banyak kasus, gaya bahasa tertentu diinterpretasikan sebagai sinyal sosial mengenai siapa pengguna tersebut, misalnya usia, latar profesional, atau tingkat pendidikan.

Ketika instruksi ditulis dengan gaya santai, model sering kali memberikan respons yang lebih ringkas atau lebih langsung. Sebaliknya, ketika instruksi ditulis dengan gaya formal, respons yang dihasilkan cenderung lebih berhati-hati dan mengikuti struktur penjelasan akademis. Perbedaan respons ini bukan sekadar akibat gaya penulisan, tetapi akibat inferensi sosial yang dilakukan model berdasarkan pola komunikasi dalam data pelatihan.

Fenomena ini menunjukkan bahwa pemrosesan bahasa oleh LLM memiliki dimensi sosial yang signifikan. Instruksi diperlakukan bukan hanya sebagai teks, tetapi sebagai bentuk interaksi manusia yang membawa sinyal identitas. Sensitivitas terhadap sinyal ini merupakan salah satu alasan mengapa \textit{user persona} dapat memengaruhi penalaran, struktur respons, maupun kecenderungan bias dalam keluaran model.

% --2.2 Persona dalam Interaksi Model Bahasa --
\section{Persona dalam Interaksi Model Bahasa}

\subsection{Definisi dan Ruang Lingkup Persona}

Dalam kajian sistem bahasa alami, \textit{persona} merujuk pada serangkaian atribut yang digunakan untuk menggambarkan identitas atau karakteristik pengguna. Atribut tersebut dapat berupa informasi sosial, demografis, profesional, atau gaya komunikasi yang merepresentasikan cara seseorang berinteraksi dalam percakapan. Persona berfungsi sebagai konteks tambahan yang dapat memengaruhi bagaimana sebuah sistem dialog memahami maksud pengguna dan membentuk respons.

Dalam konteks \textit{large language model}, persona tidak hanya dipandang sebagai label identitas, tetapi juga sebagai bagian dari sinyal yang terkandung dalam bahasa. Karena model belajar dari data pelatihan yang mencerminkan cara manusia berkomunikasi, model juga mempelajari keterkaitan antara gaya bahasa dan identitas sosial. Dengan demikian, persona tidak hanya bekerja sebagai informasi eksplisit, tetapi dapat tersirat melalui variasi linguistik seperti pilihan kata, nada, struktur kalimat, atau keformalan tulisan.

Ruang lingkup persona dalam sistem bahasa mencakup berbagai kategori identitas, seperti gender, usia, minat, latar profesional, afiliasi budaya, ataupun preferensi komunikasi. Representasi persona tersebut tidak selalu hadir dalam bentuk pernyataan langsung, tetapi sering kali dinyatakan melalui konteks linguistik yang halus tanpa deklarasi eksplisit mengenai siapa pengguna tersebut.

\subsection{Persona Eksplisit dan Persona Implisit}

Fenomena persona dalam interaksi dengan model bahasa dapat dibagi menjadi dua bentuk utama, yaitu persona eksplisit dan persona implisit. Keduanya memberikan sinyal identitas, tetapi melalui mekanisme dan intensitas yang berbeda.

Persona eksplisit muncul ketika identitas pengguna dinyatakan secara langsung dalam instruksi atau konteks percakapan. Contohnya adalah ketika pengguna menuliskan “Saya adalah mahasiswa teknik informatika” atau “Sebagai seorang dokter, saya ingin memahami...”. Ungkapan seperti ini memberikan sinyal yang jelas kepada model mengenai latar pengguna, sehingga model dapat menyesuaikan struktur respons agar lebih sesuai dengan karakteristik tersebut. Gupta et al. \parencite{gupta2024biasrunsdeep} menunjukkan bahwa penugasan persona eksplisit semacam ini dapat mengubah hasil penalaran model, meskipun tugas yang diberikan tidak berkaitan dengan identitas sosial pengguna. Perubahan respons tidak hanya menyangkut gaya bahasa, tetapi juga dapat memengaruhi kesimpulan logis yang diberikan model.

Sebaliknya, persona implisit muncul ketika identitas pengguna tidak dinyatakan secara langsung, tetapi disimpulkan oleh model berdasarkan isyarat linguistik. Penelitian Tseng et al. \parencite{tseng2024twotales} menunjukkan bahwa model memiliki kecenderungan melakukan inferensi identitas pengguna dari gaya penulisan, struktur kalimat, pilihan kata, atau tingkat formalitas. Fenomena ini dapat terjadi meskipun pengguna tidak bermaksud menyampaikan identitas tertentu. Sebagai contoh, gaya penulisan formal dengan istilah akademis sering diasosiasikan dengan latar pendidikan tertentu, sedangkan gaya penulisan santai dapat diasosiasikan dengan kategori usia atau tingkat kedekatan sosial.

Inferensi identitas tersebut bukan hasil dari aturan yang ditetapkan secara eksplisit dalam model, tetapi merupakan konsekuensi dari pola komunikasi manusia yang terserap selama proses pelatihan. Model mempelajari bahwa gaya bahasa tertentu sering muncul bersama atribut sosial tertentu, sehingga ketika gaya tersebut muncul dalam instruksi, model cenderung mengaktifkan pola respons yang sesuai dengan kategori identitas yang diasosiasikan. Fenomena ini menjadi dasar penting bagi studi mengenai pengaruh persona implisit terhadap perilaku dan penalaran model.

\subsection{Peran Persona dalam Interaksi dengan LLM}

Persona, baik eksplisit maupun implisit, berperan sebagai sinyal kontekstual yang memengaruhi interpretasi dan respons model bahasa. Ketika identitas pengguna muncul dalam bentuk atribut sosial atau gaya komunikasi tertentu, model akan memperlakukannya sebagai bagian dari konteks yang relevan. Konteks ini kemudian membentuk representasi internal yang memengaruhi bagaimana model memahami pertanyaan, menafsirkan maksud, dan menyusun jawaban.

Peran persona dalam interaksi ini dapat dilihat dari dua dimensi utama. Pertama, persona dapat memengaruhi aspek linguistik respons, seperti pilihan kata, tingkat formalitas, pola argumentasi, atau struktur penjelasan. Model cenderung menyesuaikan respons agar selaras dengan gaya komunikasi yang diasosiasikan dengan persona tertentu. Kedua, persona dapat memengaruhi penalaran model melalui apa yang disebut sebagai \textit{reasoning shift}, yaitu perubahan struktur penalaran yang terjadi akibat variasi identitas pengguna meskipun subtansi tugas tetap sama.

Sebagai ilustrasi, suatu pertanyaan logika sederhana yang diajukan oleh pengguna dengan persona profesional tertentu dapat memicu model untuk memberikan respons yang lebih sistematis atau lebih berhati-hati. Sebaliknya, pertanyaan yang diajukan dengan gaya informal dapat menghasilkan respons yang lebih ringkas dengan struktur penalaran minimal. Perubahan ini menunjukkan bahwa persona berfungsi sebagai variabel kondisi yang membentuk dinamika interaksi antara pengguna dan model.












\label{chap:studi-literatur}
\section{Penulisan Gambar, Tabel, Rumus, dan Kode}
\lipsum[1]

\subsection{Gambar}
Contoh gambar dapat dilihat pada Gambar \ref{gambar:jaringan}. Gambar dan judulnya diposisikan di tengah. Nomor gambar tidak diakhiri tanda titik. Gambar tersebut dibuat menggunakan aplikasi draw.io dan disimpan ke format PNG setelah dengan zoom setting pada angka 300\%. Ukuran gambar yang ditampilkan dapat diatur dengan mengubah nilai \textit{width} dalam sintaks \textit{includegraphics}.

\begin{figure}[t] % pilihan opsi yang disarankan: t = top, b = bottom, h = here
	\centering
  \captionsetup{justification=centering}
    	\includegraphics[width=0.7\textwidth]{image/gambar1.png}
	\caption{Contoh gambar jaringan}
	\label{gambar:jaringan}
\end{figure}

Gambar umumnya tidak jelas atau kabur jika gambar tersebut:
\begin{enumerate}[a.]
  \item diperoleh dari hasil cropping pada suatu halaman buku atau situs web;
  \item hasil pembesaran gambar yang gambar aslinya sebenarnya berukuran kecil; atau
  \item disimpan dalam resolusi kecil
\end{enumerate}
Ketidakjelasan gambar ini dapat dilihat pada garis-garis diagram yang tidak tegas dan tulisan-tulisan dalam gambar yang tampak kabur dan kurang jelas terbaca.

Untuk mendapatkan gambar yang tidak kabur (\textit{blur}), langkah-langkah berikut dapat digunakan:
\begin{enumerate}[(a)]
\item Gambar yang didapat di suatu pustaka atau referensi sebaiknya digambar ulang, misalnya menggunakan PowerPoint, Canva, Figma, draw.io, atau yang lainnya.
\item Jika diagram atau ilustrasi digambar menggunakan draw.io, saat gambar disimpan ke format PNG atau JPG (\textit{export as}), lakukan \textit{zoom} ke minimal 300\% (\textit{the default value is} 100\%). 
\item Jika diagram digambar dengan menggunakan PowerPoint, gambar dapat langsung di-\textit{copy-paste} ke Word.
\end{enumerate}

\subsection{Tabel}
Tabel ada dua jenis, yaitu tabel yang bisa termuat dalam satu halaman dan tabel yang sangat panjang sehingga tidak muat dalam satu halaman.
\subsubsection{Tabel yang Muat dalam Satu Halaman}
Contoh tabel dapat dilihat pada Tabel \ref{tbl:harga1} dan \ref{tbl:harga2}. Tabel dan judulnya dibuat rata kiri dan judul tabel diletakkan di atas tabel. Usahakan tabel dapat ditulis dalam satu halaman, tidak terpotong ke halaman berikutnya.

\begin{table}[t] % pilihan opsi yang disarankan: t = top, b = bottom, h = here
  \begin{tabular}{ | p{2cm} | p{2cm} | p{3cm} |}
	\hline
	Nama 	& Satuan 		& Harga \\
	\hline
	Buku 	& Exemplar	& 25000 \\
	Komputer	& Unit		& 2500000 \\
	Pensil	& Buah		& 118900 \\
	\hline
	\end{tabular}
\caption{Tabel harga bahan pokok}
\label{tbl:harga1}
\end{table}



\begin{table}[t] % pilihan opsi yang disarankan: t = top, b = bottom, h = here
	\begin{tabular}{ | l | c | r | }
	\hline
	Nama 	& Satuan 		& Harga \\
	\hline
	Buku 	& Exemplar	& 25000 \\
	Komputer	& Unit		& 2500000 \\
	Pensil	& Buah		& 118900 \\
	\hline
	\end{tabular}
\caption{Tabel harga bahan sekunder}
\label{tbl:harga2}
\end{table}

% -- Example of importing table from external file --
\subsubsection{Mengimpor Tabel dari Berkas Eksternal}

Tabel \ref{tbl:harga3} diimpor dari berkas eksternal \textit{table/tabel1.tex} menggunakan perintah \textit{input}. 
Dengan demikian, jika tabel tersebut perlu diubah, cukup mengubah pada berkas eksternal tersebut tanpa perlu mengubah pada berkas utama ini.

\input table/tabel1.tex


% -- Example of long table --
\subsubsection{Tabel yang Sangat Panjang}
Jika tabel terlalu panjang sehingga tidak muat dalam satu halaman, gunakan paket 
\textit{longtable} untuk membuat tabel yang dapat terpotong ke halaman berikutnya, 
seperti pada Tabel \ref{tbl:longtable1}.

\begin{longtable}{@{\extracolsep{\fill}} l c r r}
\caption{Comprehensive Data Table Example}\label{tbl:longtable1} \\
\toprule
\textbf{ID} & \textbf{Name} & \textbf{Score} & \textbf{Rank} \\
\midrule
\endfirsthead

\caption{Comprehensive Data Table Example (lanjutan)} \\
\toprule
\textbf{ID} & \textbf{Name} & \textbf{Score} & \textbf{Rank} \\
\midrule
\endhead

\midrule
\multicolumn{4}{r}{\textit{Bersambung ke halaman berikutnya}} \\
%\bottomrule
\endfoot

\bottomrule
\endlastfoot

% === Table Data ===
1 & Alice Smith & 89 & 5 \\
2 & Bob Johnson & 93 & 3 \\
3 & Carol Davis & 95 & 2 \\
4 & Daniel Wilson & 88 & 6 \\
5 & Eve Thompson & 97 & 1 \\
6 & Frank Brown & 85 & 7 \\
7 & Grace Lee & 91 & 4 \\
8 & Henry Miller & 80 & 9 \\
9 & Irene Garcia & 83 & 8 \\
10 & Jack Robinson & 78 & 10 \\
% Repeat with more rows to make it long
11 & Kevin Harris & 76 & 11 \\
12 & Laura Martin & 75 & 12 \\
13 & Michael Clark & 74 & 13 \\
14 & Natalie Lewis & 73 & 14 \\
15 & Olivia Walker & 72 & 15 \\
16 & Peter Hall & 71 & 16 \\
17 & Quinn Allen & 70 & 17 \\
18 & Rachel Young & 69 & 18 \\
19 & Samuel King & 68 & 19 \\
20 & Tina Wright & 67 & 20 \\
21 & Uma Scott & 66 & 21 \\
22 & Victor Green & 65 & 22 \\
23 & Wendy Adams & 64 & 23 \\
24 & Xavier Nelson & 63 & 24 \\
25 & Yolanda Carter & 62 & 25 \\
26 & Zachary Perez & 61 & 26 \\
27 & Amelia Baker & 60 & 27 \\
28 & Benjamin Rivera & 59 & 28 \\
29 & Charlotte Rogers & 58 & 29 \\
30 & David Murphy & 57 & 30 \\
31 & Ethan Cooper & 56 & 31 \\
32 & Fiona Reed & 55 & 32 \\
33 & George Bailey & 54 & 33 \\
34 & Hannah Cox & 53 & 34 \\
35 & Isaac Howard & 52 & 35 \\
36 & Julia Ward & 51 & 36 \\
37 & Kyle Flores & 50 & 37 \\
38 & Lily Bell & 49 & 38 \\
39 & Mason Sanders & 48 & 39 \\
40 & Nora Patterson & 47 & 40 \\
41 & Owen Ramirez & 46 & 41 \\
42 & Penelope Torres & 45 & 42 \\
43 & Quentin Foster & 44 & 43 \\
44 & Rebecca Gonzales & 43 & 44 \\
45 & Sebastian Bryant & 42 & 45 \\
46 & Taylor Alexander & 41 & 46 \\
47 & Ursula Russell & 40 & 47 \\
48 & Vincent Griffin & 39 & 48 \\
49 & William Diaz & 38 & 49 \\
50 & Zoe Simmons & 37 & 50 \\
% (You can easily extend this list to hundreds of rows)
\end{longtable}

\subsubsection{Beberapa Contoh Penulisan Rumus atau Persamaan Matematika Menggunakan LaTeX Termasuk Penomorannya}
Contoh rumus matematika dapat ditulis seperti pada Persamaan \ref{eq:contoh1} di bawah ini. 
Penomoran persamaan diletakkan di sebelah kanan, dan rumus ditulis dalam mode \textit{display math}.
\begin{equation}
E = mc^2
\label{eq:contoh1}
\end{equation}

Contoh lain penulisan rumus matematika yang lebih kompleks dapat ditulis seperti pada Persamaan \ref{eq:rumus2}.

\begin{align}
f(x) &= ax^2 + bx + c \\
f'(x) &= \frac{d}{dx}(ax^2 + bx + c) \notag \\ % tidak menampilkan nomor pada baris ini
      &= 2ax + b \label{eq:rumus2}
\end{align}

Jika rumus terlalu panjang untuk ditulis dalam satu baris, gunakan lingkungan \textit{multline} seperti pada Persamaan \ref{eq:rumus3} di bawah ini.
\begin{multline} 
y = a_0 + a_1x + a_2x^2 + a_3x^3 + a_4x^4 + a_5x^5 + a_6x^6 + a_7x^7 \\
    + a_8x^8 + a_9x^9 + a_{10}x^{10} \label{eq:rumus3}
\end{multline}

Jika ada penurunan rumus yang terdiri dari beberapa baris, namun tidak memerlukan penomoran pada setiap baris, gunakan lingkungan \textit{align*}, misalnya:

\begin{align*} 
S &= \sum_{i=1}^{n} i^2 \\
  &= 1^2 + 2^2 + 3^2 + \cdots + n^2 \\
  &= \frac{n(n + 1)(2n + 1)}{6}
\intertext{Contoh lainnya adalah rumus untuk mencari nilai rata-rata fungsi $f(x)$ pada interval $[p, q]$:}
\bar{f} &= \frac{1}{q - p} \int_{p}^{q} f(x) \, dx \\
        &= \frac{1}{q - p} \int_{p}^{q} (ax^2 + bx + c) \, dx \\
        &= \frac{1}{q - p} \left[ \frac{a}{3}x^3 + \frac{b}{2}x^2 + cx \right]_p^q \\
        &= \frac{a(q^3 - p^3)}{3(q - p)} + \frac{b(q^2 - p^2)}{2(q - p)} + c \label{eq:rumus4}
\end{align*}



\subsection{Algoritma, Pseudocode, atau Kode}
Contoh penulisan algoritma atau pseudocode dapat ditulis seperti pada Kode \ref{alg:contoh1} di bawah ini. 
Gunakan paket \textit{listings} untuk menulis source code dalam bahasa pemrograman tertentu, seperti pada Kode \ref{lst:contoh1}. 


% -- Example of pseudocode and source code listing --
% -- Gunakan minipage agar listing tidak terpotong ke halaman berikutnya --
\begin{minipage}{\textwidth} 
\begin{lstlisting}[frame=lines, captionpos=t, caption={Contoh pseudocode}, label={alg:contoh1}]
ALGORITHM HelloWorld
   PRINT "Hello, World!"
END ALGORITHM
\end{lstlisting}
\end{minipage}

\begin{minipage}{\textwidth}
\begin{lstlisting}[language=Python, frame=single, caption={Contoh source code Python}, captionpos=t, label={lst:contoh1}]
def hello_world():
    print("Hello, World!")       
hello_world()
\end{lstlisting}
\end{minipage}


\section{Beberapa Kesalahan Penulisan yang Sering Terjadi}
\subsection{Penggunaan Kata "di mana" atau "dimana"}
Banyak yang menuliskan kata "di mana" atau "dimana" sebagai pengganti kata "which" dalam bahasa Inggris. 
Padahal, penggunaan kata "di mana" atau "dimana" tidak tepat dalam konteks tersebut. Demikian juga untuk kata serupa, misalnya "yang mana".
Kata "di mana" atau "dimana" ini harus diganti dengan kata lain, seperti "dengan", "tempat", "yang", dan sebagainya tergantung kalimatnya.
Penjelasan lengkap dapat dilihat pada \autocite{BPBI}.

\subsection{Penggunaan Kata "sedangkan" dan "sehingga"}

\begin{table}[t]
  \begin{tabular}{|c|l|l|}
  \hline
  Kata & Salah & Benar \\ \hline
  sedangkan & \begin{tabular}[c]{@{}c@{}}Sedangkan sistem lama masih\\ digunakan oleh banyak pengguna.\end{tabular} & \begin{tabular}[c]{@{}c@{}}Sistem lama masih digunakan\\ oleh banyak pengguna,\\ sedangkan sistem baru belum siap.\end{tabular} \\ \hline
  sehingga & \begin{tabular}[c]{@{}c@{}}Sehingga sistem lama masih\\ digunakan oleh banyak pengguna.\end{tabular} & \begin{tabular}[c]{@{}c@{}}Sistem lama masih digunakan\\ oleh banyak pengguna sehingga\\ sistem baru belum siap.\end{tabular} \\ \hline
  \end{tabular}
  \caption{Contoh penggunaan kata "sedangkan" dan "sehingga"}
  \label{tbl:sedangkan_sehingga}
\end{table}

Kata "sedangkan" dan "sehingga" adalah kata hubung atau konjungsi. 
Konjungsi adalah kata atau ungkapan yang menghubungkan satuan bahasa 
(kata, frasa, klausa, dan kalimat). 
Konjungsi dapat dibagi menjadi konjungsi intrakalimat dan antarkalimat.  
Kata "sedangkan" menghubungkan dua klausa yang bersifat kontrasif, 
sedangkan "sehingga" menghubungkan dua klausa yang bersifat kausal. 
Dalam ragam formal, kata hubung “sedangkan” dan “sehingga” hanya dapat digunakan 
sebagai konjungsi intrakalimat sehingga kedua konjungsi itu \textbf{tidak dapat diletakkan pada awal kalimat}.
Selain itu, penggunaan kata "sedangkan" harus didahului oleh koma (,), sedangkan kata "sehingga" tidak perlu didahului oleh koma (,).
Contoh penggunaan yang benar dan salah dapat dilihat pada Tabel \ref{tbl:sedangkan_sehingga}.


\subsection{Penggunaan Istilah yang Tidak Baku}
Ada beberapa istilah yang sering digunakan dalam pembicaraan sehari-hari, tetapi tidak baku dalam penulisan ilmiah.
Beberapa istilah tersebut antara lain:
\begin{enumerate}
  \item analisa $\rightarrow$ analisis
  \item eksisting atau existing $\rightarrow$ yang ada atau saat ini
  \item bisnis proses $\rightarrow$ proses bisnis
  \item user $\rightarrow$ pengguna
  \item system $\rightarrow$ sistem
  \item database $\rightarrow$ basis data
  \item aktifitas $\rightarrow$ aktivitas
  \item efektifitas $\rightarrow$ efektivitas
  \item sosial media $\rightarrow$ media sosial
\end{enumerate}

\subsection{Pemisah Desimal dan Ribuan}
Tanda pemisah desimal dalam bahasa Indonesia adalah tanda koma, contoh:
\begin{enumerate}
  \item (Salah) Akurasi naik menjadi 50.6\% 
  \item (Benar) Akurasi naik menjadi 50,6\% 
\end{enumerate}

\subsection{Daftar atau \textit{List}}
Ada beberapa aturan penulisan daftar atau \textit{list} yang perlu diperhatikan, antara lain:
\begin{enumerate}[a)]
\item Jika memungkinkan, hindari penggunaan “bullet points” atau sejenisnya. Sebaiknya, gunakan angka (1, 2, 3, ...) atau huruf (a, b, c, …). Dengan demikian, pembaca dapat dengan mudah melihat jumlah \textit{item} atau \textit{list}. 
\item Jika dalam daftar hanya ada satu item, tidak perlu menggunakan nomor urut.
\item Penjelasan atau deskripsi suatu item sebaiknya menyatu dengan judul item tersebut, tidak berbeda halaman. Contoh yang salah: judul item ada di halaman 10, namun deskripsinya di halaman 11. Sebaiknya pindahkan judul tersebut ke halaman 11.
\item Jika penjelasan atau deskripsi suatu item cukup panjang, misalnya lebih dari 1 halaman atau terdiri atas beberapa paragraf, sebaiknya setiap item tersebut dijadikan judul subbab, kecuali jika level subbab sudah mencapai level 4. 
\end{enumerate}

\subsection{Penggunaan Kata "masing-masing" dan "setiap"}
Kata “masing-masing” digunakan di belakang kata yang diterangkan, misalnya 
"Setiap proses menggunakan algoritma masing-masing". Kata “tiap-tiap” atau “setiap”
ditempatkan di depan kata yang diterangkan, misalnya
"Setiap proses menggunakan algoritma tertentu".
