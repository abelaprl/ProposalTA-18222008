% ==========================================
% BAB II STUDI LITERATUR
% ==========================================
\chapter{STUDI LITERATUR}

Bab ini membahas konsep dan penelitian terdahulu yang menjadi landasan bagi analisis pengaruh \textit{user persona} terhadap perilaku \textit{large language model}. Pembahasan disusun secara bertahap, dimulai dari uraian mengenai model bahasa modern, mekanisme pemrosesan instruksi, konsep dasar persona, serta temuan empiris mengenai sensitivitas model terhadap identitas pengguna. Selain itu, bab ini meninjau isu bias dan metode evaluasi penalaran yang relevan bagi perancangan penelitian ini.

% -------------------------------------------------------------
\section{Large Language Model}

Perkembangan \textit{large language model} (LLM) dalam beberapa tahun terakhir telah mengubah pendekatan sistem kecerdasan buatan dalam memahami dan menghasilkan bahasa alami. LLM modern tidak hanya berfungsi sebagai model prediksi teks, tetapi juga sebagai sistem yang mampu melakukan penalaran, mengikuti instruksi kompleks, dan menyesuaikan perilaku berdasarkan konteks sosial percakapan. Bagian ini menguraikan perkembangan LLM yang relevan untuk memahami bagaimana model memproses identitas pengguna, menafsirkan \textit{persona}, dan menunjukkan variasi perilaku pada tugas-tugas penalaran.

\subsection{LLM sebagai \textit{Foundation Models}}

LLM dikategorikan sebagai \textit{foundation models}, yaitu model dengan skala parameter sangat besar yang dilatih menggunakan data dalam jumlah masif dan kemudian dapat diadaptasi untuk berbagai aplikasi \parencite{bommasani2021opportunities}. Konsep \textit{foundation models} menekankan bahwa representasi bahasa yang dipelajari tidak spesifik pada satu tugas, tetapi bersifat generik dan berfungsi sebagai dasar bagi berbagai perilaku yang muncul setelah pelatihan.

Karakteristik multi-fungsi ini membuat LLM mampu menggeneralisasi ke berbagai bentuk instruksi, gaya bahasa, serta skenario penggunaan. Namun, di balik fleksibilitas tersebut terdapat keterbatasan: model menjadi sensitif terhadap variasi kecil dalam konteks, termasuk informasi mengenai identitas pengguna. Kondisi ini menjadikan \textit{foundation models} bukan hanya sistem statistis, tetapi representasi kompleks yang dapat mengodekan aspek sosial linguistik.

\subsection{Scaling Laws dan Fenomena Kemampuan Emergent}

Peningkatan kapasitas model—baik dari sisi jumlah parameter, volume data, maupun durasi pelatihan—berkaitan dengan fenomena kemampuan emergent, yaitu kemampuan baru yang tidak muncul pada model berukuran kecil \parencite{bommasani2021opportunities}. Contohnya termasuk penyelesaian soal matematika multilangkah, pemahaman instruksi panjang, serta kemampuan dalam mengikuti gaya komunikasi pengguna.

Fenomena kemampuan emergent ini relevan dengan penelitian mengenai \textit{persona} karena sensitivitas terhadap identitas pengguna tidak terlihat pada model kecil, tetapi mulai muncul pada model berukuran besar. Ketika kapasitas model meningkat, LLM mulai menunjukkan perilaku adaptif terhadap konteks sosial, termasuk gaya bahasa pengguna, latar belakang identitas, dan \textit{framing} percakapan. Hal ini membuka peluang sekaligus tantangan dalam memahami bagaimana \textit{persona} memengaruhi alur penalaran model.

\subsection{\textit{Instruction Tuning} dan Perilaku Adaptif}

Model bahasa modern memperoleh kemampuan mengikuti instruksi melalui proses \textit{instruction tuning}, yang terdiri dari pelatihan menggunakan pasangan instruksi–jawaban serta optimisasi berbasis umpan balik manusia (\textit{reinforcement learning from human feedback}). Proses ini bertujuan menghasilkan perilaku yang lebih aman, membantu pengguna, dan selaras dengan ekspektasi interaksi alami.

Namun, proses tuning ini tidak dirancang untuk menghilangkan sensitivitas model terhadap identitas pengguna. Justru, penyesuaian yang dihasilkan membuat model semakin peka terhadap nuansa sosial dalam instruksi. Model dapat meniru gaya komunikasi pengguna, menyesuaikan tingkat kehati-hatian, atau mengubah bentuk penalaran berdasarkan sinyal kontekstual yang dihasilkan pengguna. Dalam konteks \textit{persona}, hal ini berarti bahwa model tidak sekadar menjawab perintah, tetapi juga beradaptasi terhadap profil pengguna yang tersurat maupun tersirat.

\subsection{Sensitivitas Model terhadap Konteks Linguistik}

Penelitian menunjukkan bahwa LLM sangat sensitif terhadap variasi kecil pada cara instruksi disampaikan. Zhou et al. \parencite{zhou2023largemodelsensitive} menunjukkan bahwa perbedaan kata, urutan kalimat, atau gaya penulisan dapat menghasilkan perubahan signifikan dalam prediksi model. Sensitivitas terhadap \textit{framing} ini menandakan bahwa model menggunakan seluruh konteks linguistik—bukan hanya isi instruksi semantik—untuk memutuskan bentuk respons.

Sensitivitas terhadap konteks menjadi dasar penting dalam analisis \textit{persona}. Identitas pengguna, baik eksplisit maupun implisit, merupakan bagian dari konteks. Ketika \textit{persona} dimasukkan dalam prompt, model memproses identitas tersebut sebagai sinyal yang memengaruhi representasi internal. Hal ini menjelaskan mengapa \textit{persona} dapat memicu perubahan gaya bahasa, preferensi jawaban, hingga langkah penalaran pada tugas yang seharusnya netral.

\subsection{Kemampuan Penalaran dan Keterbatasannya}

Kemampuan penalaran merupakan salah satu aspek yang menonjol dalam perkembangan LLM. Model mampu menyelesaikan tugas matematis (seperti GSM8K), menjawab soal pengetahuan umum (MMLU), atau memberikan penilaian moral dan sosial. Namun, penalaran ini tidak selalu stabil. Turpin et al. \parencite{turpin2023language} menemukan bahwa \textit{chain-of-thought} yang dihasilkan model sering kali tidak konsisten dan sensitif terhadap variasi konteks, sehingga model dapat memberikan jawaban berbeda meskipun instruksi hampir sama.

Ketidakstabilan ini relevan untuk penelitian mengenai \textit{persona}. Penalaran yang berubah akibat identitas pengguna dapat memperlihatkan bentuk \textit{reasoning bias} yang tidak tampak pada evaluasi biasa. Dengan demikian, \textit{persona} dapat berfungsi sebagai lensa untuk mengidentifikasi area di mana model tidak konsisten dalam memproses informasi.

\subsection{LLM sebagai Sistem yang Memodelkan Interaksi Sosial}

Seiring meningkatnya kapasitas, model bahasa kini tidak hanya memahami teks sebagai data linguistik, tetapi juga sebagai representasi interaksi manusia. Weidinger et al. \parencite{weidinger2021ethical} menunjukkan bahwa model besar mulai meniru pola sosial, norma, serta bias yang terkandung dalam data pelatihan. Model dapat menginferensi identitas sosial pengguna seperti usia, gaya komunikasi, atau latar belakang budaya melalui sinyal linguistik—bahkan ketika tidak dinyatakan secara eksplisit.

Perkembangan ini memperlihatkan bahwa LLM tidak hanya merupakan sistem komputasi bahasa, tetapi juga sistem inferensi sosial. Kondisi ini memperkuat relevansi penelitian terhadap \textit{persona}, karena interaksi pengguna tidak lagi diproses sebagai teks murni, tetapi sebagai bagian dari struktur sosial percakapan yang dipelajari oleh model.

\subsection{Kaitan Perkembangan LLM dengan Studi Persona}

Pertumbuhan LLM dalam kapasitas, kemampuan adaptif, dan sensitivitas terhadap konteks linguistik menjadikan \textit{persona} sebagai faktor yang semakin penting untuk dipahami. Perubahan kecil dalam identitas pengguna—baik eksplisit maupun implisit—berpotensi memengaruhi keluaran model secara signifikan. Oleh karena itu, perkembangan LLM secara langsung membuka kebutuhan untuk mengevaluasi bagaimana \textit{persona} memengaruhi penalaran, kualitas respons, dan kecenderungan \textit{human bias} pada model.

%


\label{chap:studi-literatur}
\section{Penulisan Gambar, Tabel, Rumus, dan Kode}
\lipsum[1]

\subsection{Gambar}
Contoh gambar dapat dilihat pada Gambar \ref{gambar:jaringan}. Gambar dan judulnya diposisikan di tengah. Nomor gambar tidak diakhiri tanda titik. Gambar tersebut dibuat menggunakan aplikasi draw.io dan disimpan ke format PNG setelah dengan zoom setting pada angka 300\%. Ukuran gambar yang ditampilkan dapat diatur dengan mengubah nilai \textit{width} dalam sintaks \textit{includegraphics}.

\begin{figure}[t] % pilihan opsi yang disarankan: t = top, b = bottom, h = here
	\centering
  \captionsetup{justification=centering}
    	\includegraphics[width=0.7\textwidth]{image/gambar1.png}
	\caption{Contoh gambar jaringan}
	\label{gambar:jaringan}
\end{figure}

Gambar umumnya tidak jelas atau kabur jika gambar tersebut:
\begin{enumerate}[a.]
  \item diperoleh dari hasil cropping pada suatu halaman buku atau situs web;
  \item hasil pembesaran gambar yang gambar aslinya sebenarnya berukuran kecil; atau
  \item disimpan dalam resolusi kecil
\end{enumerate}
Ketidakjelasan gambar ini dapat dilihat pada garis-garis diagram yang tidak tegas dan tulisan-tulisan dalam gambar yang tampak kabur dan kurang jelas terbaca.

Untuk mendapatkan gambar yang tidak kabur (\textit{blur}), langkah-langkah berikut dapat digunakan:
\begin{enumerate}[(a)]
\item Gambar yang didapat di suatu pustaka atau referensi sebaiknya digambar ulang, misalnya menggunakan PowerPoint, Canva, Figma, draw.io, atau yang lainnya.
\item Jika diagram atau ilustrasi digambar menggunakan draw.io, saat gambar disimpan ke format PNG atau JPG (\textit{export as}), lakukan \textit{zoom} ke minimal 300\% (\textit{the default value is} 100\%). 
\item Jika diagram digambar dengan menggunakan PowerPoint, gambar dapat langsung di-\textit{copy-paste} ke Word.
\end{enumerate}

\subsection{Tabel}
Tabel ada dua jenis, yaitu tabel yang bisa termuat dalam satu halaman dan tabel yang sangat panjang sehingga tidak muat dalam satu halaman.
\subsubsection{Tabel yang Muat dalam Satu Halaman}
Contoh tabel dapat dilihat pada Tabel \ref{tbl:harga1} dan \ref{tbl:harga2}. Tabel dan judulnya dibuat rata kiri dan judul tabel diletakkan di atas tabel. Usahakan tabel dapat ditulis dalam satu halaman, tidak terpotong ke halaman berikutnya.

\begin{table}[t] % pilihan opsi yang disarankan: t = top, b = bottom, h = here
  \begin{tabular}{ | p{2cm} | p{2cm} | p{3cm} |}
	\hline
	Nama 	& Satuan 		& Harga \\
	\hline
	Buku 	& Exemplar	& 25000 \\
	Komputer	& Unit		& 2500000 \\
	Pensil	& Buah		& 118900 \\
	\hline
	\end{tabular}
\caption{Tabel harga bahan pokok}
\label{tbl:harga1}
\end{table}



\begin{table}[t] % pilihan opsi yang disarankan: t = top, b = bottom, h = here
	\begin{tabular}{ | l | c | r | }
	\hline
	Nama 	& Satuan 		& Harga \\
	\hline
	Buku 	& Exemplar	& 25000 \\
	Komputer	& Unit		& 2500000 \\
	Pensil	& Buah		& 118900 \\
	\hline
	\end{tabular}
\caption{Tabel harga bahan sekunder}
\label{tbl:harga2}
\end{table}

% -- Example of importing table from external file --
\subsubsection{Mengimpor Tabel dari Berkas Eksternal}

Tabel \ref{tbl:harga3} diimpor dari berkas eksternal \textit{table/tabel1.tex} menggunakan perintah \textit{input}. 
Dengan demikian, jika tabel tersebut perlu diubah, cukup mengubah pada berkas eksternal tersebut tanpa perlu mengubah pada berkas utama ini.

\input table/tabel1.tex


% -- Example of long table --
\subsubsection{Tabel yang Sangat Panjang}
Jika tabel terlalu panjang sehingga tidak muat dalam satu halaman, gunakan paket 
\textit{longtable} untuk membuat tabel yang dapat terpotong ke halaman berikutnya, 
seperti pada Tabel \ref{tbl:longtable1}.

\begin{longtable}{@{\extracolsep{\fill}} l c r r}
\caption{Comprehensive Data Table Example}\label{tbl:longtable1} \\
\toprule
\textbf{ID} & \textbf{Name} & \textbf{Score} & \textbf{Rank} \\
\midrule
\endfirsthead

\caption{Comprehensive Data Table Example (lanjutan)} \\
\toprule
\textbf{ID} & \textbf{Name} & \textbf{Score} & \textbf{Rank} \\
\midrule
\endhead

\midrule
\multicolumn{4}{r}{\textit{Bersambung ke halaman berikutnya}} \\
%\bottomrule
\endfoot

\bottomrule
\endlastfoot

% === Table Data ===
1 & Alice Smith & 89 & 5 \\
2 & Bob Johnson & 93 & 3 \\
3 & Carol Davis & 95 & 2 \\
4 & Daniel Wilson & 88 & 6 \\
5 & Eve Thompson & 97 & 1 \\
6 & Frank Brown & 85 & 7 \\
7 & Grace Lee & 91 & 4 \\
8 & Henry Miller & 80 & 9 \\
9 & Irene Garcia & 83 & 8 \\
10 & Jack Robinson & 78 & 10 \\
% Repeat with more rows to make it long
11 & Kevin Harris & 76 & 11 \\
12 & Laura Martin & 75 & 12 \\
13 & Michael Clark & 74 & 13 \\
14 & Natalie Lewis & 73 & 14 \\
15 & Olivia Walker & 72 & 15 \\
16 & Peter Hall & 71 & 16 \\
17 & Quinn Allen & 70 & 17 \\
18 & Rachel Young & 69 & 18 \\
19 & Samuel King & 68 & 19 \\
20 & Tina Wright & 67 & 20 \\
21 & Uma Scott & 66 & 21 \\
22 & Victor Green & 65 & 22 \\
23 & Wendy Adams & 64 & 23 \\
24 & Xavier Nelson & 63 & 24 \\
25 & Yolanda Carter & 62 & 25 \\
26 & Zachary Perez & 61 & 26 \\
27 & Amelia Baker & 60 & 27 \\
28 & Benjamin Rivera & 59 & 28 \\
29 & Charlotte Rogers & 58 & 29 \\
30 & David Murphy & 57 & 30 \\
31 & Ethan Cooper & 56 & 31 \\
32 & Fiona Reed & 55 & 32 \\
33 & George Bailey & 54 & 33 \\
34 & Hannah Cox & 53 & 34 \\
35 & Isaac Howard & 52 & 35 \\
36 & Julia Ward & 51 & 36 \\
37 & Kyle Flores & 50 & 37 \\
38 & Lily Bell & 49 & 38 \\
39 & Mason Sanders & 48 & 39 \\
40 & Nora Patterson & 47 & 40 \\
41 & Owen Ramirez & 46 & 41 \\
42 & Penelope Torres & 45 & 42 \\
43 & Quentin Foster & 44 & 43 \\
44 & Rebecca Gonzales & 43 & 44 \\
45 & Sebastian Bryant & 42 & 45 \\
46 & Taylor Alexander & 41 & 46 \\
47 & Ursula Russell & 40 & 47 \\
48 & Vincent Griffin & 39 & 48 \\
49 & William Diaz & 38 & 49 \\
50 & Zoe Simmons & 37 & 50 \\
% (You can easily extend this list to hundreds of rows)
\end{longtable}

\subsubsection{Beberapa Contoh Penulisan Rumus atau Persamaan Matematika Menggunakan LaTeX Termasuk Penomorannya}
Contoh rumus matematika dapat ditulis seperti pada Persamaan \ref{eq:contoh1} di bawah ini. 
Penomoran persamaan diletakkan di sebelah kanan, dan rumus ditulis dalam mode \textit{display math}.
\begin{equation}
E = mc^2
\label{eq:contoh1}
\end{equation}

Contoh lain penulisan rumus matematika yang lebih kompleks dapat ditulis seperti pada Persamaan \ref{eq:rumus2}.

\begin{align}
f(x) &= ax^2 + bx + c \\
f'(x) &= \frac{d}{dx}(ax^2 + bx + c) \notag \\ % tidak menampilkan nomor pada baris ini
      &= 2ax + b \label{eq:rumus2}
\end{align}

Jika rumus terlalu panjang untuk ditulis dalam satu baris, gunakan lingkungan \textit{multline} seperti pada Persamaan \ref{eq:rumus3} di bawah ini.
\begin{multline} 
y = a_0 + a_1x + a_2x^2 + a_3x^3 + a_4x^4 + a_5x^5 + a_6x^6 + a_7x^7 \\
    + a_8x^8 + a_9x^9 + a_{10}x^{10} \label{eq:rumus3}
\end{multline}

Jika ada penurunan rumus yang terdiri dari beberapa baris, namun tidak memerlukan penomoran pada setiap baris, gunakan lingkungan \textit{align*}, misalnya:

\begin{align*} 
S &= \sum_{i=1}^{n} i^2 \\
  &= 1^2 + 2^2 + 3^2 + \cdots + n^2 \\
  &= \frac{n(n + 1)(2n + 1)}{6}
\intertext{Contoh lainnya adalah rumus untuk mencari nilai rata-rata fungsi $f(x)$ pada interval $[p, q]$:}
\bar{f} &= \frac{1}{q - p} \int_{p}^{q} f(x) \, dx \\
        &= \frac{1}{q - p} \int_{p}^{q} (ax^2 + bx + c) \, dx \\
        &= \frac{1}{q - p} \left[ \frac{a}{3}x^3 + \frac{b}{2}x^2 + cx \right]_p^q \\
        &= \frac{a(q^3 - p^3)}{3(q - p)} + \frac{b(q^2 - p^2)}{2(q - p)} + c \label{eq:rumus4}
\end{align*}



\subsection{Algoritma, Pseudocode, atau Kode}
Contoh penulisan algoritma atau pseudocode dapat ditulis seperti pada Kode \ref{alg:contoh1} di bawah ini. 
Gunakan paket \textit{listings} untuk menulis source code dalam bahasa pemrograman tertentu, seperti pada Kode \ref{lst:contoh1}. 


% -- Example of pseudocode and source code listing --
% -- Gunakan minipage agar listing tidak terpotong ke halaman berikutnya --
\begin{minipage}{\textwidth} 
\begin{lstlisting}[frame=lines, captionpos=t, caption={Contoh pseudocode}, label={alg:contoh1}]
ALGORITHM HelloWorld
   PRINT "Hello, World!"
END ALGORITHM
\end{lstlisting}
\end{minipage}

\begin{minipage}{\textwidth}
\begin{lstlisting}[language=Python, frame=single, caption={Contoh source code Python}, captionpos=t, label={lst:contoh1}]
def hello_world():
    print("Hello, World!")       
hello_world()
\end{lstlisting}
\end{minipage}


\section{Beberapa Kesalahan Penulisan yang Sering Terjadi}
\subsection{Penggunaan Kata "di mana" atau "dimana"}
Banyak yang menuliskan kata "di mana" atau "dimana" sebagai pengganti kata "which" dalam bahasa Inggris. 
Padahal, penggunaan kata "di mana" atau "dimana" tidak tepat dalam konteks tersebut. Demikian juga untuk kata serupa, misalnya "yang mana".
Kata "di mana" atau "dimana" ini harus diganti dengan kata lain, seperti "dengan", "tempat", "yang", dan sebagainya tergantung kalimatnya.
Penjelasan lengkap dapat dilihat pada \autocite{BPBI}.

\subsection{Penggunaan Kata "sedangkan" dan "sehingga"}

\begin{table}[t]
  \begin{tabular}{|c|l|l|}
  \hline
  Kata & Salah & Benar \\ \hline
  sedangkan & \begin{tabular}[c]{@{}c@{}}Sedangkan sistem lama masih\\ digunakan oleh banyak pengguna.\end{tabular} & \begin{tabular}[c]{@{}c@{}}Sistem lama masih digunakan\\ oleh banyak pengguna,\\ sedangkan sistem baru belum siap.\end{tabular} \\ \hline
  sehingga & \begin{tabular}[c]{@{}c@{}}Sehingga sistem lama masih\\ digunakan oleh banyak pengguna.\end{tabular} & \begin{tabular}[c]{@{}c@{}}Sistem lama masih digunakan\\ oleh banyak pengguna sehingga\\ sistem baru belum siap.\end{tabular} \\ \hline
  \end{tabular}
  \caption{Contoh penggunaan kata "sedangkan" dan "sehingga"}
  \label{tbl:sedangkan_sehingga}
\end{table}

Kata "sedangkan" dan "sehingga" adalah kata hubung atau konjungsi. 
Konjungsi adalah kata atau ungkapan yang menghubungkan satuan bahasa 
(kata, frasa, klausa, dan kalimat). 
Konjungsi dapat dibagi menjadi konjungsi intrakalimat dan antarkalimat.  
Kata "sedangkan" menghubungkan dua klausa yang bersifat kontrasif, 
sedangkan "sehingga" menghubungkan dua klausa yang bersifat kausal. 
Dalam ragam formal, kata hubung “sedangkan” dan “sehingga” hanya dapat digunakan 
sebagai konjungsi intrakalimat sehingga kedua konjungsi itu \textbf{tidak dapat diletakkan pada awal kalimat}.
Selain itu, penggunaan kata "sedangkan" harus didahului oleh koma (,), sedangkan kata "sehingga" tidak perlu didahului oleh koma (,).
Contoh penggunaan yang benar dan salah dapat dilihat pada Tabel \ref{tbl:sedangkan_sehingga}.


\subsection{Penggunaan Istilah yang Tidak Baku}
Ada beberapa istilah yang sering digunakan dalam pembicaraan sehari-hari, tetapi tidak baku dalam penulisan ilmiah.
Beberapa istilah tersebut antara lain:
\begin{enumerate}
  \item analisa $\rightarrow$ analisis
  \item eksisting atau existing $\rightarrow$ yang ada atau saat ini
  \item bisnis proses $\rightarrow$ proses bisnis
  \item user $\rightarrow$ pengguna
  \item system $\rightarrow$ sistem
  \item database $\rightarrow$ basis data
  \item aktifitas $\rightarrow$ aktivitas
  \item efektifitas $\rightarrow$ efektivitas
  \item sosial media $\rightarrow$ media sosial
\end{enumerate}

\subsection{Pemisah Desimal dan Ribuan}
Tanda pemisah desimal dalam bahasa Indonesia adalah tanda koma, contoh:
\begin{enumerate}
  \item (Salah) Akurasi naik menjadi 50.6\% 
  \item (Benar) Akurasi naik menjadi 50,6\% 
\end{enumerate}

\subsection{Daftar atau \textit{List}}
Ada beberapa aturan penulisan daftar atau \textit{list} yang perlu diperhatikan, antara lain:
\begin{enumerate}[a)]
\item Jika memungkinkan, hindari penggunaan “bullet points” atau sejenisnya. Sebaiknya, gunakan angka (1, 2, 3, ...) atau huruf (a, b, c, …). Dengan demikian, pembaca dapat dengan mudah melihat jumlah \textit{item} atau \textit{list}. 
\item Jika dalam daftar hanya ada satu item, tidak perlu menggunakan nomor urut.
\item Penjelasan atau deskripsi suatu item sebaiknya menyatu dengan judul item tersebut, tidak berbeda halaman. Contoh yang salah: judul item ada di halaman 10, namun deskripsinya di halaman 11. Sebaiknya pindahkan judul tersebut ke halaman 11.
\item Jika penjelasan atau deskripsi suatu item cukup panjang, misalnya lebih dari 1 halaman atau terdiri atas beberapa paragraf, sebaiknya setiap item tersebut dijadikan judul subbab, kecuali jika level subbab sudah mencapai level 4. 
\end{enumerate}

\subsection{Penggunaan Kata "masing-masing" dan "setiap"}
Kata “masing-masing” digunakan di belakang kata yang diterangkan, misalnya 
"Setiap proses menggunakan algoritma masing-masing". Kata “tiap-tiap” atau “setiap”
ditempatkan di depan kata yang diterangkan, misalnya
"Setiap proses menggunakan algoritma tertentu".
