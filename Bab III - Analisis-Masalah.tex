% ============================================================================================
% BAB III ANALISIS MASALAH
% Pembagian subbab tidak rigid dan dapat bervariasi. Bab ini minimal berisi analisis kebutuhan
% fungsional dan nonfungsional, analisis berbagai alternatif solusi yang dapat ditawarkan, dan
% metode pemilihan solusi yang diusulkan.
% ============================================================================================
\chapter{ANALISIS MASALAH}
\label{chap:analisis-masalah}
\section{Analisis Kondisi Saat Ini}

Perkembangan \textit{large language model} (LLM) dalam beberapa tahun terakhir mendorong pemanfaatan model bahasa dalam berbagai konteks, mulai dari penjawab pertanyaan, agen percakapan, hingga sistem pendukung pengambilan keputusan \parencite{bommasani2021opportunities}. Seiring dengan meluasnya penggunaan tersebut, muncul kebutuhan untuk memahami bagaimana model bereaksi terhadap variasi identitas dan karakteristik pengguna, bukan hanya terhadap instruksi tugas. Hal ini berkaitan dengan cara model memproses konteks interaksi yang memuat informasi tentang siapa yang berinteraksi dengan model, dalam kapasitas apa, dan dengan gaya komunikasi seperti apa.

Penelitian mengenai persona pada LLM sejauh ini banyak berfokus pada pemberian identitas kepada model sebagai agen percakapan. Tseng et al.\ mengkaji berbagai pendekatan \textit{role-playing} dan \textit{personalization} yang umumnya memposisikan persona pada sisi model, misalnya melalui instruksi sistem yang mendeskripsikan karakter, gaya bicara, atau peran yang harus diambil oleh model \parencite{tseng2024twotales}. Pada pengaturan ini, model diminta untuk bertindak sebagai tenaga profesional, tokoh tertentu, atau asisten dengan gaya komunikasi spesifik, dan evaluasi dilakukan dengan melihat konsistensi gaya respons maupun kesesuaian perilaku dengan persona yang diberikan.

Di luar \textit{role-playing} tersebut, sejumlah studi menunjukkan bahwa penyisipan persona eksplisit dapat memengaruhi penalaran model bahkan pada tugas yang dirancang sebagai soal penalaran abstrak dan tidak secara eksplisit memuat dimensi sosial. Gupta et al.\ menunjukkan bahwa identitas yang dilekatkan pada konteks dapat menggeser cara model melakukan penalaran dan memilih jawaban, termasuk pada soal yang dirancang untuk menguji penalaran formal \parencite{gupta2024biasrunsdeep}. Temuan ini mengindikasikan bahwa persona tidak hanya memengaruhi gaya bahasa, tetapi juga struktur langkah penalaran yang dihasilkan model.

Pada saat yang sama, struktur penalaran LLM terbukti sensitif terhadap variasi kecil pada instruksi. Turpin et al.\ memperlihatkan bahwa perubahan ringan dalam formulasi \textit{prompt} dapat menghasilkan rantai penalaran yang berbeda meskipun pertanyaannya sama \parencite{turpin2023language}. Studi lain mengenai sensitivitas model terhadap framing dan gaya penulisan menunjukkan bahwa cara sebuah instruksi disusun dapat memengaruhi isi dan gaya jawaban \parencite{zhou2023largemodelsensitive}. Kondisi ini membuat analisis persona menjadi lebih kompleks, karena persona, framing, dan gaya bahasa sering kali hadir secara bersamaan di dalam konteks interaksi, sehingga sulit memisahkan pengaruh masing-masing faktor.

Isu bias menambah lapisan kompleksitas dalam memahami perilaku model. Weidinger et al.\ menunjukkan bahwa LLM dapat mereproduksi dan memperkuat pola bias sosial yang tercermin dalam data pelatihan \parencite{weidinger2021ethical}. Ketika identitas sosial tertentu, misalnya terkait gender, profesi, atau latar budaya, dimasukkan ke dalam konteks, respons model berpotensi mencerminkan bias representasional maupun inferensial yang sudah tertanam di dalam parameter model. Dalam konteks persona, hal ini berarti bahwa perbedaan respons akibat variasi identitas pengguna tidak selalu mencerminkan perubahan kemampuan penalaran, tetapi juga dapat berkaitan dengan bias yang telah terinternalisasi.

Sebagian besar studi persona yang ada menempatkan persona pada sisi model, bukan pada sisi pengguna. Instruksi yang mengubah peran model sebagai agen percakapan berbeda dengan skenario di mana konteks interaksi menyatakan bahwa pengguna memiliki identitas atau latar belakang tertentu. Riset mengenai pemodelan pengguna mulai berkembang, misalnya melalui pendekatan \textit{user language model} yang mempelajari distribusi bahasa berdasarkan karakteristik pengguna \parencite{naous2025userlm}, tetapi penelitian yang secara sistematis mengkaji dampak \textit{user persona} eksplisit maupun implisit terhadap penalaran dan kualitas jawaban pada berbagai tugas masih relatif terbatas.

Dari sisi infrastruktur evaluasi, banyak studi sebelumnya masih mengandalkan eksekusi manual atau setengah otomatis ketika menjalankan eksperimen yang melibatkan variasi pengguna. Naous et al.\ menyoroti pentingnya pendekatan yang lebih terstruktur ketika mengevaluasi model dalam konteks variasi pengguna, termasuk pengelolaan konfigurasi, pencatatan hasil, serta konsistensi skenario pengujian \parencite{naous2025userlm}. Tanpa kerangka evaluasi yang terdokumentasi dengan jelas, eksperimen yang melibatkan banyak model, banyak persona, dan berbagai jenis tugas menjadi sulit direplikasi dan rawan ketidakkonsistenan.

Berdasarkan kondisi tersebut, masalah-masalah utama yang mendasari perumusan penelitian ini dapat diringkas pada Tabel~\ref{tab:daftar-masalah-llm-persona}.

\begin{table}[htbp]
  \centering
  \caption{Daftar masalah penelitian terkait \textit{user persona} pada LLM}
  \label{tab:daftar-masalah-llm-persona}
  \begin{tabular}{p{1.5cm}p{5.5cm}p{7.0cm}}
    \toprule
    Kode & Uraian masalah & Dampak terhadap penelitian \\
    \midrule
    M-01 &
    Persona pada LLM umumnya diterapkan pada sisi model, bukan pada sisi pengguna. &
    Belum ada pemahaman yang sistematis mengenai bagaimana \textit{user persona} eksplisit maupun implisit memengaruhi penalaran dan kualitas jawaban pada berbagai tugas. \\[0.3cm]
    
    M-02 &
    Efek persona sulit dipisahkan dari efek framing dan gaya penulisan \textit{prompt}. &
    Perubahan performa atau pola penalaran dapat berasal dari variasi formulasi instruksi, bukan semata akibat perubahan \textit{user persona}, sehingga interpretasi hasil menjadi tidak pasti. \\[0.3cm]
    
    M-03 &
    LLM membawa bias sosial yang terinternalisasi dari data pelatihan. &
    Ketika identitas pengguna memuat atribut sosial tertentu, respons model berpotensi mencerminkan bias representasional maupun inferensial, sehingga perbedaan jawaban bisa berkaitan dengan bias yang sudah ada di model. \\[0.3cm]
    
    M-04 &
    Cakupan model dan tugas pada studi terdahulu masih terbatas. &
    Analisis sensitivitas terhadap persona sering kali hanya mencakup sedikit model atau jenis tugas, sehingga belum memberikan gambaran yang cukup luas mengenai variasi perilaku LLM di berbagai konteks. \\
    \bottomrule
  \end{tabular}
\end{table}

Masalah M-01 berkaitan dengan dominasi pendekatan yang menempatkan persona pada sisi model. Tseng et al.\ membahas bagaimana persona digunakan untuk mengubah peran dan gaya respons model melalui instruksi sistem atau deskripsi karakter \parencite{tseng2024twotales}. Pendekatan ini berbeda dengan skenario di mana identitas dan karakteristik pengguna dinyatakan secara eksplisit atau implisit pada konteks interaksi. Akibatnya, pengaruh \textit{user persona} terhadap penalaran dan kualitas jawaban belum banyak dikaji secara sistematis.

Masalah M-02 muncul karena struktur penalaran LLM sangat sensitif terhadap variasi kecil dalam formulasi instruksi. Turpin et al.\ menunjukkan bahwa perubahan ringan pada susunan \textit{prompt} dapat menghasilkan rantai penalaran yang berbeda meskipun pertanyaannya sama \parencite{turpin2023language}. Zhou et al.\ juga menunjukkan bahwa framing dan gaya penulisan instruksi dapat memengaruhi isi dan gaya jawaban \parencite{zhou2023largemodelsensitive}. Dalam konteks ini, efek \textit{user persona} berpotensi tercampur dengan efek framing, sehingga diperlukan desain eksperimen yang mampu membedakan keduanya.

Masalah M-03 berhubungan dengan bias sosial yang sudah tertanam di dalam model. Weidinger et al.\ menunjukkan bahwa LLM dapat mereproduksi dan memperkuat pola bias dari data pelatihan \parencite{weidinger2021ethical}. Ketika \textit{user persona} memuat atribut sosial seperti gender, profesi, atau latar budaya, respons model terhadap persona tersebut dapat dipengaruhi oleh bias yang telah ada sebelumnya. Hal ini menyulitkan interpretasi hasil, karena perbedaan jawaban bisa berasal dari kombinasi antara penyesuaian terhadap persona dan bias yang sudah terinternalisasi di dalam model.

Masalah M-04 menyoroti keterbatasan cakupan model dan tugas pada studi-studi terdahulu. Banyak penelitian persona hanya menguji sedikit model atau fokus pada satu jenis tugas, sehingga belum memberikan gambaran yang cukup luas mengenai bagaimana variasi \textit{user persona} memengaruhi perilaku model pada spektrum tugas penalaran dan percakapan yang lebih beragam \parencite{gupta2024biasrunsdeep, tseng2024twotales}. Keterbatasan ini membuka peluang untuk merancang eksperimen yang melibatkan kombinasi multi model dan multi persona pada beberapa kategori tugas yang terpilih.
\section{Analisis Kebutuhan}

Bagian ini menjabarkan kebutuhan penelitian yang diturunkan dari masalah M-01 sampai M-04 pada analisis kondisi saat ini. Kebutuhan tersebut mencakup kebutuhan konseptual dan teknis yang harus dipenuhi agar eksperimen mengenai pengaruh \textit{user persona} eksplisit dan implisit terhadap penalaran, kualitas jawaban, dan kecenderungan \textit{human bias} pada beberapa \textit{large language model} dapat dilaksanakan secara terstruktur.

\subsection{Identifikasi Masalah Pengguna}

Dalam konteks tugas akhir ini, pengguna yang dimaksud adalah peneliti yang ingin mengevaluasi perilaku model bahasa di bawah variasi \textit{user persona}. Berdasarkan analisis pada Bagian Analisis Kondisi Saat Ini, beberapa permasalahan yang dihadapi pengguna dapat diidentifikasi sebagai berikut.

\begin{enumerate}
    \item Definisi dan pengorganisasian \textit{user persona} eksplisit dan \textit{user persona} implisit belum terdokumentasi secara terstruktur. Sebagian besar contoh yang tersedia berfokus pada persona di sisi model, sehingga perumusan persona di sisi pengguna harus disusun sendiri oleh peneliti.
    \item Perbedaan keluaran model berpotensi dipengaruhi oleh variasi formulasi instruksi dan framing \textit{prompt}, sehingga tidak selalu jelas apakah perubahan respons model disebabkan oleh variasi \textit{user persona} atau oleh perubahan cara pertanyaan disampaikan.
    \item Eksperimen yang melibatkan beberapa model dan beberapa jenis tugas menuntut adanya cara yang terkelola untuk menjalankan skenario yang sama dan mencatat hasilnya secara konsisten, agar dapat dilakukan analisis perbandingan yang sistematis.
\end{enumerate}

Permasalahan-permasalahan tersebut menjadi dasar penyusunan kebutuhan fungsional dan kebutuhan nonfungsional pada penelitian ini.

\subsection{Kebutuhan Fungsional}

Kebutuhan fungsional menggambarkan kemampuan utama yang harus didukung oleh rancangan eksperimen agar permasalahan pada subbagian sebelumnya dapat ditangani. Ringkasan kebutuhan fungsional ditunjukkan pada Tabel~\ref{tab:kebutuhan-fungsional-llm-persona}.

\begin{table}[htbp]
  \centering
  \caption{Kebutuhan fungsional penelitian}
  \label{tab:kebutuhan-fungsional-llm-persona}
  \begin{tabular}{p{1.5cm}p{8cm}p{3.5cm}}
    \toprule
    Kode & Uraian kebutuhan fungsional & Terkait masalah \\
    \midrule
    KF-01 &
    Tersedia cara yang terstruktur untuk mendefinisikan \textit{user persona} eksplisit dan \textit{user persona} implisit dalam bentuk skenario teks, sehingga variasi persona dapat dirancang secara konsisten dan digunakan kembali. &
    M-01 \\[0.2cm]
    
    KF-02 &
    Tersedia mekanisme untuk menjalankan pertanyaan yang sama pada beberapa \textit{user persona} dan beberapa model bahasa, serta menyimpan keluaran model beserta informasi persona, model, dan jenis tugas yang digunakan. &
    M-02, M-04 \\[0.2cm]
    
    KF-03 &
    Tersedia format pencatatan hasil yang memungkinkan penilaian sederhana terhadap jawaban model, misalnya penandaan benar atau salah dan indikasi adanya \textit{human bias}, sehingga hasil dapat dianalisis secara sistematis. &
    M-03, M-04 \\
    \bottomrule
  \end{tabular}
\end{table}

KF-01 berhubungan dengan kebutuhan untuk merepresentasikan persona secara eksplisit, sehingga skenario eksperimen dapat direplikasi. KF-02 menekankan pentingnya eksekusi skenario yang sama pada beberapa model dan persona dengan pencatatan hasil yang terstruktur. KF-03 memastikan bahwa keluaran model terdokumentasi dalam bentuk yang mendukung analisis kuantitatif maupun kualitatif tanpa menuntut skema penilaian yang terlalu kompleks.

\subsection{Kebutuhan Nonfungsional}

Kebutuhan nonfungsional berkaitan dengan kualitas pelaksanaan eksperimen, terutama dari sisi keterulangan, kesederhanaan implementasi, dan kemampuan pengembangan. Ringkasan kebutuhan nonfungsional ditunjukkan pada Tabel~\ref{tab:kebutuhan-nonfungsional-llm-persona}.

\begin{table}[htbp]
  \centering
  \caption{Kebutuhan nonfungsional penelitian}
  \label{tab:kebutuhan-nonfungsional-llm-persona}
  \begin{tabular}{p{1.5cm}p{3.5cm}p{8cm}}
    \toprule
    Kode & Jenis kebutuhan & Uraian kebutuhan \\
    \midrule
    KNF-01 &
    Reproducibility &
    Proses eksperimen dapat diulang melalui skrip atau konfigurasi yang terdokumentasi, sehingga skenario persona, model, dan tugas dapat dijalankan kembali dengan pengaturan yang sama. \\[0.2cm]
    
    KNF-02 &
    Simplicity &
    Implementasi eksperimen tetap sederhana dan dapat dijalankan dengan sumber daya komputasi yang wajar, misalnya melalui pemanggilan API tanpa memerlukan infrastruktur tambahan yang kompleks. \\[0.2cm]
    
    KNF-03 &
    Extensibility &
    Rancangan eksperimen memungkinkan penambahan model atau \textit{user persona} baru tanpa perubahan besar pada struktur keseluruhan, sehingga dapat menyesuaikan dengan ketersediaan model dan kebutuhan analisis lanjutan. \\
    \bottomrule
  \end{tabular}
\end{table}

\section{Analisis Pemilihan Solusi}

Bagian ini membahas alternatif pendekatan yang dapat digunakan untuk melaksanakan eksperimen \textit{multi model} dan \textit{multi persona}, kemudian menjelaskan dasar pemilihan solusi yang digunakan dalam penelitian. Analisis dilakukan dengan mempertimbangkan kebutuhan struktur representasi \textit{user persona}, konsistensi eksekusi lintas model dan lintas tugas, kemudahan pencatatan hasil untuk analisis, serta tingkat kerumitan implementasi.

\subsection{Alternatif Solusi}

Berdasarkan kebutuhan yang telah dirumuskan pada analisis kebutuhan, beberapa alternatif solusi yang dapat diidentifikasi adalah sebagai berikut.

\begin{enumerate}
    \item Pendekatan evaluasi manual berbasis antarmuka percakapan. Pada alternatif ini, interaksi dengan \textit{large language model} dilakukan langsung melalui antarmuka percakapan yang disediakan oleh penyedia layanan. \textit{User persona} disisipkan ke dalam konteks, pertanyaan diajukan satu per satu, dan jawaban dicatat secara manual ke dalam dokumen atau lembar kerja. Setiap kombinasi model, persona, dan tugas dieksekusi secara terpisah. Pendekatan ini mudah dimulai karena tidak memerlukan pengembangan skrip, tetapi sangat bergantung pada prosedur manual dan kurang terstruktur ketika jumlah kombinasi skenario menjadi besar. Selain itu, reproduksi eksperimen menjadi bergantung pada kedisiplinan pencatatan dan rentan terhadap kesalahan manusia.
    
    \item Skrip eksperimen semi terotomatisasi berbasis konfigurasi. Pada alternatif ini, definisi \textit{user persona} eksplisit dan implisit, daftar model yang dievaluasi, serta kumpulan tugas dari \textit{benchmark} seperti GSM8K dan MMLU-redux disimpan dalam berkas konfigurasi yang terstruktur. Skrip eksperimen membaca konfigurasi tersebut, membentuk \textit{prompt} berdasarkan kombinasi model, persona, dan tugas, kemudian mengirim \textit{prompt} ke model melalui antarmuka pemrograman aplikasi. Keluaran model, beserta metadata seperti nama model, jenis persona, jenis tugas, dan identitas soal, disimpan dalam berkas JSON pada direktori log. Tahap berikutnya, skrip analisis mengolah berkas JSON menjadi berkas CSV yang lebih ringkas untuk perhitungan metrik dan analisis lanjutan. Pendekatan ini menuntut pengembangan skrip, tetapi memberikan struktur yang jelas dan memudahkan pelaksanaan eksperimen berskala besar.
    
    \item Kerangka evaluasi umum yang dapat digunakan kembali. Pada alternatif ini, dibangun sebuah kerangka evaluasi yang lebih umum, misalnya berupa pustaka atau layanan yang dirancang agar dapat digunakan kembali untuk berbagai studi terkait \textit{user persona} pada \textit{large language model}. Kerangka tersebut tidak hanya mencakup skrip eksekusi eksperimen berbasis konfigurasi, tetapi juga modul moduler untuk penjadwalan eksekusi, pengelolaan versi konfigurasi, penilaian otomatis, dan visualisasi hasil. Pendekatan ini berpotensi mendukung penggunaan jangka panjang dan kolaborasi yang lebih luas, namun memerlukan usaha perancangan dan implementasi yang lebih besar dibandingkan kebutuhan minimum untuk sebuah studi tugas akhir.
\end{enumerate}

\subsection{Analisis Penentuan Solusi}

Penentuan solusi dilakukan dengan membandingkan ketiga alternatif berdasarkan beberapa kriteria utama, yaitu kemampuan merepresentasikan \textit{user persona} dan skenario eksperimen secara terstruktur dan dapat digunakan kembali, konsistensi eksekusi lintas model dan lintas tugas, dukungan pencatatan hasil dan metadata untuk analisis kuantitatif dan kualitatif, keterulangan (\textit{reproducibility}) proses eksperimen, serta tingkat kerumitan implementasi dan pemeliharaan. Ringkasan perbandingan alternatif ditunjukkan pada Tabel~\ref{tab:analisis-pemilihan-solusi}, dengan skala kualitatif rendah, sedang, dan tinggi.

\begin{table}[htbp]
  \centering
  \caption{Perbandingan alternatif solusi}
  \label{tab:analisis-pemilihan-solusi}
  \begin{tabular}{p{5cm}p{3cm}p{3cm}p{3cm}}
    \toprule
    Kriteria &
    Evaluasi manual &
    Skrip semi terotomatisasi &
    Kerangka evaluasi umum \\
    \midrule
    Representasi \textit{user persona} dan skenario yang terstruktur &
    Rendah &
    Tinggi &
    Tinggi \\[0.2cm]
    
    Konsistensi eksekusi lintas model dan tugas &
    Rendah &
    Tinggi &
    Tinggi \\[0.2cm]
    
    Pencatatan hasil dan metadata untuk analisis &
    Rendah &
    Tinggi &
    Tinggi \\[0.2cm]
    
    Keterulangan (\textit{reproducibility}) proses eksperimen &
    Rendah &
    Tinggi &
    Tinggi \\[0.2cm]
    
    Kerumitan implementasi dan pemeliharaan &
    Rendah &
    Sedang &
    Tinggi \\[0.2cm]
    
    Kemudahan penambahan model atau persona baru &
    Rendah &
    Tinggi &
    Tinggi \\
    \bottomrule
  \end{tabular}
\end{table}

Pendekatan evaluasi manual relatif mudah digunakan pada tahap eksplorasi awal, tetapi tidak memadai untuk eksperimen \textit{multi model} dan \textit{multi persona} dengan jumlah kombinasi yang besar. Keterbatasan utama muncul pada konsistensi eksekusi, keterulangan eksperimen, serta pencatatan hasil yang sistematis.

Pendekatan kerangka evaluasi umum memberikan dukungan yang kuat terhadap struktur dan keterulangan, namun menuntut upaya perancangan arsitektur dan pengembangan perangkat lunak yang cukup besar. Beban tersebut berpotensi mengalihkan fokus dari tujuan utama penelitian, yaitu analisis empiris pengaruh \textit{user persona} terhadap penalaran, kualitas jawaban, dan kecenderungan \textit{human bias}.

Pendekatan skrip eksperimen semi terotomatisasi berbasis konfigurasi memberikan keseimbangan yang lebih sesuai. Representasi model, persona, dan tugas dapat diatur dalam direktori konfigurasi yang terpisah dari kode, sementara skrip eksekusi dan analisis ditempatkan dalam direktori tersendiri. Keluaran eksperimen disimpan sebagai berkas JSON pada direktori log dan diolah lebih lanjut menjadi berkas CSV pada direktori hasil. Struktur ini mendukung konsistensi eksekusi, keterulangan eksperimen, dan analisis terukur tanpa memerlukan pembangunan kerangka evaluasi yang terlalu umum.

Berdasarkan pertimbangan tersebut, penelitian ini memilih pendekatan skrip eksperimen semi terotomatisasi berbasis konfigurasi sebagai solusi utama untuk melaksanakan eksperimen \textit{multi model} dan \textit{multi persona}.
