% ==========================================
% BAB I PENDAHULUAN
% ==========================================
\chapter{PENDAHULUAN}
\label{chap:pendahuluan}
% --- Latar Belakang ---
\section{Latar Belakang}

Kemajuan dalam pengembangan \textit{large language model} telah menghasilkan peningkatan kemampuan dalam pemrosesan bahasa alami, pemahaman konteks, serta penalaran. Model seperti GPT, LLaMA, Mistral, dan Gemini digunakan secara luas dalam sistem dialog, agen percakapan, dan berbagai aplikasi berbasis teks. Walaupun demikian, kinerja model sering kali menunjukkan variasi yang bergantung pada identitas atau karakteristik pengguna yang tersirat dalam instruksi atau konteks percakapan.

Berbagai penelitian menunjukkan bahwa \textit{large language model} bersifat sensitif terhadap informasi identitas pengguna yang disisipkan dalam percakapan. Penelitian mengenai bias penalaran implisit menunjukkan bahwa perubahan kecil pada identitas pengguna dapat mengubah penalaran model pada tugas yang tidak memiliki dimensi sosial eksplisit \parencite{gupta2024biasrunsdeep}. Selain itu, studi mengenai identitas dan percakapan mengidentifikasi bahwa informasi identitas yang disampaikan melalui gaya, framing, atau konteks dapat memengaruhi gaya bahasa, tingkat kehati-hatian, atau preferensi jawaban \parencite{tseng2024twotales}. Penelitian mengenai pemodelan pengguna juga menunjukkan bahwa variasi atribut pengguna—seperti usia, latar belakang profesional, atau kelompok sosial—dapat memengaruhi keluaran model dalam aspek penalaran, stabilitas respons, dan kecenderungan bias \parencite{naous2025userlm}.

Berdasarkan literatur tersebut, semakin jelas bahwa \textit{large language model} tidak hanya memproses konten instruksi, tetapi juga bereaksi terhadap identitas pengguna yang diberikan secara eksplisit maupun yang tersirat dalam konteks. Dengan demikian, pengaruh \textit{user persona} menjadi aspek penting untuk dipelajari. Penelitian ini berfokus pada dua bentuk \textit{user persona}, yaitu \textit{user persona} eksplisit yang diberikan melalui deskripsi identitas yang jelas, serta \textit{user persona} implisit yang muncul melalui framing, gaya penulisan, atau narasi kontekstual tanpa instruksi langsung mengenai identitas pengguna.

Walaupun berbagai penelitian telah menunjukkan adanya sensitivitas model terhadap identitas pengguna, sebagian besar studi hanya mengevaluasi satu atau dua model, cakupan persona yang terbatas, atau rentang tugas penalaran yang sempit. Belum tersedia kerangka evaluasi yang memungkinkan analisis sistematis mengenai bagaimana \textit{user persona} memengaruhi penalaran, perilaku keluaran, dan \textit{human bias} pada berbagai model secara bersamaan. Kondisi tersebut menimbulkan kebutuhan untuk merancang pendekatan evaluasi yang lebih menyeluruh dan terstruktur.

Penelitian ini disusun untuk mengevaluasi pengaruh \textit{user persona} eksplisit dan \textit{user persona} implisit terhadap penalaran, perilaku keluaran, dan kecenderungan \textit{human bias} pada berbagai \textit{large language model} melalui pendekatan \textit{multi model} dan \textit{multi persona}. Penelitian ini dimaksudkan untuk memberikan pemahaman empiris yang lebih komprehensif mengenai sensitivitas model terhadap identitas pengguna.


% --- Rumusan Masalah ---
\section{Rumusan Masalah}

Rumusan masalah berikut disusun berdasarkan kebutuhan untuk memahami bagaimana \textit{user persona} memengaruhi perilaku dan penalaran model bahasa. Penelitian sebelumnya menunjukkan bahwa identitas pengguna, baik yang diberikan secara eksplisit maupun implisit, dapat memengaruhi penalaran, kualitas keluaran, dan kecenderungan bias model \parencite{gupta2024biasrunsdeep, tseng2024twotales, naous2025userlm}. Namun, cakupan penelitian terdahulu masih terbatas pada sedikit model, sedikit persona, dan variasi tugas yang sempit.

Berdasarkan kondisi tersebut, rumusan masalah penelitian ini adalah sebagai berikut.

\begin{enumerate}
    \item Bagaimana pengaruh \textit{user persona} eksplisit dan \textit{user persona} implisit terhadap performa penalaran pada berbagai jenis tugas pada sejumlah \textit{large language model}.
    \item Bagaimana kedua jenis \textit{user persona} tersebut memengaruhi perilaku keluaran model pada skenario interaksi yang berbeda.
    \item Bagaimana pola \textit{human bias} muncul dan berubah sebagai akibat variasi \textit{user persona}.
    \item Sejauh mana sensitivitas terhadap \textit{user persona} berbeda pada berbagai \textit{large language model}, serta model mana yang menunjukkan tingkat \textit{robustness} yang lebih tinggi terhadap variasi tersebut.
\end{enumerate}



% --- Tujuan ---
\section{Tujuan Penelitian}

Tujuan penelitian ditetapkan untuk menjawab permasalahan yang telah dirumuskan. Penelitian ini diarahkan untuk menghasilkan pemahaman yang lebih komprehensif mengenai pengaruh \textit{user persona} terhadap perilaku model bahasa dalam tugas penalaran dan skenario percakapan. Secara khusus, penelitian ini bertujuan untuk:

\begin{enumerate}
    \item Menganalisis pengaruh \textit{user persona} eksplisit dan \textit{user persona} implisit terhadap performa penalaran pada sejumlah \textit{large language model}.
    \item Mengidentifikasi perubahan perilaku keluaran model yang diinduksi oleh variasi \textit{user persona} pada berbagai konteks.
    \item Menganalisis pola \textit{human bias} yang muncul akibat variasi \textit{user persona}.
    \item Menyusun perbandingan sensitivitas dan \textit{robustness} berbagai model terhadap variasi \textit{user persona}.
    \item Mengembangkan rancangan \textit{evaluation pipeline} yang memungkinkan pelaksanaan eksperimen \textit{multi model} dan \textit{multi persona} secara terotomatisasi.
\end{enumerate}

% --- Batasan Masalah ---
\section{Batasan Masalah}
Tuliskan batasan-batasan yang diambil dalam pelaksanaan tugas akhir. Batasan ini dapat dihindari (bersifat opsional, tidak perlu ada) jika topik atau judul tugas akhir dibuat cukup spesifik.

% --- Metodologi Pengerjaan TA ---
\section{Metodologi}
Tuliskan semua tahapan yang akan dilalui selama pelaksanaan tugas akhir. Tahapan ini spesifik untuk menyelesaikan persoalan tugas akhir. Khusus untuk penyusunan proposal ini, jelaskan secara detail:
\begin{enumerate}
\item	Tahapan investigasi pengumpulan fakta di latar belakang untuk merumuskan masalah.
\item	Langkah-langkah pencarian, pengelompokan, dan penapisan literatur atau sumber informasi untuk mengumpulkan informasi yang relevan tentang topik yang diangkat, termasuk teori (konsep atau teori apa saja yang perlu dicari), hal-hal yang telah dicapai oleh orang lain (cara mencari dan kata kuncinya), dan berbagai informasi pendukung, untuk mencari solusi terhadap masalah yang dibahas. Gunakan metodologi yang tepat dalam menggali informasi dan dokumentasikan prosesnya (termasuk rekaman wawancara atau survei) di dalam Lampiran, termasuk tautan ke video atau foto. Hasil penggalian informasi ini akan dijelaskan secara sistematis di Bab II Studi Literatur.
\end{enumerate}