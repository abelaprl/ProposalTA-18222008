% ==========================================
% BAB I PENDAHULUAN
% ==========================================
\chapter{PENDAHULUAN}
\label{chap:pendahuluan}
% --- Latar Belakang ---
\section{Latar Belakang}

Kemajuan signifikan dalam pengembangan \textit{large language model} telah menghasilkan peningkatan kemampuan dalam pemrosesan bahasa alami, pemahaman konteks, dan penalaran. Model seperti GPT, LLaMA, Mistral, dan Gemini digunakan secara luas dalam sistem dialog, agen percakapan, dan berbagai aplikasi yang bergantung pada interaksi bahasa. Meskipun performanya meningkat, sejumlah tantangan metodologis masih muncul, khususnya terkait sensitivitas model terhadap identitas pengguna atau konfigurasi \textit{persona} yang diberikan selama proses interaksi.

Beberapa studi menunjukkan bahwa \textit{large language model} bersifat responsif terhadap \textit{persona}-yang-disematkan kepada model. Penelitian mengenai bias penalaran implisit menunjukkan bahwa penugasan \textit{persona} tertentu dapat mengubah hasil penalaran pada tugas yang tidak mengandung informasi sosial \parencite{gupta2024biasrunsdeep}. Studi lain mengkaji pengaruh \textit{persona} eksplisit dan \textit{persona} implisit terhadap gaya respons dan perilaku model dalam skenario percakapan \parencite{tseng2024twotales}. Selain itu, penelitian mengenai pemodelan pengguna menunjukkan bahwa variasi karakteristik pengguna dapat memengaruhi keluaran model dalam aspek penalaran, preferensi, dan stabilitas respons \parencite{naous2025userlm}. Temuan tersebut mengindikasikan bahwa \textit{large language model} tidak hanya memproses instruksi, tetapi juga bereaksi terhadap atribut identitas yang diberikan melalui \textit{persona}.

Walaupun demikian, penelitian yang ada masih memiliki batasan metodologis. Mayoritas studi mengevaluasi jumlah model yang terbatas, jumlah \textit{persona} yang sempit, atau rentang tugas yang tidak mencakup variasi penalaran yang relevan. Selain itu, belum tersedia kerangka evaluasi yang memungkinkan analisis sistematis terhadap pengaruh \textit{persona} pada banyak model dan banyak kategori tugas secara bersamaan. Kondisi ini menimbulkan kebutuhan untuk merancang pendekatan evaluasi yang komprehensif dan mampu menghasilkan pemahaman empiris yang lebih luas mengenai sensitivitas model terhadap \textit{persona}.

Berdasarkan kebutuhan tersebut, penelitian ini disusun untuk menginvestigasi pengaruh \textit{persona} terhadap penalaran, kualitas keluaran, dan kecenderungan \textit{human bias} pada berbagai \textit{large language model} melalui pendekatan \textit{multi model} dan \textit{multi persona} yang terstruktur. Penelitian ini bertujuan untuk mengisi kesenjangan penelitian yang belum ditangani oleh studi sebelumnya dengan menetapkan kontribusi utama sebagai berikut:

\begin{enumerate}
    \item Merancang dan membangun kerangka evaluasi terotomatisasi yang memungkinkan pelaksanaan eksperimen berskala besar pada berbagai model dan berbagai \textit{persona}.
    \item Mengevaluasi dampak persona terhadap performa penalaran dan kualitas keluaran model pada beragam kategori tugas, termasuk penalaran numerik, penalaran logis, dan respons berbasis skenario sosial.
    \item Menganalisis pola human bias yang muncul akibat penugasan persona, termasuk bias yang memengaruhi penalaran, gaya bahasa, dan struktur respons.
    \item Membandingkan sensitivitas dan robustnes berbagai model terhadap variasi persona untuk mengidentifikasi model yang menunjukkan konsistensi yang lebih tinggi dalam konteks penggunaan yang heterogen.
\end{enumerate}


% --- Rumusan Masalah ---
\section{Rumusan Masalah}
Rumusan Masalah berisi masalah utama yang dibahas dalam tugas akhir. Rumusan masalah yang baik memiliki struktur sebagai berikut:
\begin{enumerate}
\item	Pokok persoalan dari kondisi atau situasi yang ada saat ini. Dengan kata lain, jelaskan kelemahan atau kekurangan dari kondisi, situasi, atau solusi yang dijelaskan pada latar belakang. Ini merupakan pokok rumusan masalah.
\item	Elaborasi lebih lanjut urgensi penyelesaian masalah tersebut (mengapa penting untuk diselesaikan dan akibat yang dapat terjadi jika tidak diselesaikan).
\item	Usulan singkat terkait dengan solusi yang ditawarkan untuk menyelesaikan persoalan.
Penting untuk diperhatikan bahwa persoalan yang dideskripsikan pada subbab ini akan dipertanggungjawabkan di bab Evaluasi (apakah terselesaikan atau tidak).
\end{enumerate}

% --- Tujuan ---
\section{Tujuan}
Tuliskan tujuan utama dan/atau tujuan detail yang akan dicapai dalam pelaksanaan tugas akhir. Fokuskan pada hasil akhir yang ingin diperoleh setelah tugas akhir diselesaikan, terkait dengan penyelesaian persoalan pada rumusan masalah. Penting untuk diperhatikan bahwa tujuan yang dideskripsikan pada subbab ini akan dipertanggungjawabkan di akhir pelaksanaan tugas akhir apakah tercapai atau tidak. Tuliskan kriteria keberhasilan tugas akhir ini.

% --- Batasan Masalah ---
\section{Batasan Masalah}
Tuliskan batasan-batasan yang diambil dalam pelaksanaan tugas akhir. Batasan ini dapat dihindari (bersifat opsional, tidak perlu ada) jika topik atau judul tugas akhir dibuat cukup spesifik.

% --- Metodologi Pengerjaan TA ---
\section{Metodologi}
Tuliskan semua tahapan yang akan dilalui selama pelaksanaan tugas akhir. Tahapan ini spesifik untuk menyelesaikan persoalan tugas akhir. Khusus untuk penyusunan proposal ini, jelaskan secara detail:
\begin{enumerate}
\item	Tahapan investigasi pengumpulan fakta di latar belakang untuk merumuskan masalah.
\item	Langkah-langkah pencarian, pengelompokan, dan penapisan literatur atau sumber informasi untuk mengumpulkan informasi yang relevan tentang topik yang diangkat, termasuk teori (konsep atau teori apa saja yang perlu dicari), hal-hal yang telah dicapai oleh orang lain (cara mencari dan kata kuncinya), dan berbagai informasi pendukung, untuk mencari solusi terhadap masalah yang dibahas. Gunakan metodologi yang tepat dalam menggali informasi dan dokumentasikan prosesnya (termasuk rekaman wawancara atau survei) di dalam Lampiran, termasuk tautan ke video atau foto. Hasil penggalian informasi ini akan dijelaskan secara sistematis di Bab II Studi Literatur.
\end{enumerate}