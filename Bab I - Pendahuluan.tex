% ==========================================
% BAB I PENDAHULUAN
% ==========================================
\chapter{PENDAHULUAN}
\label{chap:pendahuluan}
% --- Latar Belakang ---
\section{Latar Belakang}

Kemajuan dalam pengembangan \textit{large language model}\label{term:llm} (LLM) dalam beberapa tahun terakhir telah mengubah cara sistem komputasi memahami, memproses, dan menghasilkan bahasa alami. Model seperti GPT, LLaMA, Grok, dan Gemini dilatih menggunakan korpus berskala besar dan mampu menyelesaikan berbagai tugas mulai dari penalaran numerik hingga interpretasi skenario sosial \parencite{jurafsky2023slp3}. Pada sejumlah benchmark\label{term:benchmark} terstandardisasi, model-model tersebut dapat memberikan jawaban yang akurat dan relevan. Namun, peningkatan kemampuan ini belum sepenuhnya diikuti oleh konsistensi perilaku model dalam percakapan. Perubahan kecil dalam cara pertanyaan disampaikan sering kali menghasilkan respons yang berbeda, meskipun tugas yang diberikan tetap sama \parencite{zhou2023largemodelsensitive}.

Fenomena lain yang semakin banyak dibahas dalam penelitian mutakhir adalah bahwa perilaku model tidak hanya dipengaruhi oleh isi instruksi, tetapi juga oleh cara model memersepsi identitas pengguna\label{term:persona}. Studi mengenai bias\label{term:bias} penalaran implisit menunjukkan bahwa deskripsi singkat mengenai pengguna dapat mengubah pola penalaran\label{term:reasoning} model, termasuk pada tugas-tugas yang tidak memiliki muatan sosial, seperti penalaran numerik atau penyelesaian masalah dasar \parencite{gupta2024biasrunsdeep}. Perubahan tersebut mencakup variasi langkah penyelesaian, tingkat kehati-hatian, ataupun kecenderungan preferensi tertentu terhadap kelompok sosial.

Selain persona yang dinyatakan secara langsung\label{term:persona-eksplisit}, beberapa penelitian menemukan bahwa model dapat mengasosiasikan isyarat linguistik halus\label{term:persona-implisit}—seperti pilihan kata, tingkat formalitas, atau gaya pertanyaan—dengan karakteristik tertentu dari pengguna \parencite{tseng2024twotales}. Asosiasi ini kemudian berpotensi memengaruhi strategi penyelesaian yang dipilih model, termasuk variasi pada langkah-langkah penalaran yang biasanya tercermin dalam \textit{chain-of-thought}\label{term:cot}.

Penelitian dalam pemodelan pengguna juga menunjukkan bahwa identitas pengguna—meliputi usia, latar belakang profesional, maupun pengalaman tertentu—dapat memberikan pengaruh terhadap pola respons model \parencite{naous2025userlm}. Dalam penelitian ini, identitas pengguna direpresentasikan melalui persona yang dibentuk secara eksplisit maupun implisit di dalam prompt\label{term:prompt}. Pendekatan tersebut digunakan untuk mengkaji bagaimana model membangun asumsi mengenai pengguna dan bagaimana asumsi tersebut tercermin pada keluaran model dalam berbagai skenario tugas.

Meskipun terdapat sejumlah temuan penting, penelitian terdahulu masih memiliki keterbatasan. Sebagian besar hanya melibatkan jumlah model yang terbatas, ruang persona yang sempit, atau cakupan tugas yang relatif kecil. Belum banyak penelitian yang secara sistematis membandingkan persona eksplisit dan implisit pada berbagai model dan berbagai jenis penalaran dalam kerangka eksperimen yang konsisten\label{term:multi-persona}\label{term:multi-model}. Selain itu, penelitian mengenai perbedaan antara pendekatan persona berbasis pengguna (``your user is...'') dan pendekatan berbasis model (``you are...'') juga masih terbatas, padahal kedua bentuk framing tersebut berpotensi menghasilkan respons yang berbeda\label{term:human-bias}. Dalam konteks ini, studi seperti HELM \parencite{liang2023helm} menegaskan bahwa model sensitif\label{term:sensitivity} terhadap variasi konteks yang tampak kecil, sehingga evaluasi terstruktur menjadi semakin penting.

Di sisi lain, penelitian ini juga perlu mempertimbangkan aspek teknis model yang digunakan. Sebagian besar model modern yang relevan dengan penelitian ini mengikuti arsitektur \textit{decoder-only}\label{term:decoder-only}, yang menghasilkan keluaran secara autoregresif. Arsitektur ini dominan dalam model mutakhir seperti GPT dan LLaMA, dan menjadi dasar bagi eksperimen multi-model dalam penelitian ini. Pemilihan arsitektur ini penting untuk menjaga konsistensi evaluasi dan menghindari perbedaan perilaku yang berasal dari variasi struktur model.

Konteks evaluasi juga memerlukan pemilihan benchmark yang tepat. Dalam penelitian ini, GSM8K\label{term:gsm8k} digunakan untuk menguji penalaran numerik dasar, sedangkan MMLU-Redux\label{term:mmlu-redux} digunakan untuk mengevaluasi penalaran multi-topik. Kedua benchmark tersebut membantu mengamati bagaimana persona memengaruhi keluaran model pada tipe penalaran yang berbeda, dari yang bersifat prosedural hingga kontekstual.

Selain itu, variasi hasil antar-\textit{run} pada tugas yang sama menunjukkan perlunya mekanisme evaluasi yang terstruktur dan dapat direproduksi\label{term:robustness} \parencite{turpin2023language, cobbe2021gsm8k}. 

Untuk memenuhi kebutuhan evaluasi yang konsisten tersebut, penelitian ini menggunakan pendekatan \textit{spec-driven experiment orchestration}\label{term:spec-driven}. Pendekatan ini menyusun eksperimen berdasarkan sebuah spesifikasi formal yang mendefinisikan kombinasi persona, model, dan benchmark secara eksplisit. Spesifikasi tersebut kemudian dijalankan melalui pipeline\label{term:pipeline} yang terotomatisasi sehingga setiap konfigurasi pengujian dieksekusi dengan alur yang sama. Dengan cara ini, penelitian dapat mengurangi variasi yang tidak diperlukan, menjaga ketertelusuran setiap percobaan, serta memastikan bahwa perbandingan antar-model dan antar-persona dilakukan secara adil dan dapat direproduksi.

Berangkat dari kebutuhan tersebut, penelitian ini disusun untuk mengevaluasi pengaruh persona eksplisit dan implisit melalui eksperimen terstruktur pada berbagai model dan jenis tugas penalaran. Dengan pendekatan ini, penelitian diharapkan dapat memberikan gambaran yang lebih jelas mengenai bagaimana model menafsirkan identitas pengguna dan bagaimana penafsiran tersebut memengaruhi jawaban dalam berbagai konteks tugas.



% --- Rumusan Masalah ---
\section{Rumusan Masalah}

Penelitian sebelumnya menunjukkan bahwa persona, baik yang diberikan secara eksplisit maupun yang tersirat dari gaya bahasa, dapat memengaruhi cara model menyusun penalaran dan menghasilkan jawaban \parencite{gupta2024biasrunsdeep, tseng2024twotales, naous2025userlm}. Namun, kajian yang ada masih terbatas pada jumlah model yang sedikit, ragam persona yang sempit, serta jenis tugas yang belum cukup mencerminkan variasi penalaran yang lebih luas. Kondisi ini menunjukkan perlunya evaluasi yang lebih menyeluruh untuk memahami bagaimana persona memengaruhi perilaku model dalam konteks multi-tugas dan multi-model.

Berdasarkan uraian pada bagian sebelumnya, penelitian ini merumuskan beberapa pertanyaan utama sebagai berikut.
\begin{enumerate}
    \item Sejauh mana persona yang diberikan secara eksplisit maupun yang muncul secara implisit memengaruhi proses penalaran model pada berbagai jenis tugas, khususnya penalaran numerik dan tugas multi-topik?
    \item Bagaimana variasi persona tersebut membentuk karakter keluaran model dan memunculkan pola bias tertentu, termasuk bias sosial maupun preferensi jawaban?
    \item Bagaimana perbedaan respons antar model dapat menggambarkan tingkat sensitivitas dan ketahanan masing-masing model terhadap variasi persona dalam suatu kerangka evaluasi yang disusun secara terstruktur?
\end{enumerate}

% --- Tujuan ---
\section{Tujuan Penelitian}

Tujuan penelitian ini disusun sebagai tindak lanjut dari rumusan masalah yang telah dijelaskan sebelumnya. Secara umum, penelitian ini bertujuan memperoleh pemahaman yang lebih jelas mengenai bagaimana persona memengaruhi perilaku dan penalaran \textit{large language model}. Berbeda dari kajian yang hanya bersifat deskriptif, penelitian ini dilaksanakan melalui serangkaian eksperimen terstruktur yang melibatkan beberapa model, variasi persona, dan jenis tugas penalaran.

Secara khusus, penelitian ini bertujuan untuk:

\begin{enumerate}
    \item Menyelenggarakan eksperimen yang menguji sejauh mana persona eksplisit maupun persona implisit memengaruhi proses \textit{reasoning} model pada berbagai jenis tugas.
    
    \item Mengidentifikasi perubahan karakter keluaran model serta pola \textit{bias} yang muncul sebagai akibat dari variasi persona dalam prompt.

    \item Membandingkan respons antar model untuk menilai tingkat \textit{sensitivity} dan \textit{robustness} masing-masing model terhadap perubahan persona dalam suatu pengaturan eksperimen yang konsisten dan dapat diulang.
\end{enumerate}


% --- Batasan Masalah ---
\section{Batasan Masalah}

Batasan masalah diperlukan agar ruang lingkup penelitian tetap jelas dan terarah. Penelitian ini tidak mencakup seluruh aspek perilaku \textit{large language model}, tetapi memfokuskan kajian pada bagaimana variasi persona memengaruhi respons model pada sejumlah tugas penalaran. Adapun batasan penelitian ini adalah sebagai berikut.

\begin{enumerate}
    \item Penelitian hanya mempertimbangkan dua bentuk persona yang berorientasi pada pengguna, yaitu persona eksplisit yang dinyatakan secara langsung di dalam prompt, serta persona implisit yang muncul dari variasi gaya bahasa dan cara pengguna menyampaikan pertanyaan. Kajian ini tidak mencakup \textit{role-playing persona} yang menetapkan identitas tertentu pada model, maupun pendekatan \textit{personalization} yang bergantung pada riwayat atau profil pengguna.

    \item Model yang digunakan pada penelitian ini terbatas pada model bahasa berbasis teks dengan arsitektur \textit{decoder-only} yang tersedia melalui antarmuka API\label{term:api}. Model encoder–decoder, model multimodal, maupun model yang memerlukan proses \textit{fine-tuning} atau pelatihan ulang tidak termasuk dalam ruang lingkup penelitian.

    \item Evaluasi dilakukan pada tugas-tugas berbasis teks, meliputi penalaran numerik, penalaran logis, pertanyaan pengetahuan umum, serta skenario sosial dan moral. Penelitian ini tidak membahas tugas multimodal maupun tugas berbasis \textit{speech}.

    \item Penilaian terhadap respons model dilakukan melalui evaluasi otomatis dan analisis komparatif. Penelitian tidak melibatkan penilaian dengan partisipan manusia.

    \item Seluruh eksperimen dijalankan melalui pendekatan \textit{prompt-based evaluation}\label{term:prompt-based} tanpa melakukan perubahan terhadap parameter internal model.

    \item Analisis bias dibatasi pada \textit{human bias} yang muncul sebagai konsekuensi variasi persona. Penelitian tidak mengevaluasi bias yang berasal dari data pelatihan model atau faktor struktural model lainnya.
\end{enumerate}


% --- Metodologi Pengerjaan TA ---
\section{Metodologi Penelitian}

Penelitian ini menggunakan pendekatan eksperimental berbasis pemanggilan model melalui prompt untuk melihat bagaimana persona memengaruhi respons sejumlah \textit{large language model}. Metodologi dirancang agar alur evaluasi jelas dan dapat dijalankan kembali apabila diperlukan. Tahapan penelitian disajikan sebagai berikut.

\begin{enumerate}
    \item Perumusan spesifikasi eksperimen.
    
    Tahap ini diawali dengan menyusun dokumen spesifikasi yang memetakan kombinasi persona, model, bentuk interaksi, dan jenis tugas yang akan diuji. Spesifikasi tersebut dipakai sebagai acuan sehingga pelaksanaan eksperimen berjalan dengan alur yang tetap.

    \item Penyusunan persona eksplisit dan implisit.
    
    Persona eksplisit dituliskan secara langsung di dalam prompt, sedangkan persona implisit dibangun melalui variasi gaya bahasa pengguna tanpa menyebutkan identitas secara eksplisit. Kedua bentuk persona digunakan untuk melihat bagaimana model memahami karakter pengguna dari konteks yang berbeda.

    \item Pemilihan model dan ruang evaluasi.
    
    Penelitian menggunakan beberapa model bahasa berbasis teks yang tersedia melalui API tanpa proses \textit{fine-tuning}. Tugas yang digunakan mencakup penalaran numerik, penalaran logis, pertanyaan pengetahuan umum, serta skenario sosial dan moral.

    \item Pelaksanaan eksperimen terotomatisasi.
    
    Setiap kombinasi persona, model, dan tugas dieksekusi menggunakan pendekatan \textit{prompt-based evaluation}. Seluruh proses dijalankan secara otomatis untuk mengurangi variasi yang tidak diperlukan dan menjaga alur pengujian tetap seragam.

    \item Pengolahan respons dan analisis perbandingan.
    
    Respons model dicatat dan dianalisis berdasarkan ketepatan jawaban serta pola perubahan respons yang muncul akibat perbedaan persona. Perbandingan antar model dilakukan untuk melihat sejauh mana masing-masing model peka terhadap perubahan persona.

    \item Analisis bias.
    
    Analisis difokuskan pada \textit{human bias} yang muncul selama proses tanya jawab akibat variasi persona. Penelitian ini tidak meninjau bias yang berasal dari data pelatihan atau arsitektur model.
\end{enumerate}

Metodologi ini menjadi dasar untuk pelaksanaan eksperimen dan pembahasan pada bab selanjutnya.
