% ==========================================
% BAB I PENDAHULUAN
% ==========================================
\chapter{PENDAHULUAN}
\label{chap:pendahuluan}
% --- Latar Belakang ---
\section{Latar Belakang}

Kemajuan dalam pengembangan \textit{large language model} telah menghasilkan peningkatan kemampuan dalam pemrosesan bahasa alami, pemahaman konteks, serta penalaran. Model seperti GPT, LLaMA, Mistral, dan Gemini digunakan secara luas dalam sistem dialog, agen percakapan, dan berbagai aplikasi berbasis teks. Walaupun demikian, kinerja model sering kali menunjukkan variasi yang bergantung pada identitas atau karakteristik pengguna yang tersirat dalam instruksi atau konteks percakapan.

Berbagai penelitian menunjukkan bahwa \textit{large language model} bersifat sensitif terhadap informasi identitas pengguna yang disisipkan dalam percakapan. Penelitian mengenai bias penalaran implisit menunjukkan bahwa perubahan kecil pada identitas pengguna dapat mengubah penalaran model pada tugas yang tidak memiliki dimensi sosial eksplisit \parencite{gupta2024biasrunsdeep}. Selain itu, studi mengenai identitas dan percakapan mengidentifikasi bahwa informasi identitas yang disampaikan melalui gaya, framing, atau konteks dapat memengaruhi gaya bahasa, tingkat kehati-hatian, atau preferensi jawaban \parencite{tseng2024twotales}. Penelitian mengenai pemodelan pengguna juga menunjukkan bahwa variasi atribut pengguna—seperti usia, latar belakang profesional, atau kelompok sosial—dapat memengaruhi keluaran model dalam aspek penalaran, stabilitas respons, dan kecenderungan bias \parencite{naous2025userlm}.

Berdasarkan literatur tersebut, semakin jelas bahwa \textit{large language model} tidak hanya memproses konten instruksi, tetapi juga bereaksi terhadap identitas pengguna yang diberikan secara eksplisit maupun yang tersirat dalam konteks. Dengan demikian, pengaruh \textit{user persona} menjadi aspek penting untuk dipelajari. Penelitian ini berfokus pada dua bentuk \textit{user persona}, yaitu \textit{user persona} eksplisit yang diberikan melalui deskripsi identitas yang jelas, serta \textit{user persona} implisit yang muncul melalui framing, gaya penulisan, atau narasi kontekstual tanpa instruksi langsung mengenai identitas pengguna.

Walaupun berbagai penelitian telah menunjukkan adanya sensitivitas model terhadap identitas pengguna, sebagian besar studi hanya mengevaluasi satu atau dua model, cakupan persona yang terbatas, atau rentang tugas penalaran yang sempit. Belum tersedia kerangka evaluasi yang memungkinkan analisis sistematis mengenai bagaimana \textit{user persona} memengaruhi penalaran, perilaku keluaran, dan \textit{human bias} pada berbagai model secara bersamaan. Kondisi tersebut menimbulkan kebutuhan untuk merancang pendekatan evaluasi yang lebih menyeluruh dan terstruktur.

Penelitian ini disusun untuk mengevaluasi pengaruh \textit{user persona} eksplisit dan \textit{user persona} implisit terhadap penalaran, perilaku keluaran, dan kecenderungan \textit{human bias} pada berbagai \textit{large language model} melalui pendekatan \textit{multi model} dan \textit{multi persona}. Penelitian ini dimaksudkan untuk memberikan pemahaman empiris yang lebih komprehensif mengenai sensitivitas model terhadap identitas pengguna.


% --- Rumusan Masalah ---
\section{Rumusan Masalah}

Rumusan masalah berikut disusun berdasarkan kebutuhan untuk memahami bagaimana \textit{user persona} memengaruhi perilaku dan penalaran model bahasa. Penelitian sebelumnya menunjukkan bahwa identitas pengguna, baik yang diberikan secara eksplisit maupun implisit, dapat memengaruhi penalaran, kualitas keluaran, dan kecenderungan bias model \parencite{gupta2024biasrunsdeep, tseng2024twotales, naous2025userlm}. Namun, cakupan penelitian terdahulu masih terbatas pada sedikit model, sedikit persona, dan variasi tugas yang sempit.

Berdasarkan kondisi tersebut, rumusan masalah penelitian ini adalah sebagai berikut.

\begin{enumerate}
    \item Bagaimana pengaruh \textit{user persona} eksplisit dan \textit{user persona} implisit terhadap performa penalaran pada berbagai jenis tugas pada sejumlah \textit{large language model}.
    \item Bagaimana kedua jenis \textit{user persona} tersebut memengaruhi perilaku keluaran model pada skenario interaksi yang berbeda.
    \item Bagaimana pola \textit{human bias} muncul dan berubah sebagai akibat variasi \textit{user persona}.
    \item Sejauh mana sensitivitas terhadap \textit{user persona} berbeda pada berbagai \textit{large language model}, serta model mana yang menunjukkan tingkat \textit{robustness} yang lebih tinggi terhadap variasi tersebut.
\end{enumerate}



% --- Tujuan ---
\section{Tujuan Penelitian}

Tujuan penelitian ditetapkan untuk menjawab permasalahan yang telah dirumuskan. Penelitian ini diarahkan untuk menghasilkan pemahaman yang lebih komprehensif mengenai pengaruh \textit{user persona} terhadap perilaku model bahasa dalam tugas penalaran dan skenario percakapan. Secara khusus, penelitian ini bertujuan untuk:

\begin{enumerate}
    \item Menganalisis pengaruh \textit{user persona} eksplisit dan \textit{user persona} implisit terhadap performa penalaran pada sejumlah \textit{large language model}.
    \item Mengidentifikasi perubahan perilaku keluaran model yang diinduksi oleh variasi \textit{user persona} pada berbagai konteks.
    \item Menganalisis pola \textit{human bias} yang muncul akibat variasi \textit{user persona}.
    \item Menyusun perbandingan sensitivitas dan \textit{robustness} berbagai model terhadap variasi \textit{user persona}.
    \item Mengembangkan rancangan \textit{evaluation pipeline} yang memungkinkan pelaksanaan eksperimen \textit{multi model} dan \textit{multi persona} secara terotomatisasi.
\end{enumerate}

% --- Batasan Masalah ---
\section{Batasan Masalah}

Batasan masalah ditetapkan agar ruang lingkup penelitian terkelola dan selaras dengan tujuan penelitian. Penelitian ini tidak bertujuan mengevaluasi seluruh aspek perilaku model bahasa, tetapi fokus pada pengaruh \textit{user persona}. Batasan penelitian ini adalah sebagai berikut.

\begin{enumerate}
    \item Penelitian hanya menganalisis dua jenis \textit{user persona}, yaitu \textit{user persona} eksplisit dan \textit{user persona} implisit. Penelitian tidak mencakup \textit{role-playing persona} yang memberikan identitas kepada model maupun mekanisme \textit{personalization} berbasis histori pengguna.
    \item Pengujian terbatas pada model bahasa berbasis teks yang dapat diakses melalui API. Model multimodal, model yang memerlukan \textit{fine-tuning}, atau model yang memerlukan pelatihan ulang tidak termasuk dalam cakupan penelitian.
    \item Evaluasi dibatasi pada tugas berbasis teks, termasuk penalaran numerik, penalaran logis, pertanyaan pengetahuan umum, skenario sosial, dan skenario moral. Tugas vision-language atau \textit{speech} tidak dibahas.
    \item Penilaian kualitas keluaran dilakukan melalui evaluasi terotomatisasi dan analisis komparatif. Penilaian berbasis partisipan manusia tidak dilakukan.
    \item Penelitian menggunakan \textit{evaluation pipeline} berbasis eksekusi prompt tanpa melakukan modifikasi pada parameter internal model.
    \item Analisis bias terbatas pada \textit{human bias} yang muncul sebagai akibat variasi \textit{user persona}, dan tidak mencakup bias makro yang bersumber dari data pelatihan model.
\end{enumerate}

% --- Metodologi Pengerjaan TA ---
\section{Metodologi}

Metodologi pada tahap penyusunan proposal ini disusun untuk memastikan bahwa proses perumusan masalah, penentuan ruang lingkup penelitian, dan penyusunan kerangka teoretis dilakukan secara sistematis. Metodologi ini tidak mencakup tahapan implementasi eksperimen, yang akan dijabarkan pada Bab III, melainkan berfokus pada kegiatan awal yang diperlukan untuk menghasilkan proposal penelitian yang terarah dan berbasis kajian ilmiah.

\subsection{Tahap 1: Investigasi Awal dan Pengumpulan Fakta}

Tahap awal dilakukan untuk memahami konteks permasalahan dan mengidentifikasi isu ilmiah yang relevan dengan topik penelitian. Langkah yang dilakukan meliputi:
\begin{enumerate}
    \item Mengidentifikasi fenomena sensitivitas \textit{large language model} terhadap identitas pengguna berdasarkan contoh kasus, laporan empiris, dan temuan penelitian sebelumnya.
    \item Meninjau keluaran awal beberapa model bahasa melalui eksplorasi terbatas untuk mengamati indikasi pengaruh \textit{user persona} eksplisit dan \textit{user persona} implisit terhadap penalaran dan gaya respons.
    \item Menyimpulkan pola permasalahan yang muncul untuk kemudian dirumuskan sebagai pokok masalah penelitian.
\end{enumerate}

\subsection{Tahap 2: Pencarian, Pengelompokan, dan Penapisan Literatur}

Tahap ini dilakukan untuk memperoleh landasan ilmiah yang kuat dalam menyusun kerangka teoretis dan menentukan arah penelitian. Kegiatan yang dilakukan mencakup:
\begin{enumerate}
    \item Melakukan pencarian literatur menggunakan mesin pencarian akademik seperti Google Scholar, Semantic Scholar, arXiv, dan ACL Anthology dengan kata kunci antara lain \textit{user persona}, \textit{implicit persona}, \textit{identity-conditioned prompting}, \textit{LLM sensitivity}, \textit{reasoning evaluation}, dan \textit{bias in LLM}.
    \item Menyeleksi publikasi yang relevan, termasuk penelitian mengenai pengaruh identitas pengguna terhadap keluaran model bahasa, teori penalaran pada model bahasa, evaluasi berbasis prompt, dan bias implisit.
    \item Mengelompokkan literatur ke dalam kategori konseptual, yaitu:  
    (a) konsep dasar \textit{large language model},  
    (b) teori dan klasifikasi \textit{persona} eksplisit dan implisit,  
    (c) penelitian terdahulu mengenai identitas pengguna dan pengaruhnya terhadap keluaran model,  
    (d) metode evaluasi penalaran dan analisis bias.
    \item Menganalisis dan merangkum kontribusi, metodologi, serta keterbatasan setiap publikasi yang terpilih untuk memastikan bahwa kerangka teoretis proposal didasarkan pada referensi yang valid dan mutakhir.
    \item Mendokumentasikan seluruh proses penelusuran literatur, termasuk daftar kata kunci, sumber pencarian, dan kriteria penapisan yang digunakan. Dokumentasi tambahan, seperti rekaman proses eksplorasi awal atau catatan observasi, akan dicantumkan pada bagian lampiran.
\end{enumerate}

Tahap-tahap tersebut menghasilkan landasan konseptual dan rumusan permasalahan yang digunakan dalam penyusunan proposal tugas akhir. Hasil kajian literatur secara rinci akan disajikan pada Bab II Studi Literatur.
