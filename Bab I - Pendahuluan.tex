% ==========================================
% BAB I PENDAHULUAN
% ==========================================
\chapter{PENDAHULUAN}
\label{chap:pendahuluan}
% --- Latar Belakang ---
\section{Latar Belakang}
Kemajuan signifikan dalam pengembangan \textit{large language model} telah menghasilkan peningkatan kemampuan dalam pemrosesan bahasa alami, pemahaman konteks, dan penalaran. Model seperti GPT, LLaMA, Mistral, dan Gemini digunakan secara luas dalam sistem dialog, agen percakapan, dan berbagai aplikasi yang bergantung pada interaksi bahasa. Meskipun performanya meningkat, sejumlah tantangan metodologis masih muncul, khususnya terkait sensitivitas model terhadap identitas pengguna atau konfigurasi \textit{persona} yang diberikan selama proses interaksi.

Beberapa studi menunjukkan bahwa \textit{large language model} bersifat responsif terhadap \textit{persona}-yang-disematkan kepada model. Penelitian mengenai bias penalaran implisit menunjukkan bahwa penugasan \textit{persona} tertentu dapat mengubah hasil penalaran pada tugas yang tidak mengandung informasi sosial \parencite{gupta2024biasrunsdeep}. Studi lain mengkaji pengaruh \textit{persona} eksplisit dan \textit{persona} implisit terhadap gaya respons dan perilaku model dalam skenario percakapan \parencite{tseng2024twotales}. Selain itu, penelitian mengenai pemodelan pengguna menunjukkan bahwa variasi karakteristik pengguna dapat memengaruhi keluaran model dalam aspek penalaran, preferensi, dan stabilitas respons \parencite{naous2025userlm}. Temuan tersebut mengindikasikan bahwa \textit{large language model} tidak hanya memproses instruksi, tetapi juga bereaksi terhadap atribut identitas yang diberikan melalui \textit{persona}.

Walaupun demikian, penelitian yang ada masih memiliki batasan metodologis. Mayoritas studi mengevaluasi jumlah model yang terbatas, jumlah \textit{persona} yang sempit, atau rentang tugas yang tidak mencakup variasi penalaran yang relevan. Selain itu, belum tersedia kerangka evaluasi yang memungkinkan analisis sistematis terhadap pengaruh \textit{persona} pada banyak model dan banyak kategori tugas secara bersamaan. Kondisi ini menimbulkan kebutuhan untuk merancang pendekatan evaluasi yang komprehensif dan mampu menghasilkan pemahaman empiris yang lebih luas mengenai sensitivitas model terhadap \textit{persona}.

Berdasarkan kebutuhan tersebut, penelitian ini disusun untuk menginvestigasi pengaruh \textit{persona} terhadap penalaran, kualitas keluaran, dan kecenderungan \textit{human bias} pada berbagai \textit{large language model} melalui pendekatan \textit{multi model} dan \textit{multi persona} yang terstruktur. Penelitian ini bertujuan untuk mengisi kesenjangan penelitian yang belum ditangani oleh studi sebelumnya dengan menetapkan kontribusi utama sebagai berikut:

\begin{enumerate}
    \item Mengevaluasi dampak \textit{persona} terhadap performa penalaran dan kualitas keluaran model pada beragam kategori tugas, termasuk penalaran numerik, penalaran logis, dan respons berbasis skenario sosial.
    \item Menganalisis pola \textit{human bias} yang muncul akibat penugasan \textit{persona}, termasuk bias yang memengaruhi penalaran, gaya bahasa, dan struktur respons.
    \item Membandingkan sensitivitas dan \textit{robustness} berbagai model terhadap variasi \textit{persona} untuk mengidentifikasi model yang menunjukkan konsistensi yang lebih tinggi dalam konteks penggunaan yang heterogen.
\end{enumerate}



% --- Rumusan Masalah ---
\section{Rumusan Masalah}

Rumusan masalah dalam penelitian ini disusun berdasarkan kebutuhan untuk memahami bagaimana \textit{large language model} bereaksi terhadap variasi \textit{persona} pada berbagai kategori tugas. Penelitian sebelumnya menunjukkan bahwa penugasan \textit{persona} dapat memengaruhi penalaran, bias implisit, kualitas keluaran, serta stabilitas respons model \parencite{gupta2024biasrunsdeep, tseng2024twotales, naous2025userlm}. Temuan tersebut mengindikasikan bahwa \textit{large language model} tidak hanya memproses instruksi secara literal, tetapi juga bereaksi terhadap atribut identitas yang disematkan melalui \textit{persona}.

Walaupun demikian, penelitian yang ada masih terbatas pada jumlah model yang sedikit, cakupan \textit{persona} yang sempit, dan jenis tugas yang tidak mencerminkan keragaman kebutuhan evaluasi penalaran. Selain itu, belum tersedia kerangka evaluasi yang mampu menganalisis pengaruh \textit{persona} pada banyak model dan banyak kategori tugas secara konsisten. Kekurangan tersebut menimbulkan kebutuhan untuk merancang pendekatan yang lebih sistematis dan terstruktur untuk mengevaluasi dampak \textit{persona}.

Berdasarkan kebutuhan tersebut, rumusan masalah dalam penelitian ini adalah sebagai berikut.

\begin{enumerate}
    \item Bagaimana pengaruh variasi \textit{persona} terhadap performa penalaran pada berbagai tugas yang dievaluasi pada sejumlah \textit{large language model}.
    \item Bagaimana variasi \textit{persona} memengaruhi kualitas keluaran model.
    \item Bagaimana pola \textit{human bias} muncul dan berubah ketika model diberikan \textit{persona} tertentu.
    \item Sejauh mana sensitivitas terhadap \textit{persona} berbeda pada berbagai \textit{large language model}, dan model mana yang menunjukkan tingkat \textit{robustness} yang lebih tinggi terhadap variasi tersebut.
\end{enumerate}


% --- Tujuan ---
\section{Tujuan Penelitian}

Tujuan penelitian dirumuskan berdasarkan pokok persoalan yang telah dijelaskan pada rumusan masalah. Penelitian ini bertujuan untuk menghasilkan pemahaman yang lebih sistematis dan komprehensif mengenai sejauh mana \textit{persona} memengaruhi performa dan perilaku \textit{large language model} pada berbagai kategori tugas. Untuk mencapai tujuan tersebut, penelitian ini menetapkan beberapa sasaran utama sebagai berikut.

\begin{enumerate}
    \item Menghasilkan analisis empiris mengenai pengaruh \textit{persona} terhadap performa penalaran yang ditunjukkan oleh berbagai \textit{large language model} pada beragam jenis tugas.
    \item Mengidentifikasi perubahan kualitas keluaran model yang muncul akibat penugasan \textit{persona} pada skenario interaksi yang berbeda.
    \item Menganalisis pola \textit{human bias} yang diinduksi oleh \textit{persona} tertentu serta implikasinya terhadap konsistensi dan stabilitas respons model.
    \item Menyusun perbandingan sensitivitas berbagai model terhadap variasi \textit{persona} guna menilai tingkat \textit{robustness} yang ditunjukkan oleh setiap model.
    \item Mengembangkan rancangan \textit{evaluation pipeline} yang memungkinkan pelaksanaan eksperimen \textit{multi model} dan \textit{multi persona} secara terotomatisasi.
\end{enumerate}

% --- Batasan Masalah ---
\section{Batasan Masalah}
Tuliskan batasan-batasan yang diambil dalam pelaksanaan tugas akhir. Batasan ini dapat dihindari (bersifat opsional, tidak perlu ada) jika topik atau judul tugas akhir dibuat cukup spesifik.

% --- Metodologi Pengerjaan TA ---
\section{Metodologi}
Tuliskan semua tahapan yang akan dilalui selama pelaksanaan tugas akhir. Tahapan ini spesifik untuk menyelesaikan persoalan tugas akhir. Khusus untuk penyusunan proposal ini, jelaskan secara detail:
\begin{enumerate}
\item	Tahapan investigasi pengumpulan fakta di latar belakang untuk merumuskan masalah.
\item	Langkah-langkah pencarian, pengelompokan, dan penapisan literatur atau sumber informasi untuk mengumpulkan informasi yang relevan tentang topik yang diangkat, termasuk teori (konsep atau teori apa saja yang perlu dicari), hal-hal yang telah dicapai oleh orang lain (cara mencari dan kata kuncinya), dan berbagai informasi pendukung, untuk mencari solusi terhadap masalah yang dibahas. Gunakan metodologi yang tepat dalam menggali informasi dan dokumentasikan prosesnya (termasuk rekaman wawancara atau survei) di dalam Lampiran, termasuk tautan ke video atau foto. Hasil penggalian informasi ini akan dijelaskan secara sistematis di Bab II Studi Literatur.
\end{enumerate}