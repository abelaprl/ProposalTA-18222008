% ==========================================
% BAB I PENDAHULUAN
% ==========================================
\chapter{PENDAHULUAN}
\label{chap:pendahuluan}
% --- Latar Belakang ---
\section{Latar Belakang}

Kemajuan dalam pengembangan \textit{large language model} dalam beberapa tahun terakhir telah mengubah cara sistem komputasi memahami, memproses, dan menghasilkan bahasa alami. Model seperti GPT, LLaMA, Mistral, dan Gemini dilatih menggunakan korpus dalam skala masif dan mampu menyelesaikan berbagai tugas mulai dari penalaran numerik hingga interpretasi skenario sosial. Dalam banyak kasus, model menunjukkan kemampuan yang mendekati atau bahkan melampaui performa manusia pada benchmark tertentu. Walaupun demikian, peningkatan kapabilitas ini tidak sepenuhnya diikuti oleh stabilitas perilaku model dalam konteks interaksi dunia nyata.

Salah satu fenomena yang semakin banyak diamati dalam penelitian mutakhir adalah bahwa perilaku \textit{large language model} tidak hanya dipengaruhi oleh isi instruksi, tetapi juga oleh identitas pengguna yang tersirat atau dinyatakan secara eksplisit dalam konteks percakapan. Studi mengenai bias penalaran implisit menunjukkan bahwa perubahan kecil pada deskripsi identitas pengguna dapat menyebabkan variasi signifikan pada hasil penalaran, bahkan untuk tugas yang tidak memiliki aspek sosial eksplisit \parencite{gupta2024biasrunsdeep}. Variasi ini mencakup perubahan langkah penalaran, perbedaan tingkat kehati-hatian, hingga munculnya bias tertentu terhadap kelompok sosial.

Selain \textit{user persona} eksplisit yang dituliskan secara langsung dalam instruksi, penelitian menunjukkan bahwa model juga sensitif terhadap \textit{user persona} implisit yang muncul melalui gaya bahasa, framing naratif, struktur pertanyaan, atau atribut linguistik lainnya \parencite{tseng2024twotales}. Dalam kondisi tersebut, model tidak menerima instruksi tentang identitas pengguna, tetapi tetap membentuk asumsi internal mengenai siapa pengguna dan menyesuaikan respons sesuai asumsi tersebut. Sensitivitas ini menandakan bahwa model melakukan inferensi identitas pengguna berdasarkan sinyal linguistik yang tampak sepele, yang berimplikasi pada stabilitas penalaran dan keadilan respons.

Penelitian pada bidang pemodelan pengguna menunjukkan bahwa variasi identitas pengguna—seperti usia, latar belakang profesional, afiliasi budaya, atau posisi sosial—dapat memengaruhi keluaran model dalam berbagai dimensi, termasuk penalaran, preferensi jawaban, dan konsistensi respons \parencite{naous2025userlm}. Hal ini menunjukkan bahwa identitas pengguna, baik eksplisit maupun implisit, berfungsi sebagai variabel laten yang memengaruhi proses generatif model. Dengan demikian, analisis terhadap \textit{user persona} menjadi penting tidak hanya untuk memahami perilaku model, tetapi juga untuk mengidentifikasi potensi bias dan ketidakstabilan yang muncul dalam interaksi manusia–AI.

Walaupun berbagai studi sebelumnya memberikan indikasi bahwa identitas pengguna memengaruhi perilaku model, penelitian yang ada masih memiliki batasan. Mayoritas studi hanya mengevaluasi satu atau dua model, cakupan persona yang terbatas, atau jenis tugas yang sempit. Selain itu, tidak banyak studi yang secara sistematis membandingkan efek \textit{user persona} eksplisit dan implisit pada berbagai model dan berbagai jenis tugas penalaran dalam satu kerangka eksperimen yang konsisten. Belum tersedia pula pendekatan evaluasi yang secara terpadu menguji sensitivitas model terhadap variasi identitas pengguna di berbagai kondisi tugas, baik numerik, logis, faktual, sosial, maupun moral.

Kekosongan penelitian ini penting untuk dijembatani, mengingat model bahasa semakin banyak digunakan pada skenario yang sensitif terhadap identitas pengguna, seperti layanan kesehatan, pendidikan, konseling, sistem rekomendasi, dan interaksi berbasis nilai. Ketidakstabilan respons akibat identitas pengguna berpotensi menimbulkan bias, mengurangi keandalan model, dan menghasilkan ketidaksetaraan dalam pengalaman pengguna. Oleh karena itu, diperlukan pendekatan evaluasi yang lebih komprehensif untuk memahami bagaimana \textit{user persona} eksplisit dan implisit memengaruhi penalaran, perilaku keluaran, dan kecenderungan \textit{human bias} pada berbagai \textit{large language model}.

Berdasarkan urgensi tersebut, penelitian ini disusun untuk melakukan evaluasi empiris terhadap pengaruh \textit{user persona} eksplisit dan \textit{user persona} implisit melalui eksperimen terstruktur pada berbagai model dan berbagai jenis tugas. Penelitian ini diharapkan memberikan pemahaman yang lebih mendalam mengenai sensitivitas model terhadap identitas pengguna serta implikasinya terhadap penalaran, bias, dan keandalan model dalam aplikasi dunia nyata.


% --- Rumusan Masalah ---
\section{Rumusan Masalah}

Rumusan masalah berikut disusun berdasarkan kebutuhan untuk memahami bagaimana \textit{user persona} memengaruhi perilaku dan penalaran model bahasa. Penelitian sebelumnya menunjukkan bahwa identitas pengguna, baik yang diberikan secara eksplisit maupun implisit, dapat memengaruhi penalaran, kualitas keluaran, dan kecenderungan bias model \parencite{gupta2024biasrunsdeep, tseng2024twotales, naous2025userlm}. Namun, cakupan penelitian terdahulu masih terbatas pada sedikit model, sedikit persona, dan variasi tugas yang sempit.

Berdasarkan kondisi tersebut, rumusan masalah penelitian ini adalah sebagai berikut.

\begin{enumerate}
    \item Bagaimana pengaruh \textit{user persona} eksplisit dan \textit{user persona} implisit terhadap performa penalaran pada berbagai jenis tugas pada sejumlah \textit{large language model}.
    \item Bagaimana kedua jenis \textit{user persona} tersebut memengaruhi perilaku keluaran model pada skenario interaksi yang berbeda.
    \item Bagaimana pola \textit{human bias} muncul dan berubah sebagai akibat variasi \textit{user persona}.
    \item Sejauh mana sensitivitas terhadap \textit{user persona} berbeda pada berbagai \textit{large language model}, serta model mana yang menunjukkan tingkat \textit{robustness} yang lebih tinggi terhadap variasi tersebut.
\end{enumerate}



% --- Tujuan ---
\section{Tujuan Penelitian}

Tujuan penelitian ditetapkan untuk menjawab permasalahan yang telah dirumuskan. Penelitian ini diarahkan untuk menghasilkan pemahaman yang lebih komprehensif mengenai pengaruh \textit{user persona} terhadap perilaku model bahasa dalam tugas penalaran dan skenario percakapan. Secara khusus, penelitian ini bertujuan untuk:

\begin{enumerate}
    \item Menganalisis pengaruh \textit{user persona} eksplisit dan \textit{user persona} implisit terhadap performa penalaran pada sejumlah \textit{large language model}.
    \item Mengidentifikasi perubahan perilaku keluaran model yang diinduksi oleh variasi \textit{user persona} pada berbagai konteks.
    \item Menganalisis pola \textit{human bias} yang muncul akibat variasi \textit{user persona}.
    \item Menyusun perbandingan sensitivitas dan \textit{robustness} berbagai model terhadap variasi \textit{user persona}.
    \item Mengembangkan rancangan \textit{evaluation pipeline} yang memungkinkan pelaksanaan eksperimen \textit{multi model} dan \textit{multi persona} secara terotomatisasi.
\end{enumerate}

% --- Batasan Masalah ---
\section{Batasan Masalah}

Batasan masalah ditetapkan agar ruang lingkup penelitian terkelola dan selaras dengan tujuan penelitian. Penelitian ini tidak bertujuan mengevaluasi seluruh aspek perilaku model bahasa, tetapi fokus pada pengaruh \textit{user persona}. Batasan penelitian ini adalah sebagai berikut.

\begin{enumerate}
    \item Penelitian hanya menganalisis dua jenis \textit{user persona}, yaitu \textit{user persona} eksplisit dan \textit{user persona} implisit. Penelitian tidak mencakup \textit{role-playing persona} yang memberikan identitas kepada model maupun mekanisme \textit{personalization} berbasis histori pengguna.
    \item Pengujian terbatas pada model bahasa berbasis teks yang dapat diakses melalui API. Model multimodal, model yang memerlukan \textit{fine-tuning}, atau model yang memerlukan pelatihan ulang tidak termasuk dalam cakupan penelitian.
    \item Evaluasi dibatasi pada tugas berbasis teks, termasuk penalaran numerik, penalaran logis, pertanyaan pengetahuan umum, skenario sosial, dan skenario moral. Tugas vision-language atau \textit{speech} tidak dibahas.
    \item Penilaian kualitas keluaran dilakukan melalui evaluasi terotomatisasi dan analisis komparatif. Penilaian berbasis partisipan manusia tidak dilakukan.
    \item Penelitian menggunakan \textit{evaluation pipeline} berbasis eksekusi prompt tanpa melakukan modifikasi pada parameter internal model.
    \item Analisis bias terbatas pada \textit{human bias} yang muncul sebagai akibat variasi \textit{user persona}, dan tidak mencakup bias makro yang bersumber dari data pelatihan model.
\end{enumerate}

% --- Metodologi Pengerjaan TA ---
\section{Metodologi}

Metodologi pada tahap penyusunan proposal ini disusun untuk memastikan bahwa proses perumusan masalah, penentuan ruang lingkup penelitian, dan penyusunan kerangka teoretis dilakukan secara sistematis. Metodologi ini tidak mencakup tahapan implementasi eksperimen, yang akan dijabarkan pada Bab III, melainkan berfokus pada kegiatan awal yang diperlukan untuk menghasilkan proposal penelitian yang terarah dan berbasis kajian ilmiah.

\subsection{Tahap 1: Investigasi Awal dan Pengumpulan Fakta}

Tahap awal dilakukan untuk memahami konteks permasalahan dan mengidentifikasi isu ilmiah yang relevan dengan topik penelitian. Langkah yang dilakukan meliputi:
\begin{enumerate}
    \item Mengidentifikasi fenomena sensitivitas \textit{large language model} terhadap identitas pengguna berdasarkan contoh kasus, laporan empiris, dan temuan penelitian sebelumnya.
    \item Meninjau keluaran awal beberapa model bahasa melalui eksplorasi terbatas untuk mengamati indikasi pengaruh \textit{user persona} eksplisit dan \textit{user persona} implisit terhadap penalaran dan gaya respons.
    \item Menyimpulkan pola permasalahan yang muncul untuk kemudian dirumuskan sebagai pokok masalah penelitian.
\end{enumerate}

\subsection{Tahap 2: Pencarian, Pengelompokan, dan Penapisan Literatur}

Tahap ini dilakukan untuk memperoleh landasan ilmiah yang kuat dalam menyusun kerangka teoretis dan menentukan arah penelitian. Kegiatan yang dilakukan mencakup:
\begin{enumerate}
    \item Melakukan pencarian literatur menggunakan mesin pencarian akademik seperti Google Scholar, Semantic Scholar, arXiv, dan ACL Anthology dengan kata kunci antara lain \textit{user persona}, \textit{implicit persona}, \textit{identity-conditioned prompting}, \textit{LLM sensitivity}, \textit{reasoning evaluation}, dan \textit{bias in LLM}.
    \item Menyeleksi publikasi yang relevan, termasuk penelitian mengenai pengaruh identitas pengguna terhadap keluaran model bahasa, teori penalaran pada model bahasa, evaluasi berbasis prompt, dan bias implisit.
    \item Mengelompokkan literatur ke dalam kategori konseptual, yaitu:  
    (a) konsep dasar \textit{large language model},  
    (b) teori dan klasifikasi \textit{persona} eksplisit dan implisit,  
    (c) penelitian terdahulu mengenai identitas pengguna dan pengaruhnya terhadap keluaran model,  
    (d) metode evaluasi penalaran dan analisis bias.
    \item Menganalisis dan merangkum kontribusi, metodologi, serta keterbatasan setiap publikasi yang terpilih untuk memastikan bahwa kerangka teoretis proposal didasarkan pada referensi yang valid dan mutakhir.
    \item Mendokumentasikan seluruh proses penelusuran literatur, termasuk daftar kata kunci, sumber pencarian, dan kriteria penapisan yang digunakan. Dokumentasi tambahan, seperti rekaman proses eksplorasi awal atau catatan observasi, akan dicantumkan pada bagian lampiran.
\end{enumerate}

Tahap-tahap tersebut menghasilkan landasan konseptual dan rumusan permasalahan yang digunakan dalam penyusunan proposal tugas akhir. Hasil kajian literatur secara rinci akan disajikan pada Bab II Studi Literatur.
