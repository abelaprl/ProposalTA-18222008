% ==========================================
% BAB IV DESAIN KONSEP SOLUSI
% ==========================================

\chapter{DESAIN KONSEP SOLUSI}
\label{chap:desain-konsep-solusi}

Bab ini memaparkan rancangan konsep solusi yang diusulkan untuk menjawab permasalahan yang telah dianalisis pada bab sebelumnya. Berdasarkan hasil analisis pemilihan solusi, pendekatan yang digunakan dalam penelitian ini adalah pengembangan sistem eksperimen terotomatisasi berbasis konfigurasi. Pembahasan dalam bab ini mencakup desain konseptual eksperimen, perancangan arsitektur perangkat lunak atau \textit{evaluation pipeline}, serta spesifikasi implementasi data dan struktur berkas. Desain ini disusun untuk memenuhi kebutuhan fungsional terkait strukturisasi \textit{user persona} dan konsistensi eksekusi lintas model.
% ==========================================
%====== BAB IV.1 ======
\section{Desain Konseptual Eksperimen}
\label{sec:desain-konseptual}

Bagian ini memaparkan dasar konseptual dari eksperimen yang dikembangkan untuk mengukur pengaruh \textit{user persona} terhadap perilaku model bahasa. Perancangan ini mencakup evaluasi terhadap keterbatasan pendekatan konvensional, prinsip desain sistem terotomatisasi yang diusulkan, serta integrasi persona, model, dan \textit{benchmark} penalaran yang digunakan. Dengan adanya desain konseptual ini, alur eksperimen yang dibahas pada subbab berikutnya menjadi lebih terarah, terukur, dan dapat direplikasi.

\subsection{Keterbatasan Model Operasional Konvensional}

Pendekatan manual yang lazim digunakan dalam penelitian persona umumnya bergantung pada penulisan instruksi langsung melalui antarmuka percakapan. Meskipun sederhana, pendekatan ini memiliki dua kelemahan metodologis utama yang mengurangi validitas internal penelitian.

Pertama, terjadi instabilitas masukan. Model bahasa sangat sensitif terhadap perubahan kecil pada struktur \textit{prompt}—seperti variasi tanda baca atau perubahan gaya kalimat—yang dapat menyebabkan keluaran berbeda secara signifikan. Turpin et al.\ menunjukkan bahwa variasi kecil dalam \textit{framing} dapat menghasilkan rantai penalaran yang tidak konsisten \parencite{turpin2023language}. Ketergantungan pada input manual membuat variasi ini tidak dapat dikendalikan.

Kedua, pendekatan manual tidak menyediakan granularitas data yang memadai. Respons model biasanya hanya dicatat dalam bentuk teks akhir tanpa metadata komputasional seperti latensi atau jumlah token. Padahal, indikator tersebut penting untuk memahami beban kognitif model serta pola penalaran yang muncul pada kondisi persona tertentu \parencite{naous2025userlm}.

\subsection{Sistem Eksperimen Terotomatisasi}

Untuk mengatasi keterbatasan tersebut, penelitian ini mengusulkan sistem eksperimen terotomatisasi dengan tiga prinsip desain inti: determinisme masukan, telemetri komprehensif, dan skalabilitas eksekusi.

Pertama, determinisme masukan dicapai dengan menyimpan seluruh persona dalam berkas konfigurasi statis. Setiap \textit{prompt} dibentuk melalui mekanisme injeksi otomatis sehingga stimulus yang diterima model identik hingga tingkat karakter.

Kedua, sistem mencatat keluaran model secara lengkap dalam format terstruktur (JSON dan CSV). Telemetri yang direkam mencakup teks jawaban, \textit{reasoning trace} (bila tersedia), jumlah token, dan latensi inferensi. Dengan demikian, analisis dapat menilai tidak hanya kebenaran jawaban, tetapi juga pola beban komputasi yang terkait dengan masing-masing persona.

Ketiga, sistem mendukung eksekusi \textit{asynchronous} sehingga ribuan permintaan dapat diproses secara paralel tanpa melampaui batas layanan. Pendekatan ini meningkatkan cakupan eksperimen sekaligus mengurangi waktu eksekusi total.

Model konseptual dari sistem usulan ditampilkan pada Gambar~\ref{fig:model-after}, yang dalam penomoran dokumen dicantumkan sebagai \textbf{Gambar IV.2 Model Konseptual Sistem Eksperimen Terotomatisasi}.

\begin{figure}[htbp]
  \centering
  \includegraphics[width=0.9\textwidth,height=0.7\textheight,keepaspectratio]{image/proposed_solution.png}
  \caption{Model konseptual sistem eksperimen terotomatisasi}
  \label{fig:model-after}
\end{figure}

\subsection{Analisis Komparatif Metodologis}

Peralihan dari pendekatan \textit{Existing} ke \textit{Proposed} bukan sekadar peningkatan efisiensi, melainkan juga penguatan validitas ilmiah. Tabel~\ref{tab:komparasi-metodologis} merangkum perbedaan metodologis utama antara kedua pendekatan tersebut.

Dengan struktur yang deterministik, terdokumentasi, dan terotomatisasi, setiap kesimpulan terkait pengaruh persona terhadap penalaran model dapat ditarik secara lebih dapat dipercaya dan dapat dipertanggungjawabkan secara ilmiah.

\begin{table}[htbp]
  \centering
  \caption{Analisis komparatif validitas metodologis}
  \label{tab:komparasi-metodologis}
  \renewcommand{\arraystretch}{1.25}
  \small
  \begin{tabularx}{\textwidth}{
    >{\raggedright\arraybackslash}p{3.6cm}
    >{\raggedright\arraybackslash}X
    >{\raggedright\arraybackslash}X}
    \toprule
    \textbf{Dimensi Analisis} &
    \textbf{Sistem Konvensional (Existing)} &
    \textbf{Sistem Usulan (Proposed)} \\
    \midrule

    Pengendalian variabel &
    Rentan terhadap gangguan input manual; sensitivitas \textit{framing} sulit dikendalikan. &
    Deterministik; konfigurasi statis dan injeksi otomatis menjamin isolasi variabel independen. \\

    Granularitas data &
    Hanya menangkap jawaban akhir tanpa metadata. &
    Telemetri lengkap; menangkap latensi, token, dan jejak penalaran. \\

    Format penyimpanan &
    Tidak terstruktur; raw text sulit diproses ulang. &
    Terstruktur (JSON/CSV) dan cocok untuk analisis lanjutan. \\

    Reproduktibilitas &
    Rendah; parameter eksperimen tidak lengkap terdokumentasi. &
    Tinggi; seluruh konfigurasi dan kode dapat diaudit dan dijalankan ulang. \\

    Cakupan eksperimen &
    Terbatas; eksekusi linear menghambat skala eksperimen. &
    Masif; mendukung ribuan kombinasi melalui \textit{asynchronous execution}. \\
    \bottomrule
  \end{tabularx}
\end{table}

\subsection{Integrasi Persona, Model, dan Benchmark}

Bagian ini memaparkan komponen yang membentuk konfigurasi eksperimen, yaitu himpunan persona, himpunan model, dan \textit{benchmark} penalaran yang digunakan.

\subsubsection{Benchmark Penalaran}

Dua \textit{benchmark} digunakan untuk mengevaluasi dua bentuk penalaran yang berbeda.

\textit{GSM8K} merupakan kumpulan soal cerita matematika tingkat sekolah menengah yang menguji penalaran numerik bertahap \parencite{cobbe2021gsm8k}. Soal-soal GSM8K memiliki struktur jawaban numerik yang jelas sehingga proses evaluasi dapat dilakukan secara deterministik.

\textit{MMLU-Redux} merupakan versi terkurasi dari MMLU yang memperbaiki ambiguitas format, ketidakkonsistenan pilihan jawaban, dan ketidakseimbangan kualitas antar-subjek \parencite{mmluRedux2024dataset}. Benchmark ini digunakan untuk menguji penalaran konseptual lintas domain dengan format pilihan ganda.

Kombinasi GSM8K dan MMLU-Redux memberikan cakupan dua jenis penalaran yang komplementer: numerik prosedural dan konseptual deklaratif.

\subsubsection{Himpunan Model}

Penelitian ini menggunakan sembilan model bahasa, yang dikelompokkan ke dalam dua kategori berikut:

\begin{enumerate}
    \item Model komersial:  
    GPT-5, GPT-5 Mini, Claude 4.5 Sonnet, Claude 4.5 Haiku, Gemini 2.5 Flash, Gemini 2.5 Pro;

    \item Model publik via OpenRouter:  
    Grok 4.1 Fast, NVIDIA Nemotron-nano-12B-v2-VL, Bert Nebulon Alpha.
\end{enumerate}

Keragaman arsitektur ini memungkinkan analisis komparatif lintas paradigma, mulai dari \textit{frontier models} hingga model publik yang tersedia secara bebas.

\subsubsection{Struktur Persona}

Persona disusun berdasarkan enam dimensi utama: gender, usia, agama, pekerjaan, kewarganegaraan, dan register bahasa. Kombinasi dimensi tersebut menghasilkan lima belas persona—baik eksplisit maupun implisit—ditambah satu kondisi pengguna netral.

Tabel~\ref{tab:user-persona} menampilkan daftar lengkap persona yang digunakan.

\begin{table}[htbp]
\centering
\caption{Daftar user persona untuk kondisi eksperimen}
\label{tab:user-persona}
\renewcommand{\arraystretch}{1.15}

\resizebox{\textwidth}{!}{
\begin{tabular}{c l l l l l l l}
\toprule
\textbf{ID} &
\textbf{Persona} &
\textbf{Mode} &
\textbf{Gender} &
\textbf{Age Group} &
\textbf{Religion} &
\textbf{Occupation} &
\textbf{Nationality / Register} \\
\midrule
P1  & Implicit male baseline              & Implicit & Male   & --          & --       & --               & Neutral \\
P2  & Implicit female baseline            & Implicit & Female & --          & --       & --               & Neutral \\
P3  & Neutral user                        & Neutral  & --     & --          & --       & --               & Neutral \\
P4  & Indonesian Muslim young woman       & Explicit & Female & Young adult & Muslim   & Healthcare worker & Indonesian / Semi-formal \\
P5  & Indonesian Muslim young man         & Implicit & Male   & Young adult & Muslim   & Healthcare worker & Indonesian / Semi-formal \\
P6  & American middle-aged male           & Explicit & Male   & Middle-aged & Christian & Engineer         & American / Formal \\
P7  & American middle-aged female         & Implicit & Female & Middle-aged & Christian & Engineer         & American / Formal \\
P8  & Indonesian Gen-Z female             & Explicit & Female & Gen-Z       & --       & Student          & Indonesian / Casual-slang \\
P9  & Indonesian Gen-Z male               & Implicit & Male   & Gen-Z       & --       & Student          & Indonesian / Casual-slang \\
P10 & Middle Eastern young adult male     & Explicit & Male   & Young adult & Muslim   & Engineer         & Middle Eastern Arabic / Formal \\
P11 & Middle Eastern young adult female   & Implicit & Female & Young adult & Muslim   & Student          & Middle Eastern Arabic / Formal \\
P12 & American atheist young male         & Explicit & Male   & Young adult & Atheist  & Student          & American / Formal \\
P13 & American atheist young female       & Implicit & Female & Young adult & Atheist  & Student          & American / Formal \\
P14 & Indonesian female healthcare worker & Explicit & Female & Young adult & Muslim   & Healthcare worker & Indonesian / Semi-formal \\
P15 & Indonesian male healthcare worker   & Implicit & Male   & Young adult & Muslim   & Healthcare worker & Indonesian / Semi-formal \\
\bottomrule
\end{tabular}
}
\end{table}
\subsubsection{Konfigurasi Eksekusi}

Kombinasi lima belas persona dan sembilan model menghasilkan seratus tiga puluh lima
konfigurasi eksperimen. Setiap konfigurasi melalui empat tahap eksekusi tetap:
pemanasan persona, penetapan konteks percakapan, eksekusi \textit{benchmark}, dan
pencatatan telemetri. Dengan demikian, setiap variasi keluaran dapat ditelusuri ulang
secara langsung kepada kondisi persona yang diberikan.
\subsubsection{Contoh Mekanisme Injeksi Persona}

Proses injeksi persona dilakukan dengan menyusun \textit{system message} 
yang mendefinisikan identitas dan karakter pengguna sebelum model menerima 
stimulus pertanyaan. Dua kategori persona yang digunakan dalam penelitian ini 
adalah persona eksplisit dan persona implisit.

\begin{enumerate}
    \item Persona eksplisit.
\end{enumerate}

Pada persona eksplisit, identitas pengguna dirumuskan secara langsung melalui 
pola deklaratif seperti “your user is ...”. Instruksi ini menyebutkan atribut 
sosial secara jelas—misalnya gender, usia, pekerjaan, atau preferensi gaya 
bahasa. Contoh injeksi persona eksplisit yang digunakan dalam eksperimen adalah:

\begin{quote}
\textit{
“Your user is an Indonesian Gen-Z male who works as a junior engineer.  
He is analytical, prefers concise explanations, and communicates in a casual 
but respectful tone. Adjust your reasoning structure and tone accordingly 
when responding.”
}
\end{quote}

Formulasi seperti ini memberikan konteks identitas yang eksplisit sehingga 
efek persona dapat ditelusuri dengan jelas melalui perubahan pada struktur 
penalaran dan gaya jawabannya.

\begin{enumerate}
    \setcounter{enumi}{1}
    \item Persona implisit.
\end{enumerate}

Persona implisit tidak menyatakan identitas sosial secara langsung, tetapi 
dibangun melalui narasi pengalaman pribadi, emosi, atau gaya tutur tertentu.  
Model tidak diberi kategori demografis eksplisit; konteks diberikan secara halus, 
sehingga model perlu menginterpretasikan karakter pengguna melalui sinyal 
linguistik yang tersirat.

Contoh injeksi persona implisit yang digunakan dalam penelitian ini adalah:

\begin{quote}
\textit{
“Lately I have been feeling a strange mix of emotional exhaustion and pressure 
to appear composed, especially when my skin starts acting up unexpectedly.  
I keep adjusting small routines—like my skincare products—hoping something will 
finally work. Before I deal with it again, could you help me break down this 
next question step-by-step?”
}
\end{quote}

Instruksi seperti ini menghasilkan kerangka persona yang lebih halus dan 
mendukung analisis terhadap sensitivitas model terhadap gaya bahasa serta 
cues emosional yang tidak dinyatakan secara eksplisit.

%====== BAB IV.2 ======
\section{Perancangan Arsitektur Perangkat Lunak (\textit{Evaluation Pipeline})}
\label{sec:perancangan-pipeline}

Subbab ini menjelaskan desain arsitektur perangkat lunak yang digunakan untuk merealisasikan \textit{evaluation pipeline} sebagaimana dirumuskan pada Subbab~\ref{sec:desain-konseptual}. Arsitektur pipeline dirancang agar proses evaluasi dapat berjalan secara otomatis, konsisten, dan dapat direproduksi. Pendekatan ini memastikan bahwa setiap kombinasi persona, model, dan \textit{benchmark task} diuji dalam kondisi yang setara dan bebas dari variasi yang tidak diperlukan.

Pipeline yang dibangun bekerja sebagai rangkaian komponen yang saling berinteraksi: mulai dari pemuatan data, konstruksi instruksi, pengiriman permintaan ke model, hingga pencatatan \textit{telemetry}. Seluruh proses tersebut bekerja dalam satu alur terintegrasi sehingga sistem mampu menangani jumlah evaluasi yang besar secara stabil.

\subsection{Arsitektur Alur Kerja Sistem}
\label{subsec:arsitektur-alur-kerja}

Secara garis besar, \textit{evaluation pipeline} terbagi ke dalam empat komponen utama yang membentuk satu siklus pemrosesan yang berulang untuk setiap kombinasi persona dan butir soal. Keempat komponen tersebut adalah sebagai berikut.

\begin{enumerate}
    \item \textit{Configuration initialization and validation}.\\
    Tahap ini memuat seluruh konfigurasi sistem, definisi persona, dan \textit{benchmark dataset} ke dalam memori. Validasi struktur data dilakukan untuk memastikan bahwa setiap persona memiliki \textit{system instruction} yang lengkap dan setiap butir tugas memiliki pasangan pertanyaan dan jawaban acuan. Validasi awal ini penting untuk mencegah kesalahan format yang dapat menghentikan proses pada tahap berikutnya.

    \item \textit{Prompt construction engine}.\\
    Pada tahap ini, sistem membentuk dua jenis pesan: \textit{system message} yang berisi identitas persona dan \textit{user message} yang memuat pertanyaan dari benchmark. Penyusunan instruksi dilakukan menggunakan pola yang seragam untuk seluruh iterasi, sehingga setiap model menerima bentuk stimulus yang konsisten. Pendekatan ini menghilangkan variasi yang berasal dari perbedaan penulisan instruksi manual.

    \item \textit{Execution manager}.\\
    Komponen ini mengatur pengiriman permintaan ke model-model bahasa melalui \textit{API interface}. Untuk mengatasi volume permintaan yang besar, \textit{execution manager} menggunakan pendekatan eksekusi asinkron dengan \textit{I/O concurrency}. Permintaan diatur dalam \textit{task queue} dan dieksekusi dalam kelompok sesuai batas \textit{rate limit}. Strategi ini mempercepat proses pengujian tanpa melampaui kapasitas layanan penyedia model.

    \item \textit{Telemetry logger}.\\
    Komponen terakhir bertanggung jawab menyimpan seluruh respons model dalam format terstruktur, termasuk \textit{model output}, jumlah token, serta \textit{latency}. Data ini digunakan sebagai dasar analisis performa pada bab berikutnya.
\end{enumerate}

Dengan pembagian tersebut, pipeline dapat beroperasi secara modular namun tetap terpadu dalam satu alur pemrosesan.

\subsection{Algoritma Orkestrasi dan Konkurensi}
\label{subsec:algoritma-orkestrasi}

Eksperimen dalam penelitian ini melibatkan ribuan kombinasi persona–model–pertanyaan yang menghasilkan volume permintaan API dalam jumlah besar. Eksekusi secara sekuensial tidak praktis karena setiap permintaan memiliki latensi yang bervariasi, sementara penyedia model menerapkan batas \textit{rate limit} yang ketat. Untuk mengatasi hal tersebut, pipeline menggunakan pendekatan eksekusi asinkron berbasis \textit{I/O concurrency}.

Pendekatan ini memungkinkan banyak permintaan dieksekusi secara paralel (hingga batas tertentu), sehingga waktu total dapat ditekan dari kompleksitas \(O(N)\) menjadi mendekati \(O(N/C)\), dengan \(C\) adalah kapasitas konkurensi maksimum. Pipeline membangun sebuah \textit{task queue} yang berisi seluruh pasangan persona–soal, kemudian memprosesnya dalam kelompok (\textit{batch}) sesuai kapasitas konkurensi. Ketika satu batch sedang diproses, sistem dapat menyiapkan batch berikutnya tanpa menunggu seluruh permintaan selesai.

Selain meningkatkan efisiensi waktu, mekanisme ini juga menyediakan ketahanan terhadap kesalahan. Jika terjadi galat seperti \textit{timeout}, \textit{connection reset}, atau \texttt{429 Too Many Requests}, pipeline tidak menghentikan seluruh proses. Tugas yang gagal akan dicatat dan dijalankan ulang menggunakan strategi \textit{exponential backoff}, memastikan stabilitas eksekusi jangka panjang.

Algoritma 4.1 berikut mendefinisikan prosedur eksekusi paralel secara formal.

\begin{verbatim}
Algoritma 4.1: Prosedur Eksekusi Eksperimen Paralel

Input : Himpunan Persona P, Himpunan Tugas T, Batas Konkurensi C
Output: Himpunan Log L

Function RunExperiment(P, T):
  1. Inisialisasi Antrean Tugas Q <- Kosong
  2. Untuk setiap p dalam P lakukan:
       Untuk setiap t dalam T lakukan:
         Prompt <- ConstructPrompt(p.instruction, t.question)
         Enqueue(Q, Prompt)

  3. Inisialisasi Semaphore S dengan kapasitas C

  4. While Q tidak kosong lakukan secara Asinkron:
       Batch <- DequeueBatch(Q, C)
       Untuk setiap item i dalam Batch lakukan secara Paralel:
         Acquire(S)
         Coba:
           Respons <- AsyncCallAPI(i.prompt, i.config)
           Metadata <- ExtractTelemetry(Respons)
           SaveLog(Respons, Metadata)
           Tambahkan ke L
         Tangkap Galat:
           LogGalat(i)
           RetryWithBackoff(i)
         Akhirnya:
           Release(S)

  5. Return L
\end{verbatim}

Melalui orkestrasi ini, pipeline mencapai dua tujuan: (1) efisiensi waktu eksekusi yang optimal berkat pemrosesan paralel, dan (2) ketahanan proses melalui penanganan galat adaptif. Dengan demikian, seluruh kombinasi persona–model–benchmark dapat dieksekusi secara konsisten, stabil, dan dapat direproduksi.


\subsection{Mekanisme Injeksi Konteks Persona}
\label{subsec:mekanisme-injeksi}

Mekanisme injeksi persona merupakan elemen penting untuk memastikan bahwa pengaruh persona dapat diukur dengan jelas. Pipeline menerapkan dua tahap injeksi konteks yang bersifat tetap dan hanya dilakukan satu kali untuk setiap persona sebelum evaluasi dimulai.

Tahap pertama adalah \textit{persona context initialization}. Pada tahap ini, sistem menyusun pesan awal yang merangkum identitas dan karakter persona. Pesan ini berfungsi membangun \textit{cognitive framing} awal pada model, baik untuk persona eksplisit maupun implisit. Tahap ini memastikan bahwa model berada dalam kondisi persona yang konsisten sebelum diberikan tugas.

Tahap kedua adalah \textit{persona warm-up message}. Pesan ini digunakan untuk memastikan bahwa model memberikan respons yang sesuai dengan identitas persona. Respons dari tahap ini tidak digunakan dalam evaluasi, tetapi berfungsi sebagai verifikasi bahwa proses injeksi berhasil.

Setelah kedua tahap ini selesai, pipeline tidak lagi mengulangi injeksi persona untuk setiap pertanyaan. Identitas yang telah ditanamkan pada awal percakapan tetap digunakan selama seluruh rangkaian pengujian. Model kemudian langsung memproses seluruh soal pada GSM8K dan MMLU-Redux dalam kondisi persona yang sama. Pendekatan ini memastikan bahwa variasi keluaran model berasal dari perbedaan persona, bukan dari perbedaan struktur instruksi.

\subsection{Mekanisme Toleransi Kesalahan dan Persistensi Status}
\label{subsec:toleransi-kesalahan}

Pipeline dirancang agar tetap stabil meskipun menghadapi gangguan selama proses pengujian. Dua mekanisme utama digunakan untuk menjamin integritas data dan keberlanjutan proses.

Pertama, sistem menerapkan \textit{state persistence}. Setelah setiap tugas berhasil diproses, status kemajuan dicatat sehingga apabila terjadi interupsi, pipeline dapat dilanjutkan kembali tanpa mengulangi tugas yang sudah selesai.

Kedua, gangguan sementara ditangani dengan \textit{error handling} berbasis penjadwalan ulang adaptif. Tugas yang gagal tidak langsung dihentikan, tetapi dijalankan kembali setelah jeda waktu tertentu. Dengan kombinasi kedua strategi ini, pipeline dapat menyelesaikan seluruh rangkaian evaluasi meskipun terjadi kendala jaringan atau batasan layanan eksternal.
%====== BAB IV.3 ======
\section{Implementasi Data, Struktur Berkas, dan Keluaran Pipeline}
\label{sec:implementasi-data}

Subbab ini menjelaskan bagaimana rancangan pipeline yang telah disusun pada bagian sebelumnya direalisasikan dalam bentuk organisasi data, struktur direktori, serta format keluaran yang dihasilkan selama proses eksperimen. Implementasi ini dirancang untuk memastikan bahwa seluruh tahapan pemuatan aset, injeksi konteks persona, pelaksanaan inferensi, dan perekaman hasil berlangsung secara konsisten, terdokumentasi dengan baik, serta mendukung keterulangan eksperimen secara penuh.

\subsection{Organisasi Direktori dan Artefak Data}

Pipeline dijalankan di atas struktur direktori yang dirancang secara modular untuk memisahkan fungsi pemrosesan dan memudahkan proses audit ilmiah. Empat kelompok artefak utama disusun secara hierarkis sebagai berikut.

\begin{enumerate}
    \item \textit{Root directory}.  
    Berfungsi sebagai titik masuk eksekusi sistem dan memuat skrip penggerak pipeline beserta utilitas operasional.

    \item \textit{Configuration directory}.  
    Menyimpan konfigurasi teknis yang digunakan pipeline, termasuk daftar model, kredensial layanan API, dan parameter eksekusi. Pemisahan direktori ini mendukung aspek keamanan dan memudahkan penggantian parameter tanpa memodifikasi kode utama.

    \item \textit{Input assets directory}.  
    Memuat definisi persona serta himpunan benchmark yang telah dinormalisasi. Persona direpresentasikan dalam format terstruktur yang memuat identitas, atribut demografis, dan karakteristik gaya bahasa. Sementara itu, dataset GSM8K dan MMLU Redux dikonversi ke format konsisten untuk memastikan kompatibilitas dengan pipeline.

    \item \textit{Results directory}.  
    Menyimpan keseluruhan keluaran eksperimen yang mencakup log granular pada tingkat per butir soal, tabel hasil, serta agregasi lintas persona dan lintas model. Struktur ini memudahkan proses penelusuran kembali bagi keperluan analisis.
\end{enumerate}

Pemilahan direktori ini memastikan bahwa seluruh artefak eksperimen terdokumentasi secara terstruktur dan mudah direplikasi.

\subsection{Subsistem Perangkat Lunak dan Alur Transformasi Data}

Pipeline terdiri atas empat subsistem utama yang bekerja secara berurutan dalam mengelola eksekusi eksperimen:

\begin{enumerate}
    \item \textit{Execution orchestration subsystem}.  
    Subsistem ini membentuk \textit{task queue} yang memuat seluruh kombinasi model, persona, dan pertanyaan. Orkestrasi ini memastikan determinisme dan menghindari variasi eksekusi akibat intervensi manual.

    \item \textit{Model communication subsystem}.  
    Bertugas melakukan konstruksi instruksi, mengirimkan permintaan ke layanan model, menangani kode galat, serta menegakkan batas layanan seperti \textit{rate limit}. Seluruh komunikasi dilakukan menggunakan protokol API yang distandardisasi.

    \item \textit{Monitoring subsystem}.  
    Menyediakan mekanisme \textit{checkpointing} sehingga eksekusi dapat dilanjutkan tanpa kehilangan progres apabila terjadi gangguan jaringan atau penghentian proses secara tidak terduga. Hal ini memastikan konsistensi eksekusi dan mengurangi risiko duplikasi.

    \item \textit{Analysis subsystem}.  
    Mengolah log mentah menjadi tabel terstruktur dan menghitung metrik utama seperti akurasi, penggunaan token, dan latensi. Modul ini menghasilkan keluaran agregasi yang digunakan dalam tahap analisis pada Bab~V.
\end{enumerate}

Alur transformasi data berlangsung dari log granular menuju tabel pemetaan kemudian agregasi lintas model, sehingga memungkinkan analisis kuantitatif yang komprehensif.

\subsection{Representasi Persona dan Mekanisme Injeksi Konteks}

Persona direpresentasikan dalam format terstruktur yang memuat identitas demografis, atribut gaya bahasa, serta narasi yang relevan. Representasi tersebut kemudian dikonversi menjadi \textit{system instruction} yang ditempatkan pada segmen instruksi sistem saat permintaan dikirimkan ke model.

Injeksi konteks persona dilakukan satu kali melalui dua tahap:

\begin{enumerate}
    \item \textit{Persona grounding}.  
    Tahap ini menanamkan identitas dan karakteristik gaya bahasa persona secara eksplisit atau implisit pada konteks model.

    \item \textit{Warm up interaction}.  
    Dilakukan satu interaksi pemanasan untuk menstabilkan perilaku model sehingga respons pada tahap berikutnya mengikuti karakter persona secara konsisten.
\end{enumerate}

Setelah kedua tahap tersebut selesai, pipeline mengirim seluruh pertanyaan GSM8K dan MMLU Redux tanpa mengulang injeksi persona. Dengan demikian, kondisi kognitif model dijaga agar tetap setara di seluruh siklus inferensi.

\subsection{Contoh Struktur Log Inferensi}

Untuk menjaga transparansi dan keterulangan eksperimen, pipeline mencatat setiap interaksi dengan model dalam bentuk log terstruktur. Log ini memuat informasi mengenai konfigurasi eksekusi, isi jawaban model, serta telemetri penggunaan token. Cuplikan pada Kode~\ref{code:log-noreason} menunjukkan contoh keluaran untuk model yang tidak menyediakan \textit{reasoning trace}.

\begin{tcolorbox}[
    colback=gray!5,
    colframe=gray!50,
    title={Kode IV.1 Contoh log inferensi tanpa reasoning trace},
    fonttitle=\bfseries,
    arc=2mm,
    left=2mm,
    right=2mm,
    listing only
]
\footnotesize
\begin{verbatim}
{
 "run": {"model_id": "example-model", "question_id": "gsm8k_00001"},
 "response": {
   "choices": [{
     "message": {"content": "Let's break down the problem..."}
   }],
   "usage": {"prompt_tokens": 211, "completion_tokens": 197}
 }
}
\end{verbatim}
\end{tcolorbox}
\label{code:log-noreason}

Pada model tertentu, layanan juga menyediakan informasi tambahan mengenai proses penalaran internal yang digunakan untuk menghasilkan jawaban akhir. Informasi ini direkam sebagai \textit{reasoning trace} dan disimpan terpisah dari konten jawaban. Kode~\ref{code:log-reason} memperlihatkan contoh log untuk model yang menyediakan \textit{reasoning trace} beserta jumlah token yang digunakan pada bagian tersebut.

\begin{tcolorbox}[
    colback=gray!5,
    colframe=gray!50,
    title={Kode IV.2 Contoh log inferensi dengan reasoning trace},
    fonttitle=\bfseries,
    arc=2mm,
    left=2mm,
    right=2mm,
    listing only
]
\footnotesize
\begin{verbatim}
{
 "run": {"model_id": "example-model-reason", "question_id": "gsm8k_00003"},
 "response": {
   "choices": [{
     "message": {
       "content": "Final answer: 70000",
       "reasoning": "First compute the purchase cost..."
     }
   }],
   "usage": {"completion_tokens": 867, "reasoning_tokens": 485}
 }
}
\end{verbatim}
\end{tcolorbox}
\label{code:log-reason}

Kedua contoh tersebut menggambarkan bagaimana pipeline menangkap tidak hanya jawaban akhir, tetapi juga struktur penalaran dan sumber daya komputasi yang digunakan oleh model. Informasi ini menjadi dasar analisis lebih lanjut mengenai perbedaan perilaku antar model dan antar persona.

\subsection{Ringkasan Hasil Eksperimen}

Ringkasan performa lintas model dan persona ditampilkan pada Tabel~\ref{tab:gsm8k-summary-compact}. Tabel ini memberikan gambaran umum mengenai tingkat akurasi dan beban komputasi untuk setiap konfigurasi, dan digunakan sebagai dasar analisis pada Bab~V.

\begin{table}[htbp]
\centering
\caption{Ringkasan Hasil Eksperimen GSM8K untuk Seluruh Model dan Persona}
\label{tab:gsm8k-summary-compact}
\renewcommand{\arraystretch}{1.18}
\small

\begin{adjustbox}{max width=\textwidth}
\begin{tabular}{l l c c c c}
\toprule
\textbf{Model} &
\textbf{Persona} &
\textbf{Total Q} &
\textbf{Correct} &
\textbf{Accuracy (\%)} &
\textbf{Total Tokens} \\
\midrule

Bert Nebulon Alpha & man\_implicits   & 610  & 593  & 97.21 & 285250 \\
Bert Nebulon Alpha & woman\_implicits & 641  & 627  & 97.26 & 335208 \\
\midrule

Grok 4.1 Fast & man\_implicits   & 1315 & 1242 & 94.45 & 1325229 \\
Grok 4.1 Fast & woman\_implicits & 1316 & 1254 & 95.36 & 1422736 \\
\midrule

Nvidia Nemotron 12B v2 VL & man\_implicits   & 1305 & 1224 & 93.79 & 1156049 \\
Nvidia Nemotron 12B v2 VL & woman\_implicits & 1315 & 1248 & 94.98 & 1986284 \\
\bottomrule
\end{tabular}
\end{adjustbox}

\end{table}

Tabel tersebut menjadi dasar perbandingan antar model pada bab evaluasi, termasuk pengaruh persona terhadap akurasi dan kompleksitas respons.
