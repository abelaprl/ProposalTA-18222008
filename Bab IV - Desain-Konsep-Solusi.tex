% ==========================================
% BAB IV DESAIN KONSEP SOLUSI
% ==========================================

\chapter{DESAIN KONSEP SOLUSI}
\label{chap:desain-konsep-solusi}

Bab ini memaparkan rancangan konsep solusi yang diusulkan untuk menjawab permasalahan yang telah dianalisis pada bab sebelumnya. Berdasarkan hasil analisis pemilihan solusi, pendekatan yang digunakan dalam penelitian ini adalah pengembangan sistem eksperimen terotomatisasi berbasis konfigurasi. Pembahasan dalam bab ini mencakup desain konseptual eksperimen, perancangan arsitektur perangkat lunak atau \textit{evaluation pipeline}, serta spesifikasi implementasi data dan struktur berkas. Desain ini disusun untuk memenuhi kebutuhan fungsional terkait strukturisasi \textit{user persona} dan konsistensi eksekusi lintas model.
% ==========================================

%====== BAB IV.1 ======
\section{Desain Konseptual Eksperimen}
\label{sec:desain-konseptual}

Perancangan eksperimen dalam penelitian ini bertujuan membangun sebuah mekanisme evaluasi yang deterministik, terukur, dan dapat direplikasi untuk menganalisis pengaruh \textit{user persona} terhadap perilaku model bahasa. Bagian ini menjelaskan landasan metodologis, keterbatasan pendekatan konvensional, rancangan sistem terotomatisasi, serta integrasi persona, model, dan \textit{benchmark} yang membentuk struktur eksperimen secara menyeluruh. Dengan demikian, hubungan antara desain konseptual dan analisis yang dilakukan pada bab selanjutnya menjadi lebih eksplisit.

\subsection{Keterbatasan Model Operasional Konvensional}

Evaluasi berbasis interaksi manual melalui antarmuka percakapan masih menjadi praktik umum dalam penelitian persona dan perilaku model bahasa. Dalam pendekatan ini, instruksi persona dituliskan secara langsung ke dalam \textit{prompt} oleh peneliti untuk setiap percobaan. Meskipun tampak sederhana, pendekatan tersebut memiliki dua permasalahan metodologis utama yang menurunkan validitas internal eksperimen.

Permasalahan pertama ialah instabilitas masukan. Model bahasa menunjukkan sensitivitas tinggi terhadap variasi sintaksis dan pola penulisan; perubahan kecil seperti tanda baca, diksi, atau struktur kalimat dapat menghasilkan rantai penalaran yang berbeda, sebagaimana ditunjukkan oleh Turpin dan rekan \parencite{turpin2023language}. Ketergantungan pada pengetikan manual menyebabkan variasi tidak terkontrol sulit dihindari, sehingga efek persona sulit dipisahkan dari artefak sintaks.

Permasalahan kedua ialah rendahnya granularitas data. Respons model yang direkam secara manual umumnya hanya mencakup teks keluaran tanpa metadata komputasional seperti durasi eksekusi dan jumlah token. Padahal, informasi tersebut penting sebagai indikator beban kognitif model serta performa penalaran berorientasi pengguna \parencite{naous2025userlm}. Keterbatasan ini mengurangi ketelitian analisis dan menghambat kemampuan reproduksi penelitian.

\begin{figure}[htbp]
  \centering
  % \includegraphics{<file-gambar-existing>} % opsional jika ada ilustrasi
  \caption{Model konseptual sistem eksperimen konvensional}
  \label{fig:model-before}
\end{figure}

\subsection{Sistem Eksperimen Terotomatisasi}

Untuk mengatasi keterbatasan tersebut, penelitian ini merancang sistem eksperimen terotomatisasi dengan tiga prinsip utama, yaitu determinisme masukan berbasis konfigurasi, telemetri komprehensif, dan skalabilitas eksekusi.

Pertama, determinisme masukan dicapai dengan menyimpan seluruh persona sebagai objek data statis dalam berkas konfigurasi. Setiap \textit{prompt} dibangun melalui proses injeksi terprogram sehingga model menerima stimulus yang identik hingga tingkat karakter. Pendekatan ini menghilangkan variasi akibat pengetikan manual dan memastikan bahwa perubahan keluaran model hanya berasal dari perbedaan persona.

Kedua, telemetri komprehensif diterapkan dengan mencatat seluruh respons model dalam berkas terstruktur yang memuat teks jawaban, panjang keluaran, jejak penalaran, jumlah token, serta latensi inferensi. Dengan demikian, kinerja linguistik dan komputasional dapat dianalisis secara simultan.

Ketiga, sistem memungkinkan eksekusi paralel melalui pemrosesan asinkron untuk menangani banyak permintaan secara bersamaan. Pendekatan ini meningkatkan cakupan eksperimen tanpa mengurangi ketelitian metodologis.

Model konseptual sistem eksperimen terotomatisasi yang diusulkan ditunjukkan pada Gambar~\ref{fig:model-after}. Dalam penomoran dokumen, ilustrasi ini direferensikan sebagai Gambar~IV.2 \textit{Model Konseptual Sistem Eksperimen Terotomatisasi}.

\begin{figure}[htbp]
  \centering
  % \includegraphics{<file-gambar-proposed>} % opsional jika ada ilustrasi
  \caption{Model konseptual sistem eksperimen terotomatisasi}
  \label{fig:model-after}
\end{figure}

\subsection{Analisis Komparatif Metodologis}

Transformasi dari model konseptual \textit{Existing} ke \textit{Proposed} bukan sekadar peningkatan efisiensi operasional, melainkan juga peningkatan validitas metodologis penelitian. Tabel~\ref{tab:komparasi-metodologis} menyajikan analisis komparatif mendalam mengenai implikasi ilmiah dari kedua pendekatan tersebut yang ditinjau dari dimensi pengendalian variabel, integritas data, dan reproduktibilitas.

Melalui desain ini, penelitian memastikan bahwa setiap kesimpulan mengenai bias atau perubahan penalaran yang ditarik pada bab hasil nantinya didasarkan pada data yang berintegritas tinggi, lengkap, dan diperoleh melalui prosedur yang dapat dipertanggungjawabkan secara ilmiah.

\begin{table}[htbp]
  \centering
  \caption{Analisis komparatif validitas metodologis}
  \label{tab:komparasi-metodologis}
  \renewcommand{\arraystretch}{1.28}
  \small
  \begin{tabularx}{\textwidth}{
    >{\raggedright\arraybackslash}p{3.5cm}
    >{\raggedright\arraybackslash}X
    >{\raggedright\arraybackslash}X
  }
    \toprule
    \textbf{Dimensi Analisis} &
    \textbf{Sistem Konvensional (\textit{Existing})} &
    \textbf{Sistem Usulan (\textit{Proposed})} \\
    \midrule

    Pengendalian variabel &
    Stokastik; rentan terhadap gangguan input manual yang dapat mendistorsi kausalitas efek persona akibat sensitivitas \textit{framing}. &
    Deterministik; konfigurasi statis dan injeksi otomatis menjamin isolasi variabel independen yang presisi dan konsisten. \\

    Granularitas data &
    Dangkal (tekstual); hanya menangkap hasil akhir sehingga kehilangan nuansa proses internal model seperti latensi dan efisiensi token. &
    Dalam (telemetri); menangkap metadata performa yang mengindikasikan beban kognitif dari adopsi persona secara granular. \\

    Format penyimpanan &
    Tidak terstruktur; teks mentah dalam lembar kerja yang sulit diproses ulang secara komputasi dan rentan kesalahan salin. &
    Terstruktur (JSON/CSV); data siap olah yang mendukung analisis statistik otomatis dan deteksi pola bias yang sistematis. \\

    Reproduktibilitas &
    Rendah; parameter eksperimen sering kali tidak terdokumentasi dengan baik sehingga menyulitkan verifikasi pihak ketiga. &
    Tinggi; seluruh kondisi eksperimen terenkapsulasi dalam kode sumber dan berkas konfigurasi yang dapat diaudit dan dijalankan ulang. \\

    Cakupan eksperimen &
    Terbatas; kendala waktu eksekusi linear membatasi jumlah kombinasi model dan persona yang dapat diuji secara layak. &
    Masif; mendukung evaluasi skala besar untuk memetakan perilaku model pada spektrum persona yang luas. \\
    \bottomrule
  \end{tabularx}
\end{table}

\subsection{Integrasi Persona, Model, dan Benchmark}

Bagian ini menjelaskan struktur persona, himpunan model, serta dua \textit{benchmark} penalaran yang digunakan dalam penelitian. Ketiga komponen tersebut menjadi fondasi konfigurasi eksperimen sehingga pengaruh persona terhadap penalaran model dapat dipetakan secara sistematis.

\subsubsection{Benchmark Penalaran}

Penelitian ini menggunakan dua \textit{benchmark} utama sebagai dasar evaluasi penalaran, yaitu GSM8K dan MMLU-Redux. Keduanya dipilih karena bersifat netral terhadap identitas sosial serta menyediakan format jawaban yang terstandarisasi sehingga memungkinkan isolasi pengaruh persona pada tahap instruksi sistem.

Benchmark pertama adalah GSM8K, yaitu korpus soal cerita matematika tingkat sekolah menengah yang dirancang untuk mengevaluasi penalaran numerik bertahap \parencite{cobbe2021gsm8k}. Penelitian ini menggunakan berkas \texttt{gsm8k\_train.json} dan \texttt{gsm8k\_test.json} sebagai sumber data utama, yang telah diunduh dan dinormalisasi sesuai kebutuhan sistem. Seluruh soal mempertahankan struktur berpasangan antara pertanyaan dan jawaban numerik sehingga proses ekstraksi jawaban akhir dapat dilakukan secara deterministik.

Benchmark kedua adalah MMLU-Redux, yaitu korpus pilihan ganda lintas domain yang telah dikurasi ulang oleh Edinburgh Dataset Analytics Working Group \parencite{mmluRedux2024dataset}. Versi Redux digunakan karena memperbaiki ketidakstabilan yang hadir pada MMLU asli, seperti inkonsistensi format, ambiguitas pilihan jawaban, dan perbedaan kualitas antardomain \parencite{hendryckstest2021}. Penelitian ini menggunakan berkas \texttt{mmlu\_redux.json} sebagai sumber data utama, yang berisi representasi terstruktur untuk seluruh subjek yang diujikan.

Penggunaan kedua \textit{benchmark} tersebut sekaligus memungkinkan evaluasi dua jenis penalaran berbeda, yaitu penalaran numerik berbasis langkah dan penalaran konseptual berbasis pilihan ganda. Hal ini memberikan gambaran yang lebih komprehensif mengenai variasi perilaku model di bawah pengaruh persona.

\subsubsection{Himpunan Model}

Penelitian ini menggunakan tiga belas model bahasa yang mewakili dua kategori besar, yaitu model komersial berkinerja tinggi dan model publik atau sumber terbuka. Daftar lengkap model adalah sebagai berikut:

\begin{enumerate}
    \item Model komersial:
    GPT-4.1, GPT-4.1 Mini, Claude 3.7 Sonnet, Claude 3.7 Haiku, 
    Gemini 2.0 Flash Thinking, Gemini 2.0 Pro Experimental;

    \item Model publik via OpenRouter:
    Grok 4.1 Fast, NVIDIA Nemotron-nano-12B-v2-VL, Bert Nebulon Alpha.
\end{enumerate}

Keragaman arsitektur ini memungkinkan analisis komparatif lintas paradigma representasi dan karakteristik inferensi, mulai dari model \textit{frontier} komersial hingga model publik yang dapat diakses secara lebih luas.

\subsubsection{Struktur Persona}

Struktur persona disusun berdasarkan enam dimensi demografis dan stilistik, yaitu gender, usia, agama, pekerjaan, kewarganegaraan, dan register bahasa. Kombinasi dimensi tersebut menghasilkan lima belas persona eksplisit dan implisit, ditambah satu kondisi pengguna netral. Dua persona dasar implisit dirancang untuk menguji sensitivitas gender melalui gaya linguistik maskulin dan feminin tanpa deklarasi identitas eksplisit.

Seluruh persona yang digunakan disajikan pada Tabel~\ref{tab:user-persona}.

\begin{table}[htbp]
\centering
\caption{Daftar \textit{user persona} untuk kondisi eksperimen}
\label{tab:user-persona}
\renewcommand{\arraystretch}{1.15}

\resizebox{\textwidth}{!}{
\begin{tabular}{c l l l l l l l}
\toprule
\textbf{ID} &
\textbf{Persona} &
\textbf{Mode} &
\textbf{Gender} &
\textbf{Age Group} &
\textbf{Religion} &
\textbf{Occupation} &
\textbf{Nationality / Register} \\
\midrule
P1  & Implicit male baseline              & Implicit & Male   & --          & --       & --                & Neutral \\
P2  & Implicit female baseline            & Implicit & Female & --          & --       & --                & Neutral \\
P3  & Neutral user                        & Neutral  & --     & --          & --       & --                & Neutral \\
P4  & Indonesian Muslim young woman       & Explicit & Female & Young adult & Muslim   & Healthcare worker & Indonesian / Semi-formal \\
P5  & Indonesian Muslim young man         & Implicit & Male   & Young adult & Muslim   & Healthcare worker & Indonesian / Semi-formal \\
P6  & American middle-aged male           & Explicit & Male   & Middle-aged & Christian & Engineer          & American / Formal \\
P7  & American middle-aged female         & Implicit & Female & Middle-aged & Christian & Engineer          & American / Formal \\
P8  & Indonesian Gen-Z female, casual     & Explicit & Female & Gen-Z       & --       & Student           & Indonesian / Casual--slang \\
P9  & Indonesian Gen-Z male, casual       & Implicit & Male   & Gen-Z       & --       & Student           & Indonesian / Casual--slang \\
P10 & Middle Eastern young adult male     & Explicit & Male   & Young adult & Muslim   & Engineer          & Middle Eastern Arabic-speaking / Formal \\
P11 & Middle Eastern young adult female   & Implicit & Female & Young adult & Muslim   & Student           & Middle Eastern Arabic-speaking / Formal \\
P12 & American atheist young male         & Explicit & Male   & Young adult & Atheist  & Student           & American / Formal \\
P13 & American atheist young female       & Implicit & Female & Young adult & Atheist  & Student           & American / Formal \\
P14 & Indonesian female healthcare worker & Explicit & Female & Young adult & Muslim   & Healthcare worker & Indonesian / Semi-formal \\
P15 & Indonesian male healthcare worker   & Implicit & Male   & Young adult & Muslim   & Healthcare worker & Indonesian / Semi-formal \\
\bottomrule
\end{tabular}
}
\end{table}

\subsubsection{Konfigurasi Eksekusi}

Kombinasi seluruh persona dan ketiga belas model menghasilkan ratusan konfigurasi eksperimen dengan variasi konteks sosial dan karakteristik arsitektur yang beragam. Setiap konfigurasi dieksekusi melalui empat tahap tetap, yaitu pemanasan persona, penetapan konteks percakapan, eksekusi \textit{benchmark}, dan pencatatan telemetri. Tahapan ini memastikan bahwa perubahan respons model dapat ditelusuri secara langsung dari variabel persona dan bukan dari faktor lain di luar kendali eksperimen.

%====== BAB IV.2 ======
\section{Perancangan Arsitektur Perangkat Lunak (\textit{Evaluation Pipeline})}
\label{sec:perancangan-pipeline}

Subbab ini menjelaskan desain arsitektur perangkat lunak yang digunakan untuk merealisasikan \textit{evaluation pipeline} sebagaimana dirumuskan pada Subbab~\ref{sec:desain-konseptual}. Arsitektur pipeline dirancang agar proses evaluasi dapat berjalan secara otomatis, konsisten, dan dapat direproduksi. Pendekatan ini memastikan bahwa setiap kombinasi persona, model, dan \textit{benchmark task} diuji dalam kondisi yang setara dan bebas dari variasi yang tidak diperlukan.

Pipeline yang dibangun bekerja sebagai rangkaian komponen yang saling berinteraksi: mulai dari pemuatan data, konstruksi instruksi, pengiriman permintaan ke model, hingga pencatatan \textit{telemetry}. Seluruh proses tersebut bekerja dalam satu alur terintegrasi sehingga sistem mampu menangani jumlah evaluasi yang besar secara stabil.

\subsection{Arsitektur Alur Kerja Sistem}
\label{subsec:arsitektur-alur-kerja}

Secara garis besar, \textit{evaluation pipeline} terbagi ke dalam empat komponen utama yang membentuk satu siklus pemrosesan yang berulang untuk setiap kombinasi persona dan butir soal. Keempat komponen tersebut adalah sebagai berikut.

\begin{enumerate}
    \item \textit{Configuration initialization and validation}.\\
    Tahap ini memuat seluruh konfigurasi sistem, definisi persona, dan \textit{benchmark dataset} ke dalam memori. Validasi struktur data dilakukan untuk memastikan bahwa setiap persona memiliki \textit{system instruction} yang lengkap dan setiap butir tugas memiliki pasangan pertanyaan dan jawaban acuan. Validasi awal ini penting untuk mencegah kesalahan format yang dapat menghentikan proses pada tahap berikutnya.

    \item \textit{Prompt construction engine}.\\
    Pada tahap ini, sistem membentuk dua jenis pesan: \textit{system message} yang berisi identitas persona dan \textit{user message} yang memuat pertanyaan dari benchmark. Penyusunan instruksi dilakukan menggunakan pola yang seragam untuk seluruh iterasi, sehingga setiap model menerima bentuk stimulus yang konsisten. Pendekatan ini menghilangkan variasi yang berasal dari perbedaan penulisan instruksi manual.

    \item \textit{Execution manager}.\\
    Komponen ini mengatur pengiriman permintaan ke model-model bahasa melalui \textit{API interface}. Untuk mengatasi volume permintaan yang besar, \textit{execution manager} menggunakan pendekatan eksekusi asinkron dengan \textit{I/O concurrency}. Permintaan diatur dalam \textit{task queue} dan dieksekusi dalam kelompok sesuai batas \textit{rate limit}. Strategi ini mempercepat proses pengujian tanpa melampaui kapasitas layanan penyedia model.

    \item \textit{Telemetry logger}.\\
    Komponen terakhir bertanggung jawab menyimpan seluruh respons model dalam format terstruktur, termasuk \textit{model output}, jumlah token, serta \textit{latency}. Data ini digunakan sebagai dasar analisis performa pada bab berikutnya.
\end{enumerate}

Dengan pembagian tersebut, pipeline dapat beroperasi secara modular namun tetap terpadu dalam satu alur pemrosesan.

\subsection{Algoritma Orkestrasi dan Konkurensi}
\label{subsec:algoritma-orkestrasi}

Eksperimen dalam penelitian ini melibatkan ribuan kombinasi persona–model–pertanyaan yang menghasilkan volume permintaan API dalam jumlah besar. Eksekusi secara sekuensial tidak praktis karena setiap permintaan memiliki latensi yang bervariasi, sementara penyedia model menerapkan batas \textit{rate limit} yang ketat. Untuk mengatasi hal tersebut, pipeline menggunakan pendekatan eksekusi asinkron berbasis \textit{I/O concurrency}.

Pendekatan ini memungkinkan banyak permintaan dieksekusi secara paralel (hingga batas tertentu), sehingga waktu total dapat ditekan dari kompleksitas \(O(N)\) menjadi mendekati \(O(N/C)\), dengan \(C\) adalah kapasitas konkurensi maksimum. Pipeline membangun sebuah \textit{task queue} yang berisi seluruh pasangan persona–soal, kemudian memprosesnya dalam kelompok (\textit{batch}) sesuai kapasitas konkurensi. Ketika satu batch sedang diproses, sistem dapat menyiapkan batch berikutnya tanpa menunggu seluruh permintaan selesai.

Selain meningkatkan efisiensi waktu, mekanisme ini juga menyediakan ketahanan terhadap kesalahan. Jika terjadi galat seperti \textit{timeout}, \textit{connection reset}, atau \texttt{429 Too Many Requests}, pipeline tidak menghentikan seluruh proses. Tugas yang gagal akan dicatat dan dijalankan ulang menggunakan strategi \textit{exponential backoff}, memastikan stabilitas eksekusi jangka panjang.

Algoritma 4.1 berikut mendefinisikan prosedur eksekusi paralel secara formal.

\begin{verbatim}
Algoritma 4.1: Prosedur Eksekusi Eksperimen Paralel

Input : Himpunan Persona P, Himpunan Tugas T, Batas Konkurensi C
Output: Himpunan Log L

Function RunExperiment(P, T):
  1. Inisialisasi Antrean Tugas Q <- Kosong
  2. Untuk setiap p dalam P lakukan:
       Untuk setiap t dalam T lakukan:
         Prompt <- ConstructPrompt(p.instruction, t.question)
         Enqueue(Q, Prompt)

  3. Inisialisasi Semaphore S dengan kapasitas C

  4. While Q tidak kosong lakukan secara Asinkron:
       Batch <- DequeueBatch(Q, C)
       Untuk setiap item i dalam Batch lakukan secara Paralel:
         Acquire(S)
         Coba:
           Respons <- AsyncCallAPI(i.prompt, i.config)
           Metadata <- ExtractTelemetry(Respons)
           SaveLog(Respons, Metadata)
           Tambahkan ke L
         Tangkap Galat:
           LogGalat(i)
           RetryWithBackoff(i)
         Akhirnya:
           Release(S)

  5. Return L
\end{verbatim}

Melalui orkestrasi ini, pipeline mencapai dua tujuan: (1) efisiensi waktu eksekusi yang optimal berkat pemrosesan paralel, dan (2) ketahanan proses melalui penanganan galat adaptif. Dengan demikian, seluruh kombinasi persona–model–benchmark dapat dieksekusi secara konsisten, stabil, dan dapat direproduksi.


\subsection{Mekanisme Injeksi Konteks Persona}
\label{subsec:mekanisme-injeksi}

Mekanisme injeksi persona merupakan elemen penting untuk memastikan bahwa pengaruh persona dapat diukur dengan jelas. Pipeline menerapkan dua tahap injeksi konteks yang bersifat tetap dan hanya dilakukan satu kali untuk setiap persona sebelum evaluasi dimulai.

Tahap pertama adalah \textit{persona context initialization}. Pada tahap ini, sistem menyusun pesan awal yang merangkum identitas dan karakter persona. Pesan ini berfungsi membangun \textit{cognitive framing} awal pada model, baik untuk persona eksplisit maupun implisit. Tahap ini memastikan bahwa model berada dalam kondisi persona yang konsisten sebelum diberikan tugas.

Tahap kedua adalah \textit{persona warm-up message}. Pesan ini digunakan untuk memastikan bahwa model memberikan respons yang sesuai dengan identitas persona. Respons dari tahap ini tidak digunakan dalam evaluasi, tetapi berfungsi sebagai verifikasi bahwa proses injeksi berhasil.

Setelah kedua tahap ini selesai, pipeline tidak lagi mengulangi injeksi persona untuk setiap pertanyaan. Identitas yang telah ditanamkan pada awal percakapan tetap digunakan selama seluruh rangkaian pengujian. Model kemudian langsung memproses seluruh soal pada GSM8K dan MMLU-Redux dalam kondisi persona yang sama. Pendekatan ini memastikan bahwa variasi keluaran model berasal dari perbedaan persona, bukan dari perbedaan struktur instruksi.

\subsection{Mekanisme Toleransi Kesalahan dan Persistensi Status}
\label{subsec:toleransi-kesalahan}

Pipeline dirancang agar tetap stabil meskipun menghadapi gangguan selama proses pengujian. Dua mekanisme utama digunakan untuk menjamin integritas data dan keberlanjutan proses.

Pertama, sistem menerapkan \textit{state persistence}. Setelah setiap tugas berhasil diproses, status kemajuan dicatat sehingga apabila terjadi interupsi, pipeline dapat dilanjutkan kembali tanpa mengulangi tugas yang sudah selesai.

Kedua, gangguan sementara ditangani dengan \textit{error handling} berbasis penjadwalan ulang adaptif. Tugas yang gagal tidak langsung dihentikan, tetapi dijalankan kembali setelah jeda waktu tertentu. Dengan kombinasi kedua strategi ini, pipeline dapat menyelesaikan seluruh rangkaian evaluasi meskipun terjadi kendala jaringan atau batasan layanan eksternal.
%====== BAB IV.3 ======
\section{Implementasi Data, Struktur Berkas, dan Keluaran Pipeline}
\label{sec:implementasi-data}

Subbab ini menjelaskan bagaimana rancangan pipeline yang telah disusun pada bagian sebelumnya direalisasikan dalam bentuk organisasi data, struktur direktori, serta format keluaran yang dihasilkan selama proses eksperimen. Implementasi ini dirancang untuk memastikan bahwa seluruh tahapan pemuatan aset, injeksi konteks persona, pelaksanaan inferensi, dan perekaman hasil berlangsung secara konsisten, terdokumentasi dengan baik, serta mendukung keterulangan eksperimen secara penuh.

\subsection{Organisasi Direktori dan Artefak Data}

Pipeline dijalankan di atas struktur direktori yang dirancang secara modular untuk memisahkan fungsi pemrosesan dan memudahkan proses audit ilmiah. Empat kelompok artefak utama disusun secara hierarkis sebagai berikut.

\begin{enumerate}
    \item \textit{Root directory}.  
    Berfungsi sebagai titik masuk eksekusi sistem dan memuat skrip penggerak pipeline beserta utilitas operasional.

    \item \textit{Configuration directory}.  
    Menyimpan konfigurasi teknis yang digunakan pipeline, termasuk daftar model, kredensial layanan API, dan parameter eksekusi. Pemisahan direktori ini mendukung aspek keamanan dan memudahkan penggantian parameter tanpa memodifikasi kode utama.

    \item \textit{Input assets directory}.  
    Memuat definisi persona serta himpunan benchmark yang telah dinormalisasi. Persona direpresentasikan dalam format terstruktur yang memuat identitas, atribut demografis, dan karakteristik gaya bahasa. Sementara itu, dataset GSM8K dan MMLU Redux dikonversi ke format konsisten untuk memastikan kompatibilitas dengan pipeline.

    \item \textit{Results directory}.  
    Menyimpan keseluruhan keluaran eksperimen yang mencakup log granular pada tingkat per butir soal, tabel hasil, serta agregasi lintas persona dan lintas model. Struktur ini memudahkan proses penelusuran kembali bagi keperluan analisis.
\end{enumerate}

Pemilahan direktori ini memastikan bahwa seluruh artefak eksperimen terdokumentasi secara terstruktur dan mudah direplikasi.

\subsection{Subsistem Perangkat Lunak dan Alur Transformasi Data}

Pipeline terdiri atas empat subsistem utama yang bekerja secara berurutan dalam mengelola eksekusi eksperimen:

\begin{enumerate}
    \item \textit{Execution orchestration subsystem}.  
    Subsistem ini membentuk \textit{task queue} yang memuat seluruh kombinasi model, persona, dan pertanyaan. Orkestrasi ini memastikan determinisme dan menghindari variasi eksekusi akibat intervensi manual.

    \item \textit{Model communication subsystem}.  
    Bertugas melakukan konstruksi instruksi, mengirimkan permintaan ke layanan model, menangani kode galat, serta menegakkan batas layanan seperti \textit{rate limit}. Seluruh komunikasi dilakukan menggunakan protokol API yang distandardisasi.

    \item \textit{Monitoring subsystem}.  
    Menyediakan mekanisme \textit{checkpointing} sehingga eksekusi dapat dilanjutkan tanpa kehilangan progres apabila terjadi gangguan jaringan atau penghentian proses secara tidak terduga. Hal ini memastikan konsistensi eksekusi dan mengurangi risiko duplikasi.

    \item \textit{Analysis subsystem}.  
    Mengolah log mentah menjadi tabel terstruktur dan menghitung metrik utama seperti akurasi, penggunaan token, dan latensi. Modul ini menghasilkan keluaran agregasi yang digunakan dalam tahap analisis pada Bab~V.
\end{enumerate}

Alur transformasi data berlangsung dari log granular menuju tabel pemetaan kemudian agregasi lintas model, sehingga memungkinkan analisis kuantitatif yang komprehensif.

\subsection{Representasi Persona dan Mekanisme Injeksi Konteks}

Persona direpresentasikan dalam format terstruktur yang memuat identitas demografis, atribut gaya bahasa, serta narasi yang relevan. Representasi tersebut kemudian dikonversi menjadi \textit{system instruction} yang ditempatkan pada segmen instruksi sistem saat permintaan dikirimkan ke model.

Injeksi konteks persona dilakukan satu kali melalui dua tahap:

\begin{enumerate}
    \item \textit{Persona grounding}.  
    Tahap ini menanamkan identitas dan karakteristik gaya bahasa persona secara eksplisit atau implisit pada konteks model.

    \item \textit{Warm up interaction}.  
    Dilakukan satu interaksi pemanasan untuk menstabilkan perilaku model sehingga respons pada tahap berikutnya mengikuti karakter persona secara konsisten.
\end{enumerate}

Setelah kedua tahap tersebut selesai, pipeline mengirim seluruh pertanyaan GSM8K dan MMLU Redux tanpa mengulang injeksi persona. Dengan demikian, kondisi kognitif model dijaga agar tetap setara di seluruh siklus inferensi.

\subsection{Contoh Struktur Log Inferensi}

Untuk menjaga transparansi dan keterulangan eksperimen, pipeline mencatat setiap interaksi dengan model dalam bentuk log terstruktur. Log ini memuat informasi mengenai konfigurasi eksekusi, isi jawaban model, serta telemetri penggunaan token. Cuplikan pada Kode~\ref{code:log-noreason} menunjukkan contoh keluaran untuk model yang tidak menyediakan \textit{reasoning trace}.

\begin{tcolorbox}[
    colback=gray!5,
    colframe=gray!50,
    title={Kode IV.1 Contoh log inferensi tanpa reasoning trace},
    fonttitle=\bfseries,
    arc=2mm,
    left=2mm,
    right=2mm,
    listing only
]
\footnotesize
\begin{verbatim}
{
 "run": {"model_id": "example-model", "question_id": "gsm8k_00001"},
 "response": {
   "choices": [{
     "message": {"content": "Let's break down the problem..."}
   }],
   "usage": {"prompt_tokens": 211, "completion_tokens": 197}
 }
}
\end{verbatim}
\end{tcolorbox}
\label{code:log-noreason}

Pada model tertentu, layanan juga menyediakan informasi tambahan mengenai proses penalaran internal yang digunakan untuk menghasilkan jawaban akhir. Informasi ini direkam sebagai \textit{reasoning trace} dan disimpan terpisah dari konten jawaban. Kode~\ref{code:log-reason} memperlihatkan contoh log untuk model yang menyediakan \textit{reasoning trace} beserta jumlah token yang digunakan pada bagian tersebut.

\begin{tcolorbox}[
    colback=gray!5,
    colframe=gray!50,
    title={Kode IV.2 Contoh log inferensi dengan reasoning trace},
    fonttitle=\bfseries,
    arc=2mm,
    left=2mm,
    right=2mm,
    listing only
]
\footnotesize
\begin{verbatim}
{
 "run": {"model_id": "example-model-reason", "question_id": "gsm8k_00003"},
 "response": {
   "choices": [{
     "message": {
       "content": "Final answer: 70000",
       "reasoning": "First compute the purchase cost..."
     }
   }],
   "usage": {"completion_tokens": 867, "reasoning_tokens": 485}
 }
}
\end{verbatim}
\end{tcolorbox}
\label{code:log-reason}

Kedua contoh tersebut menggambarkan bagaimana pipeline menangkap tidak hanya jawaban akhir, tetapi juga struktur penalaran dan sumber daya komputasi yang digunakan oleh model. Informasi ini menjadi dasar analisis lebih lanjut mengenai perbedaan perilaku antar model dan antar persona.

\subsection{Ringkasan Hasil Eksperimen}

Ringkasan performa lintas model dan persona ditampilkan pada Tabel~\ref{tab:gsm8k-summary-compact}. Tabel ini memberikan gambaran umum mengenai tingkat akurasi dan beban komputasi untuk setiap konfigurasi, dan digunakan sebagai dasar analisis pada Bab~V.

\begin{table}[htbp]
\centering
\caption{Ringkasan Hasil Eksperimen GSM8K untuk Seluruh Model dan Persona}
\label{tab:gsm8k-summary-compact}
\renewcommand{\arraystretch}{1.18}
\small

\begin{adjustbox}{max width=\textwidth}
\begin{tabular}{l l c c c c}
\toprule
\textbf{Model} &
\textbf{Persona} &
\textbf{Total Q} &
\textbf{Correct} &
\textbf{Accuracy (\%)} &
\textbf{Total Tokens} \\
\midrule

Bert Nebulon Alpha & man\_implicits   & 610  & 593  & 97.21 & 285250 \\
Bert Nebulon Alpha & woman\_implicits & 641  & 627  & 97.26 & 335208 \\
\midrule

Grok 4.1 Fast & man\_implicits   & 1315 & 1242 & 94.45 & 1325229 \\
Grok 4.1 Fast & woman\_implicits & 1316 & 1254 & 95.36 & 1422736 \\
\midrule

Nvidia Nemotron 12B v2 VL & man\_implicits   & 1305 & 1224 & 93.79 & 1156049 \\
Nvidia Nemotron 12B v2 VL & woman\_implicits & 1315 & 1248 & 94.98 & 1986284 \\
\bottomrule
\end{tabular}
\end{adjustbox}

\end{table}

Tabel tersebut menjadi dasar perbandingan antar model pada bab evaluasi, termasuk pengaruh persona terhadap akurasi dan kompleksitas respons.
