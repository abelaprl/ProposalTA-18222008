% ==========================================
% BAB IV DESAIN KONSEP SOLUSI
% ==========================================

\chapter{DESAIN KONSEP SOLUSI}
\label{chap:desain-konsep-solusi}

Bab ini memaparkan rancangan konsep solusi yang diusulkan untuk menjawab permasalahan yang telah dianalisis pada bab sebelumnya. Berdasarkan hasil analisis pemilihan solusi, pendekatan yang digunakan dalam penelitian ini adalah pengembangan sistem eksperimen terotomatisasi berbasis konfigurasi. Pembahasan dalam bab ini mencakup desain konseptual eksperimen, perancangan arsitektur perangkat lunak atau \textit{evaluation pipeline}, serta spesifikasi implementasi data dan struktur berkas. Desain ini disusun untuk memenuhi kebutuhan fungsional terkait strukturisasi \textit{user persona} dan konsistensi eksekusi lintas model.
\section{Desain Konseptual Eksperimen}
\label{sec:desain-konseptual}

Bagian ini menjelaskan dasar konseptual dari sistem eksperimen yang dikembangkan untuk mengkaji pengaruh \textit{user persona} terhadap perilaku model bahasa. Desain konseptual ini berfungsi sebagai landasan arsitektur yang menghubungkan permasalahan metodologis pada Bab III dengan implementasi teknis yang dijelaskan pada Subbab~\ref{sec:perancangan-pipeline} dan Subbab~\ref{sec:implementasi-data}. Pendekatan yang digunakan menekankan pentingnya evaluasi yang terstruktur, terukur, dan bebas dari variasi input yang tidak relevan, sehingga setiap perubahan keluaran model dapat diinterpretasikan secara valid.

\subsection{Dekonstruksi Model Operasional Konvensional}
\label{subsec:existing-model}

Secara umum, praktik pengujian persona pada model bahasa masih banyak dilakukan menggunakan pendekatan manual atau semi-manual melalui \textit{conversational interface}. Pada pendekatan ini, persona dituangkan langsung ke dalam \textit{prompt} dan diberikan berulang-ulang setiap kali model diuji. Meskipun mudah diterapkan, pendekatan tersebut mengandung beberapa keterbatasan metodologis yang berdampak langsung pada validitas hasil eksperimen.

Keterbatasan pertama berkaitan dengan ketidakstabilan input atau \textit{input framing variance}. Penelitian sebelumnya menunjukkan bahwa model bahasa sangat sensitif terhadap variasi kecil pada redaksi instruksi, termasuk perubahan tanda baca, panjang kalimat, atau urutan penyampaian \parencite{turpin2023language, zhou2023largemodelsensitive}. Dalam skenario pengujian manual, variasi ini sulit dikendalikan sepenuhnya sehingga dapat menimbulkan perbedaan respons yang bukan berasal dari persona, tetapi dari ketidakteraturan input.

Keterbatasan kedua adalah minimnya \textit{observability}. Pengujian manual umumnya hanya menyimpan teks keluaran model, sedangkan informasi penting seperti \textit{latency}, jumlah token, atau struktur penalaran tidak tercatat. Hal ini menyulitkan analisis mengenai bagaimana persona memengaruhi beban komputasi atau pola respons model, sebagaimana disorot oleh Naous et al. \parencite{naous2025userlm}.

Keterbatasan ketiga adalah aspek reproduktibilitas. Karena interaksi dilakukan melalui antarmuka percakapan, sulit untuk menjamin bahwa percobaan yang sama dapat dijalankan ulang dengan kondisi yang benar-benar identik. Hal ini bertentangan dengan prinsip penelitian empiris yang menuntut transparansi dan konsistensi prosedur.

\subsection{Konstruksi Model Sistem Terotomatisasi}
\label{subsec:proposed-model}

Untuk mengatasi berbagai keterbatasan tersebut, penelitian ini mengusulkan model eksperimen terotomatisasi yang mengubah proses evaluasi dari pendekatan manual menjadi pendekatan berbasis data dan \textit{automated orchestration}. Desain konseptual ini bertumpu pada tiga pilar utama.

Pilar pertama adalah \textit{deterministic input configuration}. Setiap persona dan setiap pertanyaan benchmark diperlakukan sebagai objek data yang disimpan dalam format terstruktur. Pipeline menyusun instruksi secara programatik sehingga seluruh \textit{byte-level input} konsisten untuk setiap iterasi. Dengan cara ini, variabel independen benar-benar terbatas hanya pada variasi persona.

Pilar kedua adalah peningkatan \textit{data granularity}. Sistem merekam keluaran model beserta \textit{telemetry} seperti \textit{latency}, \textit{token usage}, dan jejak penalaran apabila tersedia. Informasi ini memungkinkan analisis lebih komprehensif terhadap dampak persona, termasuk aspek efisiensi komputasi dan kecenderungan struktur respons.

Pilar ketiga adalah \textit{scalable execution}. Mengingat jumlah kombinasi persona, model, dan butir soal yang besar, pipeline menerapkan eksekusi paralel dengan \textit{asynchronous processing}. Hal ini memungkinkan eksperimen diselesaikan dalam durasi yang lebih singkat tanpa mengorbankan konsistensi prosedural.

\subsection{Analisis Komparatif Metodologis}
\label{subsec:komparasi-metodologis}

Transformasi dari pendekatan manual menuju pipeline terotomatisasi membawa implikasi metodologis yang signifikan. Tabel berikut merangkum perbedaan utama antara kedua pendekatan dari beberapa dimensi analisis penting.

\begin{table}[htbp]
  \centering
  \caption{Perbandingan Validitas Metodologis antara Model Konvensional dan Model Terotomatisasi}
  \label{tab:komparasi-metodologis}
  \renewcommand{\arraystretch}{1.3}
  \begin{tabular}{p{3.8cm} p{5.2cm} p{5.2cm}}
    \toprule
    \textbf{Dimensi Analisis} &
    \textbf{Pendekatan Konvensional} &
    \textbf{Pendekatan Terotomatisasi} \\
    \midrule
    \textit{Input Control} &
    Rentan terhadap variasi redaksi instruksi dan ketidakkonsistenan manual. &
    Instruksi dirakit secara programatik sehingga konsisten pada seluruh iterasi. \\
    \midrule
    \textit{Data Granularity} &
    Hanya menyimpan teks keluaran. &
    Merekam \textit{latency}, token, dan metadata lainnya untuk analisis mendalam. \\
    \midrule
    \textit{Storage Format} &
    Tidak terstruktur dan sulit diproses ulang. &
    Menggunakan format JSON atau CSV yang siap dianalisis secara otomatis. \\
    \midrule
    \textit{Reproducibility} &
    Sulit menjamin kondisi eksperimen identik. &
    Seluruh parameter eksperimen terdokumentasi dan dapat diulang. \\
    \midrule
    \textit{Scalability} &
    Eksekusi linear dan memakan waktu. &
    Mendukung pemrosesan paralel berskala besar. \\
    \bottomrule
  \end{tabular}
\end{table}

Melalui perancangan ini, sistem eksperimen yang dikembangkan mampu menghasilkan data yang lebih stabil, konsisten, dan transparan. Dengan demikian, setiap perbedaan performa model dapat ditelusuri kembali secara lebih jelas ke persona yang sedang diuji.

%====== BAB IV.2 ======
\section{Perancangan Arsitektur Perangkat Lunak (\textit{Evaluation Pipeline})}
\label{sec:perancangan-pipeline}

Subbab ini menjelaskan desain arsitektur perangkat lunak yang digunakan untuk merealisasikan \textit{evaluation pipeline} sebagaimana dirumuskan pada Subbab~\ref{sec:desain-konseptual}. Arsitektur pipeline dirancang agar proses evaluasi dapat berjalan secara otomatis, konsisten, dan dapat direproduksi. Pendekatan ini memastikan bahwa setiap kombinasi persona, model, dan \textit{benchmark task} diuji dalam kondisi yang setara dan bebas dari variasi yang tidak diperlukan.

Pipeline yang dibangun bekerja sebagai rangkaian komponen yang saling berinteraksi: mulai dari pemuatan data, konstruksi instruksi, pengiriman permintaan ke model, hingga pencatatan \textit{telemetry}. Seluruh proses tersebut bekerja dalam satu alur terintegrasi sehingga sistem mampu menangani jumlah evaluasi yang besar secara stabil.

\subsection{Arsitektur Alur Kerja Sistem}
\label{subsec:arsitektur-alur-kerja}

Secara garis besar, \textit{evaluation pipeline} terbagi ke dalam empat komponen utama yang membentuk satu siklus pemrosesan yang berulang untuk setiap kombinasi persona dan butir soal. Keempat komponen tersebut adalah sebagai berikut.

\begin{enumerate}
    \item \textit{Configuration initialization and validation}.\\
    Tahap ini memuat seluruh konfigurasi sistem, definisi persona, dan \textit{benchmark dataset} ke dalam memori. Validasi struktur data dilakukan untuk memastikan bahwa setiap persona memiliki \textit{system instruction} yang lengkap dan setiap butir tugas memiliki pasangan pertanyaan dan jawaban acuan. Validasi awal ini penting untuk mencegah kesalahan format yang dapat menghentikan proses pada tahap berikutnya.

    \item \textit{Prompt construction engine}.\\
    Pada tahap ini, sistem membentuk dua jenis pesan: \textit{system message} yang berisi identitas persona dan \textit{user message} yang memuat pertanyaan dari benchmark. Penyusunan instruksi dilakukan menggunakan pola yang seragam untuk seluruh iterasi, sehingga setiap model menerima bentuk stimulus yang konsisten. Pendekatan ini menghilangkan variasi yang berasal dari perbedaan penulisan instruksi manual.

    \item \textit{Execution manager}.\\
    Komponen ini mengatur pengiriman permintaan ke model-model bahasa melalui \textit{API interface}. Untuk mengatasi volume permintaan yang besar, \textit{execution manager} menggunakan pendekatan eksekusi asinkron dengan \textit{I/O concurrency}. Permintaan diatur dalam \textit{task queue} dan dieksekusi dalam kelompok sesuai batas \textit{rate limit}. Strategi ini mempercepat proses pengujian tanpa melampaui kapasitas layanan penyedia model.

    \item \textit{Telemetry logger}.\\
    Komponen terakhir bertanggung jawab menyimpan seluruh respons model dalam format terstruktur, termasuk \textit{model output}, jumlah token, serta \textit{latency}. Data ini digunakan sebagai dasar analisis performa pada bab berikutnya.
\end{enumerate}

Dengan pembagian tersebut, pipeline dapat beroperasi secara modular namun tetap terpadu dalam satu alur pemrosesan.

\subsection{Algoritma Orkestrasi dan Konkurensi}
\label{subsec:algoritma-orkestrasi}

Jumlah kombinasi persona, model, dan \textit{benchmark tasks} menghasilkan volume permintaan yang sangat besar. Oleh karena itu, pipeline menerapkan mekanisme eksekusi asinkron untuk meningkatkan efisiensi pemrosesan.

Pipeline pertama-tama membangun sebuah \textit{task queue} yang berisi seluruh pasangan persona–soal. Selanjutnya, \textit{task queue} diproses dalam kelompok yang ukurannya ditentukan oleh kapasitas \textit{concurrency}. Ketika satu kelompok tugas sedang diproses, sistem dapat menyiapkan kelompok berikutnya. Dengan demikian, waktu pemrosesan total dapat ditekan mendekati $O(N/C)$, di mana $N$ adalah jumlah permintaan dan $C$ adalah kapasitas konkurensi.

Apabila terjadi kegagalan seperti \textit{timeout}, \textit{connection reset}, atau batas \textit{rate limit}, pipeline tidak langsung menghentikan seluruh proses. Sebaliknya, tugas tersebut dicatat dan dieksekusi ulang dengan \textit{exponential backoff}. Pendekatan ini membuat pipeline tetap stabil meskipun dijalankan dalam waktu yang panjang.

\subsection{Mekanisme Injeksi Konteks Persona}
\label{subsec:mekanisme-injeksi}

Mekanisme injeksi persona merupakan elemen penting untuk memastikan bahwa pengaruh persona dapat diukur dengan jelas. Pipeline menerapkan dua tahap injeksi konteks yang bersifat tetap dan hanya dilakukan satu kali untuk setiap persona sebelum evaluasi dimulai.

Tahap pertama adalah \textit{persona context initialization}. Pada tahap ini, sistem menyusun pesan awal yang merangkum identitas dan karakter persona. Pesan ini berfungsi membangun \textit{cognitive framing} awal pada model, baik untuk persona eksplisit maupun implisit. Tahap ini memastikan bahwa model berada dalam kondisi persona yang konsisten sebelum diberikan tugas.

Tahap kedua adalah \textit{persona warm-up message}. Pesan ini digunakan untuk memastikan bahwa model memberikan respons yang sesuai dengan identitas persona. Respons dari tahap ini tidak digunakan dalam evaluasi, tetapi berfungsi sebagai verifikasi bahwa proses injeksi berhasil.

Setelah kedua tahap ini selesai, pipeline tidak lagi mengulangi injeksi persona untuk setiap pertanyaan. Identitas yang telah ditanamkan pada awal percakapan tetap digunakan selama seluruh rangkaian pengujian. Model kemudian langsung memproses seluruh soal pada GSM8K dan MMLU-Redux dalam kondisi persona yang sama. Pendekatan ini memastikan bahwa variasi keluaran model berasal dari perbedaan persona, bukan dari perbedaan struktur instruksi.

\subsection{Mekanisme Toleransi Kesalahan dan Persistensi Status}
\label{subsec:toleransi-kesalahan}

Pipeline dirancang agar tetap stabil meskipun menghadapi gangguan selama proses pengujian. Dua mekanisme utama digunakan untuk menjamin integritas data dan keberlanjutan proses.

Pertama, sistem menerapkan \textit{state persistence}. Setelah setiap tugas berhasil diproses, status kemajuan dicatat sehingga apabila terjadi interupsi, pipeline dapat dilanjutkan kembali tanpa mengulangi tugas yang sudah selesai.

Kedua, gangguan sementara ditangani dengan \textit{error handling} berbasis penjadwalan ulang adaptif. Tugas yang gagal tidak langsung dihentikan, tetapi dijalankan kembali setelah jeda waktu tertentu. Dengan kombinasi kedua strategi ini, pipeline dapat menyelesaikan seluruh rangkaian evaluasi meskipun terjadi kendala jaringan atau batasan layanan eksternal.

%====== BAB IV.3 ======%====== BAB IV.3 (Final Version) ======
\section{Implementasi Data, Struktur Berkas, dan Keluaran Pipeline}
\label{sec:implementasi-data}

Subbab ini menjelaskan bagaimana rancangan pipeline yang telah disusun pada bagian sebelumnya direalisasikan dalam bentuk organisasi data, struktur direktori, serta format keluaran yang dihasilkan selama proses eksperimen. Implementasi ini dirancang untuk memastikan bahwa seluruh tahapan pemuatan aset, injeksi konteks persona, pelaksanaan inferensi, dan perekaman hasil berlangsung secara konsisten, terdokumentasi dengan baik, serta mendukung keterulangan eksperimen secara penuh.

\subsection{Organisasi Direktori dan Artefak Data}

Pipeline dijalankan di atas struktur direktori yang dirancang secara modular untuk memisahkan fungsi pemrosesan dan memudahkan proses audit ilmiah. Empat kelompok artefak utama disusun secara hierarkis sebagai berikut.

\begin{enumerate}
    \item \textit{Root directory}.  
    Berfungsi sebagai titik masuk eksekusi sistem dan memuat skrip penggerak pipeline beserta utilitas operasional.

    \item \textit{Configuration directory}.  
    Menyimpan konfigurasi teknis yang digunakan pipeline, termasuk daftar model, kredensial layanan API, dan parameter eksekusi. Pemisahan direktori ini mendukung aspek keamanan dan memudahkan penggantian parameter tanpa memodifikasi kode utama.

    \item \textit{Input assets directory}.  
    Memuat definisi persona serta himpunan benchmark yang telah dinormalisasi. Persona direpresentasikan dalam format terstruktur yang memuat identitas, atribut demografis, dan karakteristik gaya bahasa. Sementara itu, dataset GSM8K dan MMLU-Redux telah dikonversi ke format konsisten untuk memastikan kompatibilitas dengan pipeline.

    \item \textit{Results directory}.  
    Menyimpan keseluruhan keluaran eksperimen yang mencakup log granular pada tingkat per butir soal, tabel hasil, serta agregasi lintas persona dan lintas model. Struktur ini memudahkan proses penelusuran kembali bagi keperluan analisis.
\end{enumerate}

Pemilahan direktori ini memastikan bahwa seluruh artefak eksperimen terdokumentasi secara terstruktur dan mudah direplikasi.

\subsection{Subsistem Perangkat Lunak dan Alur Transformasi Data}

Pipeline terdiri atas empat subsistem utama yang bekerja secara berurutan dalam mengelola eksekusi eksperimen:

\begin{enumerate}
    \item \textit{Execution orchestration subsystem}.  
    Subsistem ini membentuk \textit{task queue} yang memuat seluruh kombinasi model, persona, dan pertanyaan. Orkestrasi ini memastikan determinisme dan menghindari variasi eksekusi akibat intervensi manual.

    \item \textit{Model communication subsystem}.  
    Bertugas melakukan konstruksi instruksi, mengirimkan permintaan ke layanan model, menangani kode galat, serta menegakkan batas layanan seperti \textit{rate limit}. Seluruh komunikasi dilakukan menggunakan protokol API yang distandardisasi.

    \item \textit{Monitoring subsystem}.  
    Menyediakan mekanisme \textit{checkpointing} sehingga eksekusi dapat dilanjutkan tanpa kehilangan progres apabila terjadi gangguan jaringan atau penghentian proses secara tidak terduga. Hal ini memastikan konsistensi eksekusi dan mengurangi risiko duplikasi.

    \item \textit{Analysis subsystem}.  
    Mengolah log mentah menjadi tabel terstruktur dan menghitung metrik utama seperti akurasi, penggunaan token, dan latensi. Modul ini menghasilkan keluaran agregasi yang digunakan dalam tahap analisis pada Bab~V.
\end{enumerate}

Alur transformasi data berlangsung dari log granular → tabel pemetaan → agregasi lintas-model, memungkinkan analisis kuantitatif yang komprehensif.

\subsection{Representasi Persona dan Mekanisme Injeksi Konteks}

Persona direpresentasikan dalam format terstruktur yang memuat identitas demografis, atribut gaya bahasa, serta narasi yang relevan. Representasi tersebut kemudian dikonversi menjadi \textit{system instruction} yang ditempatkan pada segmen instruksi sistem saat permintaan dikirimkan ke model.

Injeksi konteks persona dilakukan satu kali melalui dua tahap:

\begin{enumerate}
    \item \textit{Persona grounding}.  
    Tahap ini menanamkan identitas dan karakteristik gaya bahasa persona secara eksplisit atau implisit pada konteks model.

    \item \textit{Warm-up interaction}.  
    Dilakukan satu interaksi pemanasan untuk menstabilkan perilaku model sehingga respons pada tahap berikutnya mengikuti karakter persona secara konsisten.
\end{enumerate}

Setelah kedua tahap tersebut selesai, pipeline mengirim seluruh pertanyaan GSM8K dan MMLU-Redux tanpa mengulang injeksi persona. Dengan demikian, kondisi kognitif model dijaga agar tetap setara di seluruh siklus inferensi.

\subsection{Contoh Log Inferensi}

Untuk memastikan transparansi proses evaluasi, pipeline mencatat setiap interaksi dengan model dalam bentuk log terstruktur. Log ini memuat informasi mengenai konfigurasi eksekusi, isi jawaban model, serta telemetri penggunaan token. Cuplikan berikut menunjukkan contoh keluaran untuk model yang tidak menyediakan \textit{reasoning trace}.

\begin{tcolorbox}[
    colback=gray!5,
    colframe=gray!50,
    title={Contoh log inferensi tanpa reasoning trace},
    fonttitle=\bfseries,
    arc=2mm,
    left=2mm,
    right=2mm
]
\footnotesize
\begin{verbatim}
{
 "run": {"model_id": "example-model", "question_id": "gsm8k_00001"},
 "response": {
   "choices": [{
     "message": {"content": "Let's break down the problem..."}
   }],
   "usage": {"prompt_tokens": 211, "completion_tokens": 197}
 }
}
\end{verbatim}
\end{tcolorbox}

Cuplikan berikut menunjukkan keluaran log untuk model yang menyediakan \textit{reasoning trace}. Pada model tertentu seperti Nvidia Nemotron, informasi mengenai proses penalaran internal juga tersedia dan direkam secara terpisah. Informasi ini memungkinkan analisis lebih mendalam terhadap struktur penalaran model.

\begin{tcolorbox}[
    colback=gray!5,
    colframe=gray!50,
    title={Contoh log inferensi dengan reasoning trace},
    fonttitle=\bfseries,
    arc=2mm,
    left=2mm,
    right=2mm
]
\footnotesize
\begin{verbatim}
{
 "run": {"model_id": "example-model-reason", "question_id": "gsm8k_00003"},
 "response": {
   "choices": [{
     "message": {
       "content": "Final answer: 70000",
       "reasoning": "First compute the purchase cost..."
     }
   }],
   "usage": {"completion_tokens": 867, "reasoning_tokens": 485}
 }
}
\end{verbatim}
\end{tcolorbox}

\subsection{Ringkasan Hasil Eksperimen}

Ringkasan performa lintas model dan persona ditampilkan pada Tabel~\ref{tab:gsm8k-summary-compact}.  
Tabel ini memberikan gambaran umum mengenai tingkat akurasi dan beban komputasi pada masing-masing konfigurasi dan digunakan sebagai dasar analisis pada Bab~V.

\begin{table}[htbp]
\centering
\caption{Ringkasan Hasil Eksperimen GSM8K untuk Seluruh Model dan Persona}
\label{tab:gsm8k-summary-compact}
\renewcommand{\arraystretch}{1.18}
\small

\begin{adjustbox}{max width=\textwidth}
\begin{tabular}{l l c c c c}
\toprule
\textbf{Model} &
\textbf{Persona} &
\textbf{Total Q} &
\textbf{Correct} &
\textbf{Accuracy (\%)} &
\textbf{Total Tokens} \\
\midrule

Bert Nebulon Alpha & man\_implicits   & 610  & 593  & 97.21 & 285250 \\
Bert Nebulon Alpha & woman\_implicits & 641  & 627  & 97.26 & 335208 \\
\midrule

Grok 4.1 Fast & man\_implicits   & 1315 & 1242 & 94.45 & 1325229 \\
Grok 4.1 Fast & woman\_implicits & 1316 & 1254 & 95.36 & 1422736 \\
\midrule

Nvidia Nemotron-12B v2 VL & man\_implicits   & 1305 & 1224 & 93.79 & 1156049 \\
Nvidia Nemotron-12B v2 VL & woman\_implicits & 1315 & 1248 & 94.98 & 1986284 \\
\bottomrule
\end{tabular}
\end{adjustbox}

\end{table}

Tabel tersebut menjadi dasar perbandingan antar-model pada bab evaluasi, termasuk pengaruh persona terhadap akurasi dan kompleksitas respons.
