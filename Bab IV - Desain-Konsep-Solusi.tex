% ==========================================
% BAB IV DESAIN KONSEP SOLUSI
% ==========================================

\chapter{DESAIN KONSEP SOLUSI}
\label{chap:desain-konsep-solusi}

\section{Desain Konseptual Eksperimen}

Bagian ini menjelaskan landasan perancangan eksperimen yang digunakan dalam penelitian. Desain ini disusun untuk melihat bagaimana dua bentuk persona, yaitu persona eksplisit dan persona implisit, memengaruhi hasil keluaran pada beberapa kategori tugas penalaran dan beberapa sistem yang berbeda. Penyusunan bagian ini dimaksudkan untuk memastikan bahwa setiap variasi yang muncul dapat ditelusuri kembali pada kondisi persona yang digunakan, bukan pada perbedaan situasi pengujian atau susunan instruksi.

\subsection{Tujuan Perancangan Eksperimen}

Perancangan eksperimen dilakukan untuk menyediakan kerangka yang memungkinkan perbandingan persona secara terarah. Dua bentuk persona digunakan karena mewakili dua pola interaksi yang umum terjadi, yaitu ketika identitas pengguna dinyatakan secara langsung serta ketika identitas tersebut tersirat melalui cara bertutur. Kerangka ini juga dirancang agar dapat digunakan untuk membandingkan respons dari beberapa sistem secara konsisten pada jenis tugas yang sama.

\subsection{Komponen Utama Eksperimen}

Eksperimen yang dilakukan mengombinasikan tiga komponen utama, yaitu persona, sistem, dan tugas penalaran.  
Persona mencakup bentuk eksplisit dan implisit, yang masing-masing memberikan konteks pengguna dengan kedalaman dan cara penyampaian yang berbeda.  
Komponen sistem terdiri atas beberapa model yang tersedia melalui layanan API sehingga memungkinkan analisis lintas arsitektur.  
Tugas penalaran yang digunakan mencakup penalaran numerik dan penalaran lintas topik untuk melihat bagaimana bentuk persona memengaruhi keluaran pada sifat tugas yang berbeda.

\subsection{Prinsip Pengendalian Variabel}

Untuk menjaga kesetaraan pengujian, seluruh instruksi disampaikan menggunakan susunan yang seragam pada setiap kombinasi persona, sistem, dan tugas. Dengan demikian, unsur yang bervariasi hanyalah bentuk persona. Pendekatan ini dilakukan agar hasil yang diperoleh dapat dibandingkan secara langsung tanpa dipengaruhi oleh variasi lain di luar persona.

\subsection{Ruang Konfigurasi}

Ruang eksperimen dibentuk berdasarkan kombinasi antara persona, sistem, dan tugas penalaran. Setiap elemen didefinisikan melalui berkas konfigurasi sehingga struktur ruang eksperimen terdokumentasi dengan jelas dan dapat diperluas apabila diperlukan. Dengan adanya pengaturan ini, seluruh kondisi yang diuji dapat ditelusuri kembali dan dianalisis berdasarkan konfigurasi yang digunakan.

\subsection{Keterkaitan dengan Pelaksanaan Eksperimen}

Desain konseptual ini menjadi dasar bagi alur pelaksanaan yang dibahas pada bagian berikutnya. Dengan pemisahan antara tahap perancangan dan tahap pelaksanaan, eksperimen dapat dijalankan secara teratur, dan seluruh hasil yang diperoleh dapat dianalisis kembali pada bab selanjutnya.


%====== BAB IV.2 ======
\section{Arsitektur \textit{Evaluation Pipeline} dan Alur Pelaksanaan Eksperimen}
\label{sec:arsitektur-dan-alur}

Bagian ini menjelaskan bagaimana rancangan konseptual pada Subbab sebelumnya direalisasikan dalam bentuk arsitektur \textit{evaluation pipeline} yang terotomatisasi, serta bagaimana pipeline tersebut menjalankan alur eksperimen dari pemuatan \textit{specification} hingga diperolehnya keluaran akhir. Pendekatan ini dirancang agar proses evaluasi berjalan secara otomatis, konsisten, dan dapat direproduksi, sehingga setiap kombinasi persona, model, dan \textit{benchmark task} diuji dalam kondisi yang setara dan bebas dari variasi yang tidak diperlukan.

Pipeline bekerja sebagai rangkaian komponen yang saling berinteraksi: mulai dari pemuatan data, konstruksi instruksi, pengiriman permintaan ke model, hingga pencatatan \textit{telemetry}. Seluruh proses tersebut membentuk satu alur terintegrasi yang mampu menangani jumlah evaluasi besar secara stabil.

\subsection{Arsitektur Alur Kerja Sistem}
\label{subsec:arsitektur-alur-kerja}

Secara garis besar, \textit{evaluation pipeline} terbagi ke dalam empat komponen utama yang membentuk satu siklus pemrosesan berulang untuk setiap kombinasi persona dan butir soal. Keempat komponen tersebut adalah sebagai berikut.

\begin{enumerate}
    \item \textit{Configuration initialization and validation}.\\
    Tahap ini memuat seluruh konfigurasi sistem, definisi persona, dan \textit{benchmark dataset} ke dalam memori. Struktur data yang dibaca dari berkas \textit{specification} (persona, model, dan \textit{task}) divalidasi untuk memastikan bahwa setiap persona memiliki \textit{system instruction} yang lengkap dan setiap butir tugas memiliki pasangan pertanyaan dan jawaban acuan. Validasi awal ini penting untuk mencegah kesalahan format yang dapat menghentikan proses pada tahap berikutnya.

    \item \textit{Prompt construction engine}.\\
    Pada tahap ini, sistem membentuk dua jenis pesan utama: \textit{system message} yang berisi identitas dan karakter persona, serta \textit{user message} yang memuat pertanyaan dari \textit{benchmark}. Penyusunan instruksi dilakukan menggunakan pola yang seragam untuk seluruh iterasi, sehingga setiap model menerima bentuk stimulus yang konsisten. Pendekatan ini menghilangkan variasi yang berasal dari perbedaan penulisan instruksi manual, sehingga perubahan keluaran dapat dikaitkan pada persona, bukan pada redaksi \textit{prompt}.

    \item \textit{Execution manager}.\\
    Komponen ini mengatur pengiriman permintaan ke model-model bahasa melalui \textit{API interface}. Untuk mengatasi volume permintaan yang besar, \textit{execution manager} menggunakan pendekatan eksekusi asinkron berbasis \textit{I/O concurrency}. Permintaan disusun dalam \textit{task queue} dan dieksekusi dalam kelompok sesuai batas \textit{rate limit} dari penyedia layanan model. Strategi ini mempercepat proses pengujian tanpa melampaui kapasitas layanan.

    \item \textit{Telemetry logger}.\\
    Komponen terakhir bertanggung jawab menyimpan seluruh respons model dalam format terstruktur, termasuk keluaran teks, jawaban akhir yang diekstraksi, jumlah token yang digunakan, serta \textit{latency} inferensi. Data ini menjadi dasar analisis performa pada Bab~V, baik dari sisi akurasi maupun beban komputasi.
\end{enumerate}

Dengan pembagian tersebut, pipeline dapat beroperasi secara modular, namun tetap terpadu dalam satu alur pemrosesan yang deterministik.

\subsection{Algoritma Orkestrasi dan Konkurensi}
\label{subsec:algoritma-orkestrasi}

Eksperimen dalam penelitian ini melibatkan ribuan kombinasi persona–model–pertanyaan yang menghasilkan volume permintaan API dalam jumlah besar. Eksekusi secara sekuensial tidak praktis karena setiap permintaan memiliki latensi yang bervariasi, sementara penyedia model menerapkan batas \textit{rate limit} yang ketat. Untuk mengatasi hal tersebut, pipeline menggunakan pendekatan eksekusi asinkron berbasis \textit{I/O concurrency}.

Pendekatan ini memungkinkan banyak permintaan dieksekusi secara paralel (hingga batas tertentu), sehingga waktu total dapat ditekan dari kompleksitas \(O(N)\) menjadi mendekati \(O(N/C)\), dengan \(C\) adalah kapasitas konkurensi maksimum. Pipeline membangun sebuah \textit{task queue} yang berisi seluruh pasangan persona–soal, kemudian memprosesnya dalam kelompok (\textit{batch}) sesuai kapasitas konkurensi. Ketika satu batch sedang diproses, sistem dapat menyiapkan batch berikutnya tanpa menunggu seluruh permintaan selesai.

Selain meningkatkan efisiensi waktu, mekanisme ini juga menyediakan ketahanan terhadap kesalahan. Jika terjadi galat seperti \textit{timeout}, \textit{connection reset}, atau \texttt{429 Too Many Requests}, pipeline tidak menghentikan seluruh proses. Tugas yang gagal akan dicatat dan dijalankan ulang menggunakan strategi \textit{exponential backoff}, sehingga stabilitas eksekusi jangka panjang tetap terjaga.

Algoritma 4.1 berikut mendefinisikan prosedur eksekusi paralel secara formal.

\begin{verbatim}
Algoritma 4.1: Prosedur Eksekusi Eksperimen Paralel

Input : Himpunan Persona P, Himpunan Tugas T, Batas Konkurensi C
Output: Himpunan Log L

Function RunExperiment(P, T):
  1. Inisialisasi Antrean Tugas Q <- Kosong
  2. Untuk setiap p dalam P lakukan:
       Untuk setiap t dalam T lakukan:
         Prompt <- ConstructPrompt(p.instruction, t.question)
         Enqueue(Q, Prompt)

  3. Inisialisasi Semaphore S dengan kapasitas C

  4. While Q tidak kosong lakukan secara Asinkron:
       Batch <- DequeueBatch(Q, C)
       Untuk setiap item i dalam Batch lakukan secara Paralel:
         Acquire(S)
         Coba:
           Respons <- AsyncCallAPI(i.prompt, i.config)
           Metadata <- ExtractTelemetry(Respons)
           SaveLog(Respons, Metadata)
           Tambahkan ke L
         Tangkap Galat:
           LogGalat(i)
           RetryWithBackoff(i)
         Akhirnya:
           Release(S)

  5. Return L
\end{verbatim}

Melalui orkestrasi ini, pipeline mencapai dua tujuan sekaligus: (1) efisiensi waktu eksekusi yang optimal berkat pemrosesan paralel, dan (2) ketahanan proses melalui penanganan galat adaptif.

\subsection{Mekanisme Injeksi Konteks Persona}
\label{subsec:mekanisme-injeksi}

Mekanisme injeksi persona merupakan elemen penting untuk memastikan bahwa pengaruh persona terhadap keluaran model dapat diukur secara jelas. Pipeline menerapkan dua tahap injeksi konteks yang bersifat tetap dan hanya dilakukan satu kali untuk setiap persona sebelum rangkaian evaluasi dimulai.

Tahap pertama adalah \textit{persona context initialization}. Pada tahap ini, sistem menyusun \textit{system message} yang merangkum identitas dan karakter persona, baik dalam bentuk eksplisit maupun implisit sebagaimana didefinisikan pada Subbab~\ref{tab:user-persona}. Pesan ini berfungsi membangun \textit{cognitive framing} awal pada model sehingga konteks persona tertanam sebelum tugas utama diberikan.

Tahap kedua adalah \textit{persona warm-up message}. Pipeline mengirimkan satu interaksi pemanasan untuk memverifikasi bahwa respons model sudah mengikuti identitas dan gaya tutur persona tersebut. Respons dari tahap ini tidak digunakan dalam evaluasi, tetapi berfungsi sebagai pemeriksaan bahwa proses injeksi berhasil.

Setelah kedua tahap ini selesai, pipeline tidak lagi mengulangi injeksi persona untuk setiap pertanyaan. Identitas yang telah ditanamkan pada awal percakapan tetap digunakan selama seluruh rangkaian pengujian. Model kemudian langsung memproses seluruh soal pada GSM8K dan MMLU-Redux dalam kondisi persona yang sama. Pendekatan ini memastikan bahwa variasi keluaran model berasal dari perbedaan persona, bukan dari perbedaan struktur instruksi pada setiap soal.

\subsection{Alur Operasional Pelaksanaan Eksperimen}
\label{subsec:alur-operasional}

Secara operasional, alur pelaksanaan eksperimen mengikuti rangkaian langkah yang digambarkan pada Gambar~\ref{fig:alur-eksperimen}. Diagram tersebut menunjukkan hubungan antara pembentukan \textit{configuration}, penyusunan \textit{instruction}, eksekusi \textit{task}, dan pencatatan hasil dalam satu siklus pipeline.

\begin{figure}[htbp]
  \centering
  \includegraphics[width=0.5\textwidth]{image/alur-eksperimen.png}
  \caption{Diagram alur pelaksanaan eksperimen}
  \label{fig:alur-eksperimen}
\end{figure}

Pelaksanaan eksperimen dilakukan melalui langkah-langkah berikut.

\begin{enumerate}
    \item Memuat \textit{specification}.\\
    \textit{System} membaca berkas \textit{specification} yang memuat daftar persona, daftar model, daftar \textit{task}, serta aturan eksekusi. Informasi tersebut diproses menjadi dasar pembentukan himpunan \textit{configuration} yang akan dievaluasi.

    \item Membentuk \textit{configuration} lengkap.\\
    Seluruh kombinasi persona, model, dan \textit{task} dibentuk sebagai unit eksekusi dan dicatat untuk dijalankan selama eksperimen. Setiap \textit{configuration} menyimpan identitas persona, model, dan penanda butir soal yang terkait.

    \item Memilih satu \textit{configuration}.\\
    \textit{System} mengambil satu \textit{configuration} dari antrean tugas pada setiap siklus dan menjadwalkannya untuk dieksekusi hingga seluruh kombinasi selesai diproses.

    \item Menerapkan \textit{persona}.\\
    \textit{Persona} yang sesuai dengan \textit{configuration} tersebut diterapkan terlebih dahulu agar \textit{task} diproses dalam konteks pengguna yang telah ditetapkan. Pada beberapa kondisi, digunakan satu interaksi \textit{warmup} untuk memastikan bahwa respons awal model sudah mengikuti karakter persona sebelum rangkaian \textit{task} utama dikirimkan.

    \item Menyusun \textit{instruction} untuk \textit{task}.\\
    \textit{Instruction} dirumuskan dengan susunan yang seragam oleh \textit{prompt construction engine}, sehingga perbedaan hasil dapat dikaitkan pada variasi persona dan model, bukan pada perbedaan redaksi atau struktur penyampaian.

    \item Mengirim \textit{instruction} kepada model.\\
    \textit{Instruction} yang telah lengkap dikirimkan kepada model melalui \textit{execution manager} untuk memperoleh \textit{response} yang digunakan dalam tahap analisis.

    \item Penanganan kegagalan.\\
    Jika \textit{response} tidak diperoleh atau terjadi gangguan sementara, \textit{instruction} dijadwalkan ulang menggunakan jeda adaptif hingga \textit{response} valid diterima. Mekanisme ini memastikan seluruh \textit{configuration} menghasilkan keluaran yang dapat digunakan.

    \item Mencatat hasil \textit{response}.\\
    \textit{Response} yang diterima disimpan oleh \textit{telemetry logger} dalam berkas penyimpanan bersama informasi pendukung lainnya, seperti jumlah token dan \textit{latency}, untuk keperluan analisis.

    \item Melanjutkan ke \textit{configuration} berikutnya.\\
    Setelah satu \textit{configuration} selesai, \textit{system} beralih ke \textit{configuration} berikutnya hingga seluruh ruang eksperimen selesai dievaluasi.
\end{enumerate}

Dengan arsitektur dan alur operasional ini, eksperimen dapat dijalankan secara teratur, terukur, dan setiap hasil yang dihasilkan dapat ditelusuri kembali berdasarkan \textit{configuration} yang digunakan.



%====== BAB IV.3======

\section{Integrasi Komponen Eksperimen}

Bagian ini menjelaskan komponen-komponen yang digunakan dalam eksperimen, yang terdiri atas \textit{benchmark} penalaran, himpunan model, struktur persona, ruang \textit{configuration}, serta contoh mekanisme injeksi persona. Seluruh komponen tersebut didefinisikan melalui berkas \textit{specification} sehingga dapat digunakan secara konsisten pada seluruh tahapan eksperimen.

\subsection{Benchmark Penalaran}

Eksperimen menggunakan dua \textit{benchmark} yang mewakili dua bentuk kemampuan penalaran.

\textit{Benchmark} pertama adalah \textit{GSM8K}, yang berisi soal cerita matematika tingkat sekolah menengah. \textit{Benchmark} ini menilai kemampuan sistem dalam melakukan penalaran numerik bertahap. Setiap soal memiliki jawaban numerik yang jelas sehingga pemeriksaan hasil dapat dilakukan secara deterministik \parencite{cobbe2021gsm8k}.

\textit{Benchmark} kedua adalah \textit{MMLU-Redux}, versi terkurasi dari MMLU yang memperbaiki ketidakkonsistenan format dan pilihan jawaban. \textit{Benchmark} ini digunakan untuk menilai penalaran lintas topik dalam format pilihan ganda, meliputi bidang sains, matematika, humaniora, dan ilmu sosial \parencite{mmluRedux2024dataset}.

Penggunaan kedua \textit{benchmark} tersebut memberikan cakupan dua bentuk penalaran yang berbeda, yaitu penalaran numerik prosedural dan penalaran konseptual deklaratif.

\subsection{Himpunan Model}

Eksperimen dijalankan pada beberapa model yang tersedia melalui layanan API. Model-model tersebut dipilih untuk memberikan keragaman arsitektur sehingga perbedaan respons yang muncul dapat dibandingkan lintas sistem. Model yang digunakan meliputi:

\begin{enumerate}
    \item Model komersial  
    GPT-5 Mini, Claude 4.5 Haiku, Gemini 2.5 Flash, Llama 3.3 Nemotron Super 49B V1.5, Google Gemma 3n 4B, dan DeepSeek V3.2

    \item Model publik  
    Grok 4.1 Fast, NVIDIA Nemotron-nano-12B-v2-VL, dan Bert Nebulon Alpha.
\end{enumerate}

Keragaman ini memungkinkan analisis sensitivitas persona pada berbagai sistem dengan karakteristik yang berbeda.

\subsection{Struktur Persona}

Persona yang digunakan dalam eksperimen disusun berdasarkan enam dimensi: gender, usia, agama, pekerjaan, kewarganegaraan, dan register bahasa. Kombinasi dimensi tersebut menghasilkan lima belas persona yang mencakup persona eksplisit dan persona implisit, serta satu kondisi pengguna netral sebagai pembanding.

Tabel~\ref{tab:user-persona} menyajikan daftar lengkap persona yang digunakan.

\begin{table}[htbp]
\centering
\caption{Daftar persona pada kondisi eksperimen}
\label{tab:user-persona}
\renewcommand{\arraystretch}{1.15}

\resizebox{\textwidth}{!}{
\begin{tabular}{c l l l l l l l}
\toprule
\textbf{ID} &
\textbf{Persona} &
\textbf{Mode} &
\textbf{Gender} &
\textbf{Age Group} &
\textbf{Religion} &
\textbf{Occupation} &
\textbf{Nationality / Register} \\
\midrule
P1  & Implicit male baseline              & Implicit & Male   & -           & -        & -                & Neutral \\
P2  & Implicit female baseline            & Implicit & Female & -           & -        & -                & Neutral \\
P3  & Neutral user                        & Neutral  & -      & -           & -        & -                & Neutral \\
P4  & Indonesian Muslim young woman       & Explicit & Female & Young adult & Muslim   & Healthcare worker & Indonesian / Semi-formal \\
P5  & Indonesian Muslim young man         & Implicit & Male   & Young adult & Muslim   & Healthcare worker & Indonesian / Semi-formal \\
P6  & American middle-aged male           & Explicit & Male   & Middle-aged & Christian & Engineer         & American / Formal \\
P7  & American middle-aged female         & Implicit & Female & Middle-aged & Christian & Engineer         & American / Formal \\
P8  & Indonesian Gen-Z female             & Explicit & Female & Gen-Z       & -        & Student          & Indonesian / Casual-slang \\
P9  & Indonesian Gen-Z male               & Implicit & Male   & Gen-Z       & -        & Student          & Indonesian / Casual-slang \\
P10 & Middle Eastern young adult male     & Explicit & Male   & Young adult & Muslim   & Engineer         & Middle Eastern Arabic / Formal \\
P11 & Middle Eastern young adult female   & Implicit & Female & Young adult & Muslim   & Student          & Middle Eastern Arabic / Formal \\
P12 & American atheist young male         & Explicit & Male   & Young adult & Atheist  & Student          & American / Formal \\
P13 & American atheist young female       & Implicit & Female & Young adult & Atheist  & Student          & American / Formal \\
P14 & Indonesian female healthcare worker & Explicit & Female & Young adult & Muslim   & Healthcare worker & Indonesian / Semi-formal \\
P15 & Indonesian male healthcare worker   & Implicit & Male   & Young adult & Muslim   & Healthcare worker & Indonesian / Semi-formal \\
\bottomrule
\end{tabular}
}
\end{table}

\subsection{Ruang \textit{Configuration}}

Kombinasi lima belas persona dan sembilan model membentuk seratus tiga puluh lima \textit{configuration}. Setiap \textit{configuration} merepresentasikan satu pasangan persona dan model yang kemudian diuji pada himpunan \textit{task} yang sama. Dengan cara ini, variasi keluaran dapat dibandingkan pada dua tingkat, yaitu perbedaan antar persona dalam satu model dan perbedaan antar model pada persona yang sama.

Untuk menjaga keteraturan proses, setiap \textit{configuration} melewati urutan eksekusi yang tetap. Urutan tersebut meliputi penerapan persona pada awal percakapan, penyiapan konteks interaksi, pelaksanaan \textit{benchmark} pada himpunan soal yang telah ditetapkan, serta pencatatan hasil dan informasi pendukung. Pola yang berulang ini memudahkan penelusuran kembali setiap hasil ke persona, model, dan \textit{task} yang digunakan.

\subsection{Contoh Mekanisme Injeksi Persona}

Persona diterapkan melalui \textit{system message} yang dikirim sebelum \textit{task} utama diberikan. Dua bentuk persona digunakan dalam eksperimen, yaitu persona eksplisit dan persona implisit.

Pada persona eksplisit, identitas pengguna dinyatakan secara langsung melalui deskripsi. Instruksi ini menyebutkan atribut sosial yang relevan, seperti gender, usia, pekerjaan, atau preferensi gaya bahasa. Contoh yang digunakan dalam eksperimen adalah sebagai berikut.

\begin{quote}
\textit{
“Your user is an Indonesian Gen-Z male who works as a junior engineer.  
He is analytical, prefers concise explanations, and communicates in a casual but respectful tone.”}
\end{quote}

Formulasi seperti ini memberikan konteks identitas yang jelas sehingga perubahan pada struktur penalaran dan gaya jawaban dapat dikaitkan dengan persona yang digunakan.

Pada persona implisit, identitas tidak disebutkan secara langsung, tetapi ditampilkan melalui narasi pengalaman, ekspresi emosi, atau gaya tutur tertentu. Model menerima konteks ini sebagai bagian dari cerita pengguna dan perlu menyimpulkan sendiri karakter pengguna dari isyarat linguistik yang ada. Contoh yang digunakan dalam eksperimen adalah sebagai berikut.

\begin{quote}
\textit{
“Lately I have been feeling a strange mix of emotional exhaustion and pressure to appear composed, especially when my skin starts acting up unexpectedly. Before I deal with it again, could you help me break down this next question step-by-step?”}
\end{quote}

Kedua bentuk injeksi ini memungkinkan analisis perbedaan respons antara persona yang dinyatakan secara eksplisit dan persona yang hanya tersirat melalui cara pengguna menyampaikan situasi dan pertanyaannya.

%====== BAB IV.4 ======
\section{Perancangan Data dan Struktur Berkas}

Bagian ini menjelaskan rancangan data dan struktur berkas yang digunakan dalam eksperimen. Perancangan ini diperlukan agar keluaran dari setiap \textit{configuration} dapat dicatat secara teratur, ditelusuri kembali, dan dianalisis pada tahap berikutnya. Data yang digunakan dalam eksperimen dikelompokkan menjadi empat bagian utama, yaitu data konfigurasi, data benchmark, data masukan tambahan, dan data hasil eksekusi.

Data konfigurasi disimpan di dalam direktori \texttt{config}. Direktori ini memuat berkas \textit{specification} yang menjadi dasar pembentukan ruang eksperimen, termasuk berkas \texttt{model.keys.json} yang berisi daftar model yang tersedia melalui layanan API, serta berkas lain yang memuat daftar \textit{persona}, daftar \textit{task}, dan parameter eksekusi. Perubahan terhadap ruang eksperimen dapat dilakukan dengan memodifikasi berkas-berkas pada direktori ini tanpa perlu mengubah kode program.

Data benchmark disimpan di dalam direktori \texttt{data}. Direktori ini berisi \textit{dataset} yang digunakan dalam eksperimen, termasuk materi \textit{GSM8K} dan \textit{MMLU-Redux} dalam bentuk mentah maupun bentuk yang telah dinormalisasi untuk keperluan pemrosesan. Dengan pemisahan ini, sumber data utama yang digunakan pipeline terdokumentasi secara jelas.

Direktori \texttt{input} digunakan untuk menyimpan data pendukung yang tidak berasal dari benchmark utama tetapi dibutuhkan selama eksperimen, seperti kumpulan soal yang dihasilkan ulang, daftar pertanyaan tambahan, atau berkas uji lain yang disiapkan secara terpisah dari \textit{dataset} utama. Pemisahan antara \texttt{data} dan \texttt{input} menjaga agar data asli dan data turunan tidak tercampur, serta memudahkan pelacakan asal setiap \textit{task} yang dieksekusi.

Dokumen pendukung, seperti catatan desain, skema eksperimen, dan dokumentasi penggunaan pipeline, disimpan pada direktori \texttt{docs}. Direktori ini tidak terlibat langsung dalam proses eksekusi, tetapi membantu proses audit dan pemeliharaan sistem di kemudian hari.

Hasil eksperimen disimpan di dalam direktori \texttt{results}. Direktori ini memuat berkas \textit{JSON} yang mencatat \textit{response} lengkap untuk setiap \textit{configuration}, termasuk \textit{instruction} yang digunakan, jawaban model, serta metadata yang dihasilkan selama eksekusi. Ringkasan hasil disimpan dalam bentuk \textit{CSV} untuk mempermudah proses analisis, misalnya perbandingan jawaban akhir, tingkat akurasi, jumlah token, atau latensi jika informasi tersebut disediakan oleh layanan model.

Seluruh kode program ditempatkan dalam direktori \texttt{src}. Direktori ini berisi modul yang memuat \textit{specification}, menyusun \textit{instruction}, menjalankan \textit{task} untuk setiap \textit{configuration}, serta mencatat hasil eksekusi ke dalam \texttt{results}. Dengan pemisahan antara kode dan data, eksperimen dapat dijalankan kembali dengan pengaturan yang sama atau diperluas dengan \textit{specification} baru tanpa mengubah struktur direktori lainnya.

Dengan struktur direktori ini, setiap \textit{response} yang dihasilkan dapat ditelusuri kembali melalui \textit{persona}, model, dan \textit{task} yang digunakan. Perancangan ini mendukung kebutuhan replikasi eksperimen dan menjadi penghubung antara desain konseptual pada bagian sebelumnya dan analisis hasil pada bab berikutnya.

%====== BAB IV.5 ======
\section{Penanganan Gangguan dan Pemulihan \textit{Execution Flow}}

Proses eksekusi melibatkan sejumlah besar kombinasi persona, model, dan \textit{task} sehingga rentan terhadap berbagai bentuk gangguan, baik yang bersumber dari layanan model maupun dari kondisi jaringan. Bagian ini menjelaskan mekanisme yang digunakan untuk menjaga agar alur eksekusi tetap berlanjut meskipun terjadi hambatan, serta memastikan bahwa hasil yang diperoleh tetap dapat ditelusuri dan dianalisis tanpa kehilangan konsistensi.

Penanganan gangguan dilakukan dalam dua bentuk utama.

\begin{enumerate}
    \item \textit{Transient error handling}  
    Sistem mendeteksi gangguan sementara seperti \textit{timeout}, penolakan layanan, atau pemutusan koneksi. Apabila gangguan terjadi, \textit{instruction} dikirim ulang menggunakan jeda adaptif. Mekanisme ini mencegah penghentian proses secara keseluruhan dan memastikan setiap \textit{configuration} tetap menghasilkan keluaran yang dapat dianalisis.

    \item \textit{Execution flow recovery}  
    Untuk menjaga keberlanjutan proses, sistem mencatat status terakhir setelah setiap respons diterima. Apabila eksekusi terhenti sebelum seluruh \textit{configuration} selesai diproses, pipeline dapat dilanjutkan dari posisi terakhir tanpa mengulang bagian yang telah berhasil. Dengan cara ini, proses panjang tetap dapat diselesaikan tanpa kehilangan progres.
\end{enumerate}

Kedua mekanisme ini bekerja bersamaan untuk memastikan bahwa alur eksekusi tetap stabil pada skala besar. Pendekatan ini memungkinkan seluruh rangkaian eksperimen diselesaikan meskipun terdapat hambatan teknis, sehingga hasil yang diperoleh tetap dapat dipertanggungjawabkan dalam tahap analisis pada bab berikutnya.

%====== BAB IV.6 ======
\section{Implementasi Keluaran Pipeline}
\label{sec:implementasi-keluaran}

Bagian ini menyajikan bentuk keluaran yang dihasilkan oleh \textit{evaluation pipeline}
setelah seluruh tahapan pemrosesan dijalankan. Keluaran ini berfungsi sebagai artefak
utama yang digunakan dalam analisis pada Bab~V. Seluruh hasil disimpan dalam direktori
\texttt{results} dalam format terstruktur sehingga dapat ditelusuri kembali ke
\textit{persona}, model, dan \textit{task} yang digunakan.

\subsection{Contoh Struktur Log Inferensi}

Pipeline mencatat setiap interaksi dengan model dalam bentuk berkas \texttt{JSON}.
Log ini memuat identitas konfigurasi yang dieksekusi, jawaban model, serta
telemetri penggunaan token. Cuplikan berikut memperlihatkan struktur log
untuk model yang tidak menyediakan \textit{reasoning trace}.

\begin{verbatim}
{
  "run": {
    "model_id": "example-model",
    "question_id": "gsm8k_00001",
    "persona": "implicit_male"
  },
  "response": {
    "choices": [
      {
        "message": {
          "content": "Let's break down the problem..."
        }
      }
    ],
    "usage": {
      "prompt_tokens": 211,
      "completion_tokens": 197,
      "total_tokens": 408
    }
  },
  "meta": {
    "latency_ms": 842,
    "timestamp": "2025-01-18T12:44:10Z"
  }
}
\end{verbatim}

Struktur tersebut menunjukkan bahwa pipeline tidak hanya merekam jawaban,
tetapi juga metadata komputasional yang diperlukan dalam analisis efisiensi.

\subsection{Contoh Struktur Log dengan Reasoning Trace}

Beberapa model menyediakan tambahan berupa \textit{reasoning trace}. Bagian
penalaran ini disimpan terpisah dari jawaban akhir dan dicatat sebagai bagian
dari log. Cuplikan berikut menunjukkan contoh berkas log yang memuat
\textit{reasoning trace}.

\begin{verbatim}
{
  "run": {
    "model_id": "example-model-reason",
    "question_id": "gsm8k_00003",
    "persona": "explicit_genz_female"
  },
  "response": {
    "choices": [
      {
        "message": {
          "content": "Final answer: 70000",
          "reasoning": "First compute the purchase cost..."
        }
      }
    ],
    "usage": {
      "completion_tokens": 867,
      "reasoning_tokens": 485,
      "total_tokens": 1352
    }
  },
  "meta": {
    "latency_ms": 2134,
    "timestamp": "2025-01-18T12:52:41Z"
  }
}
\end{verbatim}

Log ini memungkinkan analisis lebih dalam mengenai gaya penalaran dan perubahan
struktur argumen yang mungkin disebabkan oleh persona tertentu.

\subsection{Ringkasan Hasil Eksperimen}

Pipeline juga menghasilkan ringkasan performa dalam bentuk tabel yang
menggabungkan metrik akurasi dan penggunaan token untuk setiap pasangan
persona–model. Berkas ini disimpan dalam format \texttt{CSV} untuk memudahkan
analisis lanjutan. Tabel berikut merupakan contoh ringkasan hasil yang dihasilkan.

\begin{table}[htbp]
\centering
\caption{Contoh Ringkasan Hasil Eksperimen GSM8K untuk Seluruh Model dan Persona}
\label{tab:gsm8k-summary-compact}
\renewcommand{\arraystretch}{1.15}
\small
\begin{tabular}{l l c c c c}
\toprule
\textbf{Model} &
\textbf{Persona} &
\textbf{Total Q} &
\textbf{Correct} &
\textbf{Accuracy (\%)} &
\textbf{Total Tokens} \\
\midrule
Bert Nebulon Alpha & man\_implicit   & 610  & 593  & 97.21 & 285250 \\
Bert Nebulon Alpha & woman\_implicit & 641  & 627  & 97.26 & 335208 \\
\midrule
Grok 4.1 Fast & man\_implicit   & 1315 & 1242 & 94.45 & 1325229 \\
Grok 4.1 Fast & woman\_implicit & 1316 & 1254 & 95.36 & 1422736 \\
\midrule
Nvidia Nemotron 12B v2 VL & man\_implicit   & 1305 & 1224 & 93.79 & 1156049 \\
Nvidia Nemotron 12B v2 VL & woman\_implicit & 1306 & 1230 & 94.18 & 1184521 \\
\bottomrule
\end{tabular}
\end{table}
