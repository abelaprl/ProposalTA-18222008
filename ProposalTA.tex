%! TEX program = xelatex
% 
% Template Proposal Tugas Akhir
% Program Studi Sistem dan Teknologi Informasi
% Sekolah Teknik Elektro dan Informatika
% Institut Teknologi Bandung
% 
% Dibuat oleh: IGB Baskara Nugraha 
% Email: baskara@itb.ac.id 
% 
% Last updated: 20 Oktober 2025
%
% Petunjuk penggunaan:
% 1. Ada 2 file utama, yaitu ProposalTA.tex (file ini) dan daftar-pustaka.bib (file daftar pustaka).
% 2. Sunting ProposalTA.tex sesuai dengan kebutuhan Anda.
% 3. Sunting atau generate isi daftar-pustaka.bib dengan referensi yang Anda gunakan, sesuai dengan format BibLaTeX.
% 4. Simpan kedua file tersebut dalam satu folder yang sama.
% 5. Kompilasi file ProposalTA.tex menggunakan XeLaTeX dan Biber (lihat urutan cara kompilasi di bawah).
% 6. Hasil kompilasi adalah file ProposalTA.pdf yang siap dicetak.
% 
% Urutan cara kompilasi (melalui command line):
% 1. xelatex ProposalTA.tex
% 2. biber ProposalTA      
% 3. xelatex  ProposalTA.tex
% 4. xelatex  ProposalTA.tex
%
% Catatan:
% - Pastikan Anda telah menginstal paket-paket LaTeX yang diperlukan, termasuk
%   biblatex-chicago dan fontspec.
% - Gunakan editor LaTeX yang mendukung XeLaTeX, seperti TeXstudio, Overleaf, atau lainnya.
% - Jika meenggunakan Visual Studio Code sebagai editor, pastikan mengatur "latex-workshop.latex.tools" dan
%   "latex-workshop.latex.recipes" untuk mendukung XeLaTeX dan Biber dengan cara menambahkan konfigurasi berikut:
%   "latex-workshop.latex.tools": [ 
%       {
%           "name": "xelatex",
%           "command": "xelatex",
%           "args": [
%               "-synctex=1",
%               "-interaction=nonstopmode",
%               "-file-line-error",
%               "%DOC%"
%           ]
%       },
%       {
%           "name": "biber",
%           "command": "biber",
%           "args": [
%               "%DOCFILE%"
%           ]
%       }
%   ],
%   "latex-workshop.latex.recipes": [
%       {
%           "name": "xelatex -> biber -> xelatex*2",
%           "tools": [
%               "xelatex",
%               "biber",
%               "xelatex",
%               "xelatex"
%           ]
%       }
%   ]
% - Untuk referensi lebih lanjut tentang penggunaan BibLaTeX dengan gaya Chicago, silakan merujuk ke dokumentasi resmi BibLaTeX.
%   https://ctan.org/pkg/biblatex-chicago
% - Untuk referensi lebih lanjut tentang penggunaan XeLaTeX dan fontspec, silakan merujuk ke dokumentasi resmi fontspec.
%   https://ctan.org/pkg/fontspec
% - Selamat menyusun proposal tugas akhir Anda!
%
\documentclass[12pt,a4paper,oneside]{book}

% ==========================================
% BASIC PACKAGES
% ==========================================
\usepackage[utf8]{inputenc} % for UTF-8 encoding
\usepackage{fontspec} % for font selection
\setmainfont{Times New Roman} % set main font to Times New Roman
\usepackage[a4paper, left=4cm, right=3cm, top=3cm, bottom=3cm]{geometry} % set page margins
\usepackage[indonesian]{babel} % untuk bahasa Indonesia
\usepackage{csquotes} % for context-sensitive quotation facilities
\usepackage{setspace} % for line spacing
\onehalfspacing % spasi 1.5
\usepackage{graphicx} % for images
\usepackage{caption} % for customizing captions
\usepackage{subcaption} % for sub-figures
\usepackage{hyperref} % for hyperlinks
\usepackage{titlesec} % for customizing titles
\usepackage{tocloft} % for customizing table of contents
\usepackage{lipsum} % for dummy text (lorem ipsum text)
\usepackage{float} % for [H] float placement
\usepackage{listings} % for code listing
\usepackage{amsmath} % for math
\usepackage{amssymb} % for math symbols
\usepackage[shortlabels]{enumitem} % for customizing lists
\setlist[enumerate]{nosep, topsep=-10pt} % mengurangi spasi antar item dan atas bawah daftar
\setlist[itemize]{nosep, topsep=-10pt} % mengurangi spasi antar item dan atas bawah daftar
\usepackage[skip=12pt]{parskip}
\usepackage{longtable}
\usepackage{booktabs}
\usepackage{tabularx} % untuk tabularx environment
\usepackage{pdflscape} % untuk halaman landscape
\usepackage{adjustbox} % untuk adjustbox environment
\usepackage{tcolorbox} % untuk tcolorbox environment
\tcbuselibrary{listings,skins} % untuk listing dalam tcolorbox
\usepackage[bahasai]{datetime2}

% ==========================================
% PENGATURAN HYPHENATION (PEMENGGALAN KATA)
% ==========================================
\hyphenpenalty=10000        % Mencegah pemenggalan kata
\exhyphenpenalty=10000      % Mencegah pemenggalan setelah tanda hubung eksplisit
\tolerance=1000             % Toleransi untuk spasi antar kata
\emergencystretch=3em       % Ruang tambahan untuk menghindari overfull box
\sloppy                     % Mode sloppy untuk menghindari kata terpotong


\setcounter{tocdepth}{4} % kedalaman daftar isi sampai subsubbab
\setcounter{secnumdepth}{4} % kedalaman penomoran sampai subsubbab


% ==========================================
% SITASI DAN DAFTAR PUSTAKA (MENGGUNAKAN CHICAGO MANUAL OF STYLE)
% ==========================================
\usepackage[
    backend=biber,
    authordate,
    language=english,
    autolang=other
]{biblatex-chicago}

\addbibresource{daftar-pustaka.bib}

% ==========================================
% Ubah istilah bahasa Inggris di daftar pustaka ke Bahasa Indonesia
% ==========================================
\DefineBibliographyStrings{english}{
  and          = {dan},
  andothers    = {dkk.},
  editor       = {penyunting},
  editors      = {penyunting},
  translator   = {penerjemah},
  byeditor     = {disunting oleh},
  bytranslator = {diterjemahkan oleh},
  in           = {dalam},
  edition      = {edisi},
  pages        = {hal.},
  page         = {hal.},
  volume       = {vol.},
  number       = {no.},
  urlseen      = {diakses pada},
  url          = {tautan},
}

% ==========================================
% Pastikan \cite() menampilkan (Penulis Tahun)
% ==========================================
\let\oldcite\cite
\renewcommand{\cite}{\parencite}

% ==========================================
% Atur pemisah nama penulis agar lebih natural dalam Bahasa Indonesia
% ==========================================
\renewcommand*{\finalandcomma}{} % hilangkan koma sebelum 'dan'


% ==========================================
% TAMPILAN
% ==========================================
\hypersetup{
    colorlinks=true,
    linkcolor=black,
    citecolor=black,
    urlcolor=black
}

% -- No Header dan No Footer ---
\pagestyle{plain}

% --- Ubah nama daftar listing ke "DAFTAR KODE" ---
% --- Harus diletakkan sebelum \begin{document} ---
\renewcommand{\lstlistlistingname}{\centering\normalsize DAFTAR KODE} 
\renewcommand{\lstlistingname}{Kode}
\lstset{basicstyle=\ttfamily\footnotesize,breaklines=true}
%\captionsetup[lstlisting]{justification=raggedright,singlelinecheck=false}


\renewcommand \cftchapdotsep{4.5}
%\cftsetrmarg{4em}  % Ini diset agar judul subbab tidak mepet ke nomor halaman di daftar isi
% -- Atur indentasi judul bab, subbab, dan subsubbab di daftar isi ---
\setlength{\cftchapindent}{0em}
\setlength{\cftsecindent}{0em}
\setlength{\cftsubsecindent}{0em}
\setlength{\cftsubsubsecindent}{0em}

\renewcommand*{\DTMbahasaimonthname}[1]{%
  \ifcase#1 %
    \or Januari%  % 1st month
    \or Februari%  % 2nd month
    \or Maret%  % 3rd month
    \or April%  % 4th month
    \or Mei%  % 5th month
    \or Juni%  % 6th month
    \or Juli%  % 7th month
    \or Agustus%  % 8th month
    \or September%  % 9th month
    \or Oktober%  % 10th month
    \or November%  % 11th month
    \or Desember%  % 12th month
  \fi
}

% ==========================================
% AWAL DOKUMEN
% ==========================================
\begin{document}

% ==========================================
% HALAMAN JUDUL
% ==========================================
\begin{titlepage}
\begin{center}

    
    \vspace*{1cm}
    
      {\Large\bfseries EKSPERIMEN MULTI-MODEL DAN MULTI-PERSONA DENGAN PENDEKATAN \textit{SPEC-DRIVEN EXPERIMENT ORCHESTRATION} UNTUK MENGANALISIS DAMPAK PERSONA TERHADAP PENALARAN DAN \textit{HUMAN BIAS} PADA LARGE LANGUAGE MODEL}\\
     \vspace{3cm}

    {\Large \textbf{Proposal Tugas Akhir}}\\


    \vspace{2cm}
    
    
    {\large Oleh}\\[0.3cm]
    \textbf{
    {\large Abel Apriliani}\\
    {\large 18222008}
    }\\

    \vspace{2cm}
    
    \begin{figure}[h]
    \centering
    \includegraphics[width=0.2\textwidth]{image/ganesha.jpg}
    \end{figure}
    
    
    % \vspace{1cm}
    \vfill

    \textbf{
    {\large PROGRAM STUDI SISTEM DAN TEKNOLOGI INFORMASI}\\
    {\large SEKOLAH TEKNIK ELEKTRO DAN INFORMATIKA}\\
    {\large INSTITUT TEKNOLOGI BANDUNG}\\
    {\large \DTMlangsetup{showdayofmonth=false,showmonthname=true,showyear=true}\today}
    }
\end{center}
\end{titlepage}



% ==========================================
% LEMBAR PENGESAHAN 
% ==========================================
\newpage
\thispagestyle{empty}
\pagenumbering{gobble}
\begin{center}
  \textbf{\large LEMBAR PENGESAHAN}\\[0.8cm]
    
    {\Large\bfseries EKSPERIMEN MULTI-MODEL DAN MULTI-PERSONA DENGAN PENDEKATAN \textit{SPEC-DRIVEN EXPERIMENT ORCHESTRATION} UNTUK MENGANALISIS DAMPAK PERSONA TERHADAP PENALARAN DAN \textit{HUMAN BIAS} PADA LARGE LANGUAGE MODEL}\\
     \vspace{1cm}

  {\Large \textbf{Proposal Tugas Akhir}}\\

  \vspace{1cm}
    
  {\large Oleh}\\[0.3cm]
    \textbf{
    {\large Abel Apriliani}\\
    {\large 18222008}
  }\\
    
  \vspace{0.5cm}
 
  {\large Program Studi Sistem dan Teknologi Informasi}\\
  {\large Sekolah Teknik Elektro dan Informatika}\\
  {\large Institut Teknologi Bandung}\\

  \vspace{1cm}

  Proposal Tugas Akhir ini telah disetujui dan disahkan\\ 
  di Bandung, pada tanggal \today\\[0.8cm]

% ==========================================
% Versi 1 pembimbing (default)
% ==========================================
%	Pembimbing  \\[3cm]
%	Dr. Ir. John Doe, M.T.   \\[0.2cm]
%	NIP. 123456789 
% ==========================================

% ==========================================
% Jika ada 2 pembimbing TA, uncomment dan edit 
% tabular di bawah ini. Kemudian, comment out atau hapus
% bagian versi 1 pembimbing di atas.
% ==========================================

\begin{tabular}{p{7cm}p{7cm}}
   Pembimbing 1 & Pembimbing 2 \\[2.5cm]
   Dr. Eng. Ayu Purwarianti, S.T., M.T. & Dr. Alham Fikri Aji, S.T., M.Sc., Ph.D. \\[0.2cm]
   NIP.  197701272008012011
\end{tabular}

\end{center}



% -- Change page number style to roman ---
\pagenumbering{roman} 


% ==========================================
% DAFTAR ISI, TABEL, GAMBAR
% ==========================================
% --- DAFTAR ISI ---
\makeatletter
\renewcommand{\tableofcontents}{%
  \clearpage
  \thispagestyle{plain}% no header
  \begin{center}
    {\large\bfseries\MakeUppercase{\contentsname}\par}
  \end{center}
  \vskip 1em
  \@starttoc{toc}%
}
\makeatother

\newpage
\renewcommand{\cfttoctitlefont}{\hfill\large\bfseries\MakeUppercase}
\renewcommand{\cftaftertoctitle}{\hfill}
\tableofcontents
%\addcontentsline{toc}{chapter}{DAFTAR ISI}

% --- DAFTAR GAMBAR ---
\newpage
\renewcommand{\cftloftitlefont}{\hfill\large\bfseries\MakeUppercase}
\renewcommand{\cftafterloftitle}{\hfill}
\listoffigures
\addcontentsline{toc}{chapter}{DAFTAR GAMBAR}

% --- DAFTAR TABEL ---
\newpage
\renewcommand{\cftlottitlefont}{\hfill\large\bfseries\MakeUppercase}
\renewcommand{\cftafterlottitle}{\hfill}
\listoftables
\addcontentsline{toc}{chapter}{DAFTAR TABEL}

% --- DAFTAR LISTING (ALGORITMA, PSEUDOCODE, SOURCE CODE) ---
\newpage

\lstlistoflistings
\addcontentsline{toc}{chapter}{\lstlistlistingname}

% --- DAFTAR ISTILAH ---
\newpage
{\centering\large\bfseries DAFTAR ISTILAH\par}
\addcontentsline{toc}{chapter}{DAFTAR ISTILAH}
\vspace{1em}

\begin{longtable}{>{\raggedright}p{3.5cm} p{9cm} >{\centering\arraybackslash}p{1.5cm}}
\toprule
\textbf{Istilah} & \textbf{Deskripsi} & \textbf{Hal.} \\
\midrule
\endfirsthead

\toprule
\textbf{Istilah} & \textbf{Deskripsi} & \textbf{Hal.} \\
\midrule
\endhead

\midrule
\multicolumn{3}{r}{\textit{Berlanjut ke halaman berikutnya...}} \\
\endfoot

\bottomrule
\endlastfoot

API (\textit{Application Programming Interface}) & Antarmuka pemrograman yang digunakan untuk mengakses dan memanggil model bahasa secara terprogram dalam pelaksanaan eksperimen. & \pageref{term:api} \\

Arsitektur decoder-only & Arsitektur model bahasa yang menghasilkan token secara autoregresif, yaitu memprediksi token berikutnya berdasarkan rangkaian token sebelumnya. Seluruh model yang digunakan dalam penelitian ini termasuk dalam kategori ini. & \pageref{term:decoder-only} \\

Benchmark & Kumpulan tugas atau dataset terstandardisasi yang digunakan untuk mengevaluasi dan membandingkan kinerja model bahasa. & \pageref{term:benchmark} \\

Bias & Kecenderungan atau penyimpangan tertentu pada keluaran model yang tidak sepenuhnya disebabkan oleh isi pertanyaan, tetapi juga oleh cara model memersepsi pengguna atau konteks. & \pageref{term:bias} \\

\textit{Chain-of-thought} & Rangkaian penjelasan langkah penalaran yang dihasilkan model sebelum memberikan jawaban akhir. & \pageref{term:cot} \\

\textit{Evaluation pipeline} & Rangkaian proses terotomatisasi yang mengatur pemanggilan model, penerapan persona, eksekusi tugas, pencatatan respons, dan pengolahan hasil untuk keperluan analisis. & \pageref{term:pipeline} \\

GSM8K & Benchmark penalaran numerik yang terdiri atas soal cerita matematika tingkat sekolah dasar hingga menengah, digunakan untuk mengukur kemampuan penalaran langkah demi langkah. & \pageref{term:gsm8k} \\

\textit{Human bias} & Bias yang berkaitan dengan identitas atau karakteristik manusia, seperti gender, usia, latar belakang sosial, atau stereotip tertentu, yang tercermin dalam keluaran model. & \pageref{term:human-bias} \\

Large Language Model (LLM) & Model bahasa berskala besar yang dilatih pada korpus teks dalam jumlah besar dan mampu menyelesaikan berbagai tugas, seperti penalaran, tanya jawab, dan penyusunan teks. & \pageref{term:llm} \\

MMLU-Redux & Versi terkurasi dari benchmark \textit{Massive Multitask Language Understanding} yang mencakup berbagai topik pengetahuan umum dan tugas penalaran multi-topik. & \pageref{term:mmlu-redux} \\

Multi-model evaluation & Pengaturan eksperimen yang melibatkan lebih dari satu model bahasa untuk membandingkan perilaku, performa, dan sensitivitas masing-masing model. & \pageref{term:multi-model} \\

Multi-persona evaluation & Pengaturan eksperimen yang melibatkan beberapa persona untuk melihat bagaimana variasi persona memengaruhi keluaran model pada tugas yang sama. & \pageref{term:multi-persona} \\

Penalaran (\textit{reasoning}) & Proses penyusunan langkah pemikiran oleh model untuk menyelesaikan suatu tugas, misalnya penalaran numerik, logis, atau berbasis skenario sosial. & \pageref{term:reasoning} \\

Persona & Representasi identitas atau karakter pengguna yang dimasukkan ke dalam prompt, baik secara eksplisit maupun implisit, dan dapat memengaruhi cara model menyusun respons. & \pageref{term:persona} \\

Persona eksplisit & Persona yang dinyatakan secara langsung di dalam prompt, misalnya dengan menyebutkan latar belakang, profesi, atau karakter pengguna secara eksplisit. & \pageref{term:persona-eksplisit} \\

Persona implisit & Persona yang tersirat dari gaya bahasa, pilihan kata, tingkat formalitas, dan cara pengguna menyampaikan pertanyaan tanpa penyebutan identitas secara langsung. & \pageref{term:persona-implisit} \\

Prompt & Teks masukan yang diberikan kepada model untuk memicu dan mengarahkan respons yang dihasilkan. & \pageref{term:prompt} \\

Prompt-based evaluation & Pendekatan evaluasi yang menggunakan variasi prompt untuk menguji perilaku model tanpa melakukan pelatihan ulang atau perubahan parameter internal. & \pageref{term:prompt-based} \\

Robustness & Tingkat ketahanan model terhadap variasi prompt atau persona, yaitu sejauh mana model tetap memberikan respons yang konsisten pada tugas yang sama. & \pageref{term:robustness} \\

Sensitivitas model (\textit{sensitivity}) & Tingkat kepekaan model terhadap perubahan pada prompt atau persona, yang tercermin dari seberapa besar variasi respons yang dihasilkan. & \pageref{term:sensitivity} \\

\textit{Spec-driven experiment orchestration} & Pendekatan perancangan eksperimen yang berbasis pada dokumen spesifikasi formal. Spesifikasi tersebut mendefinisikan kombinasi model, persona, skenario prompt, dan benchmark yang dieksekusi secara otomatis melalui pipeline, sehingga evaluasi dapat dilakukan secara konsisten, terstruktur, dan dapat direproduksi. & \pageref{term:spec-driven} \\

\end{longtable}

\mainmatter
% --- FORMAT TAMPILAN JUDUL BAB, SUBBAB, JUDUL GAMBAR DAN TABEL ---
% --- Judul Bab ---
\titleformat{\chapter}[display]
      {\centering\normalfont\large\bfseries} % Commands for the entire chapter title
      {\MakeUppercase \chaptertitlename\ \thechapter}{0pt}{\large} % Chapter number format
\renewcommand\thechapter{\Roman{chapter}}
% --- Judul Subbab dan Subsubbab ---
\titleformat{\section}
	{\normalfont\bfseries}
	{\thesection}{1em}{}
\titleformat{\subsection}
	{\normalfont\bfseries}
	{\thesubsection}{1em}{}
\titleformat{\subsubsection}
	{\normalfont\bfseries}
	{\thesubsubsection}{1em}{}


% --- Format judul gambar dan tabel ---
\captionsetup[figure]{labelsep=space}
\captionsetup[table]{labelsep=space}
\captionsetup[lstlisting]{labelsep=space}

% --- Atur indentasi paragraf ---
\setlength{\parindent}{0pt}
% -- Change page number style to arabic ---
\pagenumbering{arabic} 

% ==========================================
% BAB I PENDAHULUAN
% ==========================================
\chapter{PENDAHULUAN}
\label{chap:pendahuluan}
% --- Latar Belakang ---
\section{Latar Belakang}

Kemajuan dalam pengembangan \textit{large language model} telah menghasilkan peningkatan kemampuan dalam pemrosesan bahasa alami, pemahaman konteks, serta penalaran. Model seperti GPT, LLaMA, Mistral, dan Gemini digunakan secara luas dalam sistem dialog, agen percakapan, dan berbagai aplikasi berbasis teks. Walaupun demikian, kinerja model sering kali menunjukkan variasi yang bergantung pada identitas atau karakteristik pengguna yang tersirat dalam instruksi atau konteks percakapan.

Berbagai penelitian menunjukkan bahwa \textit{large language model} bersifat sensitif terhadap informasi identitas pengguna yang disisipkan dalam percakapan. Penelitian mengenai bias penalaran implisit menunjukkan bahwa perubahan kecil pada identitas pengguna dapat mengubah penalaran model pada tugas yang tidak memiliki dimensi sosial eksplisit \parencite{gupta2024biasrunsdeep}. Selain itu, studi mengenai identitas dan percakapan mengidentifikasi bahwa informasi identitas yang disampaikan melalui gaya, framing, atau konteks dapat memengaruhi gaya bahasa, tingkat kehati-hatian, atau preferensi jawaban \parencite{tseng2024twotales}. Penelitian mengenai pemodelan pengguna juga menunjukkan bahwa variasi atribut pengguna—seperti usia, latar belakang profesional, atau kelompok sosial—dapat memengaruhi keluaran model dalam aspek penalaran, stabilitas respons, dan kecenderungan bias \parencite{naous2025userlm}.

Berdasarkan literatur tersebut, semakin jelas bahwa \textit{large language model} tidak hanya memproses konten instruksi, tetapi juga bereaksi terhadap identitas pengguna yang diberikan secara eksplisit maupun yang tersirat dalam konteks. Dengan demikian, pengaruh \textit{user persona} menjadi aspek penting untuk dipelajari. Penelitian ini berfokus pada dua bentuk \textit{user persona}, yaitu \textit{user persona} eksplisit yang diberikan melalui deskripsi identitas yang jelas, serta \textit{user persona} implisit yang muncul melalui framing, gaya penulisan, atau narasi kontekstual tanpa instruksi langsung mengenai identitas pengguna.

Walaupun berbagai penelitian telah menunjukkan adanya sensitivitas model terhadap identitas pengguna, sebagian besar studi hanya mengevaluasi satu atau dua model, cakupan persona yang terbatas, atau rentang tugas penalaran yang sempit. Belum tersedia kerangka evaluasi yang memungkinkan analisis sistematis mengenai bagaimana \textit{user persona} memengaruhi penalaran, perilaku keluaran, dan \textit{human bias} pada berbagai model secara bersamaan. Kondisi tersebut menimbulkan kebutuhan untuk merancang pendekatan evaluasi yang lebih menyeluruh dan terstruktur.

Penelitian ini disusun untuk mengevaluasi pengaruh \textit{user persona} eksplisit dan \textit{user persona} implisit terhadap penalaran, perilaku keluaran, dan kecenderungan \textit{human bias} pada berbagai \textit{large language model} melalui pendekatan \textit{multi model} dan \textit{multi persona}. Penelitian ini dimaksudkan untuk memberikan pemahaman empiris yang lebih komprehensif mengenai sensitivitas model terhadap identitas pengguna.


% --- Rumusan Masalah ---
\section{Rumusan Masalah}

Rumusan masalah berikut disusun berdasarkan kebutuhan untuk memahami bagaimana \textit{user persona} memengaruhi perilaku dan penalaran model bahasa. Penelitian sebelumnya menunjukkan bahwa identitas pengguna, baik yang diberikan secara eksplisit maupun implisit, dapat memengaruhi penalaran, kualitas keluaran, dan kecenderungan bias model \parencite{gupta2024biasrunsdeep, tseng2024twotales, naous2025userlm}. Namun, cakupan penelitian terdahulu masih terbatas pada sedikit model, sedikit persona, dan variasi tugas yang sempit.

Berdasarkan kondisi tersebut, rumusan masalah penelitian ini adalah sebagai berikut.

\begin{enumerate}
    \item Bagaimana pengaruh \textit{user persona} eksplisit dan \textit{user persona} implisit terhadap performa penalaran pada berbagai jenis tugas pada sejumlah \textit{large language model}.
    \item Bagaimana kedua jenis \textit{user persona} tersebut memengaruhi perilaku keluaran model pada skenario interaksi yang berbeda.
    \item Bagaimana pola \textit{human bias} muncul dan berubah sebagai akibat variasi \textit{user persona}.
    \item Sejauh mana sensitivitas terhadap \textit{user persona} berbeda pada berbagai \textit{large language model}, serta model mana yang menunjukkan tingkat \textit{robustness} yang lebih tinggi terhadap variasi tersebut.
\end{enumerate}



% --- Tujuan ---
\section{Tujuan Penelitian}

Tujuan penelitian ditetapkan untuk menjawab permasalahan yang telah dirumuskan. Penelitian ini diarahkan untuk menghasilkan pemahaman yang lebih komprehensif mengenai pengaruh \textit{user persona} terhadap perilaku model bahasa dalam tugas penalaran dan skenario percakapan. Secara khusus, penelitian ini bertujuan untuk:

\begin{enumerate}
    \item Menganalisis pengaruh \textit{user persona} eksplisit dan \textit{user persona} implisit terhadap performa penalaran pada sejumlah \textit{large language model}.
    \item Mengidentifikasi perubahan perilaku keluaran model yang diinduksi oleh variasi \textit{user persona} pada berbagai konteks.
    \item Menganalisis pola \textit{human bias} yang muncul akibat variasi \textit{user persona}.
    \item Menyusun perbandingan sensitivitas dan \textit{robustness} berbagai model terhadap variasi \textit{user persona}.
    \item Mengembangkan rancangan \textit{evaluation pipeline} yang memungkinkan pelaksanaan eksperimen \textit{multi model} dan \textit{multi persona} secara terotomatisasi.
\end{enumerate}

% --- Batasan Masalah ---
\section{Batasan Masalah}
Tuliskan batasan-batasan yang diambil dalam pelaksanaan tugas akhir. Batasan ini dapat dihindari (bersifat opsional, tidak perlu ada) jika topik atau judul tugas akhir dibuat cukup spesifik.

% --- Metodologi Pengerjaan TA ---
\section{Metodologi}
Tuliskan semua tahapan yang akan dilalui selama pelaksanaan tugas akhir. Tahapan ini spesifik untuk menyelesaikan persoalan tugas akhir. Khusus untuk penyusunan proposal ini, jelaskan secara detail:
\begin{enumerate}
\item	Tahapan investigasi pengumpulan fakta di latar belakang untuk merumuskan masalah.
\item	Langkah-langkah pencarian, pengelompokan, dan penapisan literatur atau sumber informasi untuk mengumpulkan informasi yang relevan tentang topik yang diangkat, termasuk teori (konsep atau teori apa saja yang perlu dicari), hal-hal yang telah dicapai oleh orang lain (cara mencari dan kata kuncinya), dan berbagai informasi pendukung, untuk mencari solusi terhadap masalah yang dibahas. Gunakan metodologi yang tepat dalam menggali informasi dan dokumentasikan prosesnya (termasuk rekaman wawancara atau survei) di dalam Lampiran, termasuk tautan ke video atau foto. Hasil penggalian informasi ini akan dijelaskan secara sistematis di Bab II Studi Literatur.
\end{enumerate}
% ==========================================
% BAB II STUDI LITERATUR
% ==========================================
\chapter{STUDI LITERATUR}

% -------------------------------------------------------------

\section{Large Language Models}

\textit{Large language models} merupakan fondasi utama dari sistem generatif modern yang digunakan dalam penelitian ini. Perkembangan model berskala besar seperti GPT-3 yang diperkenalkan oleh \textcite{brown2020language} menunjukkan bahwa peningkatan kapasitas model dan jumlah data pelatihan secara signifikan meningkatkan kemampuan representasi serta menghasilkan keluaran yang semakin selaras dengan instruksi dan lebih terstruktur pada berbagai tugas pemrosesan bahasa.

Pemahaman terhadap mekanisme internal \textit{large language models} menjadi penting dalam konteks penelitian ini karena perilaku persona dan variasi penalaran yang diamati merupakan konsekuensi langsung dari sifat probabilistik, struktur arsitektural, dan tujuan pelatihan model. Model bahasa tidak melakukan penalaran simbolik, melainkan mempelajari distribusi token dari data dan menghasilkan keluaran melalui estimasi probabilitas token berikutnya. Dengan demikian, fenomena seperti pergeseran gaya respons, koherensi argumen, atau sensitivitas terhadap persona berakar pada mekanisme internal tersebut.

Selain itu, model berskala besar membawa bias yang terdapat dalam data pelatihan. Analisis oleh \textcite{bender2021stochastic} menunjukkan bahwa data berukuran sangat besar yang tidak terkurasi dapat merepresentasikan ketidakseimbangan sosial, kultur, dan bahasa. Konsekuensinya, \textit{large language models} dapat menginternalisasi dan mereproduksi bias tersebut. Pemahaman mengenai dasar matematis dan arsitektural menjadi penting untuk menjelaskan bagaimana bias tersebut muncul serta bagaimana persona dapat mengubah pola keluaran model.

Subbagian berikut membahas dasar matematis dari \textit{autoregressive language modeling} sebagai komponen fundamental dari sebagian besar \textit{large language models}.

\subsection{Autoregressive Language Modeling}

\textit{Autoregressive language modeling} digunakan oleh sebagian besar \textit{large language models} untuk membentuk distribusi probabilitas atas urutan token melalui prediksi token berikutnya berdasarkan seluruh konteks sebelumnya. Pendekatan ini menyediakan kerangka matematis yang menjelaskan bagaimana keluaran model terbentuk, bagaimana representasi internal berubah ketika konteks dimodifikasi, serta bagaimana instruksi awal seperti persona dapat menghasilkan variasi pola respons.

\subsubsection{Formulasi Probabilistik dan Next-Token Prediction}

Pada \textit{autoregressive language modeling}, probabilitas urutan token $x_1, x_2, \dots, x_T$ difaktorisasi menjadi
\begin{equation}
p(x_1, x_2, \dots, x_T)
=
\prod_{t=1}^{T} p(x_t \mid x_{<t}).
\label{eq:autoregressive-factorization}
\end{equation}

Pendekatan ini diperkenalkan oleh \textcite{bengio2003nlm} dan menjadi fondasi bagi model bahasa berbasis jaringan saraf. Model menghasilkan distribusi token melalui mekanisme \textit{next-token prediction}, di mana setiap prediksi dibentuk dari representasi konteks dalam hidden state. Setiap hidden state merupakan hasil transformasi berulang dari embedding token sebelumnya, sehingga konteks awal seperti instruksi persona secara langsung menentukan bentuk representasi yang mengalir ke langkah-langkah berikutnya.

Distribusi token dihitung melalui fungsi softmax atas nilai logit yang dihasilkan oleh model. Karena fungsi softmax bersifat sensitif terhadap perbedaan kecil pada logit, perubahan kecil pada hidden state akibat instruksi persona dapat menghasilkan pergeseran yang signifikan dalam distribusi probabilitas token berikutnya. Dengan demikian, efek persona muncul sebagai fenomena matematis berupa perubahan representasi konteks yang memodulasi arah prediksi token.

\textcite{brown2020language} menunjukkan bahwa model berskala besar mampu menampilkan pola respons yang mengikuti struktur instruksi pengguna. Dalam \textit{instruction following}, model menghasilkan keluaran yang konsisten dengan pola instruksi dalam data pelatihan. Struktur respons yang mengikuti instruksi tercapai karena model mempelajari hubungan statistik antara bentuk perintah dan rentang respons yang berasosiasi dengannya.

Model juga menghasilkan rangkaian token yang tampak sebagai penjelasan berurutan ketika diberikan tugas tertentu. Pada \textit{contextual reasoning}, urutan token yang dihasilkan membentuk struktur langkah-langkah yang selaras dengan konteks sebelumnya. Struktur ini muncul dari kecocokan probabilistik antartoken dalam embedding space dan tidak bergantung pada mekanisme penalaran eksplisit. Token dipilih berdasarkan kedekatannya secara distribusional terhadap konteks, sehingga rangkaian yang terbentuk tampak menyerupai penalaran.

Efek persona terhadap distribusi token dapat diilustrasikan melalui pergeseran embedding cluster. Instruksi persona dengan gaya formal menghasilkan hidden state yang memberi skor logit lebih tinggi bagi token dengan register formal, sehingga token tersebut menjadi lebih mungkin muncul. Sebaliknya, persona santai menghasilkan distribusi yang memberi preferensi terhadap token informal. Pergeseran ini terjadi pada level representasi, bukan pada perubahan struktur arsitektural.

Selain itu, proses inferensi bersifat autoregresif dan tidak menggunakan token benar seperti pada pelatihan. Ketika model menghasilkan tokennya sendiri, distribusi prediksi dapat mengalami deviasi yang semakin besar seiring panjang urutan, sebuah ketidaksesuaian yang dikenal dengan istilah training–inference mismatch. Kondisi ini memperbesar sensitivitas terhadap konteks awal, sehingga pengaruh persona menjadi lebih menonjol.

\subsubsection{Cross-Entropy Loss dan Implikasi Pelatihan}

Model dilatih dengan mengoptimalkan \textit{cross-entropy loss}, yang mengukur seberapa baik distribusi prediksi model mendekati distribusi token benar dalam data. Objektif ini diformulasikan sebagai
\begin{equation}
\mathcal{L}
=
- \sum_{t=1}^{T} \log p_{\theta}(x_t \mid x_{<t}).
\label{eq:cross-entropy-loss}
\end{equation}

Sebagaimana dijelaskan oleh \textcite{goodfellow2016deep}, optimasi terhadap \textit{cross-entropy loss} mendorong model untuk menyesuaikan parameter sehingga meningkatkan probabilitas token yang benar. Pelatihan ini tidak dirancang untuk mengoptimalkan koherensi semantik atau struktur argumentatif, melainkan untuk meniru distribusi token dalam corpus pelatihan.

Konsekuensi penting dari pendekatan ini adalah terinternalisasinya bias distribusional yang terdapat dalam data pelatihan. \textcite{bender2021stochastic} menunjukkan bahwa corpus berskala besar sering kali memuat ketidakseimbangan representasi linguistik dan sosial. Karena model melakukan estimasi probabilitas berdasarkan pola distribusional tersebut, bias yang tertanam dalam data dapat muncul kembali dalam keluaran model.

Sensitivitas mekanisme autoregresif terhadap konteks awal memperkuat pengaruh persona. Instruksi persona yang muncul pada bagian awal masukan membentuk hidden state awal dan memodulasi jalur prediksi token, sehingga menghasilkan perbedaan konsisten dalam gaya argumentasi, tingkat ketegasan, dan struktur penjelasan meskipun instruksi utamanya sama. Fenomena ini menjadi landasan bagi penelitian ini dalam mengevaluasi bagaimana persona mempengaruhi keluaran model dan persepsi pengguna.

\subsection{Arsitektur Transformer}

Arsitektur Transformer merupakan dasar bagi sebagian besar \textit{large language models} modern. Arsitektur ini dirancang untuk memproses urutan secara efisien melalui mekanisme \textit{self-attention}, yang memungkinkan model membentuk representasi konteks secara global tanpa hambatan ketergantungan sekuensial. Mekanisme ini memiliki peran penting dalam menentukan bagaimana informasi mengalir di dalam model, bagaimana representasi konteks diperbarui di setiap lapisan, serta bagaimana instruksi persona memodulasi distribusi token selama proses prediksi. Secara ringkas, struktur komponen utama Transformer ditunjukkan pada Gambar~\ref{fig:transformer-architecture}.

\begin{figure}[H]
    \centering
    \includegraphics[width=0.7\textwidth]{image/arsitektur-transformer.png}
    \caption{Struktur umum arsitektur Transformer \parencite{vaswani2017attention}}
    \label{fig:transformer-architecture}
\end{figure}

Gambar~\ref{fig:transformer-architecture} menunjukkan aliran informasi melalui mekanisme \textit{attention}, \textit{multi-head integration}, \textit{positional encoding}, \textit{feed-forward block}, \textit{residual connection}, dan \textit{layer normalization} pada setiap lapisan.

\subsubsection{Mekanisme Self-Attention}

Mekanisme \textit{self-attention} menghitung hubungan antartoken melalui representasi query, key, dan value yang diproyeksikan dari token masukan. Formulasi ini dijelaskan oleh \textcite{vaswani2017attention} sebagai
\begin{equation}
\text{Attention}(Q, K, V)
=
\text{softmax}\left(
\frac{QK^{T}}{\sqrt{d_k}}
\right)V,
\label{eq:self-attention}
\end{equation}
di mana $d_k$ adalah dimensi key. Operasi ini memberikan bobot perhatian berdasarkan kecocokan distribusional antara query dan key. Bobot tersebut mengatur kontribusi value dalam pembentukan representasi token, sehingga token dengan relevansi lebih tinggi terhadap konteks akan memiliki pengaruh lebih besar.

Fungsi softmax yang digunakan pada perhatian sensitif terhadap variasi kecil pada nilai logit, sehingga perubahan kecil pada hidden state awal—seperti akibat instruksi persona—dapat menghasilkan perubahan nontrivial pada bobot perhatian. Dengan demikian, persona mempengaruhi jalur informasi sejak lapisan pertama dengan memodifikasi representasi konteks yang digunakan untuk membangun distribusi token selanjutnya.

\subsubsection{Multi-Head Attention dan Layer-Wise Representations}

Komponen \textit{multi-head attention} memperluas mekanisme perhatian dengan memproses beberapa proyeksi query, key, dan value secara paralel. Setiap head mempelajari pola ketergantungan yang berbeda dalam urutan, seperti hubungan sintaktis, asosiasi semantik, koherensi wacana, atau struktur respons yang berulang dalam data pelatihan.

Representasi yang dihasilkan oleh setiap head kemudian digabungkan untuk membentuk layer-wise representations, yaitu representasi token yang diperbarui di setiap lapisan berdasarkan kombinasi informasi dari seluruh head. Lapisan-lapisan Transformer menyusun hierarchical representations yang semakin kaya, karena representasi pada lapisan berikutnya memanfaatkan konteks yang telah diperkaya oleh lapisan sebelumnya.

Dalam konteks persona, modifikasi kecil pada hidden state awal dapat memengaruhi sensitivitas head tertentu terhadap pola bahasa tertentu. Sebagai contoh, persona formal dapat memperkuat kontribusi head yang secara statistik lebih sering terkait dengan struktur kalimat baku, sedangkan persona santai dapat menggeser perhatian ke pola yang lebih percakapan. Efek ini terpropagasi sepanjang lapisan dan berdampak langsung pada distribusi logit yang menentukan token berikutnya.

\subsubsection{Positional Encoding dan Bias Struktural}

Karena \textit{self-attention} tidak mengandung informasi posisi secara inheren, Transformer menggunakan \textit{positional encoding} untuk menggabungkan informasi urutan ke dalam representasi token. Encoding ini memastikan bahwa model dapat membedakan token berdasarkan posisinya, yang penting untuk menjaga struktur urutan bahasa.

Kajian oleh \textcite{liu2024lost} menunjukkan bahwa penggunaan \textit{positional encoding} dan struktur perhatian menyebabkan beberapa bias struktural, termasuk:
\begin{itemize}
    \item \textit{recency bias}, yaitu kecenderungan model memberi bobot perhatian lebih besar pada token yang muncul di akhir konteks,
    \item \textit{positional bias}, yakni sensitivitas yang berbeda terhadap token di posisi tertentu,
    \item penurunan pemanfaatan informasi pada bagian tengah urutan (\textit{lost in the middle}).
\end{itemize}

Bias ini relevan terhadap fenomena persona karena instruksi persona biasanya berada di awal konteks. Representasi awal tersebut tetap berpengaruh kuat terhadap representasi selanjutnya meskipun bagian lain dari urutan berada lebih jauh.




\subsubsection{Feed-Forward Networks, Residual Connection, dan Layer Normalization}

Setiap lapisan Transformer mencakup feed-forward block yang menerapkan transformasi nonlinier pada setiap representasi token. Komponen ini memperkaya representasi dengan menambahkan nonlinieritas dan meningkatkan kapasitas model untuk mempelajari pola yang lebih kompleks.

Residual connection memungkinkan informasi dari lapisan sebelumnya tetap dipertahankan dan membantu stabilitas propagasi sinyal di sepanjang jaringan. Layer normalization menjaga distribusi aktivasi tetap stabil selama pelatihan, sehingga setiap lapisan dapat membentuk representasi token yang konsisten dan dapat diprediksi.

Interaksi antara feed-forward block, residual connection, dan layer normalization membentuk hierarchical representations yang digunakan untuk menghitung skor logit pada setiap langkah prediksi token. Modifikasi kecil pada representasi awal—misalnya akibat persona—akan terpropagasi melalui seluruh lapisan dan menghasilkan pola respons yang konsisten dengan karakter persona tersebut.

\subsection{Pelatihan dan Inferensi}

Proses pelatihan dan inferensi pada \textit{large language models} memiliki perbedaan mendasar dalam distribusi konteks yang digunakan untuk menghasilkan prediksi. Perbedaan ini menentukan stabilitas keluaran, sensitivitas terhadap instruksi awal, serta konsistensi struktur respons. Memahami dinamika ini penting untuk menjelaskan bagaimana persona dapat memengaruhi pola prediksi model.

\subsubsection{Teacher Forcing dan Exposure Bias}

Selama pelatihan, model menggunakan token benar dari data sebagai konteks pada setiap langkah prediksi melalui prosedur \textit{teacher forcing}. Distribusi yang dipelajari model pada langkah ke-$t$ didasarkan pada probabilitas
\begin{equation}
p_{\theta}(x_t \mid x_{<t}),
\end{equation}
yang dihitung dengan mengondisikan representasi pada token yang benar. Prosedur ini mempercepat konvergensi tetapi menciptakan ketergantungan kuat terhadap distribusi konteks yang tidak muncul pada saat inferensi.

Berbeda dari pelatihan, pada proses inferensi model tidak lagi menerima token benar; model menggunakan token yang dihasilkannya sendiri sebagai konteks berikutnya. Ketidaksesuaian antara kondisi pelatihan dan inferensi ini menimbulkan \textit{exposure bias}, yaitu akumulasi deviasi akibat kesalahan kecil pada tahap awal. Akumulasi ini memperkuat pengaruh konteks awal, termasuk instruksi persona, karena representasi yang terbentuk pada awal urutan digunakan berulang kali pada langkah-langkah selanjutnya.

\subsubsection{Training–Inference Mismatch}

Optimasi selama pelatihan dilakukan dengan meminimalkan \textit{cross-entropy loss}:
\begin{equation}
\mathcal{L}
=
- \sum_{t=1}^{T} \log p_{\theta}(x_t \mid x_{<t}),
\end{equation}
yang mengukur kecocokan model terhadap distribusi token benar. Namun distribusi yang digunakan pada inferensi adalah distribusi yang dibentuk oleh token prediksi model sendiri. Karena token prediksi tersebut dapat menyimpang dari token benar, model bekerja pada konteks yang secara statistik berbeda dari konteks yang digunakan untuk melatihnya. Ketidaksesuaian ini membuat keluaran model sangat sensitif terhadap variasi kecil pada konteks awal, termasuk modifikasi representasi akibat persona.

\subsubsection{Strategi Decoding dan Dampaknya pada Pola Keluaran}

Inferensi memerlukan pemilihan token dari distribusi probabilitas yang dihitung oleh model. Pemilihan ini ditentukan oleh strategi decoding, yang memainkan peran signifikan dalam membentuk struktur dan koherensi keluaran.

Pendekatan deterministik seperti \textit{greedy decoding} memilih token dengan probabilitas tertinggi pada setiap langkah, menghasilkan respons yang stabil namun kurang variatif. Metode pencarian seperti \textit{beam search} mengevaluasi beberapa kandidat urutan sekaligus sehingga meningkatkan koherensi, meskipun sensitivitas terhadap konteks awal tetap tinggi. Pendekatan berbasis sampling, seperti \textit{top-$k$ sampling} atau \textit{nucleus sampling}, memilih token dari distribusi terpotong dan menghasilkan variasi respons yang lebih besar.

Perbedaan strategi ini memengaruhi struktur keluaran yang tampak seperti reasoning. Respons yang tampak lebih linier dan terkontrol sering muncul pada decoding deterministik, sementara respons yang lebih bervariasi muncul pada pendekatan sampling. Kedua pola tersebut merupakan hasil dinamika probabilistik, bukan hasil dari mekanisme penalaran eksplisit.

\subsubsection{Implikasi Terhadap Pengaruh Persona}

Dinamika pelatihan dan inferensi serta variasi strategi decoding menjelaskan mengapa persona memiliki pengaruh kuat terhadap keluaran model. Instruksi persona ditempatkan pada awal konteks dan langsung membentuk representasi awal pada hidden state. Ketika model memasuki tahap inferensi dan menggunakan keluarannya sendiri sebagai konteks, perbedaan kecil dalam representasi awal akibat persona dapat terakumulasi dan menghasilkan variasi respons yang konsisten. Strategi decoding berbasis sampling memperbesar variasi tersebut, sedangkan pendekatan deterministik memperkuat konsistensi gaya persona tetapi dengan rentang ekspresi yang lebih sempit. Fenomena ini menjadi dasar penjelasan bagaimana persona dapat menghasilkan keluaran yang berbeda secara sistematis meskipun instruksi utama tidak berubah.

\subsection{Sumber Bias dalam LLM}

Bias pada \textit{large language models} muncul sebagai konsekuensi dari proses pelatihan berbasis data skala besar, struktur arsitektural model, tujuan optimasi, dan prosedur penyelarasan. Bias tersebut tidak muncul secara eksplisit sebagai keputusan, melainkan sebagai hasil dari pemodelan distribusi data yang tidak seimbang atau prosedur pelatihan yang tidak simetris. Pemahaman mengenai sumber bias ini penting untuk menjelaskan bagaimana model membentuk pola keluaran tertentu dan bagaimana instruksi persona dapat memperkuat atau memodulasi kecenderungan tersebut.

\subsubsection{Bias Berbasis Data}

Corpus pelatihan untuk LLM sering kali mencakup miliaran token yang dikumpulkan dari berbagai sumber daring. \textcite{bender2021stochastic} menekankan bahwa data semacam ini tidak terhindarkan dari ketidakseimbangan representasi sosial, linguistik, maupun kultural. Ketidakseimbangan tersebut terekam dalam distribusi token, sehingga model mempelajari korelasi yang merefleksikan bias dalam data.

Karena model mengoptimalkan kecocokan terhadap distribusi tersebut, token atau pola yang sering muncul dalam corpus memiliki probabilitas lebih tinggi untuk diprediksi. Akibatnya, perbedaan gaya, perspektif, atau struktur diskursus yang dominan dalam corpus dapat muncul kembali dalam keluaran model. Ketika persona diperkenalkan, persona tersebut berinteraksi dengan bias data karena representasi awalnya memodulasi kecenderungan yang sudah tertanam dalam distribusi model.

\subsubsection{Bias Struktural}

Arsitektur Transformer secara inheren memiliki struktur yang memengaruhi pola keluaran model. Kajian oleh \textcite{liu2024lost} menunjukkan adanya \textit{recency bias}, yaitu kecenderungan model memberikan perhatian lebih besar pada token yang muncul di bagian akhir konteks. Selain itu, adanya \textit{positional encoding} dan struktur perhatian yang terdistribusi menghasilkan \textit{positional bias}, yaitu sensitivitas yang berbeda terhadap token berdasarkan posisinya.

Fenomena \textit{lost in the middle} juga menjadi salah satu bentuk bias struktural. Model menunjukkan penurunan performa dalam memanfaatkan informasi yang berada pada posisi tengah dalam urutan panjang. Bias struktural ini dapat berinteraksi dengan persona: instruksi persona pada awal konteks membentuk representasi awal yang stabil, sementara bagian tengah konteks dapat tereduksi kontribusinya. Dengan demikian, persona memperoleh pengaruh kuat dalam jalur prediksi model.

\subsubsection{Bias Objektif Pelatihan}

Proses pelatihan LLM berfokus pada optimasi \textit{cross-entropy loss}, yang mendorong model untuk memprediksi token yang paling konsisten dengan distribusi corpus. Tujuan optimasi ini tidak mempertimbangkan kebenaran faktual, keadilan representasional, atau keseimbangan perspektif. Fokus tunggal pada kecocokan distribusi membuat model mereplikasi pola teks yang paling sering muncul, termasuk bias distribusional yang tidak disengaja.

Selain itu, karena pelatihan dilakukan dengan prosedur \textit{teacher forcing}, distribusi konteks selama pelatihan berbeda dari kondisi inferensi. Ketidaksesuaian tersebut dapat memperkuat pola bias, terutama jika token yang diprediksi pada tahap awal menyebabkan pergeseran representasi yang terpropagasi sepanjang urutan. Persona yang ditempatkan di awal konteks dapat memperbesar efek ini karena modifikasi representasi awal akan terakumulasi melalui mekanisme autoregresif.

\subsubsection{Bias Alignment}

Prosedur penyelarasan model dengan instruksi manusia, seperti \textit{reinforcement learning from human feedback} (RLHF), memperkenalkan bias tambahan. \textcite{ouyang2022training} menunjukkan bahwa preferensi anotator manusia membentuk pola respons tertentu yang dianggap lebih sesuai. Proses ini tidak netral, karena distribusi preferensi anotator dapat mencerminkan bias budaya, gaya komunikasi, atau norma tertentu.

Selama penyelarasan, model mempelajari pola respons yang diberi skor lebih tinggi oleh anotator, sehingga memperkuat gaya tertentu dan melemahkan gaya yang lain. Interaksi antara alignment bias dan persona menjadi relevan karena persona yang berusaha meniru gaya tertentu dapat bertemu dengan bias bawaan dari prosedur RLHF, menghasilkan respons yang berbeda dari persona yang secara nominal memiliki instruksi sama.

Secara keseluruhan, kombinasi bias data, bias struktural, bias objektif pelatihan, dan bias alignment membentuk spektrum kecenderungan dalam model. Persona tidak menciptakan bias baru, tetapi memodulasi bias yang sudah ada melalui perubahan representasi awal yang digunakan dalam proses prediksi token.

%==== 2.2.1

\section{Persona sebagai Konstruksi Linguistik dalam Interaksi LLM}

Persona dalam konteks \textit{large language models} tidak merujuk pada sifat psikologis atau karakter manusia, tetapi pada instruksi linguistik yang ditempatkan di awal konteks untuk mengarahkan model menghasilkan respons dengan gaya, register, atau struktur tertentu. Instruksi tersebut berfungsi sebagai sinyal kondisional yang memodulasi representasi awal dalam hidden state, sehingga memengaruhi jalur prediksi token selama proses autoregresif.

Penelitian mengenai \textit{prompt-based conditioning} menunjukkan bahwa perubahan kecil pada formulasi konteks dapat menghasilkan keluaran yang berbeda secara signifikan \parencite{schick2021pet, zhao2021calibration}. Model bahasa merespons pola instruksi berdasarkan distribusi representasi yang telah dipelajari selama pra-pelatihan dan penyelarasan. Oleh karena itu, pemahaman tentang persona penting untuk memastikan bahwa penggunaan persona sebagai variabel eksperimen memiliki landasan teoretis yang jelas.

Subbagian berikut membahas definisi persona dalam model bahasa serta mekanisme teknis yang membuat persona mampu memodulasi keluaran model.

\subsection{Definisi Persona dalam Konteks Model Bahasa}

Persona dalam model bahasa dipahami sebagai instruksi linguistik yang membingkai cara model menghasilkan respons. Instruksi ini dapat berupa pernyataan peran (\textit{role prompt}), deskripsi gaya komunikasi, atau konteks mengenai karakteristik pengguna yang ditempatkan pada awal masukan. Instruksi tersebut memengaruhi representasi awal yang terbentuk melalui embedding token dan hidden state pertama, sehingga mengubah distribusi probabilitas token pada langkah-langkah berikutnya selama generasi.

Literatur mengenai \textit{prompt-based learning} menunjukkan bahwa model sangat sensitif terhadap struktur dan formulasi instruksi yang diberikan \parencite{schick2021pet}. Pola respons yang selaras dengan instruksi bukan merupakan bentuk penalaran laten, tetapi hasil dari kecocokan distribusional antara instruksi dan representasi yang dipelajari selama pra-pelatihan. Dengan kata lain, persona berperan sebagai sinyal yang menggeser distribusi prediktif model tanpa membangun struktur kepribadian atau preferensi yang stabil.

Penelitian mengenai kalibrasi konteks menunjukkan bahwa bahkan variasi kecil dalam deskripsi atau framing dapat menghasilkan perbedaan yang berarti dalam keluaran model \parencite{zhao2021calibration}. Instruksi seperti “You are a formal academic assistant” dan “Your user is a university student” bekerja melalui mekanisme yang sama: keduanya memodifikasi representasi awal, yang kemudian menentukan pola prediksi token sepanjang proses autoregresif. Dengan demikian, persona dipandang sebagai alat linguistik yang mempengaruhi keluaran melalui perubahan kondisi awal, bukan sebagai entitas dengan perilaku internal.

\subsection{Mekanisme Persona dalam Model Autoregresif}

Efek persona dalam model autoregresif muncul dari cara model membentuk representasi konteks pada langkah awal inferensi. Instruksi persona dimasukkan sebagai token pertama dan diproses melalui embedding layer serta lapisan awal Transformer. Proses ini menghasilkan hidden state awal yang digunakan untuk menghitung distribusi probabilitas token pertama, dan hidden state tersebut menjadi dasar bagi seluruh langkah prediksi berikutnya. Perubahan kecil pada representasi awal dapat menghasilkan perbedaan signifikan karena sifat propagatif dari mekanisme autoregresif.

Mekanisme \textit{self-attention} memperkuat efek ini. Setiap token dalam instruksi persona berkontribusi pada perhitungan perhatian melalui operasi $\text{softmax}\!\left(\frac{QK^{T}}{\sqrt{d_k}}\right)$, sehingga memodulasi bobot yang menentukan bagaimana representasi konteks dibangun pada setiap lapisan. Perubahan distribusi perhatian tersebut menyebabkan pergeseran representasi yang memengaruhi logit pada seluruh langkah generasi. Hasilnya, persona tidak hanya mengubah gaya bahasa, tetapi juga memengaruhi struktur respons dan pola penjelasan yang dihasilkan model.

Konsistensi efek persona diperkuat oleh sifat autoregresif model: token yang dihasilkan pada langkah awal menjadi bagian dari konteks untuk langkah berikutnya. Fenomena ini sejalan dengan \textit{training–inference mismatch} dan \textit{exposure bias}, di mana deviasi kecil pada konteks awal diperkuat sepanjang urutan \parencite{liu2023pretrain}. Akibatnya, persona dapat menghasilkan pergeseran sistematis dalam keluaran meskipun instruksi tugas tetap sama.

Temuan empiris mengenai teknik prompting, seperti \textit{chain-of-thought prompting} \parencite{wei2022chain}, menunjukkan bahwa modifikasi konteks awal berdampak langsung pada struktur penjelasan dan pola reasoning yang ditampilkan model. Hal ini mendukung pemahaman bahwa persona adalah sinyal kondisional yang bekerja melalui mekanisme representasi awal dan propagasi token dalam model autoregresif. Oleh karena itu, analisis mekanisme ini menjadi dasar penting bagi penggunaan persona dalam penelitian ini.

\subsection{Klasifikasi Persona dalam Literatur}

Penelitian mengenai persona pada \textit{large language models} menunjukkan bahwa pemberian identitas atau gaya pengguna pada konteks awal dapat menggeser pola penalaran, struktur respons, serta tingkat kehati-hatian model \parencite{gupta2024biasrunsdeep}. Survei komprehensif mengenai persona dan personalisasi pada LLM membedakan antara persona di sisi pengguna, persona yang menetapkan peran pada model, dan skema personalisasi jangka panjang \parencite{tseng2024twotales}. Berdasarkan batasan metodologis penelitian ini, hanya persona di sisi pengguna yang digunakan, yaitu persona eksplisit, persona implisit, dan persona netral sebagai baseline.

\subsubsection{Persona eksplisit}

Persona eksplisit menyatakan identitas pengguna secara langsung melalui deskripsi identitas pada system message. Identitas dirumuskan berdasarkan beberapa dimensi yang relevan seperti gender, rentang usia, agama, pekerjaan, kewarganegaraan, dan register bahasa \parencite{gupta2024biasrunsdeep}. Dalam penelitian ini, persona eksplisit direalisasikan melalui deskripsi identitas yang ditempatkan pada awal konteks, tanpa memberikan informasi tambahan terkait jawaban atau strategi penyelesaian soal. Format teknis seperti “Your user is …” digunakan sebagai implementasi praktis dari \textit{identity descriptor} yang dijelaskan pada penelitian persona-assigned models, meskipun tidak berasal dari satu paper tertentu. Deskripsi identitas ini bekerja sebagai sinyal kondisional yang memodulasi representasi awal sehingga mempengaruhi struktur penjelasan atau langkah-langkah reasoning yang dipilih model.

\subsubsection{Persona implisit melalui gaya tutur}

Persona implisit tidak menyebutkan identitas pengguna secara eksplisit, tetapi dibentuk melalui narasi pengalaman, pilihan diksi, atau gaya tutur pada masukan pengguna. Survei \textcite{tseng2024twotales} menunjukkan bahwa LLM dapat menginferensi persona dari isyarat linguistik semacam ini, sehingga gaya tutur dapat berfungsi sebagai bentuk persona tersirat. Penelitian mengenai sensitivitas model terhadap framing prompt menemukan bahwa variasi kecil dalam formulasi bahasa dapat menghasilkan perbedaan yang sistematis dalam gaya respons atau tingkat perincian penjelasan \parencite{zhou2023largemodelsensitive, zhao2021calibrate}. Dalam penelitian ini, persona implisit diberikan melalui paragraf naratif yang merepresentasikan sudut pandang pengguna. Representasi tersirat ini mendorong model untuk menyesuaikan register dan pola penalaran berdasarkan interpretasinya terhadap karakter pengguna yang muncul dari gaya bahasanya.

\subsubsection{Persona netral}

Persona netral digunakan sebagai baseline ketika tidak ada sinyal identitas atau gaya tambahan yang diberikan. Pada kondisi ini, system message hanya berfokus pada instruksi tugas tanpa menyebutkan gender, usia, pekerjaan, atau atribut sosial lain. Baseline diperlukan untuk memisahkan efek persona eksplisit dan implisit dari variasi yang mungkin muncul akibat struktur instruksi atau noise dalam proses decoding. Studi mengenai ketidaksetiaan penjelasan model pada reasoning \parencite{turpin2023language} menekankan pentingnya baseline yang jelas ketika mengevaluasi pergeseran pola reasoning, sehingga persona netral menjadi komponen metodologis penting dalam penelitian ini.

Ruang lingkup penelitian ini tidak mencakup role-playing persona yang menetapkan identitas tertentu pada model, maupun pendekatan personalisasi jangka panjang yang melibatkan penyimpanan profil pengguna. Survei \textcite{tseng2024twotales} serta kajian risiko etis pada model bahasa \parencite{weidinger2021ethical, bommasani2021opportunities} menunjukkan bahwa personalisasi jangka panjang dan role-playing persona membawa implikasi metodologis serta risiko bias yang berbeda dari persona berbasis konteks linguistik yang digunakan dalam penelitian ini.

\subsection{Persona sebagai Variabel Eksperimental dalam Penelitian Ini}

Dalam penelitian ini, persona diperlakukan sebagai variabel eksperimen yang memodulasi kondisi awal pada proses generasi token tanpa mengubah isi soal, struktur tugas, atau informasi kunci yang dibutuhkan untuk menjawab pertanyaan. Persona hanya memengaruhi framing identitas pengguna dan gaya komunikasi, sehingga setiap variasi keluaran dapat dikaitkan secara langsung dengan perbedaan konteks linguistik.

\subsubsection{Konfigurasi persona dan dimensi identitas}

Persona disusun berdasarkan enam dimensi identitas yang muncul dalam penelitian sebelumnya, yaitu gender, rentang usia, agama, pekerjaan, kewarganegaraan, dan register bahasa \parencite{gupta2024biasrunsdeep}. Dimensi tersebut digunakan untuk membentuk himpunan persona eksplisit dan implisit yang konsisten, terstruktur, dan dapat direplikasi. Pada persona eksplisit, seluruh dimensi dituliskan secara langsung dalam system message sebagai deskripsi identitas. Pada persona implisit, dimensi tersebut direpresentasikan secara tersirat melalui narasi pengguna sehingga model perlu menginferensikannya dari gaya tutur \parencite{tseng2024twotales}.

\subsubsection{Integrasi persona dalam pipeline eksperimen}

Penerapan persona dilakukan melalui dua tahap, yaitu persona context initialization dan persona warm-up message. Tahap pertama membentuk konteks identitas melalui system message. Tahap kedua berupa satu interaksi pemanasan yang digunakan untuk memastikan bahwa model mengikuti gaya persona sebelum mengerjakan soal. Pendekatan kalibrasi konteks ini sejalan dengan temuan bahwa performa few-shot LLM sangat sensitif terhadap formulasi instruksi dan framing awal \parencite{zhao2021calibrate, liu2023pretrain}.

Setelah kalibrasi, seluruh soal dalam benchmark dijalankan dalam kondisi persona yang sama. Pelaksanaan kombinasi persona–model–benchmark diatur melalui pendekatan \textit{spec-driven experiment orchestration}, yaitu eksperimen yang disusun terlebih dahulu di dalam berkas spesifikasi formal sebelum dijalankan secara otomatis oleh \textit{pipeline}. Penelitian ini menggunakan GSM8K untuk menguji penalaran aritmetika berbasis soal cerita \parencite{cobbe2021gsm8k}, serta MMLU-Redux 2.0 untuk mengevaluasi kemampuan penalaran multi-bidang \parencite{hendryckstest2021, gema2024arewedonewithmmlu, mmluRedux2024dataset}.

\subsubsection{Kontrol struktur prompt dan pengaruh framing}

Struktur prompt dibuat seragam pada seluruh model dan seluruh persona agar variabel yang berubah hanyalah konteks identitas dan gaya tutur. Instruksi tugas tidak diubah dan berada dalam format yang sama, sedangkan bagian yang bervariasi hanya system message untuk persona eksplisit dan gaya tutur pada masukan pengguna untuk persona implisit. Penelitian mengenai sensitivitas model terhadap framing \parencite{zhou2023largemodelsensitive} menegaskan bahwa desain prompt harus dikontrol ketat untuk memastikan bahwa perbedaan keluaran memang berasal dari persona, bukan variasi teknis lain.

Dengan desain ini, persona berfungsi sebagai faktor kondisional yang memengaruhi representasi awal pada hidden state, sesuai dengan mekanisme autoregresif yang dijelaskan pada Subbab~2.1. Variasi keluaran pada benchmark dapat dianalisis sebagai konsekuensi langsung dari perubahan konteks linguistik di awal interaksi, bukan dari modifikasi parameter model atau struktur tugas.

\subsection{Efek Persona terhadap Keluaran Model}

Persona berfungsi sebagai sinyal kondisional yang membentuk representasi awal pada hidden state, sehingga memodulasi jalur prediksi token selama proses autoregresif. Efek persona muncul bukan karena model memiliki pemahaman tentang identitas pengguna, tetapi karena model menafsirkan deskripsi identitas atau gaya tutur sebagai bagian dari konteks linguistik yang mempengaruhi pembobotan perhatian, pemilihan token, dan struktur penjelasan. Penelitian mengenai reasoning bias pada persona-assigned models menunjukkan bahwa perubahan kecil dalam deskripsi identitas dapat menyebabkan pergeseran sistematis dalam pola penalaran \parencite{gupta2024biasrunsdeep}. Selain itu, sensitivitas LLM terhadap framing instruksi \parencite{zhou2023largemodelsensitive} dan pentingnya kalibrasi konteks awal \parencite{zhao2021calibrate} mendukung bahwa persona berpotensi mempengaruhi keluaran meskipun tugas yang diberikan tetap sama.

Efek persona dalam penelitian ini dianalisis melalui tiga mekanisme utama, yaitu pergeseran register dan gaya respons, perubahan struktur penjelasan, dan modifikasi jalur reasoning.

\subsubsection{Pergeseran register dan gaya respons}

Persona eksplisit dan implisit dapat mengubah register bahasa, tingkat formalitas, atau preferensi gaya penyampaian model. Variasi ini terjadi karena deskripsi identitas atau gaya tutur mempengaruhi distribusi representasi pada lapisan awal Transformer. Framing linguistik yang berbeda telah terbukti menghasilkan respons yang berbeda meskipun instruksi tugas sama \parencite{zhou2023largemodelsensitive}. Dengan demikian, persona dapat menyebabkan keluaran yang lebih formal, lebih ringkas, atau lebih naratif, meskipun jawaban yang benar tidak berubah. Pergeseran gaya ini penting untuk dianalisis agar tidak disalahartikan sebagai variasi kemampuan model.

\subsubsection{Perubahan struktur penjelasan}

Persona juga dapat memodulasi kecenderungan model untuk memberikan penjelasan panjang, ringkas, berhati-hati, atau langsung ke jawaban. Perubahan struktur penjelasan sejalan dengan temuan bahwa model dapat memberikan penjelasan yang terdengar rasional tetapi tidak selalu merefleksikan proses reasoning internal \parencite{turpin2023language}. Karena itu, persona yang mendorong gaya tertentu—seperti persona akademik atau persona yang berbicara santai—dapat mempengaruhi format penjelasan model tanpa mempengaruhi validitas langkah reasoning yang sebenarnya diperlukan untuk menyelesaikan soal.

\subsubsection{Modifikasi jalur reasoning}

Efek paling penting dari persona adalah pergeseran pada langkah-langkah reasoning yang dipilih model. Penelitian \textcite{gupta2024biasrunsdeep} menunjukkan bahwa deskripsi identitas dapat mengubah preferensi model terhadap pola reasoning tertentu, seperti memilih penjelasan yang lebih hati-hati, lebih sistematis, atau lebih cepat menuju jawaban. Dalam konteks penelitian ini, tugas penalaran pada GSM8K dan MMLU-Redux sangat sensitif terhadap perubahan konteks awal karena model mengakumulasi informasi secara autoregresif. Persona implisit melalui narasi gaya tutur juga dapat mendorong model menafsirkan situasi sosial atau emosi tertentu sebelum memulai reasoning, sebagaimana ditunjukkan oleh temuan mengenai inferensi persona tersirat \parencite{tseng2024twotales}. Hal ini dapat menggeser struktur reasoning meskipun konten soal identik.

\subsubsection{Implikasi terhadap evaluasi model}

Efek persona harus diperlakukan sebagai variabel eksperimental yang mempengaruhi keluaran model, bukan sebagai indikator perubahan kemampuan. Evaluasi holistik \parencite{liang2023helm} menekankan bahwa model harus diuji pada berbagai kondisi untuk memahami sensitivitasnya terhadap variasi konteks. Dengan demikian, analisis persona dalam penelitian ini tidak hanya mengevaluasi apakah model memperoleh jawaban yang benar, tetapi juga bagaimana perubahan framing identitas dan gaya tutur mempengaruhi keandalan reasoning, kestabilan respons, dan konsistensi performa lintas model.

Dengan kerangka ini, persona dipahami sebagai faktor linguistik yang menggeser dinamika prediksi token, sehingga mengubah jalur reasoning dan struktur respons tanpa mengubah akses model terhadap informasi atau kemampuan umum yang dimilikinya.

%======= 2.3
\section{Evaluasi Benchmark}

Benchmark digunakan sebagai instrumen evaluasi yang memberikan ukuran terstandarisasi terhadap kemampuan penalaran \textit{large language models}. Melalui benchmark, performa model dapat dibandingkan secara konsisten lintas persona, model, dan skenario instruksi. Evaluasi berbasis benchmark juga penting dalam konteks penelitian ini karena keluaran reasoning LLM dapat dipengaruhi oleh framing linguistik, termasuk variasi persona yang diberikan di awal konteks \parencite{zhou2023largemodelsensitive}. 

Selain itu, penelitian menunjukkan bahwa penjelasan yang dihasilkan model tidak selalu mencerminkan proses penalaran internal, tetapi dapat berupa penjelasan yang tidak setia (\textit{unfaithful}) terhadap mekanisme prediksi yang sebenarnya digunakan \parencite{turpin2023language}. Oleh karena itu, benchmark diperlukan untuk menyediakan dasar evaluasi yang objektif ketika menganalisis bagaimana persona memengaruhi struktur reasoning dan keputusan model.

Dalam penelitian ini digunakan dua benchmark yang saling melengkapi, yaitu GSM8K dan MMLU-Redux. Keduanya mewakili dua bentuk penalaran yang berbeda: penalaran numerik prosedural dan penalaran konseptual berbasis pengetahuan.

\subsection{GSM8K}

GSM8K merupakan benchmark untuk mengevaluasi \textit{numerical reasoning} melalui soal cerita matematika tingkat sekolah dasar \parencite{cobbe2021gsm8k}. Setiap soal membutuhkan identifikasi informasi penting, penyusunan langkah-langkah perhitungan yang logis, serta penarikan kesimpulan secara runtut. Meskipun sederhana bagi manusia, struktur reasoning ini menantang bagi LLM karena model harus menghasilkan urutan token yang menyerupai alur penyelesaian multi-langkah.

GSM8K relevan dalam konteks persona karena penalaran numerik yang bersifat prosedural terbukti sensitif terhadap variasi framing instruksi. Penelitian mengenai sensitivitas LLM terhadap perubahan gaya prompt menunjukkan bahwa perbedaan kecil dalam formulasi konteks dapat menggeser struktur langkah reasoning yang dihasilkan \parencite{zhou2023largemodelsensitive}. Hal ini memungkinkan persona memengaruhi panjang penjelasan, tingkat kehati-hatian, atau bentuk argumentasi yang ditampilkan model selama menyelesaikan soal numerik.

\subsection{MMLU-Redux}

MMLU-Redux adalah versi kurasi ulang dari benchmark MMLU yang mengevaluasi kemampuan penalaran konseptual dan pemahaman lintas disiplin \parencite{mmluRedux2024dataset}. Benchmark ini mencakup berbagai bidang seperti sains, humaniora, hukum, kedokteran, dan ilmu sosial. Berbeda dari GSM8K, tugas dalam MMLU-Redux disajikan dalam format \textit{multiple-choice}, sehingga model harus memilih jawaban yang paling tepat berdasarkan representasi pengetahuan dan pemahaman konsep.

Karena format evaluasi bersifat tertutup, MMLU-Redux memudahkan pengamatan terhadap pergeseran preferensi jawaban yang muncul akibat variasi persona. Sensitivitas terhadap framing telah dibahas dalam literatur \parencite{zhou2023largemodelsensitive}, sehingga perubahan gaya linguistik pada konteks awal dapat memengaruhi kecenderungan model dalam memilih opsi tertentu meskipun informasi faktual tidak berubah.

\subsection{Tantangan Evaluasi Berbasis Persona}

Evaluasi berbasis persona menghadapi beberapa tantangan metodologis. Tantangan pertama adalah memastikan bahwa perubahan respons disebabkan oleh persona, bukan karena variasi formulasi instruksi. Karena LLM sangat peka terhadap struktur prompt dan pilihan kata \parencite{zhao2021calibrate}, penelitian ini menggunakan format prompt yang sepenuhnya konsisten untuk seluruh eksperimen.

Tantangan kedua adalah variabilitas keluaran model. LLM dapat memberikan respons berbeda meskipun instruksi identik, terutama pada tugas yang melibatkan reasoning multi-langkah \parencite{turpin2023language}. Untuk mengurangi variabilitas tersebut, proses evaluasi diotomatisasi dan seluruh benchmark dijalankan dalam konfigurasi deterministik yang seragam.

Dengan demikian, GSM8K dan MMLU-Redux memberikan dua perspektif berbeda tentang bagaimana persona memengaruhi reasoning. GSM8K memperlihatkan efek persona pada struktur reasoning prosedural, sedangkan MMLU-Redux menunjukkan bagaimana framing identitas pengguna dapat menggeser preferensi jawaban dalam tugas konseptual. Kombinasi keduanya memberikan fondasi metodologis yang kuat untuk analisis pada Bab IV dan Bab V.

\section{Penelitian Terdahulu dan Kesenjangan Penelitian}

Penelitian mengenai persona pada \textit{large language models} menunjukkan bahwa variasi identitas pengguna, gaya tutur, dan framing instruksi dapat memengaruhi pola penalaran dan bentuk respons model. Kajian ini relevan dengan mekanisme internal LLM yang dijelaskan pada Subbab~2.1 dan sensitivitas model terhadap konteks awal yang dijelaskan pada Subbab~2.2. Meskipun demikian, literatur yang ada masih menyisakan sejumlah pertanyaan mendasar mengenai sejauh mana persona memengaruhi reasoning dalam struktur evaluasi yang terukur dan terstandarisasi.

\subsection{Ringkasan Literatur Terkait}

Gupta \parencite{gupta2024biasrunsdeep} menunjukkan bahwa persona eksplisit dapat menggeser langkah penalaran yang dihasilkan model, bahkan ketika isi tugas tetap sama. Temuan ini mengindikasikan bahwa identitas yang ditempatkan pada konteks awal tidak hanya memengaruhi gaya bahasa, tetapi juga struktur reasoning.

Tseng \parencite{tseng2024twotales} menegaskan bahwa persona tidak hanya muncul melalui deklarasi identitas, tetapi juga melalui pola bahasa yang implisit. Model dapat menafsirkan ciri pengguna dari pilihan kata dan narasi, kemudian menyesuaikan respons sesuai interpretasi tersebut. Kondisi ini konsisten dengan mekanisme pembentukan representasi awal yang dijelaskan pada Subbab~2.2.

Turpin \parencite{turpin2023language} menunjukkan bahwa reasoning yang dihasilkan LLM sering kali tidak stabil dan dapat berubah akibat variasi kecil dalam prompt. Penelitian ini memperkuat pemahaman bahwa penalaran model bukan proses simbolik, melainkan hasil dinamika distribusi token yang sangat sensitif terhadap konteks.

Selain itu, penelitian mengenai risiko bias menunjukkan bahwa LLM dapat memunculkan pola sosial yang tidak seimbang sebagai konsekuensi dari data pelatihan \parencite{weidinger2021ethical, bommasani2021opportunities}. Ketika persona tertentu diperkenalkan, bias yang sudah ada dapat teramplifikasi atau termodulasi.

Penelitian terkait sensitivitas prompt \parencite{zhou2023largemodelsensitive} dan kalibrasi konteks \parencite{zhao2021calibrate} juga menunjukkan bahwa framing linguistik di awal interaksi dapat mengubah respons model secara signifikan. Hasil ini memperkuat argumen bahwa persona, sebagai bentuk framing, dapat memengaruhi reasoning yang dihasilkan model.

\subsection{Keterbatasan Penelitian Sebelumnya}

Meskipun kontribusi penelitian terdahulu penting, beberapa keterbatasan masih terlihat jelas.

Pertama, sebagian besar penelitian persona hanya menguji sedikit model dan tidak melakukan analisis lintas-LLM. Hal ini menyebabkan sulitnya menggeneralisasi bagaimana pengaruh persona berbeda antarmodel.

Kedua, variasi persona yang digunakan pada studi sebelumnya sering kali terbatas pada beberapa contoh representatif, sehingga belum menangkap spektrum identitas pengguna yang lebih luas. Sebaliknya, penelitian ini menggunakan himpunan persona eksplisit dan implisit yang lebih beragam.

Ketiga, sedikit penelitian yang mengevaluasi persona dalam konteks benchmark reasoning yang terstandarisasi. Banyak studi berfokus pada dialog atau tugas generatif yang tidak memiliki jawaban benar salah, sehingga efek persona sulit diukur secara objektif.

Keempat, tidak semua penelitian memastikan konsistensi struktur prompt. Karena LLM sangat sensitif terhadap perubahan formulasi instruksi \parencite{zhou2023largemodelsensitive, zhao2021calibrate}, perbedaan kecil dalam prompt berpotensi mencemari hasil analisis persona.

Kelima, stabilitas reasoning jarang dievaluasi pada benchmark yang berbeda secara kognitif, misalnya penalaran numerik (GSM8K) versus penalaran konseptual (MMLU-Redux). Padahal, persona dapat berdampak berbeda pada tiap jenis tugas.

Keenam, kerangka evaluasi LLM yang umum digunakan—seperti HELM \parencite{liang2023helm}, LM Evaluation Harness, dan OpenAI Evals—belum dirancang untuk mengevaluasi pengaruh persona sebagai variabel eksperimen. Framework-framework tersebut berfokus pada pengujian model terhadap benchmark terstandarisasi dengan prompt yang statis, sehingga tidak mendukung integrasi persona eksplisit maupun implisit, \textit{warm-up} konteks, ataupun variasi framing identitas pengguna. Selain itu, kerangka tersebut tidak menyediakan mekanisme untuk mengeksekusi kombinasi \textit{multi model}~$\times$~\textit{multi persona}~$\times$~\textit{multi benchmark} secara otomatis serta tidak menyimpan keluaran lengkap yang diperlukan untuk menganalisis perubahan struktur penjelasan, gaya bahasa, atau pola \textit{bias} berbasis persona. Akibatnya, pendekatan evaluasi yang ada belum mampu menangani kebutuhan metodologis penelitian ini secara menyeluruh.

\subsection{Posisi dan Kontribusi Penelitian Ini}

Penelitian ini dirancang untuk mengisi kesenjangan tersebut melalui beberapa kontribusi utama.

Pertama, penelitian ini mengevaluasi beberapa model LLM secara paralel, sehingga memungkinkan analisis komparatif mengenai perbedaan sensitivitas persona antarmodel.

Kedua, penelitian ini menggunakan himpunan persona eksplisit dan implisit yang dirancang secara sistematis dan selaras dengan kerangka teoretis pada Subbab~2.2, sehingga variasi pengaruh persona dapat diamati secara lebih komprehensif.

Ketiga, penelitian ini menggunakan dua benchmark reasoning yang memiliki karakteristik kognitif berbeda—GSM8K untuk penalaran numerik prosedural dan MMLU-Redux untuk penalaran konseptual berbasis pilihan ganda. Pendekatan ini menyediakan analisis yang lebih kaya mengenai bagaimana persona memengaruhi bentuk reasoning yang berbeda.

Keempat, seluruh evaluasi dijalankan dalam \textit{pipeline} terotomatisasi dengan struktur prompt yang benar-benar seragam melalui pendekatan \textit{spec-driven experiment orchestration}, mengikuti rekomendasi penelitian mengenai sensitivitas prompt \parencite{zhou2023largemodelsensitive}. Hal ini memastikan bahwa variasi keluaran dapat ditelusuri secara jelas ke persona, bukan ke perbedaan instruksi.

Dengan demikian, penelitian ini tidak hanya mereplikasi studi tentang persona, tetapi memperluas ruang analisis melalui evaluasi lintas-model, lintas-persona, dan lintas-benchmark. Formulasi ini memberikan kontribusi empiris baru mengenai bagaimana identitas pengguna memengaruhi pola reasoning dalam \textit{large language models}.

% ============================================================================================
% BAB III ANALISIS MASALAH
% Pembagian subbab tidak rigid dan dapat bervariasi. Bab ini minimal berisi analisis kebutuhan
% fungsional dan nonfungsional, analisis berbagai alternatif solusi yang dapat ditawarkan, dan
% metode pemilihan solusi yang diusulkan.
% ============================================================================================
\chapter{ANALISIS MASALAH}
\label{chap:analisis-masalah}
\section{Analisis Kondisi Saat Ini}

Perkembangan \textit{large language model} (LLM) dalam beberapa tahun terakhir mendorong pemanfaatan model bahasa dalam berbagai konteks, mulai dari penjawab pertanyaan, agen percakapan, hingga sistem pendukung pengambilan keputusan \parencite{bommasani2021opportunities}. Seiring dengan meluasnya penggunaan tersebut, muncul kebutuhan untuk memahami bagaimana model bereaksi terhadap variasi identitas dan karakteristik pengguna, bukan hanya terhadap instruksi tugas. Hal ini berkaitan dengan cara model memproses konteks interaksi yang memuat informasi tentang siapa yang berinteraksi dengan model, dalam kapasitas apa, dan dengan gaya komunikasi seperti apa.

Penelitian mengenai persona pada LLM sejauh ini banyak berfokus pada pemberian identitas kepada model sebagai agen percakapan. Tseng et al.\ mengkaji berbagai pendekatan \textit{role-playing} dan \textit{personalization} yang umumnya memposisikan persona pada sisi model, misalnya melalui instruksi sistem yang mendeskripsikan karakter, gaya bicara, atau peran yang harus diambil oleh model \parencite{tseng2024twotales}. Pada pengaturan ini, model diminta untuk bertindak sebagai tenaga profesional, tokoh tertentu, atau asisten dengan gaya komunikasi spesifik, dan evaluasi dilakukan dengan melihat konsistensi gaya respons maupun kesesuaian perilaku dengan persona yang diberikan.

Di luar \textit{role-playing} tersebut, sejumlah studi menunjukkan bahwa penyisipan persona eksplisit dapat memengaruhi penalaran model bahkan pada tugas yang dirancang sebagai soal penalaran abstrak dan tidak secara eksplisit memuat dimensi sosial. Gupta et al.\ menunjukkan bahwa identitas yang dilekatkan pada konteks dapat menggeser cara model melakukan penalaran dan memilih jawaban, termasuk pada soal yang dirancang untuk menguji penalaran formal \parencite{gupta2024biasrunsdeep}. Temuan ini mengindikasikan bahwa persona tidak hanya memengaruhi gaya bahasa, tetapi juga struktur langkah penalaran yang dihasilkan model.

Pada saat yang sama, struktur penalaran LLM terbukti sensitif terhadap variasi kecil pada instruksi. Turpin et al.\ memperlihatkan bahwa perubahan ringan dalam formulasi \textit{prompt} dapat menghasilkan rantai penalaran yang berbeda meskipun pertanyaannya sama \parencite{turpin2023language}. Studi lain mengenai sensitivitas model terhadap framing dan gaya penulisan menunjukkan bahwa cara sebuah instruksi disusun dapat memengaruhi isi dan gaya jawaban \parencite{zhou2023largemodelsensitive}. Kondisi ini membuat analisis persona menjadi lebih kompleks, karena persona, framing, dan gaya bahasa sering kali hadir secara bersamaan di dalam konteks interaksi, sehingga sulit memisahkan pengaruh masing-masing faktor.

Isu bias menambah lapisan kompleksitas dalam memahami perilaku model. Weidinger et al.\ menunjukkan bahwa LLM dapat mereproduksi dan memperkuat pola bias sosial yang tercermin dalam data pelatihan \parencite{weidinger2021ethical}. Ketika identitas sosial tertentu, misalnya terkait gender, profesi, atau latar budaya, dimasukkan ke dalam konteks, respons model berpotensi mencerminkan bias representasional maupun inferensial yang sudah tertanam di dalam parameter model. Dalam konteks persona, hal ini berarti bahwa perbedaan respons akibat variasi identitas pengguna tidak selalu mencerminkan perubahan kemampuan penalaran, tetapi juga dapat berkaitan dengan bias yang telah terinternalisasi.

Sebagian besar studi persona yang ada menempatkan persona pada sisi model, bukan pada sisi pengguna. Instruksi yang mengubah peran model sebagai agen percakapan berbeda dengan skenario di mana konteks interaksi menyatakan bahwa pengguna memiliki identitas atau latar belakang tertentu. Riset mengenai pemodelan pengguna mulai berkembang, misalnya melalui pendekatan \textit{user language model} yang mempelajari distribusi bahasa berdasarkan karakteristik pengguna \parencite{naous2025userlm}, tetapi penelitian yang secara sistematis mengkaji dampak \textit{user persona} eksplisit maupun implisit terhadap penalaran dan kualitas jawaban pada berbagai tugas masih relatif terbatas.

Dari sisi infrastruktur evaluasi, banyak studi sebelumnya masih mengandalkan eksekusi manual atau setengah otomatis ketika menjalankan eksperimen yang melibatkan variasi pengguna. Naous et al.\ menyoroti pentingnya pendekatan yang lebih terstruktur ketika mengevaluasi model dalam konteks variasi pengguna, termasuk pengelolaan konfigurasi, pencatatan hasil, serta konsistensi skenario pengujian \parencite{naous2025userlm}. Tanpa kerangka evaluasi yang terdokumentasi dengan jelas, eksperimen yang melibatkan banyak model, banyak persona, dan berbagai jenis tugas menjadi sulit direplikasi dan rawan ketidakkonsistenan.

Berdasarkan kondisi tersebut, masalah-masalah utama yang mendasari perumusan penelitian ini dapat diringkas pada Tabel~\ref{tab:daftar-masalah-llm-persona}.

\begin{table}[htbp]
  \centering
  \caption{Daftar masalah penelitian terkait \textit{user persona} pada LLM}
  \label{tab:daftar-masalah-llm-persona}
  \begin{tabular}{p{1.5cm}p{5.5cm}p{7.0cm}}
    \toprule
    Kode & Uraian masalah & Dampak terhadap penelitian \\
    \midrule
    M-01 &
    Persona pada LLM umumnya diterapkan pada sisi model, bukan pada sisi pengguna. &
    Belum ada pemahaman yang sistematis mengenai bagaimana \textit{user persona} eksplisit maupun implisit memengaruhi penalaran dan kualitas jawaban pada berbagai tugas. \\[0.3cm]
    
    M-02 &
    Efek persona sulit dipisahkan dari efek framing dan gaya penulisan \textit{prompt}. &
    Perubahan performa atau pola penalaran dapat berasal dari variasi formulasi instruksi, bukan semata akibat perubahan \textit{user persona}, sehingga interpretasi hasil menjadi tidak pasti. \\[0.3cm]
    
    M-03 &
    LLM membawa bias sosial yang terinternalisasi dari data pelatihan. &
    Ketika identitas pengguna memuat atribut sosial tertentu, respons model berpotensi mencerminkan bias representasional maupun inferensial, sehingga perbedaan jawaban bisa berkaitan dengan bias yang sudah ada di model. \\[0.3cm]
    
    M-04 &
    Cakupan model dan tugas pada studi terdahulu masih terbatas. &
    Analisis sensitivitas terhadap persona sering kali hanya mencakup sedikit model atau jenis tugas, sehingga belum memberikan gambaran yang cukup luas mengenai variasi perilaku LLM di berbagai konteks. \\
    \bottomrule
  \end{tabular}
\end{table}

Masalah M-01 berkaitan dengan dominasi pendekatan yang menempatkan persona pada sisi model. Tseng et al.\ membahas bagaimana persona digunakan untuk mengubah peran dan gaya respons model melalui instruksi sistem atau deskripsi karakter \parencite{tseng2024twotales}. Pendekatan ini berbeda dengan skenario di mana identitas dan karakteristik pengguna dinyatakan secara eksplisit atau implisit pada konteks interaksi. Akibatnya, pengaruh \textit{user persona} terhadap penalaran dan kualitas jawaban belum banyak dikaji secara sistematis.

Masalah M-02 muncul karena struktur penalaran LLM sangat sensitif terhadap variasi kecil dalam formulasi instruksi. Turpin et al.\ menunjukkan bahwa perubahan ringan pada susunan \textit{prompt} dapat menghasilkan rantai penalaran yang berbeda meskipun pertanyaannya sama \parencite{turpin2023language}. Zhou et al.\ juga menunjukkan bahwa framing dan gaya penulisan instruksi dapat memengaruhi isi dan gaya jawaban \parencite{zhou2023largemodelsensitive}. Dalam konteks ini, efek \textit{user persona} berpotensi tercampur dengan efek framing, sehingga diperlukan desain eksperimen yang mampu membedakan keduanya.

Masalah M-03 berhubungan dengan bias sosial yang sudah tertanam di dalam model. Weidinger et al.\ menunjukkan bahwa LLM dapat mereproduksi dan memperkuat pola bias dari data pelatihan \parencite{weidinger2021ethical}. Ketika \textit{user persona} memuat atribut sosial seperti gender, profesi, atau latar budaya, respons model terhadap persona tersebut dapat dipengaruhi oleh bias yang telah ada sebelumnya. Hal ini menyulitkan interpretasi hasil, karena perbedaan jawaban bisa berasal dari kombinasi antara penyesuaian terhadap persona dan bias yang sudah terinternalisasi di dalam model.

Masalah M-04 menyoroti keterbatasan cakupan model dan tugas pada studi-studi terdahulu. Banyak penelitian persona hanya menguji sedikit model atau fokus pada satu jenis tugas, sehingga belum memberikan gambaran yang cukup luas mengenai bagaimana variasi \textit{user persona} memengaruhi perilaku model pada spektrum tugas penalaran dan percakapan yang lebih beragam \parencite{gupta2024biasrunsdeep, tseng2024twotales}. Keterbatasan ini membuka peluang untuk merancang eksperimen yang melibatkan kombinasi multi model dan multi persona pada beberapa kategori tugas yang terpilih.

\section{Analisis Kebutuhan}


\lipsum[4]
\subsection{Identifikasi Masalah Pengguna}
\lipsum[5]
\subsection{Kebutuhan Fungsional}
\lipsum[6]
\subsection{Kebutuhan Nonfungsional}
\lipsum[7]

\section{Analisis Pemilihan Solusi}
\subsection{Alternatif Solusi}
\lipsum[8]
\subsection{Analisis Penentuan Solusi}
\lipsum[9]
% ==========================================
% BAB IV DESAIN KONSEP SOLUSI
% ==========================================

\chapter{DESAIN KONSEP SOLUSI}
\label{chap:desain-konsep-solusi}

Bab ini memaparkan rancangan konsep solusi yang diusulkan untuk menjawab permasalahan yang telah dianalisis pada bab sebelumnya. Berdasarkan hasil analisis pemilihan solusi, pendekatan yang digunakan dalam penelitian ini adalah pengembangan sistem eksperimen terotomatisasi berbasis konfigurasi. Pembahasan dalam bab ini mencakup desain konseptual eksperimen, perancangan arsitektur perangkat lunak atau \textit{evaluation pipeline}, serta spesifikasi implementasi data dan struktur berkas. Desain ini disusun untuk memenuhi kebutuhan fungsional terkait strukturisasi \textit{user persona} dan konsistensi eksekusi lintas model.
% ==========================================
%====== BAB IV.1 ======
\section{Desain Konseptual Eksperimen}
\label{sec:desain-konseptual}

Bagian ini memaparkan dasar konseptual dari eksperimen yang dikembangkan untuk mengukur pengaruh \textit{user persona} terhadap perilaku model bahasa. Perancangan ini mencakup evaluasi terhadap keterbatasan pendekatan konvensional, prinsip desain sistem terotomatisasi yang diusulkan, serta integrasi persona, model, dan \textit{benchmark} penalaran yang digunakan. Dengan adanya desain konseptual ini, alur eksperimen yang dibahas pada subbab berikutnya menjadi lebih terarah, terukur, dan dapat direplikasi.

\subsection{Keterbatasan Model Operasional Konvensional}

Pendekatan manual yang lazim digunakan dalam penelitian persona umumnya bergantung pada penulisan instruksi langsung melalui antarmuka percakapan. Meskipun sederhana, pendekatan ini memiliki dua kelemahan metodologis utama yang mengurangi validitas internal penelitian.

Pertama, terjadi instabilitas masukan. Model bahasa sangat sensitif terhadap perubahan kecil pada struktur \textit{prompt}—seperti variasi tanda baca atau perubahan gaya kalimat—yang dapat menyebabkan keluaran berbeda secara signifikan. Turpin et al.\ menunjukkan bahwa variasi kecil dalam \textit{framing} dapat menghasilkan rantai penalaran yang tidak konsisten \parencite{turpin2023language}. Ketergantungan pada input manual membuat variasi ini tidak dapat dikendalikan.

Kedua, pendekatan manual tidak menyediakan granularitas data yang memadai. Respons model biasanya hanya dicatat dalam bentuk teks akhir tanpa metadata komputasional seperti latensi atau jumlah token. Padahal, indikator tersebut penting untuk memahami beban kognitif model serta pola penalaran yang muncul pada kondisi persona tertentu \parencite{naous2025userlm}.

\subsection{Sistem Eksperimen Terotomatisasi}

Untuk mengatasi keterbatasan tersebut, penelitian ini mengusulkan sistem eksperimen terotomatisasi dengan tiga prinsip desain inti: determinisme masukan, telemetri komprehensif, dan skalabilitas eksekusi.

Pertama, determinisme masukan dicapai dengan menyimpan seluruh persona dalam berkas konfigurasi statis. Setiap \textit{prompt} dibentuk melalui mekanisme injeksi otomatis sehingga stimulus yang diterima model identik hingga tingkat karakter.

Kedua, sistem mencatat keluaran model secara lengkap dalam format terstruktur (JSON dan CSV). Telemetri yang direkam mencakup teks jawaban, \textit{reasoning trace} (bila tersedia), jumlah token, dan latensi inferensi. Dengan demikian, analisis dapat menilai tidak hanya kebenaran jawaban, tetapi juga pola beban komputasi yang terkait dengan masing-masing persona.

Ketiga, sistem mendukung eksekusi \textit{asynchronous} sehingga ribuan permintaan dapat diproses secara paralel tanpa melampaui batas layanan. Pendekatan ini meningkatkan cakupan eksperimen sekaligus mengurangi waktu eksekusi total.

Model konseptual dari sistem usulan ditampilkan pada Gambar~\ref{fig:model-after}, yang dalam penomoran dokumen dicantumkan sebagai \textbf{Gambar IV.2 Model Konseptual Sistem Eksperimen Terotomatisasi}.

\begin{figure}[htbp]
  \centering
  \includegraphics[width=0.9\textwidth,height=0.7\textheight,keepaspectratio]{image/proposed_solution.png}
  \caption{Model konseptual sistem eksperimen terotomatisasi}
  \label{fig:model-after}
\end{figure}

\subsection{Analisis Komparatif Metodologis}

Peralihan dari pendekatan \textit{Existing} ke \textit{Proposed} bukan sekadar peningkatan efisiensi, melainkan juga penguatan validitas ilmiah. Tabel~\ref{tab:komparasi-metodologis} merangkum perbedaan metodologis utama antara kedua pendekatan tersebut.

Dengan struktur yang deterministik, terdokumentasi, dan terotomatisasi, setiap kesimpulan terkait pengaruh persona terhadap penalaran model dapat ditarik secara lebih dapat dipercaya dan dapat dipertanggungjawabkan secara ilmiah.

\begin{table}[htbp]
  \centering
  \caption{Analisis komparatif validitas metodologis}
  \label{tab:komparasi-metodologis}
  \renewcommand{\arraystretch}{1.25}
  \small
  \begin{tabularx}{\textwidth}{
    >{\raggedright\arraybackslash}p{3.6cm}
    >{\raggedright\arraybackslash}X
    >{\raggedright\arraybackslash}X}
    \toprule
    \textbf{Dimensi Analisis} &
    \textbf{Sistem Konvensional (Existing)} &
    \textbf{Sistem Usulan (Proposed)} \\
    \midrule

    Pengendalian variabel &
    Rentan terhadap gangguan input manual; sensitivitas \textit{framing} sulit dikendalikan. &
    Deterministik; konfigurasi statis dan injeksi otomatis menjamin isolasi variabel independen. \\

    Granularitas data &
    Hanya menangkap jawaban akhir tanpa metadata. &
    Telemetri lengkap; menangkap latensi, token, dan jejak penalaran. \\

    Format penyimpanan &
    Tidak terstruktur; raw text sulit diproses ulang. &
    Terstruktur (JSON/CSV) dan cocok untuk analisis lanjutan. \\

    Reproduktibilitas &
    Rendah; parameter eksperimen tidak lengkap terdokumentasi. &
    Tinggi; seluruh konfigurasi dan kode dapat diaudit dan dijalankan ulang. \\

    Cakupan eksperimen &
    Terbatas; eksekusi linear menghambat skala eksperimen. &
    Masif; mendukung ribuan kombinasi melalui \textit{asynchronous execution}. \\
    \bottomrule
  \end{tabularx}
\end{table}

\subsection{Integrasi Persona, Model, dan Benchmark}

Bagian ini memaparkan komponen yang membentuk konfigurasi eksperimen, yaitu himpunan persona, himpunan model, dan \textit{benchmark} penalaran yang digunakan.

\subsubsection{Benchmark Penalaran}

Dua \textit{benchmark} digunakan untuk mengevaluasi dua bentuk penalaran yang berbeda.

\textit{GSM8K} merupakan kumpulan soal cerita matematika tingkat sekolah menengah yang menguji penalaran numerik bertahap \parencite{cobbe2021gsm8k}. Soal-soal GSM8K memiliki struktur jawaban numerik yang jelas sehingga proses evaluasi dapat dilakukan secara deterministik.

\textit{MMLU-Redux} merupakan versi terkurasi dari MMLU yang memperbaiki ambiguitas format, ketidakkonsistenan pilihan jawaban, dan ketidakseimbangan kualitas antar-subjek \parencite{mmluRedux2024dataset}. Benchmark ini digunakan untuk menguji penalaran konseptual lintas domain dengan format pilihan ganda.

Kombinasi GSM8K dan MMLU-Redux memberikan cakupan dua jenis penalaran yang komplementer: numerik prosedural dan konseptual deklaratif.

\subsubsection{Himpunan Model}

Penelitian ini menggunakan sembilan model bahasa, yang dikelompokkan ke dalam dua kategori berikut:

\begin{enumerate}
    \item Model komersial:  
    GPT-5, GPT-5 Mini, Claude 4.5 Sonnet, Claude 4.5 Haiku, Gemini 2.5 Flash, Gemini 2.5 Pro;

    \item Model publik via OpenRouter:  
    Grok 4.1 Fast, NVIDIA Nemotron-nano-12B-v2-VL, Bert Nebulon Alpha.
\end{enumerate}

Keragaman arsitektur ini memungkinkan analisis komparatif lintas paradigma, mulai dari \textit{frontier models} hingga model publik yang tersedia secara bebas.

\subsubsection{Struktur Persona}

Persona disusun berdasarkan enam dimensi utama: gender, usia, agama, pekerjaan, kewarganegaraan, dan register bahasa. Kombinasi dimensi tersebut menghasilkan lima belas persona—baik eksplisit maupun implisit—ditambah satu kondisi pengguna netral.

Tabel~\ref{tab:user-persona} menampilkan daftar lengkap persona yang digunakan.

\begin{table}[htbp]
\centering
\caption{Daftar user persona untuk kondisi eksperimen}
\label{tab:user-persona}
\renewcommand{\arraystretch}{1.15}

\resizebox{\textwidth}{!}{
\begin{tabular}{c l l l l l l l}
\toprule
\textbf{ID} &
\textbf{Persona} &
\textbf{Mode} &
\textbf{Gender} &
\textbf{Age Group} &
\textbf{Religion} &
\textbf{Occupation} &
\textbf{Nationality / Register} \\
\midrule
P1  & Implicit male baseline              & Implicit & Male   & --          & --       & --               & Neutral \\
P2  & Implicit female baseline            & Implicit & Female & --          & --       & --               & Neutral \\
P3  & Neutral user                        & Neutral  & --     & --          & --       & --               & Neutral \\
P4  & Indonesian Muslim young woman       & Explicit & Female & Young adult & Muslim   & Healthcare worker & Indonesian / Semi-formal \\
P5  & Indonesian Muslim young man         & Implicit & Male   & Young adult & Muslim   & Healthcare worker & Indonesian / Semi-formal \\
P6  & American middle-aged male           & Explicit & Male   & Middle-aged & Christian & Engineer         & American / Formal \\
P7  & American middle-aged female         & Implicit & Female & Middle-aged & Christian & Engineer         & American / Formal \\
P8  & Indonesian Gen-Z female             & Explicit & Female & Gen-Z       & --       & Student          & Indonesian / Casual-slang \\
P9  & Indonesian Gen-Z male               & Implicit & Male   & Gen-Z       & --       & Student          & Indonesian / Casual-slang \\
P10 & Middle Eastern young adult male     & Explicit & Male   & Young adult & Muslim   & Engineer         & Middle Eastern Arabic / Formal \\
P11 & Middle Eastern young adult female   & Implicit & Female & Young adult & Muslim   & Student          & Middle Eastern Arabic / Formal \\
P12 & American atheist young male         & Explicit & Male   & Young adult & Atheist  & Student          & American / Formal \\
P13 & American atheist young female       & Implicit & Female & Young adult & Atheist  & Student          & American / Formal \\
P14 & Indonesian female healthcare worker & Explicit & Female & Young adult & Muslim   & Healthcare worker & Indonesian / Semi-formal \\
P15 & Indonesian male healthcare worker   & Implicit & Male   & Young adult & Muslim   & Healthcare worker & Indonesian / Semi-formal \\
\bottomrule
\end{tabular}
}
\end{table}
\subsubsection{Konfigurasi Eksekusi}

Kombinasi lima belas persona dan sembilan model menghasilkan seratus tiga puluh lima
konfigurasi eksperimen. Setiap konfigurasi melalui empat tahap eksekusi tetap:
pemanasan persona, penetapan konteks percakapan, eksekusi \textit{benchmark}, dan
pencatatan telemetri. Dengan demikian, setiap variasi keluaran dapat ditelusuri ulang
secara langsung kepada kondisi persona yang diberikan.
\subsubsection{Contoh Mekanisme Injeksi Persona}

Proses injeksi persona dilakukan dengan menyusun \textit{system message} 
yang mendefinisikan identitas dan karakter pengguna sebelum model menerima 
stimulus pertanyaan. Dua kategori persona yang digunakan dalam penelitian ini 
adalah persona eksplisit dan persona implisit.

\begin{enumerate}
    \item Persona eksplisit.
\end{enumerate}

Pada persona eksplisit, identitas pengguna dirumuskan secara langsung melalui 
pola deklaratif seperti “your user is ...”. Instruksi ini menyebutkan atribut 
sosial secara jelas—misalnya gender, usia, pekerjaan, atau preferensi gaya 
bahasa. Contoh injeksi persona eksplisit yang digunakan dalam eksperimen adalah:

\begin{quote}
\textit{
“Your user is an Indonesian Gen-Z male who works as a junior engineer.  
He is analytical, prefers concise explanations, and communicates in a casual 
but respectful tone. Adjust your reasoning structure and tone accordingly 
when responding.”
}
\end{quote}

Formulasi seperti ini memberikan konteks identitas yang eksplisit sehingga 
efek persona dapat ditelusuri dengan jelas melalui perubahan pada struktur 
penalaran dan gaya jawabannya.

\begin{enumerate}
    \setcounter{enumi}{1}
    \item Persona implisit.
\end{enumerate}

Persona implisit tidak menyatakan identitas sosial secara langsung, tetapi 
dibangun melalui narasi pengalaman pribadi, emosi, atau gaya tutur tertentu.  
Model tidak diberi kategori demografis eksplisit; konteks diberikan secara halus, 
sehingga model perlu menginterpretasikan karakter pengguna melalui sinyal 
linguistik yang tersirat.

Contoh injeksi persona implisit yang digunakan dalam penelitian ini adalah:

\begin{quote}
\textit{
“Lately I have been feeling a strange mix of emotional exhaustion and pressure 
to appear composed, especially when my skin starts acting up unexpectedly.  
I keep adjusting small routines—like my skincare products—hoping something will 
finally work. Before I deal with it again, could you help me break down this 
next question step-by-step?”
}
\end{quote}

Instruksi seperti ini menghasilkan kerangka persona yang lebih halus dan 
mendukung analisis terhadap sensitivitas model terhadap gaya bahasa serta 
cues emosional yang tidak dinyatakan secara eksplisit.

%====== BAB IV.2 ======
\section{Perancangan Arsitektur Perangkat Lunak (\textit{Evaluation Pipeline})}
\label{sec:perancangan-pipeline}

Subbab ini menjelaskan desain arsitektur perangkat lunak yang digunakan untuk merealisasikan \textit{evaluation pipeline} sebagaimana dirumuskan pada Subbab~\ref{sec:desain-konseptual}. Arsitektur pipeline dirancang agar proses evaluasi dapat berjalan secara otomatis, konsisten, dan dapat direproduksi. Pendekatan ini memastikan bahwa setiap kombinasi persona, model, dan \textit{benchmark task} diuji dalam kondisi yang setara dan bebas dari variasi yang tidak diperlukan.

Pipeline yang dibangun bekerja sebagai rangkaian komponen yang saling berinteraksi: mulai dari pemuatan data, konstruksi instruksi, pengiriman permintaan ke model, hingga pencatatan \textit{telemetry}. Seluruh proses tersebut bekerja dalam satu alur terintegrasi sehingga sistem mampu menangani jumlah evaluasi yang besar secara stabil.

\subsection{Arsitektur Alur Kerja Sistem}
\label{subsec:arsitektur-alur-kerja}

Secara garis besar, \textit{evaluation pipeline} terbagi ke dalam empat komponen utama yang membentuk satu siklus pemrosesan yang berulang untuk setiap kombinasi persona dan butir soal. Keempat komponen tersebut adalah sebagai berikut.

\begin{enumerate}
    \item \textit{Configuration initialization and validation}.\\
    Tahap ini memuat seluruh konfigurasi sistem, definisi persona, dan \textit{benchmark dataset} ke dalam memori. Validasi struktur data dilakukan untuk memastikan bahwa setiap persona memiliki \textit{system instruction} yang lengkap dan setiap butir tugas memiliki pasangan pertanyaan dan jawaban acuan. Validasi awal ini penting untuk mencegah kesalahan format yang dapat menghentikan proses pada tahap berikutnya.

    \item \textit{Prompt construction engine}.\\
    Pada tahap ini, sistem membentuk dua jenis pesan: \textit{system message} yang berisi identitas persona dan \textit{user message} yang memuat pertanyaan dari benchmark. Penyusunan instruksi dilakukan menggunakan pola yang seragam untuk seluruh iterasi, sehingga setiap model menerima bentuk stimulus yang konsisten. Pendekatan ini menghilangkan variasi yang berasal dari perbedaan penulisan instruksi manual.

    \item \textit{Execution manager}.\\
    Komponen ini mengatur pengiriman permintaan ke model-model bahasa melalui \textit{API interface}. Untuk mengatasi volume permintaan yang besar, \textit{execution manager} menggunakan pendekatan eksekusi asinkron dengan \textit{I/O concurrency}. Permintaan diatur dalam \textit{task queue} dan dieksekusi dalam kelompok sesuai batas \textit{rate limit}. Strategi ini mempercepat proses pengujian tanpa melampaui kapasitas layanan penyedia model.

    \item \textit{Telemetry logger}.\\
    Komponen terakhir bertanggung jawab menyimpan seluruh respons model dalam format terstruktur, termasuk \textit{model output}, jumlah token, serta \textit{latency}. Data ini digunakan sebagai dasar analisis performa pada bab berikutnya.
\end{enumerate}

Dengan pembagian tersebut, pipeline dapat beroperasi secara modular namun tetap terpadu dalam satu alur pemrosesan.

\subsection{Algoritma Orkestrasi dan Konkurensi}
\label{subsec:algoritma-orkestrasi}

Eksperimen dalam penelitian ini melibatkan ribuan kombinasi persona–model–pertanyaan yang menghasilkan volume permintaan API dalam jumlah besar. Eksekusi secara sekuensial tidak praktis karena setiap permintaan memiliki latensi yang bervariasi, sementara penyedia model menerapkan batas \textit{rate limit} yang ketat. Untuk mengatasi hal tersebut, pipeline menggunakan pendekatan eksekusi asinkron berbasis \textit{I/O concurrency}.

Pendekatan ini memungkinkan banyak permintaan dieksekusi secara paralel (hingga batas tertentu), sehingga waktu total dapat ditekan dari kompleksitas \(O(N)\) menjadi mendekati \(O(N/C)\), dengan \(C\) adalah kapasitas konkurensi maksimum. Pipeline membangun sebuah \textit{task queue} yang berisi seluruh pasangan persona–soal, kemudian memprosesnya dalam kelompok (\textit{batch}) sesuai kapasitas konkurensi. Ketika satu batch sedang diproses, sistem dapat menyiapkan batch berikutnya tanpa menunggu seluruh permintaan selesai.

Selain meningkatkan efisiensi waktu, mekanisme ini juga menyediakan ketahanan terhadap kesalahan. Jika terjadi galat seperti \textit{timeout}, \textit{connection reset}, atau \texttt{429 Too Many Requests}, pipeline tidak menghentikan seluruh proses. Tugas yang gagal akan dicatat dan dijalankan ulang menggunakan strategi \textit{exponential backoff}, memastikan stabilitas eksekusi jangka panjang.

Algoritma 4.1 berikut mendefinisikan prosedur eksekusi paralel secara formal.

\begin{verbatim}
Algoritma 4.1: Prosedur Eksekusi Eksperimen Paralel

Input : Himpunan Persona P, Himpunan Tugas T, Batas Konkurensi C
Output: Himpunan Log L

Function RunExperiment(P, T):
  1. Inisialisasi Antrean Tugas Q <- Kosong
  2. Untuk setiap p dalam P lakukan:
       Untuk setiap t dalam T lakukan:
         Prompt <- ConstructPrompt(p.instruction, t.question)
         Enqueue(Q, Prompt)

  3. Inisialisasi Semaphore S dengan kapasitas C

  4. While Q tidak kosong lakukan secara Asinkron:
       Batch <- DequeueBatch(Q, C)
       Untuk setiap item i dalam Batch lakukan secara Paralel:
         Acquire(S)
         Coba:
           Respons <- AsyncCallAPI(i.prompt, i.config)
           Metadata <- ExtractTelemetry(Respons)
           SaveLog(Respons, Metadata)
           Tambahkan ke L
         Tangkap Galat:
           LogGalat(i)
           RetryWithBackoff(i)
         Akhirnya:
           Release(S)

  5. Return L
\end{verbatim}

Melalui orkestrasi ini, pipeline mencapai dua tujuan: (1) efisiensi waktu eksekusi yang optimal berkat pemrosesan paralel, dan (2) ketahanan proses melalui penanganan galat adaptif. Dengan demikian, seluruh kombinasi persona–model–benchmark dapat dieksekusi secara konsisten, stabil, dan dapat direproduksi.


\subsection{Mekanisme Injeksi Konteks Persona}
\label{subsec:mekanisme-injeksi}

Mekanisme injeksi persona merupakan elemen penting untuk memastikan bahwa pengaruh persona dapat diukur dengan jelas. Pipeline menerapkan dua tahap injeksi konteks yang bersifat tetap dan hanya dilakukan satu kali untuk setiap persona sebelum evaluasi dimulai.

Tahap pertama adalah \textit{persona context initialization}. Pada tahap ini, sistem menyusun pesan awal yang merangkum identitas dan karakter persona. Pesan ini berfungsi membangun \textit{cognitive framing} awal pada model, baik untuk persona eksplisit maupun implisit. Tahap ini memastikan bahwa model berada dalam kondisi persona yang konsisten sebelum diberikan tugas.

Tahap kedua adalah \textit{persona warm-up message}. Pesan ini digunakan untuk memastikan bahwa model memberikan respons yang sesuai dengan identitas persona. Respons dari tahap ini tidak digunakan dalam evaluasi, tetapi berfungsi sebagai verifikasi bahwa proses injeksi berhasil.

Setelah kedua tahap ini selesai, pipeline tidak lagi mengulangi injeksi persona untuk setiap pertanyaan. Identitas yang telah ditanamkan pada awal percakapan tetap digunakan selama seluruh rangkaian pengujian. Model kemudian langsung memproses seluruh soal pada GSM8K dan MMLU-Redux dalam kondisi persona yang sama. Pendekatan ini memastikan bahwa variasi keluaran model berasal dari perbedaan persona, bukan dari perbedaan struktur instruksi.

\subsection{Mekanisme Toleransi Kesalahan dan Persistensi Status}
\label{subsec:toleransi-kesalahan}

Pipeline dirancang agar tetap stabil meskipun menghadapi gangguan selama proses pengujian. Dua mekanisme utama digunakan untuk menjamin integritas data dan keberlanjutan proses.

Pertama, sistem menerapkan \textit{state persistence}. Setelah setiap tugas berhasil diproses, status kemajuan dicatat sehingga apabila terjadi interupsi, pipeline dapat dilanjutkan kembali tanpa mengulangi tugas yang sudah selesai.

Kedua, gangguan sementara ditangani dengan \textit{error handling} berbasis penjadwalan ulang adaptif. Tugas yang gagal tidak langsung dihentikan, tetapi dijalankan kembali setelah jeda waktu tertentu. Dengan kombinasi kedua strategi ini, pipeline dapat menyelesaikan seluruh rangkaian evaluasi meskipun terjadi kendala jaringan atau batasan layanan eksternal.
%====== BAB IV.3 ======
\section{Implementasi Data, Struktur Berkas, dan Keluaran Pipeline}
\label{sec:implementasi-data}

Subbab ini menjelaskan bagaimana rancangan pipeline yang telah disusun pada bagian sebelumnya direalisasikan dalam bentuk organisasi data, struktur direktori, serta format keluaran yang dihasilkan selama proses eksperimen. Implementasi ini dirancang untuk memastikan bahwa seluruh tahapan pemuatan aset, injeksi konteks persona, pelaksanaan inferensi, dan perekaman hasil berlangsung secara konsisten, terdokumentasi dengan baik, serta mendukung keterulangan eksperimen secara penuh.

\subsection{Organisasi Direktori dan Artefak Data}

Pipeline dijalankan di atas struktur direktori yang dirancang secara modular untuk memisahkan fungsi pemrosesan dan memudahkan proses audit ilmiah. Empat kelompok artefak utama disusun secara hierarkis sebagai berikut.

\begin{enumerate}
    \item \textit{Root directory}.  
    Berfungsi sebagai titik masuk eksekusi sistem dan memuat skrip penggerak pipeline beserta utilitas operasional.

    \item \textit{Configuration directory}.  
    Menyimpan konfigurasi teknis yang digunakan pipeline, termasuk daftar model, kredensial layanan API, dan parameter eksekusi. Pemisahan direktori ini mendukung aspek keamanan dan memudahkan penggantian parameter tanpa memodifikasi kode utama.

    \item \textit{Input assets directory}.  
    Memuat definisi persona serta himpunan benchmark yang telah dinormalisasi. Persona direpresentasikan dalam format terstruktur yang memuat identitas, atribut demografis, dan karakteristik gaya bahasa. Sementara itu, dataset GSM8K dan MMLU Redux dikonversi ke format konsisten untuk memastikan kompatibilitas dengan pipeline.

    \item \textit{Results directory}.  
    Menyimpan keseluruhan keluaran eksperimen yang mencakup log granular pada tingkat per butir soal, tabel hasil, serta agregasi lintas persona dan lintas model. Struktur ini memudahkan proses penelusuran kembali bagi keperluan analisis.
\end{enumerate}

Pemilahan direktori ini memastikan bahwa seluruh artefak eksperimen terdokumentasi secara terstruktur dan mudah direplikasi.

\subsection{Subsistem Perangkat Lunak dan Alur Transformasi Data}

Pipeline terdiri atas empat subsistem utama yang bekerja secara berurutan dalam mengelola eksekusi eksperimen:

\begin{enumerate}
    \item \textit{Execution orchestration subsystem}.  
    Subsistem ini membentuk \textit{task queue} yang memuat seluruh kombinasi model, persona, dan pertanyaan. Orkestrasi ini memastikan determinisme dan menghindari variasi eksekusi akibat intervensi manual.

    \item \textit{Model communication subsystem}.  
    Bertugas melakukan konstruksi instruksi, mengirimkan permintaan ke layanan model, menangani kode galat, serta menegakkan batas layanan seperti \textit{rate limit}. Seluruh komunikasi dilakukan menggunakan protokol API yang distandardisasi.

    \item \textit{Monitoring subsystem}.  
    Menyediakan mekanisme \textit{checkpointing} sehingga eksekusi dapat dilanjutkan tanpa kehilangan progres apabila terjadi gangguan jaringan atau penghentian proses secara tidak terduga. Hal ini memastikan konsistensi eksekusi dan mengurangi risiko duplikasi.

    \item \textit{Analysis subsystem}.  
    Mengolah log mentah menjadi tabel terstruktur dan menghitung metrik utama seperti akurasi, penggunaan token, dan latensi. Modul ini menghasilkan keluaran agregasi yang digunakan dalam tahap analisis pada Bab~V.
\end{enumerate}

Alur transformasi data berlangsung dari log granular menuju tabel pemetaan kemudian agregasi lintas model, sehingga memungkinkan analisis kuantitatif yang komprehensif.

\subsection{Representasi Persona dan Mekanisme Injeksi Konteks}

Persona direpresentasikan dalam format terstruktur yang memuat identitas demografis, atribut gaya bahasa, serta narasi yang relevan. Representasi tersebut kemudian dikonversi menjadi \textit{system instruction} yang ditempatkan pada segmen instruksi sistem saat permintaan dikirimkan ke model.

Injeksi konteks persona dilakukan satu kali melalui dua tahap:

\begin{enumerate}
    \item \textit{Persona grounding}.  
    Tahap ini menanamkan identitas dan karakteristik gaya bahasa persona secara eksplisit atau implisit pada konteks model.

    \item \textit{Warm up interaction}.  
    Dilakukan satu interaksi pemanasan untuk menstabilkan perilaku model sehingga respons pada tahap berikutnya mengikuti karakter persona secara konsisten.
\end{enumerate}

Setelah kedua tahap tersebut selesai, pipeline mengirim seluruh pertanyaan GSM8K dan MMLU Redux tanpa mengulang injeksi persona. Dengan demikian, kondisi kognitif model dijaga agar tetap setara di seluruh siklus inferensi.

\subsection{Contoh Struktur Log Inferensi}

Untuk menjaga transparansi dan keterulangan eksperimen, pipeline mencatat setiap interaksi dengan model dalam bentuk log terstruktur. Log ini memuat informasi mengenai konfigurasi eksekusi, isi jawaban model, serta telemetri penggunaan token. Cuplikan pada Kode~\ref{code:log-noreason} menunjukkan contoh keluaran untuk model yang tidak menyediakan \textit{reasoning trace}.

\begin{tcolorbox}[
    colback=gray!5,
    colframe=gray!50,
    title={Kode IV.1 Contoh log inferensi tanpa reasoning trace},
    fonttitle=\bfseries,
    arc=2mm,
    left=2mm,
    right=2mm,
    listing only
]
\footnotesize
\begin{verbatim}
{
 "run": {"model_id": "example-model", "question_id": "gsm8k_00001"},
 "response": {
   "choices": [{
     "message": {"content": "Let's break down the problem..."}
   }],
   "usage": {"prompt_tokens": 211, "completion_tokens": 197}
 }
}
\end{verbatim}
\end{tcolorbox}
\label{code:log-noreason}

Pada model tertentu, layanan juga menyediakan informasi tambahan mengenai proses penalaran internal yang digunakan untuk menghasilkan jawaban akhir. Informasi ini direkam sebagai \textit{reasoning trace} dan disimpan terpisah dari konten jawaban. Kode~\ref{code:log-reason} memperlihatkan contoh log untuk model yang menyediakan \textit{reasoning trace} beserta jumlah token yang digunakan pada bagian tersebut.

\begin{tcolorbox}[
    colback=gray!5,
    colframe=gray!50,
    title={Kode IV.2 Contoh log inferensi dengan reasoning trace},
    fonttitle=\bfseries,
    arc=2mm,
    left=2mm,
    right=2mm,
    listing only
]
\footnotesize
\begin{verbatim}
{
 "run": {"model_id": "example-model-reason", "question_id": "gsm8k_00003"},
 "response": {
   "choices": [{
     "message": {
       "content": "Final answer: 70000",
       "reasoning": "First compute the purchase cost..."
     }
   }],
   "usage": {"completion_tokens": 867, "reasoning_tokens": 485}
 }
}
\end{verbatim}
\end{tcolorbox}
\label{code:log-reason}

Kedua contoh tersebut menggambarkan bagaimana pipeline menangkap tidak hanya jawaban akhir, tetapi juga struktur penalaran dan sumber daya komputasi yang digunakan oleh model. Informasi ini menjadi dasar analisis lebih lanjut mengenai perbedaan perilaku antar model dan antar persona.

\subsection{Ringkasan Hasil Eksperimen}

Ringkasan performa lintas model dan persona ditampilkan pada Tabel~\ref{tab:gsm8k-summary-compact}. Tabel ini memberikan gambaran umum mengenai tingkat akurasi dan beban komputasi untuk setiap konfigurasi, dan digunakan sebagai dasar analisis pada Bab~V.

\begin{table}[htbp]
\centering
\caption{Ringkasan Hasil Eksperimen GSM8K untuk Seluruh Model dan Persona}
\label{tab:gsm8k-summary-compact}
\renewcommand{\arraystretch}{1.18}
\small

\begin{adjustbox}{max width=\textwidth}
\begin{tabular}{l l c c c c}
\toprule
\textbf{Model} &
\textbf{Persona} &
\textbf{Total Q} &
\textbf{Correct} &
\textbf{Accuracy (\%)} &
\textbf{Total Tokens} \\
\midrule

Bert Nebulon Alpha & man\_implicits   & 610  & 593  & 97.21 & 285250 \\
Bert Nebulon Alpha & woman\_implicits & 641  & 627  & 97.26 & 335208 \\
\midrule

Grok 4.1 Fast & man\_implicits   & 1315 & 1242 & 94.45 & 1325229 \\
Grok 4.1 Fast & woman\_implicits & 1316 & 1254 & 95.36 & 1422736 \\
\midrule

Nvidia Nemotron 12B v2 VL & man\_implicits   & 1305 & 1224 & 93.79 & 1156049 \\
Nvidia Nemotron 12B v2 VL & woman\_implicits & 1315 & 1248 & 94.98 & 1986284 \\
\bottomrule
\end{tabular}
\end{adjustbox}

\end{table}

Tabel tersebut menjadi dasar perbandingan antar model pada bab evaluasi, termasuk pengaruh persona terhadap akurasi dan kompleksitas respons.

% ==========================================
% BAB V RENCANA SELANJUTNYA
% ==========================================
\chapter{RENCANA SELANJUTNYA}
\label{chap:rencana-selanjutnya}



\section{Rencana Implementasi dan Estimasi Biaya}
\label{sec:rencana-implementasi-biaya}

Rencana implementasi pada tahap berikutnya adalah menjalankan kembali
\textit{evaluation pipeline} yang telah dijelaskan pada Bab~IV dengan cakupan
penuh, mencakup sembilan model bahasa, dua \textit{benchmark} penalaran
(GSM8K dan MMLU-Redux), serta lima belas \textit{user persona} (implisit,
eksplisit, dan netral). Bagian ini merumuskan langkah implementasi teknis,
asumsi kebutuhan token, serta estimasi biaya penggunaan API berdasarkan harga
resmi masing-masing model pada platform OpenRouter%
\footcite{openrouter_gpt5mini,openrouter_claudehaiku45,openrouter_gemini25flash,openrouter_deepseek32,openrouter_llama33,openrouter_gemma3n}

Estimasi dilakukan menggunakan kurs konstan 1 USD = Rp16.000.

% -------------------------------------------------------------
% 5.1.1 Rencana Implementasi Eksperimen
% -------------------------------------------------------------
\subsection{Rencana Implementasi Eksperimen}

Implementasi eksperimen direncanakan mengikuti enam langkah utama berikut.

\begin{enumerate}
  \item Persiapan aset data.

  Sistem memuat berkas definisi 15 persona, korpus GSM8K (split
  \textit{test}), MMLU-Redux (20 subjek), kredensial API, dan konfigurasi
  model. Struktur direktori dan modul pemrosesan mengikuti rancangan pada
  Subbab~\ref{subsec:organisasi-direktori}.

  \item Inisialisasi dan \textit{warm-up} persona.

  Setiap model menerima satu pesan awal untuk menanamkan konteks persona
  sebelum mengerjakan soal pertama. Tahap ini juga berfungsi sebagai
  \textit{sanity check} bahwa model mengikuti identitas dan gaya bahasa
  persona.

  \item Eksekusi eksperimen utama.

  Setiap kombinasi model--persona menjalankan seluruh soal GSM8K dan
  MMLU-Redux menggunakan mekanisme injeksi pesan berbasis peran:
  persona pada \textit{system message} dan soal pada \textit{user message}.
  Setiap respons diharuskan mencakup penalaran langkah demi langkah.

  \item Pencatatan log granular.

  Setiap respons disimpan sebagai log JSON yang memuat isi \textit{prompt},
  jawaban mentah, \textit{token usage}, dan \textit{latency}.

  \item Agregasi dan validasi hasil.

  Data log diubah menjadi CSV agregat yang berisi akurasi, rata-rata
  latensi, dan konsumsi token. Validasi tambahan mencakup pemeriksaan pola
  jawaban dan konsistensi jumlah entri.

  \item Penanganan kegagalan.

  Kegagalan akibat \textit{timeout} atau batas \textit{rate limit} ditangani
  menggunakan \textit{retry} dengan \textit{exponential backoff}, sesuai
  mekanisme pada Bab~IV. Dengan demikian, kegagalan sebagian tidak
  mengganggu keseluruhan eksperimen.
\end{enumerate}

% -------------------------------------------------------------
% 5.1.2 Himpunan Model dan Skenario Eksekusi
% -------------------------------------------------------------
\subsection{Himpunan Model dan Skenario Eksekusi}

Eksperimen ini menggunakan sembilan model dengan rincian sebagai berikut.

\begin{enumerate}
  \item Enam model berbayar (via OpenRouter):
  \begin{enumerate}
    \item openai/gpt-5-mini
    \item anthropic/claude-haiku-4.5
    \item google/gemini-2.5-flash
    \item deepseek/deepseek-v3.2
    \item nvidia/llama-3.3-nemotron-super-49b-v1.5
    \item google/gemma-3n-e4b-it
  \end{enumerate}

  \item Tiga model yang pada saat perancangan tersedia sebagai \textit{free-tier}:
  \begin{enumerate}
    \item xai/grok-4.1-fast
    \item nvidia/nemotron-nano-12b-v2-vl
    \item openrouter/bert-nebulon-alpha
  \end{enumerate}
\end{enumerate}

Seluruh sembilan model dijalankan pada konfigurasi penuh: dua
\textit{benchmark} dan lima belas persona. Namun, estimasi biaya hanya
dihitung untuk enam model berbayar.

% -------------------------------------------------------------
% 5.1.3 Asumsi Jumlah Soal dan Kebutuhan Token
% -------------------------------------------------------------
\subsection{Asumsi Jumlah Soal dan Kebutuhan Token}

Kebutuhan token dihitung berdasarkan dua sumber utama:
GSM8K (1319 soal) dan MMLU-Redux (2000 soal).
Pada kedua \textit{benchmark}, model diarahkan untuk memberikan
penalaran lengkap sebelum jawaban akhir, sehingga konsumsi token per soal
diharapkan berada pada kisaran yang relatif tinggi.

\begin{enumerate}
  \item GSM8K.

  Total token per persona per model diestimasikan sebagai:
  \[
    T_{\text{GSM8K}} \approx
    1319 \times 1200
    = 1{,}582{,}800 \text{ token}.
  \]

  \item MMLU-Redux.

  Total token per persona per model diestimasikan sebagai:
  \[
    T_{\text{MMLU}} \approx
    2000 \times 1200
    = 2{,}400{,}000 \text{ token}.
  \]
\end{enumerate}

Total token inti per persona diperoleh dari penjumlahan keduanya:
\[
  T_{\text{base, persona}} =
  1{,}582{,}800 + 2{,}400{,}000
  = 3{,}982{,}800.
\]

Untuk mengakomodasi \textit{warm-up} dan \textit{retry}, digunakan faktor
overhead 20\%:
\[
  T_{\text{persona}} \approx
  1.2 \times 3{,}982{,}800
  = 4{,}779{,}360.
\]

Sehingga total token per model untuk 15 persona adalah:
\[
  T_{\text{model}}
  \approx 15 \times 4{,}779{,}360
  = 71{,}690{,}400
  \approx 71{,}7 \times 10^6.
\]

Komposisi token diasumsikan:
\[
  T_{\text{in}} = 0.4T_{\text{model}},\qquad
  T_{\text{out}} = 0.6T_{\text{model}}.
\]

% -------------------------------------------------------------
% 5.1.4 Estimasi Biaya per Model
% -------------------------------------------------------------
\subsection{Estimasi Biaya per Model}

Harga token per model mengacu pada dokumentasi OpenRouter%
\parencite{openrouter_gpt5mini,openrouter_claudehaiku45,openrouter_gemini25flash,
openrouter_deepseek32,openrouter_llama33,openrouter_gemma3n}.  
Biaya untuk model ke-$m$ dihitung dengan rumus:
\[
  \text{cost}_m =
  p_{\text{in},m} \times \frac{T_{\text{in}}}{10^6}
  +
  p_{\text{out},m} \times \frac{T_{\text{out}}}{10^6},
\]
dengan $p_{\text{in},m}$ dan $p_{\text{out},m}$ adalah harga per satu juta
token untuk \textit{input} dan \textit{output}.

Estimasi berikut menggunakan kurs Rp\,16.000 per USD dan total token
$T_{\text{model}} \approx 71{,}7 \times 10^6$.

\begin{table}[htbp]
\centering
\caption{Estimasi biaya enam model berbayar untuk konfigurasi penuh 15 persona}
\label{tab:estimasi_biaya_model}

\renewcommand{\arraystretch}{1.22}
\begin{tabular}{lccc}
\toprule
Model &
Total token $T_{\text{model}}$ &
Biaya (USD) &
Biaya (Rp) \\
\midrule
openai/gpt-5-mini              
& $\approx 71{,}7 \times 10^6$ & 93.20  & $\approx 1{,}491{,}000$ \\

anthropic/claude-haiku-4.5     
& $\approx 71{,}7 \times 10^6$ & 243.75 & $\approx 3{,}900{,}000$ \\

google/gemini-2.5-flash        
& $\approx 71{,}7 \times 10^6$ & 116.14 & $\approx 1{,}858{,}000$ \\

deepseek/deepseek-v3.2         
& $\approx 71{,}7 \times 10^6$ & 24.95  & $\approx   399{,}000$ \\

nvidia/llama-3.3-nemotron-super-49b-v1.5 
& $\approx 71{,}7 \times 10^6$ & 20.07  & $\approx   321{,}000$ \\

google/gemma-3n-e4b-it         
& $\approx 71{,}7 \times 10^6$ & 2.29   & $\approx    37{,}000$ \\
\midrule
Total enam model berbayar 
& -- & 500.40 & $\approx 8{,}006{,}000$ \\
\bottomrule
\end{tabular}
\end{table}

Tiga model lain yang tersedia sebagai \textit{free-tier}
(grok-4.1-fast, nemotron-nano-12b-v2-vl, dan bert-nebulon-alpha)
diperkirakan mengonsumsi token serupa tetapi tidak menimbulkan biaya
finansial langsung. Status \textit{free-tier} tersebut tetap harus
diverifikasi kembali sebelum eksperimen akhir dijalankan.

Dengan demikian, estimasi total biaya finansial untuk menjalankan seluruh
eksperimen multi-model, multi-persona, dan dua \textit{benchmark} penalaran
adalah sekitar 500,40 USD atau kurang lebih 8 juta rupiah. Angka ini
bersifat konservatif karena telah memasukkan biaya \textit{warm-up}
dan \textit{retry}, sehingga realisasi biaya dapat lebih rendah apabila
konsumsi token aktual per soal ternyata lebih kecil dari asumsi yang
digunakan dalam perhitungan ini.


Jelaskan secara detail langkah-langkah rencana selanjutnya, hal-hal yang diperlukan atau akan disiapkan, dan risiko dan mitigasinya, yang meliputi:
\begin{enumerate}
\item	Rencana implementasi, termasuk alat dan bahan yang diperlukan, lingkungan, konfigurasi, biaya, dan sebagainya.
\item	Desain pengujian dan evaluasi, misalnya metode verifikasi dan validasi.
\item	Analisis risiko dan mitigasi, misalnya tindakan selanjutnya jika ada yang tidak berjalan sesuai rencana.
\end{enumerate}

\backmatter



% ==========================================
% DAFTAR PUSTAKA
% ==========================================
\nocite{*}

\printbibliography[title={DAFTAR PUSTAKA}]

% ==========================================
% LAMPIRAN (optional)
% ==========================================
\appendix
% Uncomment baris di bawah ini jika ada lampiran
% \input{Lampiran-A.tex}
% \input{Lampiran-B.tex}

\end{document}
 