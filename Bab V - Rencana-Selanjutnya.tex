% ==========================================
% BAB V RENCANA SELANJUTNYA
% ==========================================
\chapter{RENCANA SELANJUTNYA}
\label{chap:rencana-selanjutnya}



\section{Rencana Implementasi dan Estimasi Biaya}
\label{sec:rencana-implementasi-biaya}

Rencana implementasi pada tahap berikutnya adalah menjalankan kembali
\textit{evaluation pipeline} yang telah dijelaskan pada Bab~IV dengan cakupan
penuh, mencakup sembilan model bahasa, dua \textit{benchmark} penalaran
(GSM8K dan MMLU-Redux), serta lima belas \textit{user persona} (implisit,
eksplisit, dan netral). Bagian ini merumuskan langkah implementasi teknis,
asumsi kebutuhan token, serta estimasi biaya penggunaan API berdasarkan harga
resmi masing-masing model pada platform OpenRouter%
\footcite{openrouter_gpt5mini,openrouter_claudehaiku45,openrouter_gemini25flash,openrouter_deepseek32,openrouter_llama33,openrouter_gemma3n}

Estimasi dilakukan menggunakan kurs konstan 1 USD = Rp16.000.

% -------------------------------------------------------------
% 5.1.1 Rencana Implementasi Eksperimen
% -------------------------------------------------------------
\subsection{Rencana Implementasi Eksperimen}

Implementasi eksperimen direncanakan mengikuti enam langkah utama berikut.

\begin{enumerate}
  \item Persiapan aset data.

  Sistem memuat berkas definisi 15 persona, korpus GSM8K (split
  \textit{test}), MMLU-Redux (20 subjek), kredensial API, dan konfigurasi
  model. Struktur direktori dan modul pemrosesan mengikuti rancangan pada
  Subbab~\ref{subsec:organisasi-direktori}.

  \item Inisialisasi dan \textit{warm-up} persona.

  Setiap model menerima satu pesan awal untuk menanamkan konteks persona
  sebelum mengerjakan soal pertama. Tahap ini juga berfungsi sebagai
  \textit{sanity check} bahwa model mengikuti identitas dan gaya bahasa
  persona.

  \item Eksekusi eksperimen utama.

  Setiap kombinasi model--persona menjalankan seluruh soal GSM8K dan
  MMLU-Redux menggunakan mekanisme injeksi pesan berbasis peran:
  persona pada \textit{system message} dan soal pada \textit{user message}.
  Setiap respons diharuskan mencakup penalaran langkah demi langkah.

  \item Pencatatan log granular.

  Setiap respons disimpan sebagai log JSON yang memuat isi \textit{prompt},
  jawaban mentah, \textit{token usage}, dan \textit{latency}.

  \item Agregasi dan validasi hasil.

  Data log diubah menjadi CSV agregat yang berisi akurasi, rata-rata
  latensi, dan konsumsi token. Validasi tambahan mencakup pemeriksaan pola
  jawaban dan konsistensi jumlah entri.

  \item Penanganan kegagalan.

  Kegagalan akibat \textit{timeout} atau batas \textit{rate limit} ditangani
  menggunakan \textit{retry} dengan \textit{exponential backoff}, sesuai
  mekanisme pada Bab~IV. Dengan demikian, kegagalan sebagian tidak
  mengganggu keseluruhan eksperimen.
\end{enumerate}

% -------------------------------------------------------------
% 5.1.2 Himpunan Model dan Skenario Eksekusi
% -------------------------------------------------------------
\subsection{Himpunan Model dan Skenario Eksekusi}

Eksperimen ini menggunakan sembilan model dengan rincian sebagai berikut.

\begin{enumerate}
  \item Enam model berbayar (via OpenRouter):
  \begin{enumerate}
    \item openai/gpt-5-mini
    \item anthropic/claude-haiku-4.5
    \item google/gemini-2.5-flash
    \item deepseek/deepseek-v3.2
    \item nvidia/llama-3.3-nemotron-super-49b-v1.5
    \item google/gemma-3n-e4b-it
  \end{enumerate}

  \item Tiga model yang pada saat perancangan tersedia sebagai \textit{free-tier}:
  \begin{enumerate}
    \item xai/grok-4.1-fast
    \item nvidia/nemotron-nano-12b-v2-vl
    \item openrouter/bert-nebulon-alpha
  \end{enumerate}
\end{enumerate}

Seluruh sembilan model dijalankan pada konfigurasi penuh: dua
\textit{benchmark} dan lima belas persona. Namun, estimasi biaya hanya
dihitung untuk enam model berbayar.

% -------------------------------------------------------------
% 5.1.3 Asumsi Jumlah Soal dan Kebutuhan Token
% -------------------------------------------------------------
\subsection{Asumsi Jumlah Soal dan Kebutuhan Token}

Kebutuhan token dihitung berdasarkan dua sumber utama:
GSM8K (1319 soal) dan MMLU-Redux (2000 soal).
Pada kedua \textit{benchmark}, model diarahkan untuk memberikan
penalaran lengkap sebelum jawaban akhir, sehingga konsumsi token per soal
diharapkan berada pada kisaran yang relatif tinggi.

\begin{enumerate}
  \item GSM8K.

  Total token per persona per model diestimasikan sebagai:
  \[
    T_{\text{GSM8K}} \approx
    1319 \times 1200
    = 1{,}582{,}800 \text{ token}.
  \]

  \item MMLU-Redux.

  Total token per persona per model diestimasikan sebagai:
  \[
    T_{\text{MMLU}} \approx
    2000 \times 1200
    = 2{,}400{,}000 \text{ token}.
  \]
\end{enumerate}

Total token inti per persona diperoleh dari penjumlahan keduanya:
\[
  T_{\text{base, persona}} =
  1{,}582{,}800 + 2{,}400{,}000
  = 3{,}982{,}800.
\]

Untuk mengakomodasi \textit{warm-up} dan \textit{retry}, digunakan faktor
overhead 20\%:
\[
  T_{\text{persona}} \approx
  1.2 \times 3{,}982{,}800
  = 4{,}779{,}360.
\]

Sehingga total token per model untuk 15 persona adalah:
\[
  T_{\text{model}}
  \approx 15 \times 4{,}779{,}360
  = 71{,}690{,}400
  \approx 71{,}7 \times 10^6.
\]

Komposisi token diasumsikan:
\[
  T_{\text{in}} = 0.4T_{\text{model}},\qquad
  T_{\text{out}} = 0.6T_{\text{model}}.
\]

% -------------------------------------------------------------
% 5.1.4 Estimasi Biaya per Model
% -------------------------------------------------------------
\subsection{Estimasi Biaya per Model}

Harga token per model mengacu pada dokumentasi OpenRouter%
\parencite{openrouter_gpt5mini,openrouter_claudehaiku45,openrouter_gemini25flash,
openrouter_deepseek32,openrouter_llama33,openrouter_gemma3n}.  
Biaya untuk model ke-$m$ dihitung dengan rumus:
\[
  \text{cost}_m =
  p_{\text{in},m} \times \frac{T_{\text{in}}}{10^6}
  +
  p_{\text{out},m} \times \frac{T_{\text{out}}}{10^6},
\]
dengan $p_{\text{in},m}$ dan $p_{\text{out},m}$ adalah harga per satu juta
token untuk \textit{input} dan \textit{output}.

Estimasi berikut menggunakan kurs Rp\,16.000 per USD dan total token
$T_{\text{model}} \approx 71{,}7 \times 10^6$.

\begin{table}[htbp]
\centering
\caption{Estimasi biaya enam model berbayar untuk konfigurasi penuh 15 persona}
\label{tab:estimasi_biaya_model}

\renewcommand{\arraystretch}{1.22}
\begin{tabular}{lccc}
\toprule
Model &
Total token $T_{\text{model}}$ &
Biaya (USD) &
Biaya (Rp) \\
\midrule
openai/gpt-5-mini              
& $\approx 71{,}7 \times 10^6$ & 93.20  & $\approx 1{,}491{,}000$ \\

anthropic/claude-haiku-4.5     
& $\approx 71{,}7 \times 10^6$ & 243.75 & $\approx 3{,}900{,}000$ \\

google/gemini-2.5-flash        
& $\approx 71{,}7 \times 10^6$ & 116.14 & $\approx 1{,}858{,}000$ \\

deepseek/deepseek-v3.2         
& $\approx 71{,}7 \times 10^6$ & 24.95  & $\approx   399{,}000$ \\

nvidia/llama-3.3-nemotron-super-49b-v1.5 
& $\approx 71{,}7 \times 10^6$ & 20.07  & $\approx   321{,}000$ \\

google/gemma-3n-e4b-it         
& $\approx 71{,}7 \times 10^6$ & 2.29   & $\approx    37{,}000$ \\
\midrule
Total enam model berbayar 
& -- & 500.40 & $\approx 8{,}006{,}000$ \\
\bottomrule
\end{tabular}
\end{table}

Tiga model lain yang tersedia sebagai \textit{free-tier}
(grok-4.1-fast, nemotron-nano-12b-v2-vl, dan bert-nebulon-alpha)
diperkirakan mengonsumsi token serupa tetapi tidak menimbulkan biaya
finansial langsung. Status \textit{free-tier} tersebut tetap harus
diverifikasi kembali sebelum eksperimen akhir dijalankan.

Dengan demikian, estimasi total biaya finansial untuk menjalankan seluruh
eksperimen multi-model, multi-persona, dan dua \textit{benchmark} penalaran
adalah sekitar 500,40 USD atau kurang lebih 8 juta rupiah. Angka ini
bersifat konservatif karena telah memasukkan biaya \textit{warm-up}
dan \textit{retry}, sehingga realisasi biaya dapat lebih rendah apabila
konsumsi token aktual per soal ternyata lebih kecil dari asumsi yang
digunakan dalam perhitungan ini.


Jelaskan secara detail langkah-langkah rencana selanjutnya, hal-hal yang diperlukan atau akan disiapkan, dan risiko dan mitigasinya, yang meliputi:
\begin{enumerate}
\item	Rencana implementasi, termasuk alat dan bahan yang diperlukan, lingkungan, konfigurasi, biaya, dan sebagainya.
\item	Desain pengujian dan evaluasi, misalnya metode verifikasi dan validasi.
\item	Analisis risiko dan mitigasi, misalnya tindakan selanjutnya jika ada yang tidak berjalan sesuai rencana.
\end{enumerate}