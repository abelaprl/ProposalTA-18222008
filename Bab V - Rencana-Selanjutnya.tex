% ==========================================
% BAB V RENCANA SELANJUTNYA
% ==========================================
\chapter{RENCANA SELANJUTNYA}
\label{chap:rencana-selanjutnya}


Bab ini menjelaskan langkah-langkah yang akan dilakukan pada tahap pelaksanaan, mulai dari menjalankan \textit{evaluation pipeline} yang telah dirancang hingga penyusunan analisis akhir. Penjelasan difokuskan pada bagaimana konfigurasi \textit{multi model}, \textit{multi persona}, dan \textit{multi benchmark} akan dieksekusi secara sistematis, serta bagaimana hasil yang diperoleh akan dicatat dan dievaluasi.

Selain itu, bab ini juga menguraikan estimasi kebutuhan sumber daya dan biaya penggunaan API, rencana penjadwalan pengerjaan Tugas Akhir, serta risiko teknis dan metodologis yang mungkin muncul selama pelaksanaan eksperimen. 
\section{Rencana Implementasi dan Estimasi Biaya} 
\label{sec:rencana-implementasi-biaya}

Rencana implementasi pada tahap berikutnya adalah menjalankan kembali
\textit{evaluation pipeline} yang telah dijelaskan pada Bab~IV dengan cakupan
penuh, yang meliputi sembilan model bahasa, dua \textit{benchmark} penalaran
(GSM8K dan MMLU-Redux), serta lima belas \textit{user persona} (implisit,
eksplisit, dan netral). Bagian ini merumuskan langkah implementasi teknis,
asumsi kebutuhan token, serta estimasi biaya penggunaan API berdasarkan harga
resmi masing-masing model pada platform OpenRouter%

Estimasi dilakukan menggunakan kurs konstan 1 USD = Rp16.700.

% -------------------------------------------------------------
% 5.1.1 Rencana Implementasi Eksperimen
% -------------------------------------------------------------
% -------------------------------------------------------------
% 5.1.1 Rencana Implementasi Eksperimen
% -------------------------------------------------------------
\subsection{Rencana Implementasi Eksperimen}

Pelaksanaan eksperimen direncanakan mengikuti enam langkah utama berikut.

\begin{enumerate}
  \item Persiapan aset data.\\
  Sistem memuat berkas definisi lima belas persona, korpus GSM8K (split
  \textit{test}), MMLU-Redux (20 subjek), kredensial API, serta konfigurasi
  model. Struktur direktori dan modul pemrosesan mengikuti rancangan pada
  Subbab~\ref{sec:perancangan-data-struktur}.

  \item Inisialisasi dan \textit{warm-up} persona.\\
  Setiap model menerima satu pesan awal untuk menanamkan konteks persona
  sebelum mengerjakan soal pertama. Tahap ini juga berfungsi sebagai
  \textit{sanity check} untuk memastikan bahwa model mengikuti identitas dan
  gaya bahasa persona secara konsisten.

  \item Eksekusi eksperimen utama.\\
  Setiap kombinasi model--persona menjalankan seluruh soal GSM8K dan
  MMLU-Redux menggunakan mekanisme injeksi pesan berbasis peran:
  persona pada \textit{system message} dan soal pada \textit{user message}.
  Setiap respons diharuskan menyertakan penalaran langkah demi langkah.

  \item Pencatatan log granular.\\
  Seluruh respons disimpan sebagai berkas JSON yang memuat isi \textit{prompt},
  jawaban mentah, \textit{token usage}, serta \textit{latency}. Format ini
  memastikan bahwa setiap respons dapat ditelusuri kembali ke konfigurasi yang
  digunakan.

  \item Agregasi dan validasi hasil.\\
  Log yang terkumpul diubah menjadi berkas CSV agregat yang berisi akurasi,
  rata-rata latensi, serta total konsumsi token. Validasi tambahan dilakukan
  melalui pemeriksaan pola jawaban dan konsistensi jumlah entri.

  \item Penanganan kegagalan.\\
  Kegagalan akibat \textit{timeout} atau batas \textit{rate limit} ditangani
  menggunakan mekanisme \textit{retry} dengan \textit{exponential backoff},
  sebagaimana dijelaskan pada Bab~IV. Dengan demikian, kegagalan sebagian
  tidak menghentikan keseluruhan eksperimen.
\end{enumerate}



% -------------------------------------------------------------
% 5.1.2 Himpunan Model dan Skenario Eksekusi
% -------------------------------------------------------------
\subsection{Himpunan Model dan Skenario Eksekusi}

Eksperimen ini menggunakan sembilan model dengan rincian sebagai berikut.

\begin{enumerate}
  \item Enam model berbayar (via OpenRouter):
  \begin{enumerate}
    \item openai/gpt-5-mini
    \item qwen/qwen3-vl-30b-a3b-instruct
    \item google/gemini-2.5-flash
    \item deepseek/deepseek-v3.2
    \item nvidia/llama-3.3-nemotron-super-49b-v1.5
    \item google/gemma-3n-e4b-it
  \end{enumerate}

  \item Tiga model yang pada saat perancangan tersedia sebagai \textit{free-tier}:
  \begin{enumerate}
    \item xai/grok-4.1-fast
    \item nvidia/nemotron-nano-12b-v2-vl
    \item openrouter/bert-nebulon-alpha
  \end{enumerate}
\end{enumerate}

Seluruh sembilan model dijalankan pada konfigurasi penuh: dua
\textit{benchmark} dan lima belas persona. Namun, estimasi biaya hanya
dihitung untuk enam model berbayar.

% -------------------------------------------------------------
% 5.1.3 Asumsi Jumlah Soal dan Kebutuhan Token
% -------------------------------------------------------------
\subsection{Asumsi Jumlah Soal dan Kebutuhan Token}

Kebutuhan token dihitung berdasarkan dua sumber utama:
GSM8K (1319 soal) dan MMLU-Redux (2000 soal).
Pada kedua \textit{benchmark}, model diarahkan untuk memberikan
penalaran lengkap sebelum jawaban akhir, sehingga konsumsi token per soal
diharapkan berada pada kisaran yang relatif tinggi.

\begin{enumerate}
  \item GSM8K.

  Total token per persona per model diestimasikan sebagai:
  \[
    T_{\text{GSM8K}} \approx
    1319 \times 1200
    = 1{,}582{,}800 \text{ token}.
  \]

  \item MMLU-Redux.

  Total token per persona per model diestimasikan sebagai:
  \[
    T_{\text{MMLU}} \approx
    2000 \times 1200
    = 2{,}400{,}000 \text{ token}.
  \]
\end{enumerate}

Total token inti per persona diperoleh dari penjumlahan keduanya:
\[
  T_{\text{base, persona}} =
  1{,}582{,}800 + 2{,}400{,}000
  = 3{,}982{,}800.
\]

Untuk mengakomodasi \textit{warm-up} dan \textit{retry}, digunakan faktor
overhead 20\%:
\[
  T_{\text{persona}} \approx
  1.2 \times 3{,}982{,}800
  = 4{,}779{,}360.
\]

Sehingga total token per model untuk 15 persona adalah:
\[
  T_{\text{model}}
  \approx 15 \times 4{,}779{,}360
  = 71{,}690{,}400
  \approx 71{,}7 \times 10^6.
\]

Komposisi token diasumsikan:
\[
  T_{\text{in}} = 0.4T_{\text{model}},\qquad
  T_{\text{out}} = 0.6T_{\text{model}}.
\]
% -------------------------------------------------------------
% 5.1.4 Estimasi Biaya per Model
% -------------------------------------------------------------
\subsection{Estimasi Biaya per Model}

Harga token per model mengacu pada dokumentasi OpenRouter%
\parencite{openrouter_gpt5mini,openrouter_qwen3vl30ba3b,openrouter_gemini25flash,
openrouter_deepseek32,openrouter_llama33,openrouter_gemma3n}.  
Biaya untuk model ke-$m$ dihitung dengan rumus:
\[
  \text{cost}_m =
  p_{\text{in},m} \times \frac{T_{\text{in}}}{10^6}
  +
  p_{\text{out},m} \times \frac{T_{\text{out}}}{10^6},
\]
dengan $p_{\text{in},m}$ dan $p_{\text{out},m}$ adalah harga per satu juta
token untuk \textit{input} dan \textit{output}.

Estimasi berikut menggunakan kurs Rp\,16.700 per USD dan total token
$T_{\text{model}} \approx 71{,}7 \times 10^6$.

\begin{table}[htbp]
\centering
\caption{Estimasi biaya enam model berbayar untuk konfigurasi penuh 15 persona}
\label{tab:estimasi_biaya_model}

\renewcommand{\arraystretch}{1.22}
\footnotesize
\begin{adjustbox}{max width=\textwidth}
\begin{tabular}{l c c c}
\toprule
\textbf{Model} &
\textbf{Total Token} $T_{\text{model}}$ &
\textbf{Biaya (USD)} &
\textbf{Biaya (Rp)} \\
\midrule
openai/gpt-5-mini & $\approx 71{,}7 \times 10^6$ & 93.20  & $\approx 1{,}556{,}000$ \\
qwen/qwen3-vl-30b-a3b-instruct & $\approx 71{,}7 \times 10^6$ & 30.11 & $\approx 503{,}000$ \\
google/gemini-2.5-flash & $\approx 71{,}7 \times 10^6$ & 116.14 & $\approx 1{,}939{,}000$ \\
deepseek/deepseek-v3.2 & $\approx 71{,}7 \times 10^6$ & 24.95  & $\approx 417{,}000$ \\
nvidia/llama-3.3-nemotron-super-49b-v1.5 & $\approx 71{,}7 \times 10^6$ & 20.07  & $\approx 335{,}000$ \\
google/gemma-3n-e4b-it & $\approx 71{,}7 \times 10^6$ & 2.29   & $\approx 38{,}000$ \\
\midrule
\textbf{Total enam model berbayar} & -- & \textbf{286.76} & $\approx \textbf{4{,}788{,}000}$ \\
\bottomrule
\end{tabular}
\end{adjustbox}
\end{table}

Tiga model lain yang tersedia sebagai \textit{free-tier}
(grok-4.1-fast, nemotron-nano-12b-v2-vl, dan bert-nebulon-alpha)
diperkirakan mengonsumsi token serupa tetapi tidak menimbulkan biaya
finansial langsung. Status \textit{free-tier} tersebut tetap harus
diverifikasi kembali sebelum eksperimen akhir dijalankan.

Dengan demikian, estimasi total biaya finansial untuk menjalankan seluruh
eksperimen multi-model, multi-persona, dan dua \textit{benchmark} penalaran
adalah sekitar \textbf{286.76 USD}, atau kurang lebih
\textbf{4,8 juta rupiah}. Angka ini bersifat konservatif karena telah
memasukkan biaya \textit{warm-up}, \textit{retry}, dan variasi panjang
jawaban, sehingga realisasi biaya dapat lebih rendah apabila konsumsi token
aktual per soal ternyata lebih kecil atau lebih besar dari asumsi yang digunakan dalam
perhitungan ini.

%------------------- BAB V.2 Rencana Pengerjaan dan Pengembangan -------------------

\subsection{Rencana Pengerjaan dan Pengembangan}

Pelaksanaan Tugas Akhir ini dirancang dalam beberapa tahapan yang saling
berurutan, mencakup penyusunan kerangka eksperimen, pengembangan
\textit{evaluation pipeline}, pelaksanaan evaluasi pada berbagai model dan
persona, hingga analisis hasil. Mengingat cakupan eksperimen yang luas
(\textit{multi model} × \textit{multi persona}), durasi pengerjaan diperpanjang
untuk memastikan seluruh proses dapat berjalan stabil dan menghasilkan keluaran
yang dapat dianalisis secara komprehensif. Rincian tahapan pelaksanaan Tugas
Akhir ditampilkan pada Tabel~\ref{tab:rencana-pengerjaan} berikut.

\begin{table}[H]
\centering
\caption{Rencana tahapan pelaksanaan Tugas Akhir }
\begin{tabular}{p{5cm} p{3cm} p{6cm}}
\toprule
\textbf{Kegiatan} & \textbf{Durasi} & \textbf{Output Utama} \\
\midrule

Studi literatur lanjutan dan perumusan kerangka eksperimen &
4 minggu &
Pemutakhiran tinjauan pustaka, definisi metodologi, pemetaan risiko bias
persona, serta penyusunan struktur awal \textit{evaluation pipeline}. \\

Pengembangan dan implementasi \textit{evaluation pipeline} (multi-model) &
6 minggu &
Modul pemanggilan model, integrasi API, modul persona, modul benchmark,
mekanisme \textit{logging}, serta verifikasi format keluaran model. \\

Pengujian internal, \textit{dry-run}, dan penyempurnaan pipeline &
2 minggu &
Pipeline stabil dan konsisten, validasi batas token,
pengujian kesalahan, serta standarisasi format data untuk analisis. \\

Pelaksanaan eksperimen penuh (persona × model × benchmark) &
4 minggu &
Dataset hasil eksperimen yang lengkap: keluaran GSM8K dan MMLU-Redux,
statistik token, latensi, tingkat keberhasilan respons, dan log eksekusi. \\

Replikasi eksperimen dan verifikasi ulang hasil &
1--2 minggu &
Eksperimen ulang untuk meningkatkan reliabilitas, mengurangi noise,
dan memastikan konsistensi antar-konfigurasi. \\

Analisis hasil dan penyusunan bagian laporan &
3 minggu &
Visualisasi hasil, analisis kuantitatif dan kualitatif,
interpretasi pengaruh persona terhadap performa model, serta
penulisan laporan akhir Tugas Akhir. \\

\bottomrule
\end{tabular}
\label{tab:rencana-pengerjaan}
\end{table}

\section{Desain Pengujian dan Evaluasi}
\label{sec:desain-pengujian-evaluasi}

Desain pengujian disusun untuk memastikan bahwa seluruh hasil eksperimen dapat
diverifikasi, divalidasi, dan direplikasi. Pengujian memanfaatkan artefak log
granular, telemetry token, dan pemeriksaan konsistensi yang telah ditanamkan
dalam pipeline pada Bab~IV.

\begin{enumerate}

  \item Verifikasi konsistensi eksekusi

    Tahap ini memastikan bahwa setiap model menerima stimulus yang identik pada
    seluruh soal dan persona sehingga variasi performa dapat dikaitkan secara
    langsung dengan persona atau arsitektur model.

    \begin{enumerate}[label=(\alph*)]
      \item Konsistensi konstruksi prompt

            Pemeriksaan memastikan bahwa struktur persona pada system message dan
            konten soal pada user message identik pada semua eksekusi. Variasi kecil
            seperti perbedaan tanda baca dapat mengubah penalaran model sehingga
            verifikasi dilakukan secara programatik melalui log JSON.

      \item Kesesuaian urutan eksekusi

            Urutan indeks interaksi, nomor soal, dan urutan persona diperiksa untuk
            memastikan pipeline mengikuti konfigurasi eksperimen.

      \item Keberhasilan tahap warm-up

            Respons awal model dinilai untuk melihat apakah gaya bahasa persona sudah
            terserap dengan benar. Kegagalan dianggap anomali dan dieksekusi ulang.
    \end{enumerate}

  \item Validasi keluaran model

    Validasi memastikan bahwa keluaran model memiliki format yang dapat
    dievaluasi secara otomatis pada GSM8K dan MMLU-Redux.

    \begin{enumerate}[label=(\alph*)]
      \item Validasi GSM8K

            Model harus menghasilkan jawaban numerik akhir yang dapat diekstraksi
            secara deterministik dan menyertakan penalaran langkah demi langkah.

      \item Validasi MMLU-Redux

            Model harus memilih salah satu opsi A, B, C, atau D, serta memberikan
            penjelasan penalaran sebelum menentukan jawaban.

      \item Pemeriksaan konsistensi format respons

            Pemeriksaan mencakup panjang teks, struktur, keterbacaan, dan kesesuaian
            format untuk menghindari kesalahan parsing.
    \end{enumerate}

  \item Evaluasi kuantitatif

    Evaluasi kuantitatif dilakukan untuk mengukur dampak persona terhadap
    performa model.

    \begin{enumerate}[label=(\alph*)]
      \item Akurasi jawaban

            Akurasi dihitung dengan membandingkan jawaban akhir terhadap ground
            truth dan diagregasi per model dan per persona.

      \item Konsumsi token

            Analisis mencakup token input, token output, dan token penalaran sebagai
            indikator beban komputasi serta kecenderungan verbosity.

      \item Latensi eksekusi

            Latensi dihitung berdasarkan timestamp pada log JSON untuk menilai
            stabilitas waktu respons pada eksekusi berskala besar.
    \end{enumerate}

\end{enumerate}


%========= V.3 Analisis Risiko dan Mitigasi ==========
\section{Analisis Risiko dan Mitigasi}

Pelaksanaan eksperimen pada lingkungan multi-model dan multi-persona menimbulkan
sejumlah risiko yang perlu dikelola untuk menjaga integrist hasil Tugas Akhir.
Risiko terutama berkaitan dengan reliabilitas API, stabilitas keluaran model,
dan konsistensi penyimpanan log.

\begin{enumerate}

  \item Risiko kegagalan pemanggilan API

    Risiko meliputi timeout, gangguan koneksi, dan rate limit yang dapat
    menyebabkan hilangnya data atau ketidaksinkronan indeks eksekusi.

    \begin{enumerate}[label=(\alph*)]
      \item Penggunaan mekanisme retry adaptif berbasis exponential backoff

      \item Pencatatan seluruh galat dalam log terpisah sehingga dapat dilakukan
            re-eksekusi selektif

      \item Penurunan tingkat konkurensi secara otomatis ketika laju galat
            meningkat untuk menjaga stabilitas
    \end{enumerate}

  \item Risiko lonjakan konsumsi token

    Respons model dapat menjadi terlalu panjang, terutama ketika diminta
    memberikan penalaran eksplisit, sehingga meningkatkan biaya dan durasi
    eksperimen.

    \begin{enumerate}[label=(\alph*)]
      \item Penetapan batas maximum completion length

      \item Pemantauan berkala terhadap rata-rata konsumsi token

      \item Penyesuaian minimal pada instruksi persona yang memicu keluaran
            berlebihan
    \end{enumerate}

  \item Risiko penyimpanan dan konsistensi log

    Volume log yang besar meningkatkan risiko korupsi berkas dan ketidaksesuaian
    antara indeks model, persona, dan soal.

    \begin{enumerate}[label=(\alph*)]
      \item Penyimpanan respons dalam format JSON dengan skema tetap

      \item Pemeriksaan silang jumlah entri, indeks soal, dan struktur pipeline
            selama agregasi

      \item Penerapan checkpointing untuk mencegah kehilangan data jika eksekusi
            terhenti mendadak
    \end{enumerate}

\end{enumerate}
